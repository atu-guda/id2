\documentclass[13pt,d,ruslang]{vakthesisatu}
%\documentclass[14pt]{dissatu}

\usepackage[verbose,showframe,a4paper,tmargin=20mm,bmargin=20mm,lmargin=25mm,rmargin=20mm,headsep=4pt]{geometry}
\newlength\TW
\setlength{\TW}{0.01\textwidth} % after geometry!

\usepackage[backend=biber,language=russian,sorting=none,maxnames=3,bibstyle=gost-numeric]{biblatex}

\usepackage{blox}
\usepackage[europeanresistors,americaninductors,siunitx,fulldiodes]{circuitikz}

\usetikzlibrary{calc}
\usetikzlibrary{arrows}
\usetikzlibrary{patterns}
%\usepgflibrary{shapes.geometric}

\tikzset{
  >=stealth,
  %semiRed/.style={fill=red,opacity=0.3,draw=black,thin},
  medline/.style={draw=black,line width=0.6pt},
  medlinep/.style={draw=black,line width=0.6pt,->},
  semiboldline/.style={draw=black,line width=1.2pt},
  boldline/.style={draw=black,line width=2.0pt},
  wire/.style={draw=black,line width=1.0pt},
  elelem/.style={draw=black,line width=1.5pt}
}



%\setstretch{1.2}

\newcommand{\bookname}{Моделі та  методи адаптивно-пошукової ідентифікації систем з хаотичною динамікою}
\newcommand{\bookyear}{2016}
\newcommand{\dissauthor}{Гуда~А.И.}
\renewcommand{\Rada}{Д.~08.084.01}
\renewcommand{\SekrRadi}{Селівьорстова~Т.В.}

\title{\bookname}
\author{Гуда Антон Игоревич}
\supervisor{Михалёв Александр Ильич}{д.т.н., проф.}
\speciality[математичне моделювання та обчислювальні методи]{01.05.02}[технических наук]
\udc{681.5.015}
\date{\bookyear}
\institution{Национальная металлургическая академия Украины, Министерство образования и науки Украины}{Днепропетровск}

%\author{\dissauthor}

\addbibresource{atuworks.bib}
\addbibresource{artrefs.bib}

\hypersetup{
  pdftitle={\bookname},
  pdfauthor={\dissauthor},
  colorlinks=true,
  linkcolor=blue,
  citecolor=brown
}

%\newcommand{\LinkRef}[1]{ \textit{\color{red}#1} }
\newcommand{\LinkRef}[1]{}
\newcommand{\Cmt}[1]{ {\small\color{red}#1} }
%\newcommand{\Cmt}[1]{}

\clubpenalty9990
\widowpenalty9990%

%\pagestyle{stheadings}

% ====================================================================================================
\begin{document}

\thispagestyle{empty}
% \hyperref[titul]{Titul}
\phantomsection
\pdfbookmark[0]{Titul}{titulbookmark}
%\Xmaketitle

\begin{center}

\institutionMain

\belongMain

\end{center}

\vspace{3ex}


\begin{flushright}
{
Кваліфікаційна наукова \\
праця на правах рукопису
}
\end{flushright}

\vspace{3ex}

\begin{center}
\dissauthorFullMain
\end{center}

\vspace{2ex}

\begin{flushright}
УДК \UDC
\end{flushright}

%\vspace{3ex}
\vspace{\stretch{2}}

\begin{center}
{\large\bfseries\scshape ДИСЕРТАЦІЯ }

\vspace{3ex}

{\large\scshape \bookname}

\vspace{\stretch{2}}

\dissSpecId\ --- \dissSpecMain

\vspace{1ex}

технічні науки

\vspace{3ex}

Подається на здобуття наукового ступеня доктора наук

\end{center}%

\vspace{2ex}

{
\noindent
\small
Дисертація містить результати власних досліджень. Використання ідей,
результатів і текстів інших авторів мають посилання на відповідне джерело.\\
\_\_\_\_\_\_\_\_\_\_ А.І. Гуда
}

\vspace{3ex}

\noindent
Науковий консультант \\
\superUa, доктор технічних наук, професор

\vspace{1ex}

\noindent
{\scriptsize
  \parbox[t]{0.47\textwidth}{%
    \CYRC\cyre\cyrishrt\ \cyrp\cyrr\cyri\cyrm\cyrii\cyrr\cyrn\cyri\cyrk\ %
      \cyrd\cyri\cyrs\cyre\cyrr\cyrt\cyra\cyrc\cyrii\cyrishrt\cyrn\cyro\cyryi\ \cyrr\cyro\cyrb\cyro\cyrt\cyri\ %
      \cyrii\cyrd\cyre\cyrn\cyrt\cyri\cyrch\cyrn\cyri\cyrishrt\ \cyrz\cyra\ \cyrz\cyrm\cyrii\cyrs\cyrt\cyro\cyrm\ %
      \cyrz\ \cyrii\cyrn\cyrsh\cyri\cyrm\cyri,\ \cyrp\cyro\cyrd\cyra\cyrn\cyri\cyrm\cyri\ %
      \cyrd\cyro\ \cyrs\cyrp\cyre\cyrc\cyrii\cyra\cyrl\cyrii\cyrz\cyro\cyrv\cyra\cyrn\cyro\cyryi\ %
      \cyrv\cyrch\cyre\cyrn\cyro\cyryi\ \cyrr\cyra\cyrd\cyri\ \Rada%
  }%
}%

\vspace{1ex}%
\noindent
{\scriptsize
  \CYRV\cyrch\cyre\cyrn\cyri\cyrishrt\ \cyrs\cyre\cyrk\cyrr\cyre\cyrt\cyra\cyrr\ %
    \cyrs\cyrp\cyre\cyrc\cyrr\cyra\cyrd\cyri\ \Rada%
    \hfill%
    \SekrRadi%
}

\vspace{2ex}%

\begin{center}
\cityMain\ --- \bookyear
\end{center}



% \begin{center}
% \institution
%   {\Large Титульная страница}
%   \vfill
%   \bookyear
% \end{center}


\clearpage
\tableofcontents


% ----------------------------------------------------------------------------------------------------
\subimport{p0/}{p0.tex} % p0/p0.tex  % введение

\subimport{p1/}{p1.tex} % p1/p1.tex  % обзор

\subimport{p2/}{p2.tex} % p2/p2.tex  % критерии

\subimport{p3/}{p3.tex} % p3/p3.tex  % методы

\subimport{p4/}{qontrol.tex} % p4/qontrol.tex

\subimport{p5/}{testsys.tex} % p5/testsys.tex

\subimport{p6/}{colp_real.tex} % p6/colp_real.tex

\subimport{p7/}{relax_real.tex} % p7/relax_real.tex

\clearpage
\phantomsection
\chapter*{Выводы}


У роботі вирішена науково-технічна проблема ідентифікації параметрів складних технічних
систем у режимі хаотичної динаміки
з метою забезпечення їх керованої поведінки. При цьому:

\begin{itemize}

  \item
    створено нові критерії ідентифікації, які, на відміну від тих, що
    існують, придатні для аналізу стану та динаміки
    хаотичних систем, що створить фізично зумовлене обґрунтування працездатності систем
    ідентифікації;

  \item
    створено новий клас систем ідентифікації у межах
    адаптивно-пошуковой парадигми,
    які за рахунок використання колективної динаміки
    ансамблю пошукових агентів забезпечують
    кращу швидкість ідентифікації без істотного впливу на похибку ідентифікації;

  \item
    проведена перевірка працездатності запропонованих методів
    на прикладах як відомих систем хаотичної динаміки,
    так і на декількох інших динамічних систем, які проявляють
    складну та хаотичну динаміку;

  \item
   визначено, що системи з сухим тертям з точки зору задачі ідентифікації
   при певних  умовах функціонування
   мають властивості, що поєднують їх з системами хаотичної динаміки;

 \item
  створено програмне забезпечення, придатне для моделювання як систем
  хаотичної динаміки, так і систем мультиагентної ідентифікації;

  \item
  проведено як комп'ютерне моделювання процесів ідентифікації систем
  хаотичної динаміки, так і фізичне моделювання таких, що підтверджує адекватність
  побудованих моделей.

\end{itemize}

% В целом, применение рассмотренных видов критериев, в совокупности
% с мультимодельными (мультиагентными) методами поисковой идентификации
% оказалось продуктивным и показало их
% работоспособность для рассмотренных хаотических, и в какой-то мере эквивалентным им
% сложных динамических систем.
%
% Для большинства из рассмотренных систем вид критерия
% был получен практически автоматически, из анализа полученных
% в результате моделирования зависимостей, имеющих прямое или косвенное отношение
% к энергетическим параметрам. Для системы с сухим трением,
% синтез критерия был основан на явных физических зависимостях.
% Созданные критерии позволяют реализовать процесс идентификации для случаев
% сложной или же хаотической динамики,
% когда полностью отсутствуют хоть какая-нибудь информация о
% трендовом поведении объекта, и невозможно получить аналитическую
% или же статистическую связь между параметрами и выходом системы.
%
% Одновременная настройка параметров нескольких моделей ансамблем поисковых агентов
% позволила максимально преодолеть противоречие между скоростью и точностью
% идентификации. При таком подходе время идентификации определяется
% временем реакции идентифицируемой системы на изменение параметра.
% Для систем со сложной, а тем более хаотической динамикой это время достигает
% значительных величин, по сравнению с характерными временами самой динамики системы.
%
%
% Полученные зависимости ошибок идентификации от параметров
% самой системы идентификации, с одной стороны, позволяют
% правильно настроить эти параметры для получения максимально быстрой
% и точной идентификации. С другой стороны, эти зависимости
% характеризуют связь свойств идентифицируемого объекта с
% требуемыми качествами самой системы идентификации.







\printbibliography
\label{e:atu}{~}



\end{document}

