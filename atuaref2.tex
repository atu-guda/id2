\documentclass[a4paper,12pt]{atuaref}
%\documentclass[a4paper,12pt,russian]{atuaref}
\RequirePackage[utf8]{inputenc}
\usepackage[verbose,tmargin=25mm,bmargin=25mm,lmargin=30mm,rmargin=20mm,headsep=10pt]{geometry}
\usepackage{amssymb}
\usepackage{amsmath}
\usepackage{amsthm}
\usepackage{tabularx}
\usepackage{fouriernc}
\usepackage{paratype}

\usepackage{graphicx}
%\usepackage[english,russian,ukraineb]{babel}
\usepackage[english,ukraineb,russian]{babel}

\usepackage[backend=biber,language=russian,sorting=none,maxnames=3,bibstyle=gost-numeric]{biblatex}


\newcommand{\booknameUa}{Ансамблеві пошукові моделі і методи параметричної ідентифікації систем з хаотичною поведінкою}
\newcommand{\booknameRu}{Ансамблевые поисковые модели и методы параметрической идентификации систем с хаотическим поведением}
\newcommand{\booknameEn}{Ensemble search models and methods for parametric identification of systems with chaotic behavior}
\newcommand{\bookname}{\booknameRu}

\newcommand{\bookyear}{2018}
\newcommand{\dissauthorUa}{Гуда~А.І.}
\newcommand{\dissauthorRu}{Гуда~А.И.}
\newcommand{\dissauthorEn}{Guda~A.I.}
\newcommand{\dissauthorFullRu}{Гуда Антон Игоревич}
\newcommand{\dissauthorFullUa}{Гуда Антон Ігорович}
\newcommand{\dissauthorMain}{\dissauthorRu}
\newcommand{\dissauthorAref}{\dissauthorUa}
\newcommand{\dissauthorFullMain}{\dissauthorFullRu}
\newcommand{\dissauthorFullAref}{\dissauthorFullUa}

\newcommand{\dissSpecUa}{математичне    моделювання  та обчислювальні методи}
\newcommand{\dissSpecRu}{математическое моделирование и вычислительные методы}
\newcommand{\dissSpecEn}{Mathematical Modelling and Computational Methods}
\newcommand{\dissSpecMain}{\dissSpecRu}
\newcommand{\dissSpecAref}{\dissSpecUa}
\newcommand{\dissSpecId}{01.05.02}
\newcommand{\dissScopeRu}{технических наук}
\newcommand{\dissScopeUa}{техничних наук}
\newcommand{\dissScopeMain}{\dissScopeRu}
\newcommand{\dissScopeAref}{\dissScopeUa}
\newcommand{\UDC}{004: 681.5.015}
\newcommand{\dissRada}{Д.~08.084.01}
\newcommand{\dissSekrRadi}{Селівьорстова~Т.В.}
\newcommand{\institutionRu}{Национальная металлургическая академия Украины}
\newcommand{\institutionUa}{Національна  металургійна     академія України}
\newcommand{\institutionEn}{National Metallurgical academy of Ukraine}
\newcommand{\institutionMain}{\institutionRu}
\newcommand{\institutionAref}{\institutionUa}
\newcommand{\belongRu}{Министерство образования и науки Украины}
\newcommand{\belongUa}{Міністерство освіти і науки      України}
\newcommand{\belongEn}{Ministry of Education and Science of Ukraine}
\newcommand{\belongMain}{\belongRu}
\newcommand{\belongAref}{\belongUa}
\newcommand{\cityRu}{Днепр}
\newcommand{\cityUa}{Дніпро}
\newcommand{\cityEn}{Dnipro}
\newcommand{\cityMain}{\cityRu}
\newcommand{\cityAref}{\cityUa}
\newcommand{\superRu}{Михалёв Александр Ильич}
\newcommand{\superUa}{Михальов Олександр Ілліч}
\newcommand{\superMain}{\superRu}
\newcommand{\superAref}{\superUa}



\addbibresource{atuworks.bib}

\DeclareMathOperator*{\sign}{sign}

\newcommand{\TermDef}[1]{\textit{\textbf{#1}}}
\newcommand{\TermUse}[1]{\textit{#1}}
\newcommand{\ProgName}[1]{\textsl{#1}}
\newcommand{\xsect}[1]{\medskip\begin{center}\textbf{#1}\end{center}\medskip\penalty10000}
%\newcommand{\xxsect}[1]{\smallskip\begin{center}\textbf{#1}\end{center}\smallskip\penalty10000}
\newcommand{\xxsect}[1]{\smallskip\textbf{#1}\smallskip\penalty10000}


\begin{document}

\sloppy

\thispagestyle{empty}
\begin{center}

\textbf{\belongAref}

\vspace{1ex}

\textbf{\institutionAref}

\vspace{3ex}

\textbf{Гуда Антон Ігорович}

\end{center}

\vspace{3ex}

\begin{flushright}
УДК \UDC
\end{flushright}

\vfill

\begin{center}
\textbf{\Large
\booknameUa
}

\vfill

\dissSpecId --- \dissSpecAref

\vfill

\textbf{Автореферат} \\
\textbf{  дисертації на здобуття наукового ступеня }\\
\textbf{доктора \dissScopeAref }


\vfill

\cityAref --- \bookyear

\end{center}

\clearpage

\thispagestyle{empty}
Дисертацією є рукопис

\vspace{3ex plus 2ex}

Робота виконана в Національній металургійній академії України
Міністерства освіти і науки України.


\vspace{3ex plus 2ex}

\begin{tabular}{lp{0.58\textwidth}}

\textbf{Науковий консультант:}
&
доктор технічних наук, професор,\newline
\textbf{Михальов Олександр Ілліч,}\newline
Національна металургійна академія України,
завідувач кафедри інформаційних технологій і систем; м.~Дніпро.
\\
{~} & {~}
\\
\textbf{Офіційні опоненти:} 
&
доктор технічних наук, професор,\newline
\ldots
\\
{~} & {~}
\\
{~}
&
доктор технічних наук, професор,\newline
\ldots
\\
{~} & {~}
\\
{~}
&
доктор технічних наук, професор,\newline
\ldots
\end{tabular}

\vspace{5ex plus 4ex}


\vfill

Захист відбудеться 
<<\_\_\_\_>>
~\_\_\_\_\_\_\_\_\_\_\_
\bookyear~р. о \_\_\_\_ годині
на засіданні спеціалізованої вченої ради Д 08.084.01 у Національній
металургійній академії України за адресою: 49005, м.~\cityUa,
пр.~Гагаріна,~4.


\vspace{3ex plus 2ex}
З дисертацією можна ознайомитись у бібліотеці Національної металургійної
академії України за адресою: 49600, м.~\cityUa, пр.~Гагаріна,~4.

\vspace{3ex plus 2ex}
Автореферат розісланий 
<<\_\_\_\_>>
~\_\_\_\_\_\_\_\_\_\_\_
\bookyear~р.

\vspace{3ex plus 2ex}

\begin{tabular}{p{0.44\textwidth}p{0.2\textwidth}p{0.35\textwidth}}
Вчений секретар спеціалізованої вченої ради 
&
{~}
&
\SekrRadi
\end{tabular}

\vspace{2ex}

\clearpage


\setcounter{page}{1}

\xsect{ЗАГАЛЬНА ХАРАКТЕРИСТИКА РОБОТИ}

\textbf{Актуальність роботи.}
Нелінійні динамічні системи, широко представлені в сучасних технологічних і
природних процесах, незважаючи на детермінізм їх визначення, можуть проявляти
хаотичні властивості в своїй динаміці. При цьому як завгодно малі збурення у вхідних
впливах і параметрах самої системи призводять до значних, але кінцевим збуренням
вихідного сигналу. Це призводить до певних труднощів при конструюванні,
управлінні і прогнозі поведінки таких систем.

При математичному та комп'ютерному моделюванні систем динамічного хаосу
виникають специфічні для даних систем проблеми. Перш за все -- потрібно
забезпечити наявність працездатного критерію адекватності моделі. Для задач
ідентифікації наявність такого критерію є принциповим. Найбільш часто
використовувані при моделюванні поведінки динамічних систем критерії, засновані
на звичних заходи в просторі вихідних сигналів, виявляються непрацездатними. З
іншого боку, спеціальні характеристики оцінювання хаотичних властивостей динамічних
систем, такі як фрактальна розмірність, показник Ляпунова, перетин Пуанкаре,
недостатньо інформативні для задачі ідентифікації через обмежений діапазон
змін, великою похибкою при їх вимірі для реальних систем і суттєвою
обчислювальною складністю.

У якості прототипу для синтезу нових методів ідентифікації доцільно обрати ряд
методів ідентифікації, які було розроблено та досліджено у роботах
П.~Ейкхоффа, Л.А.~Растрігіна, Л.Н.~Фіцнера, Е.Є.~Гачинського,
А.І.~Дроздова, О.І.~Михальова, зокрема пошукові та адаптивно-пошукові
методи.
Також вважається доцільним використання як основи
наступного класу двомодельних адаптивно-пошукових методів,
які було розроблено та досліджено у попередньої роботі.
Проте, результати моделювання процесів ідентифікації цими методами
технічних об'єктів, які проявляють складну та хаотичну динаміку, показали
їх повну або обмежену непридатність.


Тому проблема ідентифікації динамічних систем, які проявляють хаотичну
динаміку, або близьку до хаотичної, є
\textit{актуальною}.

\smallskip
\textbf{Зв'язок роботи з науковими програмами, планами, темами.}
Дисертаційна робота виконувалась у рамках науково-дослідних робіт
Національної металургійної академії України за держбюджетною
тематикою:

\begin{itemize}

\item XXX

\end{itemize}

Результати було впроваджено у рамках науково-практичного дослідження
``Оцінка    можливості заміни випробувань КА на стійкість до акустичного навантаження
випробуваннями широкосмугової вібрації'', згідно договору №~V-105-16-3 від 07.09.2016.

\textbf{Мета і задачі дослідження.}
Головною метою дослідження є створення нових методів ідентифікації з
використанням адаптивно-пошукових принципів настроювання параметрів, які
були б придатні для створення моделей систем, які проявляють хаотичну
та/або схожу на хаотичну динаміку. В свою чергу, окремо ставиться задача
розв'язання наукової проблеми створення адекватних моделей процесів
ідентифікації хаотичних систем з використанням запропонованих нових
методів, аналізу та дослідження їх характеристик в умовах невизначеності.

Для досягнення даної мети були поставлені наступні задачі:

\begin{itemize}

  \item
  розробити нові критерії ідентифікації, які, на відміну від тих, що
  існують, при моделюванні були б придатні для аналізу стану та динаміки
  хаотичних систем, що створить обґрунтування працездатності систем
  ідентифікації;

  \item
  розвинути існуючи та розробити нові методи пошуку, які б не мали
  обмежень тих, що існують, та у повної мірі використовували можливості
  паралельних обчислювань та переваг використання ансамблю
  синергированних моделей;

  \item
  розвинути існуючи та розробити нові методи адаптації параметрів
  системи ідентифікації, здатні пристосуватися до зміни режимів роботи
  системи;

  \item
  розробити програмне забезпечення, придатне для моделювання як систем
  хаотичної динаміки, так і систем ідентифікації;

  \item
  провести комп'ютерне моделювання процесів ідентифікації систем
  хаотичної динаміки та дослідити їх працездатність, можливості та
  характеристики.

\end{itemize}


\textbf{Об'єктом дослідження є}
технічні системи, які в процесі їх функціювання можуть
входити в хаотичні режими.

\smallskip
\textbf{Предметом дослідження є}
математичні моделі процесів та методи
адаптивно-пошукової ідентифікації технічних систем з хаотичною динамікою.

\smallskip
\textbf{Методи дослідження.}
Для вирішення поставлених задач використовувався математичний апарат
теорії управління та ідентифікації нелінійних систем, динамічного хаосу,
обчислювальних методів, нечіткої логіки, теорії інформації, тощо.

\smallskip
\textbf{Наукова новизна одержаних результатів:}
Основний науковий результат полягає в розробці нових методів ідентифікації
технічних систем хаотичної динаміки, створенні відповідних математичних
моделей та дослідженні результатів моделювання процесів
адаптивно-пошукової ідентифікації.

Основні наукові результати, що отримані в дисертаційної роботі, полягають
 в наступному:

\begin{itemize}

  \item
  уперше
  створено критерії ідентифікації нелінійних динамічних систем,
  які, на відміну від тих, що існують, дозволяють оцінити їх стан та
  хаотичну динаміку, та дають підстави для створення ефективних алгоритмів
  настроювання параметрів моделей систем ідентифікації;

  \item
  уперше
  створено методи адаптивно-пошукової ідентифікації на підставі
  адаптивно-пошукової парадигми з використанням ансамблю пошукових агентів,
  які взаємодіють проміж собою, які на відміну від методів, що використовують
  одну модель або пару моделей, значно підвищують швидкість пошуку та
  здатні за мінімальний час перестрілюватися при різкої зміні параметра, а на
  відміну від ройових алгоритмів, нові методи потребують значно меншої
  кількості моделей та забезпечують певні гарантії пошуку;

  \item
  уперше
  створено нову класифікацію систем ідентифікації динамічних систем,
  яка як вбирає у себе методи, що існують, так і дозволяє
  створювати нові методи ідентифікації за рахунок
  комбінування їх складових частин;

  \item
  набуло подальший розвиток
  методи оцінювання якості ідентифікації,
  які на відміну від тих, що існують,
  враховують використання множини агентів;


  \item
  набули подальший розвиток
  підходи до адаптації параметрів систем
  адаптивно-пошукової ідентифікації, які придатні використовувати поточну
  інформацію від ансамблю сінергированих моделей, та корегувати глобальні
  параметри пошуку;

  \item
   уперше
   визначено, що системи з сухим тертям з точки зору задачі ідентифікації
   при певних  умовах функціонування
   мають властивості, що поєднують їх з системами хаотичної динаміки, тобто з
   системами хаотичної динаміки їх поєднує суттєва залежність від початкових
   умов та вид атрактору, що також потребує використання нових методів ідентифікації;

  \item
    набула подальший розвиток
    модель генератора Копітца, яка враховує
    більшу кількість нелінійних ефектів,
    що забезпечує більш адекватні результати процесу
    ідентифікації її параметрів новими методами;

  \item
   уперше
   запропоновано модель системи хаотичної динаміки системи зв'язаних релаксаційних генераторів,
   яка відрізняєшся від існуючих відсутністю індуктивних компонентів,
   працездатністю при малих напругах та можливістю
   керування частотним діапазоном у широкому діапазоні,
   що сприяє процесу аналізу хаотичної динаміки
   фізичного об'єкту, перевірки адекватності математичної моделі
   та властивостей системи ідентифікації стосовно цієї системи.


\end{itemize}


\smallskip
\textbf{Практичне значення одержаних результатів.}
Розроблені методи ідентифікації було використано
при проектуванні, створенні, налаштовуванні параметрів
стенду дослідження вібраційного та акустичного впливу.
Аналіз результатів даних з цього стенду
дав можливість указати потрібні нелінійні властивості системи,
та діапазон параметрів, які у сукупності
забезпечують широкосмуговий спектр коливань.

Створене програмне середовище для моделювання нелінійних динамічних систем
використається при проведенні практичних робіт по дисциплінам
``Моделювання систем'',
``Сучасні системи управління'' на кафедри інформаційних технологій
і систем Національної металургійної академії України.


\smallskip
\textbf{Особистий внесок здобувача.}

Усі основні положення і результати
дисертаційної роботи, які виносяться на захист, отримано здобувачем особисто та
опубліковано в роботах [1--46]. У наукових працях, опублікованих у співавторстві,
здобувачу належать наступні результати. У роботі [1] \ldots



\smallskip
\textbf{Апробація результатів.}
Основні положення дисертаційної роботи доповідались на наукових
семінарах кафедри ІТС, \ldots

\smallskip
\textbf{Публікації.}
По темі дисертації опубліковано
46 друкованих праць:

\smallskip
\textbf{Структура і обсяг роботи.}
Дисертація складається з вступу, 7 розділів, що викладені на
XXX сторінках, висновків, списку літературних джерел з
XX найменувань,
1 додатка.
Робота проілюстрована XXX рисунками.


\xsect{ОСНОВНИЙ ЗМІСТ РОБОТИ}

У \textbf{вступі} обґрунтовано актуальність теми,
сформульованл цілі і задачі дослідження,
відзначені наукова новизна і практична цінність роботи, відображені
одержані результати, що виносяться на захист.

У \textbf{першому розділі}
проведено аналітичний огляд проблем
влстивостей систем хаотчної динаміки
з точки зору задачі ідентифікації.

Проведено аналіз публікації,
розглянуто низка сетодів ідентификації,
що існує.
Розглянуто причини,
які на дозволяють використовувоти ці методи для
ідентифицації систем динаміного хаосу,
а також складних нехаотичних систем,
які з точки зору ідентификації
мають з ними певні загальні риси.





\textbf{Другий розділ}
присвячений
сінтезу крітеріїв ідентификації,
які мають сенс при роботі с системами
хаотичної динаміки.

Розлянуто фізичні та емпірічні підстави для створення
еритеріїв. Встановлено
загальні властивості критеріїв,
які впливають на процес ідентификації.





\medskip

\textbf{У третьому розділі}
разглянуто структуру пошукових систем
ідентификації.
Веедено поняття пошукового агенту,
координатору пошуку.
Розглянуто методи,
які реалізуются пошуковим агентом для
реалізації власної динамки.
Зроблено аналіз методів, яки використовуються
координаторми пошуку для досягнення мети
задачі ідентифікації, та засоби адаптації.

Введено класификацю структури та методів систем іденіификації у цілому.

Запропановано метод оцінювання помилок ідентификації
з врахування мультиагентної структури.





\textbf{У четвертому розділі} розглядається розроблене програмне
забезпечення
для моделювання
поведінки  нелінійних,
у тому числі хаотичних динамічних систем.
Програма була створена для моделювання динаміки
нелінійних систем та проведення аналізу.

Мова програмування --- C++,
графічний інтерфейс реалізовано за допомогою
бібліотек Qt та mathGL.
Розроблено ієрархію класів для
моделювання різноманітних елементів,
у тому числі суттєво нелінійних.

Також було створено спеціалізоване
програмне забезпечення для мікроконтролерів,
щодо забазпечення взаємодії
з реальнии динаміними системами.




\textbf{У п'ятому розділі}
розглянуте практичне застосування розроблених методів при моделюванні
процесів ідентифікації ряду систем хаотичної динаміки,
ак відомих, так і тих, які розглянуто уперше.

У підрозділі 5.1 досліджуються процеси ідентифікації системи Лоренца.

У підрозділі 5.2 досліджуються процеси ідентифікації системи Sprott A.

У підрозділі 5.3 досліджуються процеси ідентифікації системи Дуффінга.

У підрозділі 5.4 досліджуються процеси ідентифікації системи Чуа.

У підрозділі 5.5 досліджуються процеси ідентифікації системи Ван-дер-Поля.

У підрозділі 5.6 досліджуються процеси ідентифікації системи Колпітца у традиційному вигляді.

У підрозділі 5.7 досліджуються процеси ідентифікації системи вібраційної системи
із зоною нечутливості у силі, що повертає.

У підрозділі 5.7 досліджуються процеси ідентифікації системи з сухим тертям.

У підрозділі 5.8 досліджуються процеси ідентифікації системи  вібраційної системи
із гістерезисом у силі, що повертає, у хаотичному режимі.

\textbf{У шостому розділі}
розглянуте практичне застосування
нових методів ідентифікації для генератора Колпітца
з використанням реального фізичного об'єкту.
Впроваджену нову математичну модель,
яка враховує більше нелінійних властивостей
BJT транзистора, та різних режимів роботи.

Створено мікроконтролерну систему, яка
дозволила отримати дані з реального
генератора, та за допомогою програмного комплексу,
який описано у розділі 4, перевірити адекватність як
нової моделі генератора, що було використано,
так і працездатність створеної для неї системи ідентифікації.



\textbf{У сьомому розділі}
запропоновано модель
системи зв'язаних релаксаційних генераторів,
які проявляють хаотичну динаміку при певних умовах.
Досліджуються  властивості цієї системи.

Створено фізичний об'єкт,
який реалізовує запропановану динаміку,
а також
створено мікроконтролерну систему, яка
дозволила перевірити адекватність математичної моделі.
Також було проведено властивостей системи ідентифікації стосовно цієї системи.

\xsect{ВИСНОВКИ}

Основні результати дисертаційної роботи полягають у наступному:

\begin{itemize}

  \item
  створено нові критерії ідентифікації, які, на відміну від тих, що
  існують, придатні для аналізу стану та динаміки
  хаотичних систем, що створить фізично зумовлене обґрунтування працездатності систем
  ідентифікації;

  \item
  створено новий клас систем ідентификації у межах
    адаптивно-пошуковой парадигми,
    які за рахунок використання колективної динаміки
    анфамблю пошукових агентів забезаечують
    кращу якість ідентифицації;

  \item
  створено програмне забезпечення, придатне для моделювання як систем
  хаотичної динаміки, так і систем мультиагентної ідентифікації;

  \item
  проведено комп'ютерне моделювання процесів ідентифікації систем
  хаотичної динаміки, підтверджено їх працездатність.


\end{itemize}


\xsect{СПИСОК ОПУБЛІКОВАНИХ ПРАЦЬ ЗА ТЕМОЮ ДИСЕРТАЦІЇ}

\nocite{*}

\printbibliography




\xsect{АННОТАЦИЯ}

\textbf{\dissauthorRu}
\textbf{\booknameRu}
\textbf{--- На правах рукописи.}

Диссертация на соискание ученой степени
доктора
\dissScopeRu\ {}
по специальности
\dissSpecId\ --- <<\dissSpecRu>>.
\institutionRu, \belongRu, \cityRu, \bookyear.


Ключевые слова:
адаптивно-поисковая идентификация, нелинейные динамические системы,
информационные оценки, математическая модель.


\xsect{АНОТАЦІЯ}

\textbf{\dissauthorUa}
\textbf{\booknameUa}
\textbf{--- На правах рукопису.}

Дисертація на здобуття наукового ступеня
доктора
\dissScopeUa\ {}
за спеціальністю
\dissSpecId\ --- <<\dissSpecUa>>.
\institutionUa, \belongUa, \cityUa, \bookyear.


Ключові слова: адаптивно-пошукова ідентифікація, нелінійні динамічні
системи, інформаційні оцінки, математична модель.



\xsect{ABSTRACT}

\textbf{\dissauthorEn}
\textbf{\booknameEn}.
\textbf{--- As Manuscript.}

Thesis for the degree of Doctor of Technical Science in Specialization
\dissSpecId\ --- <<\dissSpecEn>>.
\institutionEn, \belongEn, \cityEn, \bookyear.

The dissertation is devoted to

Keywords: adaptive-search identification, nonlinear dynamic systems,
information estimations, mathematical model.

\clearpage

{~}
\vfill

\begin{center}


Підписано до друку xx.xx.\bookyear~р.

Формат $60 \times 84/16$  Папір друкарський. Ум. др.арк.~2

Друк різограф. Замовлення \No 02/17. Наклад - 100 прим.

% ДНВП <<Системні технології>>

49006, Дніпро, пр.~Гагаріна,4

st@nmetau.edu.ua

\end{center}

\vfill


\end{document}

