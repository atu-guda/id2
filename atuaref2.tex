\documentclass[a4paper,12pt]{atuaref}
%\documentclass[a4paper,12pt,russian]{atuaref}
\RequirePackage[utf8]{inputenc}
\usepackage[verbose,tmargin=25mm,bmargin=25mm,lmargin=30mm,rmargin=20mm,headsep=10pt]{geometry}
\usepackage{amssymb}
\usepackage{amsmath}
\usepackage{amsthm}
\usepackage{tabularx}
\usepackage{fouriernc}
\usepackage{paratype}

\usepackage{graphicx}
%\usepackage[english,russian,ukraineb]{babel}
\usepackage[english,ukraineb,russian]{babel}

\usepackage[backend=biber,language=russian,sorting=none,maxnames=3,bibstyle=gost-numeric]{biblatex}


\newcommand{\booknameUa}{Моделі та  методи адаптивно-пошукової ідентифікації систем з хаотичною   динамікою}
\newcommand{\booknameRu}{Модели  и  методы адаптивно-поисковой идентификации систем с хаотической динамикой}
\newcommand{\booknameEn}{Models and methods for adaptive-search identification systems with chaotic dynamics}
\newcommand{\bookname}{\booknameRu}

\newcommand{\bookyear}{2017}
\newcommand{\dissauthorUa}{Гуда~А.І.}
\newcommand{\dissauthorRu}{Гуда~А.И.}
\newcommand{\dissauthorEn}{Guda~A.I.}
\newcommand{\dissauthor}{\dissauthorRu}
\newcommand{\dissSpecUa}{математичне    моделювання  та обчислювальні методи}
\newcommand{\dissSpecRu}{математическое моделирование и вычислительные методы}
\newcommand{\dissSpecEn}{Mathematical Modelling and Computational Methods}
\newcommand{\dissSpecMain}{\dissSpecRu}
\newcommand{\dissSpecAref}{\dissSpecUa}
\newcommand{\dissSpecId}{01.05.02}
\newcommand{\dissScopeRu}{технических наук}
\newcommand{\dissScopeUa}{техничних наук}
\newcommand{\dissScopeMain}{\dissScopeRu}
\newcommand{\dissScopeAref}{\dissScopeUa}
\newcommand{\UDC}{681.5.015}
\newcommand{\dissRada}{Д.~08.084.01}
\renewcommand{\Rada}{Д.~08.084.01}
\renewcommand{\SekrRadi}{Селівьорстова~Т.В.}
\newcommand{\institutionRu}{Национальная металлургическая академия Украины}
\newcommand{\institutionUa}{Національна  металургійна     академія України}
\newcommand{\institutionEn}{National Metallurgical academy of Ukraine}
\newcommand{\institutionMain}{\institutionRu}
\newcommand{\institutionAref}{\institutionUa}
\newcommand{\belongRu}{Министерство образования и науки Украины}
\newcommand{\belongUa}{Міністерство освіти і науки      України}
\newcommand{\belongEn}{Ministry of Education and Science of Ukraine}
\newcommand{\belongMain}{\belongRu}
\newcommand{\belongAref}{\belongUa}
\newcommand{\cityRu}{Днипро}
\newcommand{\cityUa}{Дніпро}
\newcommand{\cityEn}{Dnipro}
\newcommand{\cityMain}{\cityRu}
\newcommand{\cityAref}{\cityUa}
\newcommand{\superRu}{Михалёв Александр Ильич}
\newcommand{\superUa}{Михальов Олександр Ілліч}
\newcommand{\superMain}{\superRu}
\newcommand{\superAref}{\superUa}

\title{\booknameRu}
\author{Гуда Антон Игоревич}
\supervisor{Михалёв Александр Ильич}{д.т.н., проф.}
%\speciality[\dissSpecRu]{{01.05.02}}[технических наук]
%\udc{681.5.015}
\date{\bookyear}
%\institution{Национальная металлургическая академия Украины, Министерство образования и науки Украины}{Днипро}



\addbibresource{atuworks.bib}

\DeclareMathOperator*{\sign}{sign}

\newcommand{\TermDef}[1]{\textit{\textbf{#1}}}
\newcommand{\TermUse}[1]{\textit{#1}}
\newcommand{\ProgName}[1]{\textsl{#1}}
\newcommand{\xsect}[1]{\medskip\begin{center}\textbf{#1}\end{center}\medskip\penalty10000}
%\newcommand{\xxsect}[1]{\smallskip\begin{center}\textbf{#1}\end{center}\smallskip\penalty10000}
\newcommand{\xxsect}[1]{\smallskip\textbf{#1}\smallskip\penalty10000}


\begin{document}

\sloppy
\setcounter{page}{1}

\xsect{ЗАГАЛЬНА ХАРАКТЕРИСТИКА РОБОТИ}

\textbf{Актуальність роботи.}
Сучасні технологічні об'єкти являють собою складні динамічні
системи, що містять нелінійні елементи. Для досягнення необхідних
показників якості роботи таких систем, побудови систем керування,
діагностики працездатності й прогнозування виходів з ладу потрібне
створення достатнє адекватних моделей процесів, що відбуваються
в системі. Висока складність технологічних систем і наявність
в них істотно нелінійних елементів практично виключає
можливість побудови простих, зручних для аналізу аналітичних
моделей. У таких випадках побудова моделі базується на фізичних,
хімічних й інших принципах, що лежать в основі функціонування
системи, та потребують використання обчислювальних методів
при їх моделюванні.

Невизначеність та нелінійність моделей,
що описують сучасні  технологічні об'єкти
й системи керування, у більшості випадків не дає можливості  визначити
аналітичні залежності значень параметрів від вхідних і вихідних
сигналів.  Для ідентифікації параметрів подібних систем показали
свою ефективність пошукові методи.
При цьому наявність шумів обмежує клас
застосовних пошукових методів \TermUse{прямими методами},
які використають у процесі роботи тільки самі
значення вихідного сигналу системи,
і не вимірюють або обчислюють у явному виді його похідні.

Існуючі методи прямого пошуку екстремуму,
що були розроблені
Л.А.~Растрігіним,
Л.Н.~Фіцнером,
Е.Є.~Гачинським,
А.І.~Дроздовим,
О.І.~Михальовим,
у першу чергу використовувались для рішення задач оптимізації.
Залишаючись застосовними для рішення
задач ідентифікації, ці методи не мають достатні характеристики точності й
швидкості ідентифікації, що пов'язане зі спадкуванням тих обмежень, які
характерні для задач оптимізації, і в більшості випадків не існують для задач
нелінійної ідентифікації.

З іншого боку одержали розвиток і широке поширення \textit{непараметричні}
методи ідентифікації, наприклад нейро або нейро-фаззи методи.
На жаль, використання цих методів, що у цілому мають широкий
спектр застосовності, не дає інформації про конкретні значення параметрів
систем, що підлягають ідентифікації. Крім цього, реалізація даних методів вимагає значних
обчислювальних витрат, що також перешкоджає застосуванню систем,
що реалізують непараметричні методи ідентифікації в складі промислового устаткування.
З іншого боку, більшість пошукових методів ідентифікації
можуть бути реалізовані у
вигляді простих мікропроцесорних компонентів,
а досить часто й без використання
мікропроцесорної техніки, що підвищує надійність і швидкодію системи.

Через велике різноманіття нелінійних динамічних систем не може бути одного
методу ідентифікації, яких був би адекватний
всім можливим задачам параметричної ідентифікації.
Для оцінки складності поставленої задачі й
визначення працездатності методу ідентифікації потрібні відповідні алгоритми.
При цьому для досить складних систем
перевірити працездатність, задати припустимі
параметри самої системи ідентифікації й визначити характеристики методу можна
тільки шляхом моделювання самого процесу ідентифікації
на досить представницькій
множині вхідних сигналів і шумів.

Через високий ступінь невизначеності в апріорній інформації про параметри
системи, досить важко заздалегідь визначити діапазон припустимих значень
параметрів системи ідентифікації. Тому особливу цінність представляють
методи адаптивно-пошукової ідентифікації (\TermUse{АПІ}),
здатні підбудовувати свої параметри, ґрунтуючись на
поточній інформації про стан системи.

Таким чином,
задача моделювання
й дослідження процесів адаптивно-пошукової
ідентифікації нелінійних динамічних систем є \textit{актуальною}.

\smallskip
\textbf{Зв'язок роботи з науковими програмами, планами, темами.}
Дисертаційна робота виконувалась у рамках науково-дослідних робіт
Національної металургійної академії України за держбюджетною
тематикою:

\begin{itemize}

\item XXX

\end{itemize}

\textbf{Мета і задачі дослідження.}
Метою даного дослідження є створення моделей адаптивно-пошукових
систем ідентифікації, визначення їхніх властивостей і можливостей.
Для досягнення даної мети були поставлені наступні задачі:

\begin{itemize}

\item розробити клас методів

\item розробити

\item визначити

\item XXX

\item створити програмний комплекс для моделювання
процесів АПІ
нелінійних динамічних систем.
\end{itemize}


\textbf{Об'єктом дослідження є}
системи адаптивно-пошукової ідентифікації нелінійних динамічних систем.

\smallskip
\textbf{Предметом дослідження є}
математичні моделі систем адаптивно-пошукової ідентифікації
нелінійних динамічних об'єктів.

\smallskip
\textbf{Методи дослідження.}
Для вирішення поставлених задач
використовувався математичний апарат
теорії управління та ідентифікації,
обчислювальних методів,
теорії інформації, тощо.

\smallskip
\textbf{Наукова новизна одержаних результатів:}
\begin{itemize}

\item уперше

\item уперше

\item уперше

\item уперше

\item уперше

\item уперше


\end{itemize}


\smallskip
\textbf{Практичне значення одержаних результатів.}
Розроблені методи ідентифікації

\smallskip
\textbf{Особистий внесок здобувача.}

Усі основні положення і результати
дисертаційної роботи, які виносяться на захист, отримано здобувачем особисто та
опубліковано в роботах [1-–XX]. У наукових працях, опублікованих у співавторстві,
здобувачу належать наступні результати. У роботі [1]



\smallskip
\textbf{Апробація результатів.}
Основні положення дисертаційної роботи доповідались на наукових
семінарах кафедри ІТС,

\smallskip
\textbf{Публікації.}
По темі дисертації опубліковано
XX друкованих праць:

\smallskip
\textbf{Структура і обсяг роботи.}
Дисертація складається з вступу, 7 розділів, що викладені на
XXX сторінках, висновків, списку літературних джерел з
XX найменувань,
1 додатка.
Робота проілюстрована XXX рисунками.


\xsect{ОСНОВНИЙ ЗМІСТ РОБОТИ}

У \textbf{вступі} обґрунтовано актуальність теми,
показана необхідність побудови
математичних моделей  Сформульовані цілі і задачі дослідження,
відзначені наукова новизна і практична цінність роботи, відображені
одержані результати, що виносяться на захист.

У \textbf{першому розділі}
проведено аналітичний огляд проблем постановки задачі ідентифікації,






\textbf{Другий розділ}
присвячений розгляду питань інформаційної оцінки якості роботи
методів ідентифікації  й складності задачі ідентифікації.






\medskip

\textbf{У третьому розділі}



\textbf{У четвертому розділі} розглядається розроблене програмне
забезпечення
для моделювання
поведінки  нелінійних динамічних систем.
Програма була створена для моделювання динаміки
нелінійних систем та проведення аналізу.

Мова програмування --- C++,
графічний інтерфейс реалізовано за допомогою
бібліотеці Qt.
Розроблено ієрархію класів для
моделювання різноманітних елементів,
у тому числі суттєво нелінійних.




\textbf{У п'ятому розділі}
розглянуте практичне застосування розроблених методів при моделюванні
процесів ідентифікації




\textbf{У шостому розділі}
розглянуте практичне застосування


\textbf{У сьосому розділі}
розглянуте практичне застосування

\xsect{ВИСНОВКИ}

Основні результати дисертаційної роботи полягають у наступному:

\begin{itemize}

\item
XXX


\item
XXX


\item
XXX


\item
XXX


\item
XXX


\item
XXX


\item
XXX


\end{itemize}


\xsect{СПИСОК ОПУБЛІКОВАНИХ ПРАЦЬ ЗА ТЕМОЮ ДИСЕРТАЦІЇ}

\nocite{*}

\printbibliography




\xsect{АННОТАЦИЯ}

\textbf{Гуда~А.И.}
\textbf{Моделирование процессов адаптивно-поисковой идентификации нелинейных динамических систем.}
\textbf{--- На правах рукописи.}

Диссертация на соискание ученой степени доктора технических
наук по специальности 01.05.02
<<Математическое моделирование и вычислительные методы>>.
Национальная металлургическая академия Украины Министерства образования и науки, Днипро, 2017.


Ключевые слова:
адаптивно-поисковая идентификация, нелинейные динамические системы,
информационные оценки, математическая модель.


\xsect{АНОТАЦІЯ}

\textbf{Гуда~А.І.}
\textbf{Моделювання процесів адаптивно-пошукової ідентифікації нелінійних динамічних систем.}
\textbf{--- На правах рукопису.}

Дисертація на здобуття наукового ступеня доктора технічних наук
за спеціальністю 01.05.02 --
<<Математичне моделювання і обчислювальні методи>>.
Національна металургійна академія України
Міністерства освіти і науки України, Дніпро, 2017.


Ключові слова: адаптивно-пошукова ідентифікація, нелінійні динамічні
системи, інформаційні оцінки, математична модель.



\xsect{ABSTRACT}

\textbf{Guda~A.I.}
\textbf{Modelling of nonlinear dynamic systems adaptive-search identification processes}.
\textbf{--- As Manuscript.}

Thesis for the degree of Doctor of Technical Science in Specialization 01.05.02 –-
<<Mathematical Modelling and Computational Methods>>.
National metallurgical academy of Ukraine  of the Ministry of Education and Science if Ukraine, Dnipro, 2017.

The dissertation is devoted to

Keywords: adaptive-search identification, nonlinear dynamic systems,
information estimations, mathematical model.

\clearpage

{~}
\vfill

\begin{center}


Підписано до друку xx.xx.2017~р.

Формат $60 \times 84/16$  Папір друкарський. Ум. др.арк.~2

Друк різограф. Замовлення \No 02/17. Наклад - 100 прим.

% ДНВП <<Системні технології>>

49006, Дніпро, пр.~Гагаріна,4

st@nmetau.edu.ua

\end{center}

\vfill


\end{document}

