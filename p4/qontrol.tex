\chapter{Программно-аппаратный комплекс моделирования нелинейных динамических систем ``qontrol''}
\label{chapter_qontrol}

\section{Анализ существующих систем} % ---------------------------------------------

\subsection{Matlab, simulink}

\subsection{Scilab, XXX}

\subsection{Multisim}

\subsection{LabView}

\subsection{Выводы}



\section{Требования к системе моделирования} % ---------------------------------------------

\section{Основы построения программно-аппаратного комплекса} % ------------------------------

Для разработки комплекса в качестве базового языка был выбран C++,
как обеспечивающий совмещение гибкости и возможностей разработки
с необходимой скоростью вычислений. Также, данный язык
является полностью применимым для программирования микроконтроллеров
класса STM32, входящих в состав аппаратной части комплекса.

Тем не менее, для реализации возможностей автоматизации
внутри самой программы требуется применение языка,
не требующего предварительной компиляции, достаточно распространённого,
быстрого, поддерживающего объектную модель.
В качестве такого языка был выбран ECMAScript, известный также
как Javascript.

\subsection{Реализация интроспекции}

Для реализации задач гибкого подхода к моделированию, автоматизации
и автоматического построения интерфейсных элементов
требуется получение информации об объектах системы моделирования
в процессе работы программы (интроспекции).
В самом языке C++ в данный момент не средств реализации интроспекции.
Библиотека Qt, используемая в разработке, предоставляет
базовые возможности интроспекции, за счёт использования метакомпилятора ``moc''.

\subsection{Базовые объекты системы и структурная схема модели}



\section{Пользовательский интерфейс комплекса} % ----------------------------------------

Пользовательский интерфейс программы ``qontrol'' был создан с помощью
средств, предоставляемых библиотекой Qt.

\section{Основные подходы к } % --------------------------------------------- 

\section{Выводы по разделу} % ---------------------------------------------

Створено щось непогане.

