\phantomsection
\chapter*{Введение}

\paragraph{Актуальность темы}
\pdfbookmark[1]{Актуальность темы}{actualitybookmark}

Нелінійні динамічні системи, широко представлені в сучасних технологічних і
природних процесах, незважаючи на детермінізм їх визначення, можуть проявляти
хаотичні властивості в своїй динаміці. При цьому як завгодно малі збурення у вхідних
впливах і параметрах самої системи призводять до значних, але кінцевим збуренням
вихідного сигналу. Це призводить до певних труднощів при конструюванні,
управлінні і прогнозі поведінки таких систем.

При математичному та комп'ютерному моделюванні систем динамічного хаосу
виникають специфічні для даних систем проблеми. Перш за все -- потрібно
забезпечити наявність працездатного критерію адекватності моделі. Для задач
ідентифікації наявність такого критерію є принциповим. Найбільш часто
використовувані при моделюванні поведінки динамічних систем критерії, засновані
на звичних заходи в просторі вихідних сигналів, виявляються непрацездатними. З
іншого боку, спеціальні характеристики оцінювання хаотичних властивостей динамічних
систем, такі як фрактальна розмірність, показник Ляпунова, перетин Пуанкаре,
недостатньо інформативні для задачі ідентифікації через обмежений діапазон
змін, великою похибкою при їх вимірі для реальних систем і суттєвою
обчислювальною складністю.

У якості прототипу для синтезу нових методів ідентифікації доцільно обрати ряд
методів ідентифікації, які було розроблено та досліджено у роботах
П.~Ейкхоффа, Л.А.~Растрігіна, Л.Н.~Фіцнера, Е.Є.~Гачинського,
А.І.~Дроздова, О.І.~Михальова, зокрема пошукові та адаптивно-пошукові
методи.

Також вважається доцільним використання як основи
наступного класу двомодельних адаптивно-пошукових методів,
які було розроблено та досліджено у попередньої роботі.
Проте, результати моделювання процесів ідентифікації цими методами
технічних об'єктів, які проявляють складну та хаотичну динаміку, показали
їх повну або обмежену непридатність.


Тому проблема ідентифікації динамічних систем, які проявляють хаотичну
динаміку, або близьку до хаотичної, є
\textit{актуальною}.

Синтез систем идентификации для сложных нелинейных динамических систем
был и остаётся сложной задачей. Невозможность определения
аналитической зависимости значений идентифицируемых параметров
от выхода объекта делает неработоспособными большинство известных
методов идентификации. Наибольшую сложность для идентификации
представляют системы динамического хаоса~\cite{moon_chaotic_vibr,anisch_nonlin_eff,sprott_212}.
Даже лучшие в своем классе, методы адаптивно-поисковой идентификации, % TODO: ref
в чистом виде оказываются в данном случае неработоспособными.
Успешность построения системы идентификации для хаотических
систем определяется наличием интегрального критерия,
учитывающего динамику системы и зависящего от идентифицируемых параметров.

\paragraph{Зв'язок роботи з науковими програмами, планами, темами.}
\pdfbookmark[1]{Связь работы с ...}{sciencelinkbookmark}

Дисертаційна робота виконувалась у рамках науково-дослідних робіт
Національної металургійної академії України за держбюджетною
тематикою:

\begin{itemize}


\item
  ``Вдосконалення технології утилізації в металургійній промисловості
  матеріальних і енергетичних відходів'', №~держ. реєст. 0113U003266;

  \item
  ``Дослідження та імітаційне моделювання процесів нелінійної динаміки
  формування фрактальних структур функціональних   покриттів'',
  № держ. реєст. 0110U003240;

  \item
  ``Наукове обґрунтування та розробка ефективних тепло-масообмінних
  процесів в інноваційних металургійних технологіях'', №~держ. реєст 0115 U003176.

\end{itemize}

Результати було впроваджено у рамках науково-практичного дослідження
``Оцінка    можливості заміни випробувань КА на стійкість до акустичного навантаження
випробуваннями широкосмугової вібрації'', згідно договору №~V-105-16-3 від 07.09.2016.


\paragraph{Мета і задачі дослідження}
\pdfbookmark[1]{Цель и задания исследования}{aimandtaskbookmark}

Головною метою дослідження є створення нових методів ідентифікації з
використанням адаптивно-пошукових принципів настроювання параметрів, які
були б придатні для створення моделей систем, які проявляють хаотичну
та/або схожу на хаотичну динаміку. В свою чергу, окремо ставиться задача
розв'язання наукової проблеми створення адекватних моделей процесів
ідентифікації хаотичних систем з використанням запропонованих нових
методів, аналізу та дослідження їх характеристик в умовах невизначеності.

Для досягнення даної мети були поставлені наступні задачі:

\begin{itemize}

  \item
  розробити нові критерії ідентифікації, які, на відміну від тих, що
  існують, при моделюванні були б придатні для аналізу стану та динаміки
  хаотичних систем, що створить обґрунтування працездатності систем
  ідентифікації;

  \item
  розвинути існуючи та розробити нові методи пошуку, які б не мали
  обмежень тих, що існують, та у повної мірі використовували можливості
  паралельних обчислювань та переваг використання ансамблю
  синергированних моделей;

  \item
  розвинути існуючи та розробити нові методи адаптації параметрів
  системи ідентифікації, здатні пристосуватися до зміни режимів роботи
  системи;

  \item
  розробити програмне забезпечення, придатне для моделювання як систем
  хаотичної динаміки, так і систем ідентифікації;

  \item
  провести комп'ютерне моделювання процесів ідентифікації систем
  хаотичної динаміки та дослідити їх працездатність, можливості та
  характеристики.

\end{itemize}



\pdfbookmark[1]{Объект исследования}{objectresearchbookmark}
\textit{Об'єкт дослідження}  ---
технічні системи, які в процесі їх функціювання можуть
входити в хаотичні режими.

\textit{Предметом дослідження} ---
математичні моделі процесів та методи
мультіагентної ідентифікації технічних систем з хаотичною динамікою.

\textit{Методи дослідження} ---
Для вирішення поставлених задач використовувався математичний апарат
теорії управління та ідентифікації нелінійних систем, динамічного хаосу,
обчислювальних методів, нечіткої логіки, теорії інформації тощо.
математический аппарат теории управления и идентификации,
\ldots

\pdfbookmark[1]{Научная новизна}{scinewbookmark}
\paragraph{Наукова новизна одержаних результатів}

Основний науковий результат полягає в розв'язанні науково-практичної проблеми
синтезу методів ідентифікації
технічних систем хаотичної динаміки, створенні відповідних математичних
моделей та дослідженні результатів моделювання процесів
адаптивно-пошукової ідентифікації.

Основні наукові результати, що отримані в дисертаційної роботі, полягають
в наступному:

\noindent
вперше:

\begin{itemize}

  \item
  створено критерії ідентифікації нелінійних динамічних систем,
  які, на відміну від тих, що існують, дозволяють оцінити їх стан та
  хаотичну динаміку, та дають підстави для створення ефективних алгоритмів
  настроювання параметрів моделей систем ідентифікації;

  \item
  створено методи адаптивно-пошукової ідентифікації на підставі
  адаптивно-пошукової парадигми з використанням ансамблю пошукових агентів,
  які взаємодіють проміж собою, які на відміну від методів, що використовують
  одну модель або пару моделей, значно підвищують швидкість пошуку та
  здатні за мінімальний час перестрілюватися при різкої зміні параметра, а на
  відміну від ройових алгоритмів, нові методи потребують значно меншої
  кількості моделей та забезпечують певні гарантії пошуку;

  \item
  створено нову класифікацію систем ідентифікації динамічних систем,
  яка як вбирає у себе методи, що існують, так і дозволяє
  створювати нові методи ідентифікації за рахунок
  комбінування їх складових частин;

  \item
   визначено, що системи з сухим тертям з точки зору задачі ідентифікації
   при певних  умовах функціонування
   мають властивості, що поєднують їх з системами хаотичної динаміки, тобто з
   системами хаотичної динаміки їх поєднує суттєва залежність від початкових
   умов та вид атрактору, що також потребує використання нових методів ідентифікації;

  \item
   запропоновано модель системи хаотичної динаміки системи зв'язаних релаксаційних генераторів,
   яка відрізняєшся від існуючих відсутністю індуктивних компонентів,
   працездатністю при малих напругах та можливістю
   керування частотним діапазоном у широкому діапазоні,
   що сприяє процесу аналізу хаотичної динаміки
   фізичного об'єкту, перевірки адекватності математичної моделі
   та властивостей системи ідентифікації стосовно цієї системи.


\end{itemize}


\noindent
Набуло подальший розвиток:
\begin{itemize}

  \item
  методи оцінювання якості ідентифікації,
  які на відміну від тих, що існують,
  враховують використання множини агентів;

  \item
  підходи до адаптації параметрів систем
  адаптивно-пошукової ідентифікації, які придатні використовувати поточну
  інформацію від ансамблю сінергированих моделей, та корегувати глобальні
  параметри пошуку;

  \item
    модель генератора Копітца, яка враховує
    більшу кількість нелінійних ефектів,
    що забезпечує більш адекватні результати процесу
    ідентифікації її параметрів новими методами;

\end{itemize}


\paragraph{Практичне значення одержаних результатів}
\pdfbookmark[1]{Практическое значение}{practicalusagebookmark}

Розроблені методи ідентифікації було використано
при проектуванні, створенні, налаштовуванні параметрів
стенду дослідження вібраційного та акустичного впливу.
Аналіз результатів даних з цього стенду
дав можливість указати потрібні нелінійні властивості системи,
та діапазон параметрів, які у сукупності
забезпечують широкосмуговий спектр коливань.

Створене програмне середовище для моделювання нелінійних динамічних систем
використається при проведенні практичних робіт по дисциплінам
``Моделювання систем'',
``Сучасні системи управління'' на кафедри інформаційних технологій
і систем Національної металургійної академії України.


Все результаты внедрения подтверждены соответствующими актами
(Приложение А).

\paragraph{Особистий внесок здобувача}
\pdfbookmark[1]{Личный вклад}{personaladdbookmark}

Усі основні положення і результати
дисертаційної роботи, які виносяться на захист, отримано здобувачем особисто та
опубліковано в роботах [1--50]. У наукових працях, опублікованих у співавторстві,
здобувачу належать наступні результати. У роботі [1] \ldots

\paragraph{Апробація результатів}

Основні положення дисертаційної роботи доповідались на наукових
семінарах кафедри ІТС,
регіональному науковому семінарі Придніпровського Наукового Центру НАН України
''Сучасні проблеми управління та моделювання складних систем'',
науково-практичних коференціях:
``Інформатика та системні науки'' (ІСН-2011) Полтава--2011,
``Интеллектуальные системы принятия решений и проблемы вычислительного интеллекта'' (ISDMCI) Херсон--2011,
``Информационные технологии в управлении сложными системами'' Днепропетровск--2011,
``Автоматизация: проблемы, идеи, решения'' Севастополь-2011,
``Интеллектуальные системы принятия решений и проблемы вычислительного интеллекта'' (ISDMCI) Херсон--2012,
``Автоматизация: проблемы, идеи, решения'' Севастополь-2012,
``Интеллектуальные системы принятия решений и проблемы вычислительного интеллекта'' (ISDMCI) Херсон--2013,
``Автоматизация: проблемы, идеи, решения'' Севастополь-2013,
``Интеллектуальные системы принятия решений и проблемы вычислительного интеллекта'' (ISDMCI) Херсон--2014,
``Интеллектуальные системы принятия решений и проблемы вычислительного интеллекта'' (ISDMCI) Херсон--2015,
``Computer Sciences and Information Technologies'' (CSIT) Lviv--2015 (Scopus),
``Интеллектуальные системы принятия решений и проблемы вычислительного интеллекта'' (ISDMCI) Херсон--2016,
``Data Stream Mining and Processing'' DSMP Lviv-2016 (Scopus,Web of Science).

\paragraph{Публікації}

По темі дисертації опубліковано
49 друкована праця,
з них
24 входять до до міжнародних наукометрічних баз,
13 статей опубліковано у матеріалах конференцій.


