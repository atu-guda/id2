\chapter*{Введение}

\paragraph{Актуальность темы}.
В настоящее время ...

Синтез систем идентификации для сложных нелинейных динамических систем
был и остаётся сложной задачей. Невозможность определения
аналитической зависимости значений идентифицируемых параметров
от выхода объекта делает неработоспособными большинство известных
методов идентификации. Наибольшую сложность для идентификации
представляют системы динамического хаоса~\cite{moon_chaotic_vibr,anisch_nonlin_eff,sprott_212}.
Даже лучшие в своем классе, методы адаптивно-поисковой идентификации, % TODO: ref
в чистом виде оказываются в данном случае неработоспособными.
Успешность построения системы идентификации для хаотических
систем определяется наличием интегрального критерия,
учитывающего динамику системы и зависящего от идентифицируемых параметров.

\paragraph{Связь работы с научными программами, планами, темами.}
Диссертационная работы выполнялась в соответствии с 

\paragraph{Цель и задания исследования.}
Целью диссертационной работы является

Достижение этой цели требует решения следующих задач:

\begin{itemize}

\item

\item

\item

\end{itemize}

\textit{Объект исследования} --- 

\textit{Предмет исследования} ---

\textit{Методы исследования} ---

\paragraph{Научная новизна результатов диссертационной работы} заключается в
развитии теории 

Основные результаты, определяющие новизну диссертационной работы,
состоят в следующем:

\begin{itemize}

\item
  впервые предложен комплексный подход к оцениванию параметров

\item

\item

\end{itemize}


\paragraph{Практическое значение полученных результатов.}
Разработанные методы исследования и математические модели

Все результаты внедрения подтверждены соответствующими актами
(Приложение А).

\paragraph{Личный вклад соискателя.} Все результаты диссертационной работы
получены автором самостоятельно и опубликованы в работах [].
В работах, опубликованных в соавторстве, автору
принадлежат следующие результаты:


\paragraph{Апробация результатов диссертации.}
Основные результаты проведенных исследований были представлены, докладывались и
обсуждались на следующих научных конференциях:

\paragraph{Публикации.}
Основные результаты по теме диссертации изложены в XXX
опубликованных работах: XXX статьях, из них XXX статьи в научных
специализированных изданиях Украины, XXX статьи в международных изданиях,
XXX публикация в сборниках материалов и тезисов докладов научно-технических
конференций.



