\clearpage
\phantomsection
\chapter*{Вступ}

\paragraph{Актуальність роботи}
\pdfbookmark[1]{Актуальность темы}{actualitybookmark}

Нелінійні динамічні системи, розповсюджені в сучасних технологічних і
природних процесах, незважаючи на детермінізм їх визначення, можуть проявляти
хаотичні властивості в своїй динаміці. При цьому як завгодно малі збурення у вхідних
впливах і параметрах самої системи призводять до значних, але кінцевих збурень
вихідного сигналу. Це призводить до певних труднощів при конструюванні,
управлінні і прогнозі поведінки таких систем.

При математичному та комп'ютерному моделюванні систем динамічного хаосу
виникають специфічні для даних систем проблеми. Перш за все --- потрібно
забезпечити наявність працездатного критерію адекватності моделі. Для задач
ідентифікації наявність такого критерію є принциповою. Найчастіше
критерії, що використовуються при моделюванні поведінки динамічних систем,
засновані
на звичайних методах в просторі вихідних сигналів, виявляються непрацездатними. З
іншого боку, спеціальні характеристики оцінювання хаотичних властивостей динамічних
систем, такі як фрактальна розмірність, показник Ляпунова, перетин Пуанкаре, ---
недостатньо інформативні для задачі ідентифікації через обмежений діапазон
змін, велику похибку при їх вимірі для реальних систем і суттєву
обчислювальну складність.

У якості прототипу для синтезу нових методів ідентифікації є доцільним обрати ряд
методів ідентифікації, які було розроблено та досліджено у роботах
П.~Ейкхоффа, Л.А.~Растрігіна, Л.Н.~Фіцнера, Е.Є.~Гачинського,
А.І.~Дроздова, О.І.~Михальова, зокрема пошукові та адаптивно-пошукові
методи.
Також вважається за доцільне використання в якості основи
наступного класу двомодельних адаптивно-пошукових методів,
які було розроблено та досліджено у попередній роботі.
Проте, результати моделювання цими методами
процесів ідентифікації цими методами
технічних об'єктів, які проявляють складну та хаотичну динаміку, показали
їх повну або обмежену непридатність.


Тому вирішення проблеми
ідентифікації параметрів складних технічних систем у режимі хаотичної динаміки
є \textit{актуальним}.

% Синтез систем идентификации для сложных нелинейных динамических систем
% был и остаётся сложной задачей. Невозможность определения
% аналитической зависимости значений идентифицируемых параметров
% от выхода объекта делает неработоспособными большинство известных
% методов идентификации. Наибольшую сложность для идентификации
% представляют системы динамического хаоса~\cite{moon_chaotic_vibr,anisch_nonlin_eff,sprott_212}.
% Даже лучшие в своем классе, методы адаптивно-поисковой идентификации, % TODO: ref
% в чистом виде оказываются в данном случае неработоспособными.
% Успешность построения системы идентификации для хаотических
% систем определяется наличием интегрального критерия,
% учитывающего динамику системы и зависящего от идентифицируемых параметров.

\paragraph{Зв'язок роботи з науковими програмами, планами, темами.}
\pdfbookmark[1]{Связь работы с ...}{sciencelinkbookmark}

Дисертаційна робота виконувалась у рамках науково-дослідних робіт
Національної металургійної академії України за держбюджетною
тематикою:

\begin{itemize}


\item
  ``Вдосконалення технології утилізації в металургійній промисловості
  матеріальних і енергетичних відходів'', №~держ. реєст. 0113U003266;

  \item
  ``Дослідження та імітаційне моделювання процесів нелінійної динаміки
  формування фрактальних структур функціональних   покриттів'',
  № держ. реєст. 0110U003240;

  \item
  ``Наукове обґрунтування та розробка ефективних тепло-масообмінних
  процесів в інноваційних металургійних технологіях'', №~держ. реєст 0115U003176.

\end{itemize}

Результати було впроваджено у рамках науково-практичного дослідження
``Оцінка можливості заміни випробувань КА на стійкість до акустичного навантаження
випробуваннями широкосмугової вібрації'', згідно договору \hbox{№~V-105-16-3} від 07.09.2016
(Спеціальне конструкторсько-технологічне бюро ІТМ НАН України, м.~Дніпро),
та при ідентифікації режиму роботи вертикально-осьової вітроустановки з H-ротором Дар'є
з врахуванням складно-періодичної або хаотичної структури коливань лопатей
(Інститут транспортних систем і технологій НАН України, м.~Дніпро).


\paragraph{Мета і задачі дослідження}
\pdfbookmark[1]{Цель и задания исследования}{aimandtaskbookmark}

Головною метою дослідження є створення нових методів ідентифікації з
використанням адаптивно-пошукових принципів настроювання параметрів, які
були б придатні для створення моделей систем, які проявляють хаотичну динаміку.

Для досягнення даної мети були поставлені наступні задачі:

\begin{itemize}

  \item
  розробити нові критерії ідентифікації, які, на відміну від тих, що
  існують, були б придатні для аналізу стану та динаміки
  систем з хаотичною динамікою, і створять базис для обґрунтування працездатності систем
  ідентифікації;

  \item
  розвинути існуючі та розробити нові методи пошуку, які б
  % не мали обмежень тих, що існують, та
  у повної мірі використовували можливості
  паралельних обчислювань та переваг використання ансамблю
  синергірованних моделей;

  % \item
  % розвинути існуючи та розробити нові методи адаптації параметрів
  % системи ідентифікації, здатні пристосуватися до зміни режимів роботи
  % системи;

  \item
  створити моделі процесів
  ідентифікації хаотичних систем з використанням запропонованих методів,
  провести комп'ютерне моделювання процесів ідентифікації систем
  хаотичної динаміки та дослідити їх працездатність, можливості та
  характеристики;

  \item
  розробити нову фізичну систему з хаотичною поведінкою
  на підставі системи зв'язаних релаксаційних елементів,
  що надасть можливість дослідити та підтвердити переваги
  розроблених методів ідентифікації;

  \item
  провести натурне моделювання процесів ідентифікації запропонованого
  генератора хаосу, та порівняти з результатами комп'ютерного моделювання;

  \item
  розробити програмне забезпечення, придатне для моделювання систем
  хаотичної динаміки, систем ідентифікації, предікції та управління.

\end{itemize}



\pdfbookmark[1]{Объект исследования}{objectresearchbookmark}
\textbf{Об'єктом дослідження є}
технічні системи, які в процесі їх функціювання можуть
входити в хаотичні режими.

\smallskip
\textbf{Предметом дослідження є}
математичні моделі процесів та методи
ансамблевої ідентифікації технічних систем з хаотичною динамікою.

\smallskip
\textbf{Методи дослідження.}
Для вирішення поставлених задач використовувався математичний апарат
теорії управління та ідентифікації нелінійних систем, динамічного хаосу,
обчислювальних методів, нечіткої логіки, теорії інформації тощо.


\pdfbookmark[1]{Научная новизна}{scinewbookmark}
\paragraph{Наукова новизна одержаних результатів}

Основний науковий результат полягає у розв'язанні науково-технічної проблеми
синтезу методів ідентифікації
технічних систем хаотичної динаміки, створенні відповідних математичних
моделей та дослідженні результатів моделювання процесів
ансамблевої ідентифікації.

Основні наукові результати, що отримані в дисертаційній роботі, полягають
в наступному:

\noindent
Вперше:
%
\begin{itemize}

  \item
  створено критерії ідентифікації нелінійних динамічних систем,
  які, на відміну від тих, що існують, дозволяють оцінити їх стан та
  хаотичну динаміку, а також дають підстави для створення ефективних алгоритмів
  настроювання параметрів моделей систем ідентифікації;

  \item
  створено методи ідентифікації на підставі
  адаптивно-пошукової парадигми з використанням ансамблю пошукових агентів,
  які взаємодіють проміж собою, які на відміну від методів, що використовують
  одну модель або пару моделей, значно підвищують швидкість пошуку та
  здатні за мінімальний час  перелаштовуватися при різкій зміні параметрів, а на
  відміну від ройових алгоритмів нові методи потребують значно меншої
  кількості моделей та забезпечують певні гарантії пошуку;

  \item
  створено нову класифікацію систем ідентифікації динамічних систем,
  яка, як вбирає у себе методи, що існують, так і дозволяє
  створювати нові методи ідентифікації за рахунок
  комбінування їх складових частин;

  \item
   визначено, що системи з сухим тертям з точки зору задачі ідентифікації
   при певних  умовах функціонування
   мають властивості, що поєднують їх з системами хаотичної динаміки, тобто існує
   % системами хаотичної динаміки їх поєднує
   суттєва залежність від початкових
   умов та характерний вид атрактору; %, що також потребує використання нових методів ідентифікації;

  \item
   запропоновано модель системи хаотичної динаміки
   з використанням зв'язаних релаксаційних генераторів,
   яка відрізняється від існуючих відсутністю індуктивних компонентів,
   працездатністю при малих напругах та можливістю
   керування частотним діапазоном у широкому інтервалі,
   що сприяє процесу аналізу хаотичної динаміки
   фізичного об'єкту, перевірці адекватності математичної моделі
   та властивостей системи ідентифікації.
\end{itemize}

\noindent
Набуло подальший розвиток:
\begin{itemize}

  \item
  методи оцінювання якості ідентифікації,
  які на відміну від тих, що існують,
  враховують використання множини агентів;

  \item
  підходи до адаптації параметрів систем
  ансамблевої ідентифікації, які придатні використовувати поточну
  інформацію від ансамблю синергірованих моделей та коригувати глобальні
  параметри пошуку в умовах апріорної та поточної невизначеності;

  \item
    модель генератора Колпітца, яка враховує
    більшу кількість нелінійних ефектів,
    що забезпечує більш адекватні результати процесу
    ідентифікації її параметрів запропонованими методами;

\end{itemize}




\paragraph{Практичне значення одержаних результатів}
\pdfbookmark[1]{Практическое значение}{practicalusagebookmark}

Розроблені методи ідентифікації було використано
при проектуванні, створенні, налаштовуванні параметрів
стенду дослідження вібраційного та акустичного впливу.
Аналіз результатів даних з цього стенду
дав можливість вказати необхідні нелінійні властивості системи
та діапазон параметрів, які у сукупності
забезпечують широкосмуговий спектр коливань.

Створене програмне середовище для моделювання нелінійних динамічних систем
використовується при проведенні практичних робіт по дисциплінам
``Моделювання систем'',
``Сучасні системи управління'' на кафедрі інформаційних технологій
і систем Національної металургійної академії України.



\paragraph{Особистий внесок здобувача}
\pdfbookmark[1]{Личный вклад}{personaladdbookmark}

Усі основні положення і результати
дисертаційної роботи, які виносяться на захист, отримано здобувачем особисто та
опубліковано в роботах [1--50]. У наукових працях, опублікованих у співавторстві,
здобувачу належать наступні результати.
У роботі~[1] створено критерій ти система ідентифікації системи Дуффінга.
У роботах~[2,42] створено критерій ти система ідентифікації системи Ресслера.
У роботі~[3] досліджено критерії ідентифікації для групи хаотичних систем.
У роботах~[4,17,32] досліджено вплив збурень на процес ідентифікації.
У роботах~[5,6,40] зроблено аналіз можливих критеріїв для системи Ван-дер-Поля.
У роботах~[7,41] розроблено та використано критерій для системи Чуа.
У роботі~[8] запропонована та досліджена хаотична система релаксаційних генераторів.
У роботі~[9] створено систему ідентифікації з зоною нечутливості.
У роботах~[10,45] проаналізована динаміка базової моделі Колпітца та її ідентифікація.
У роботі~[11] досліджено влив параметрів системи ідентифікації на її характеристики.
У роботах~[12,13,15] запропоновано метод розширення робочого діапазону та якості системи ідентифікації.
У роботі~[14] розглянуто фізичні передумови для синтезу критерію.
У роботі~[16] створено один з методів роботи агенту ідентифікації.
У роботах~[18,19,29,49,50] запропоновано метод ідентифікації з множиною моделей.
У роботі~[20] запропоновано критерій ефективності.
У роботі~[21,46] досліджено вплив параметрів багатомодельної системи ідентифікації.
У роботі~[22] досліджено параметри системи багатомодельної ідентифікації.
У роботі~[23] проведено перевірку можливості використання діаграм Пуанкаре для ідентифікації.
У роботі~[24] розроблено та досліджено новий хаотичний генератор на основі системи релаксаційних елементів.
У роботі~[25] досліджено процес взаємодії трьох пошукових агентів.
У роботі~[26] проведено аналіз потрібного діапазону при вимірюванні.
У роботі~[27] досліджено вплив релаксаційного генератора на точність позиціювання.
У роботі~[28] створено критерій та система ідентифікації системи ''Sprott~A''.
У роботі~[30] запропоновано інформаційні методи оцінки складності задачі ідентифікації.
У роботі~[31,38] обґрунтовано використання фізичних принципів при створенні критерію.
У роботі~[33] проведено порівняльний аналіз критеріїв.
У роботі~[34,43,47,48] використано фізичні принципи при ідентифікації систем Лоренца.
У роботі~[35,44] досліджено хаотичну систему з гістерезисом та процес її ідентифікації.
У роботі~[36,39] проведено аналіз процесів ідентифікації параметрів для групи хаотичних систем.
У роботі~[37] досліджено рівноважні стани агентів ідентифікації поблизу екстремуму.


\paragraph{Апробація результатів}

Основні положення дисертаційної роботи доповідались на наукових
семінарах кафедри ІТС,
регіональному науковому семінарі Придніпровського Наукового Центру НАН України
''Сучасні проблеми управління та моделювання складних систем'',
науково-практичних коференціях:
``Інформатика та системні науки'' (ІСН-2011) Полтава--2011,
``Интеллектуальные системы принятия решений и проблемы вычислительного интеллекта'' (ISDMCI) Херсон--2011,
``Информационные технологии в управлении сложными системами'' Днепропетровск--2011,
``Автоматизация: проблемы, идеи, решения'' Севастополь-2011,
``Интеллектуальные системы принятия решений и проблемы вычислительного интеллекта'' (ISDMCI) Херсон--2012,
``Автоматизация: проблемы, идеи, решения'' Севастополь-2012,
``Интеллектуальные системы принятия решений и проблемы вычислительного интеллекта'' (ISDMCI) Херсон--2013,
``Автоматизация: проблемы, идеи, решения'' Севастополь-2013,
``Интеллектуальные системы принятия решений и проблемы вычислительного интеллекта'' (ISDMCI) Херсон--2014,
``Интеллектуальные системы принятия решений и проблемы вычислительного интеллекта'' (ISDMCI) Херсон--2015,
``Computer Sciences and Information Technologies'' (CSIT) Lviv--2015 (Scopus),
``Интеллектуальные системы принятия решений и проблемы вычислительного интеллекта'' (ISDMCI) Херсон--2016,
``Data Stream Mining and Processing'' DSMP Lviv-2016 (Scopus,Web of Science).

\paragraph{Публікації}

За темою дисертаційної роботи опубліковано
50 наукових праць. Основний зміст і результати досліджень
викладено у 36 друкованих працях у наукових фахових виданнях, які
рекомендовано Міністерством освіти і науки України,
29 --- включено до інших міжнародних наукометричних баз,
у тому числі 1 --- включено до бази Web of Science,
13 робіт опубліковано у збірниках наукових праць та матеріалах конференцій.

Структура та обсяг роботи. Дисертація складається із вступу, 7
розділів, висновків, списку використаних джерел, додатків.
Загальний обсяг
роботи складає 304 сторінки тексту, 295 рисунків, дві таблиці, список
використаних джерел з 155 найменувань, 3 додатка.

