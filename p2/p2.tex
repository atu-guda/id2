\chapter{Критерии идентификации}

Как уже было отмечено, для успешности процесса идентификации требуется
построение различных интегральных критериев, зависящих,
хотя бы в первом приближении
от величины идентифицируемого параметра. Конкретный вид определяется самим идентифицируемым
объектом. При этом, довольно часто (но не всегда) в качестве основы для такого критерия выступает
некая энергетическая зависимость. С учётом усреднения на требуемом интервале,
один из простых видов критерия может быть задан как
%
\begin{equation}
\od{q_{x^2}}{t}
=
\frac{1}{\tau} \left( x^2(t) - q_{x^2}(t) \right)
,
\label{atu:eq:q}
\end{equation}
%
%\noindent
где $\tau$ -- характерное время оценивания, $x(t)$ -- выбранная переменная состояния системы.
Это далеко не единственный вид определения критерия, но в данной работе предлагается именно он,
и последующие критерием будут использовать (\ref{atu:eq:q}).

\section{Выводы по разделу 2}

Выводы.

