\chapter{Критерии идентификации}

\section{Отличие задачи идентификации от поиска экстремума}

На первый взгляд, при заданном критерии идентификации, задача идентификации
сводится к классической задаче описка экстремума: надо найти такие значения
параметров, при которых критерий принимает максимальное значение.
На самом деле, существуют определённые аспекты, которые делают такое
сведение практически невозможным.

Прежде всего, в задаче поиска
экстремума предполагается, что наблюдаемая система статична:
значение критерия не зависит от времени, и проведя измерение
в точке один раз, можно к нему не возвращаться.
Напротив, идентификация динамической системы предполагает,
что значение критерия, даже после какого-либо усреднения на конечном
интервале времени, есть величина динамическая, причём динамика определяется
не только параметрами системы, но и свойствами самой системы измерения,
а также процессом взаимодействия системы измерения с моделями. При этом
возможны весьма нетривиальные результаты, такие как параметрический
резонанс, распространение параметрических волн на множестве моделей и т.д.

Также существенную роль могут играть как шумы измерения, так и побочные
эффекты от процесса фильтрации шумов. Большая часть фильтров приводит к запаздыванию
в процессе измерения, и игнорирование этих явлений может привести
к нарушению устойчивости поиска.

В задаче поиска экстремума предполагается, что не только
значение функции известно точно в каждой точке, но также известны все производные.
В реальные задачах идентификации производные непосредственно
не доступны для измерения, а их оценка требует применения специальных
методов. При этом процесс оценивания производных, как правило,
более чувствителен к шумам измерения, чес собственно измерение.

Далее, в в задачах поиска экстремума не учитывается возможность
смещения экстремума со временем. Напротив, в задачах идентификации
или же изначально предполагается вариабельность параметров, или же
имеются ограничение на время измерения.


Все эти явления делают задачу идентификации более сложной, чем
классическая задача поиска экстремума, чем и обусловлено
существование широкого спектра методов идентификации. Тем не менее,
некоторые алгоритмы, применимые при поиска экстремума, могут быть
полезны при синтезе системы идентификации.

Как уже было отмечено, для успешности процесса идентификации требуется
построение различных интегральных критериев, зависящих,
хотя бы в первом приближении
от величины идентифицируемого параметра. Конкретный вид определяется самим идентифицируемым
объектом. При этом, довольно часто (но не всегда) в качестве основы для такого критерия выступает
некая энергетическая зависимость. С учётом усреднения на требуемом интервале,
один из простых видов критерия может быть задан как
%
\begin{equation}
\od{q_{x^2}}{t}
=
\frac{1}{\tau} \left( x^2(t) - q_{x^2}(t) \right)
,
\label{atu:eq:q}
\end{equation}
%
%\noindent
где $\tau$ -- характерное время оценивания, $x(t)$ -- выбранная переменная состояния системы.
Это далеко не единственный вид определения критерия, но в данной работе предлагается именно он,
и последующие критерием будут использовать (\ref{atu:eq:q}).

\section{Выводы по разделу 2}

Выводы.

