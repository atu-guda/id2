\chapter{Критерии идентификации}

\section{Свойства и параметры критериев идентификации}

\textit{ Эта часть уйдёт в главу 1, в постановку}

Пусть задан динамический объект $ \mathbf{O}$, характеризуемый параметром $p_o(t)$.
Информация об этом параметре может быть представлена
различными способами. Например, может быть задан только диапазон
$[p_{\min}, p_{\max}]$,
возможных значений параметра,
этот же диапазон может иметь зависимость от времени:
$[p_{\min}(t), p_{\max}(t)]$.
На основании априорных исследований может
быть задана как плотность вероятности для значений параметра,
так и зависимость этой вероятности от времени. Наличие плотности вероятности
позволяет использовать информационные оценки идентификации [cipkin,self],
но не является необходимым условием для синтеза системы идентификации в целом.
Также могут быть заданы дополнительные ограничения, такие как максимальна скорость
изменения параметра и т.д.

Пусть также существует конечное множество моделей
$\mathbf{M}_i$. Структура моделей предполагается известной и одинаковой,
а параметру объекта $p_o(t)$ соответствуют параметры моделей $p_{i}(t)$.

На вход как моделей так и объекта подаётся входной сигнал $u(t)$.
Частный случай, когда ни объект, ни модели не требуют входного сигнала
не нарушает общности постановки. В некоторых случаях
не следует пренебрегать ошибкой измерения входного сигнала $w_u(t)$,
однако, при анализе динамики систем идентификации в целом,
этой величиной, как правило, пренебрегают.

Выходные сигналы моделей
$x_i(t)$ считаем известными точно, так как ошибки
представления значений при численных вычислениях заведомо
пренебрежимо малы с точностью измерений. Напротив,
выход объекта $x_{op}(t)$ считаем измеренным
с определённой ошибкой измерения $w(t)$ : $x_o(t) = x_{op}(t) + w(t)$.

Особенность динамики хаотических систем не позволяет
определить цель идентификации как задачу минимизации
какой-либо меры $\mu(x_o(t),x_i(t))$ в пространстве выходных
сигналов~[].

\textit{TODO: обоснование и картинки}

Для целей идентификации необходимо существование
заданного критерия $q(x(t))$, близость величин
которых для объекта и модели (в смысле какой-либо меры)
и позволяет говорить о достижении цели идентификации.


Физическая измеримость.

Отношение к интересующим аспектам динамики объекта.

Монотонность -- как следствие одноэкстремальность?

Следствием невозможности непосредственного использования выходных
сигналов объекта и моделей в качестве критерия идентификации, а также
требование физической измеримости критерия является
тот факт, что для получения оценки критерия, как правило,
требуется значительное время.

Зависимость (обозначение?) от времени оценивания.
Методы измерения.






\section{Отличие задачи идентификации от поиска экстремума}

На первый взгляд, при заданном критерии идентификации, задача идентификации
сводится к классической задаче описка экстремума: надо найти такие значения
параметров, при которых критерий принимает максимальное значение.
На самом деле, существуют определённые аспекты, которые делают такое
сведение практически невозможным.

Прежде всего, в задаче поиска
экстремума предполагается, что наблюдаемая система статична:
значение критерия не зависит от времени, и проведя измерение
в точке один раз, можно к нему не возвращаться.
Напротив, идентификация динамической системы предполагает,
что значение критерия, даже после какого-либо усреднения на конечном
интервале времени, есть величина динамическая, причём динамика определяется
не только параметрами системы, но и свойствами самой системы измерения,
а также процессом взаимодействия системы измерения с моделями. При этом
возможны весьма нетривиальные результаты, такие как параметрический
резонанс, распространение параметрических волн на множестве моделей и т.д.

Также существенную роль могут играть как шумы измерения, так и побочные
эффекты от процесса фильтрации шумов.
Применение практически любого фильтра данного класса приводит к запаздыванию
в процессе измерения, и игнорирование этих явлений может привести
как к нарушению устойчивости поиска, так и получению совершенно
неадекватных результатов при наличии устойчивости.

В задаче поиска экстремума предполагается, что не только
значение функции известно точно в каждой точке, но также известны все производные
(по крайней мере, необходимые для работы метода).
В реальные задачах идентификации производные непосредственно
не доступны для измерения, а их оценка требует применения специальных
методов. При этом процесс оценивания производных, как правило,
более чувствителен к шумам измерения, чем собственно измерение.

Требование конечности производных в условиях дискретного представления сигнала.

Далее, в в задачах поиска экстремума не учитывается возможность
смещения экстремума со временем. Напротив, в задачах идентификации
или же изначально предполагается вариабельность параметров, или же,
как минимум имеется ограничение на время измерения.


Все эти явления делают задачу идентификации более сложной, чем
классическая задача поиска экстремума, чем и обусловлено
существование широкого спектра методов идентификации. Тем не менее,
некоторые алгоритмы, применимые при поиска экстремума, могут быть
полезны при синтезе системы идентификации.

Как уже было отмечено, для успешности процесса идентификации требуется
построение различных интегральных критериев, зависящих,
хотя бы в первом приближении
от величины идентифицируемого параметра. Конкретный вид определяется самим идентифицируемым
объектом. При этом, довольно часто (но не всегда) в качестве основы для такого критерия выступает
некая энергетическая зависимость. С учётом усреднения на требуемом интервале,
один из простых видов критерия может быть задан как
%
\begin{equation}
\od{q_{x^2}}{t}
=
\frac{1}{\tau} \left( x^2(t) - q_{x^2}(t) \right)
,
\label{atu:eq:qlin}
\end{equation}
%
%\noindent
где $\tau$ -- характерное время оценивания, $x(t)$ -- выбранная переменная состояния системы.
Это далеко не единственный вид определения критерия, но в данной работе предлагается именно он,
и последующие критерием будут использовать (\ref{atu:eq:qlin}).

Чернавскиий~\cite{chernavskii_syn_info}.

\section{Выводы по разделу 2}

Выводы.

