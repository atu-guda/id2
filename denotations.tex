% denotations
\clearpage
\phantomsection
% \chapter*{Список условных обозначений}
\pdfbookmark[0]{Список условных обозначений}{denotationsbookmark}

\begin{center}
  \textbf{Список условных обозначений}
\end{center}


\begin{description}

  \item[$u(t)$]  --- вход объекта и моделей, $[U]$, (стр.~\pageref{atu:d:u});

  \item[$\mathbf{x}(t)$]  --- выход объекта и моделей, при необходимости разбивается на скалярные компоненты:
    $x(t)$, $y(t)$, $z(t)$ \ldots, $[X]$, (стр.~\pageref{atu:d:x});

  \item[$w(t)$]  --- ошибка измерения выходного сигнала объекта, (стр.~\pageref{atu:d:w});

  \item[$\mathcal{P}$] --- множество допустимых значений параметров (стр.~\pageref{atu:d:p_set});

  \item[$N$]  --- количество моделей, (стр.~\pageref{atu:d:N});

  \item[$p_\mathrm{id}(t)$ ] --- текущее значение параметра, полученное в результате идентификации (стр.~\pageref{atu:d:p_id});

  \item[$q(p)$]  --- зависимость значения критерия идентификации от параметра в стационарном случае, $[Q]$;

  \item[$q_o(t)$, $q_m(t)$]  --- мгновенные значения критерия для объекта и модели соответственно, $[Q]$;

  \item[$p_o(t)$, $p_m(t)$, $p_{mi}(t)$]  --- значения параметра для объекта, текущей модели, модели с индексом $i$, $[P]$;

  \item[$F(q_o,q_m) $]  --- функция качества идентификации, $[1]$, (стр.~\pageref{atu:d:F});

  \item[$\Delta p$]  --- диапазон изменения параметра, $[P]$;

  \item[$\Delta q$]  --- диапазон изменения критерия качества, $[Q]$;

  \item[$\beta(p) = \od{q(p)}{p}$]  --- текущий наклон $q(p)$, $[Q/P]$;

  \item[$\beta_a = \frac{\Delta q}{\Delta p} $]  --- усредненный наклон критерия, задаёт соотношение масштабов по $p$ и $q$, $[Q/P]$;

  \item[$ $]  --- x;

  \item[$\tau_q$]  --- характерное время усреднения критерия, $[s]$;

  \item[$a_q = \frac{1}{\tau_q} $]  --- величина, обратная $\tau_q$, $[s^{-1}]$;

  \item[$q_\gamma $]  --- величина, определяющая рабочий диапазон функции качества, $[Q]$;

  \item[$\gamma = \frac{1}{q_\gamma} $]  --- чувствительность функции качества, $[Q^{-1}]$;

  \item[$q_\beta = \frac{q_\gamma}{\beta} $]  --- рабочий диапазон функции качества в пространстве параметра, $[P]$.

  \item[$\frac{q_\gamma}{\Delta q} $]  --- x;

  \item[$\frac{q_\gamma}{(N-1)\Delta q} $]  --- x;

  \item[$p_{e} $]  --- оценка идентифицируемого параметра одним поисковым агентом, (стр.~\pageref{atu:d:p_e});

  \item[$p_{eql} $]  --- оценка $p_e$ методом с кусочно-линейной аппроксимацией $q$, (стр.~\pageref{atu:d:p_eql});

  \item[$p_{eFq} $]  --- оценка $p_e$ квадратичной аппроксимацией $F$, (стр.~\pageref{atu:eq:p_eFq});

  \item[$p_{eFc} $]  --- оценка $p_e$ методом трёхточечного COG по $F$, (стр.~\pageref{atu:eq:p_eFc});

  \item[$S$]  --- степень ``уверенности'' агента в собственной оценке $p_e$, (стр.~\pageref{atu:d:S});

  \item[$ $]  --- x;

  \item[$ $]  --- x;

  \item[$ $]  --- x;

  \item[$ $]  --- x;


\end{description}



