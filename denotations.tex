% denotations
\clearpage
\phantomsection
% \chapter*{Список условных обозначений}
\pdfbookmark[0]{Список условных обозначений}{denotationsbookmark}

\begin{center}
  \textbf{Список условных обозначений}
\end{center}


\begin{description}

  \item[$q(p)$]  -- зависимость значения критерия идентификации от параметра в стационарном случае, $[Q]$;

  \item[$q_o(t)$, $q_m(t)$]  -- мгновенные значения критерия для объекта и модели соответственно, $[P]$;

  \item[$p_o(t)$, $p_m(t)$, $p_{mi}(t)$]  -- значения параметра для объекта, текущей модели, модели с индексом $i$;

  \item[$F(q) $]  -- функция качества, $[1]$;

  \item[$\Delta p$]  -- диапазон изменения параметра, $[P]$;

  \item[$\Delta q$]  -- диапазон изменения критерия качества, $[Q]$;

  \item[$\beta(p) = \od{q(p)}{p}$]  -- текущий наклон $q(p)$, $[Q/P]$;

  \item[$\beta_a = \frac{\Delta q}{\Delta p} $]  -- усредненный наклон критерия, задаёт соотношение масштабов по $p$ и $q$, $[Q/P]$;

  \item[$ $]  -- x;

  \item[$\tau_q$]  -- характерное время усреднения критерия, $[s]$;

  \item[$a_q = \frac{1}{\tau_q} $]  -- величина, обратная $\tau_q$, $[s^{-1}]$;

  \item[$q_\gamma $]  -- величина, определяющая рабочий диапазон функции качества, $[Q]$;

  \item[$\gamma = \frac{1}{q_\gamma} $]  -- чувствительность функции качества, $[Q^{-1}]$;

  \item[$ $]  -- x;

  \item[$ $]  -- x;

  \item[$ $]  -- x;

  \item[$ $]  -- x;

  \item[$ $]  -- x;

  \item[$ $]  -- x;

  \item[$ $]  -- x;

  \item[$ $]  -- x;

  \item[$ $]  -- x;


\end{description}



