% denotations
\clearpage
\phantomsection
% \chapter*{Список условных обозначений}
\pdfbookmark[0]{Список условных обозначений}{denotationsbookmark}

\begin{center}
\textbf{Список умовних позначень}
\end{center}


\begin{description}

  \item[$u(t)$]  ---
    вхід об'єкта і моделей,
    $[U]$, (стор.~\pageref{atu:d:u});

  \item[$\mathbf{x}(t)$]  ---
  вихід об'єкта і моделей, при необхідності розбивається на скалярні компоненти:
    $x(t)$, $y(t)$, $z(t)$ \ldots, $[X]$, (стор.~\pageref{atu:d:x});

  \item[$w(t)$]  ---
    похибка вимірювання вихідного сигналу об'єкта,
    (стор.~\pageref{atu:d:w});

  \item[$\mathcal{P}$] ---
    множина допустимих значень параметрів
    (стор.~\pageref{atu:d:p_set});

  \item[$N$]  ---
    кількість моделей або агентів,
    (стор.~\pageref{atu:d:N});

  \item[$p_\mathrm{id}(t)$ ] ---
    поточне значення параметра, отримане в результаті ідентифікації
    (стор.~\pageref{atu:d:p_id});

  \item[$q(p)$]  ---
    залежність значення критерію ідентифікації від параметра в стаціонарному випадку,
    $[Q]$, (стор.~\pageref{atu:d:q_p});

  \item[$q_o(t)$, $q_m(t)$]  ---
    миттєві значення критерію для об'єкта і моделі відповідно,
    $[Q]$, (стор.~\pageref{atu:d:q_o});

  \item[$p_o(t)$, $p_m(t)$, $p_{mi}(t)$]  ---
   значення параметра для об'єкта, поточної моделі, моделі з індексом
    $i$, $[P]$, (стор.~\pageref{atu:d:p_o});

  \item[$F(q_o,q_m) $]  ---
    функція якості ідентифікації,
    $[1]$, (стор.~\pageref{atu:d:F});

  \item[$\Delta p$]  ---
    діапазон зміни параметра,
    $[P]$, (стор.~\pageref{atu:d:Delta_q});

  \item[$\Delta q$]  ---
    діапазон зміни критерію якості,
    $[Q]$, (стор.~\pageref{atu:d:Delta_q});

  \item[$\beta(p) = \od{q(p)}{p}$]  ---
    поточний нахил графіка критерію
    $q(p)$, $[Q/P]$;

  \item[$\beta_a = \frac{\Delta q}{\Delta p} $]  ---
    усереднений нахил критерію, задає співвідношення масштабів по
    $p$ та $q$, $[Q/P]$;

  \item[$\tau_q$]  ---
    характерний час усереднення критерію,
    $[s]$, (стор.~\pageref{atu:d:tau_q});

  \item[$a_q = \frac{1}{\tau_q} $]  ---
    характерна частота зрізу критерію, величина, зворотня до
    $\tau_q$, $[s^{-1}]$, (стор.~\pageref{atu:d:a_q});

  \item[$q_\gamma $]  ---
    величина, яка визначає робочий діапазон функції якості,
    $[Q]$, (стор.~\pageref{atu:d:q_gamma});

  \item[$\gamma = \frac{1}{q_\gamma} $]  ---
    чутливість функції якості,
    $[Q^{-1}]$, (стор.~\pageref{atu:d:gamma});

  \item[$q_\beta = \frac{q_\gamma}{\beta} $]  ---
    робочий діапазон функції якості в просторі параметра,
    $[P]$.

  \item[$ \sigma_q (\tau_q)  $]  ---
   співвідношення між часом оцінювання $\tau_q $
   і середньоквадратичної похибкою вимірювання критерію
    $[Q]$, (стор.~\pageref{atu:d:sigma_q});

  % \item[$\frac{q_\gamma}{\Delta q} $]  ---
  %   x;
  %
  % \item[$\frac{q_\gamma}{(N-1)\Delta q} $]  ---
  %   x;

  \item[$p_{e} $]  ---
    оцінка ідентифікованого параметра одним пошуковим агентом,
    $[P]$, (стор.~\pageref{atu:d:p_e});

  \item[$p_{eql} $]  ---
   оцінка $ p_e $ методом з кусочно-лінійною апроксимацією $q$,
   $[P]$, (стор.~\pageref{atu:d:p_eql});

  \item[$p_{eFq} $]  ---
    оцінка $ p_e $ квадратичною апроксимацією $F$,
    (стор.~\pageref{atu:eq:p_eFq});

  \item[$p_{eFc} $]  ---
    оцінка $ p_e $ методом триточкового COG по $ F $,
    (стор.~\pageref{atu:eq:p_eFc});

  \item[$S$]  ---
  ступінь ``впевненості'' агента в своїй оцінці $ p_e $,
    $[1]$, (стор.~\pageref{atu:d:S});

  \item[$W$]  ---
    worthiness агенту в при оценці $p_e$,
    $[1]$, (стор.~\pageref{atu:d:W});

  % \item[$ $]  --- x;
  %
  % \item[$ $]  --- x;
  %
  % \item[$ $]  --- x;
  %
  % \item[$ $]  --- x;


\end{description}



