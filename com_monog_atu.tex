\documentclass[a4paper,paratype,12pt,fouriernc]{cmonogra}
% atutext dejavu droid paratype fouriernc

\usepackage[verbose,dvips,a4paper,tmargin=20mm,bmargin=25mm,lmargin=20mm,rmargin=20mm,headsep=4pt,includeheadfoot]{geometry}


\usepackage[nodisplayskipstretch]{setspace}
%\usepackage{amsmath}
\usepackage{mathtools}
\usepackage{indentfirst}
\usepackage{titling}

\usepackage{graphicx}
\usepackage{placeins} % FloatBarrier
\usepackage{siunitx}

\usepackage[backend=biber,sorting=none,maxnames=100,bibstyle=gost-numeric,babel=other]{biblatex}

\usepackage{tikz}
\usetikzlibrary{arrows}
\usetikzlibrary{patterns}

\usepackage[europeanresistors,americaninductors,siunitx,fulldiodes]{circuitikz}

\usepackage[english,ukraineb,russian]{babel}

\usepackage{currfile}

\setcounter{topnumber}{3}
\renewcommand\topfraction{.7}
\setcounter{bottomnumber}{2}
\renewcommand\bottomfraction{.6}
\setcounter{totalnumber}{7}
\renewcommand\textfraction{.1}
\renewcommand\floatpagefraction{.5}
\setcounter{dbltopnumber}{2}
\renewcommand\dbltopfraction{.7}
\renewcommand\dblfloatpagefraction{.6}

\clubpenalty9500
\widowpenalty9500%


\newlength{\TW}
\setlength{\TW}{0.01\textwidth}

\DeclareMathOperator*{\sign}{sign}



\usepackage{blox}
\usepackage[europeanresistors,americaninductors,siunitx,fulldiodes]{circuitikz}

\usetikzlibrary{calc}
\usetikzlibrary{arrows}
\usetikzlibrary{patterns}
%\usepgflibrary{shapes.geometric}

\tikzset{
  >=stealth,
  %semiRed/.style={fill=red,opacity=0.3,draw=black,thin},
  medline/.style={draw=black,line width=0.6pt},
  medlinep/.style={draw=black,line width=0.6pt,->},
  semiboldline/.style={draw=black,line width=1.2pt},
  boldline/.style={draw=black,line width=2.0pt},
  wire/.style={draw=black,line width=1.0pt},
  elelem/.style={draw=black,line width=1.5pt}
}



\setstretch{1.2}


\newcommand{\booknameUa}{Моделі та  методи адаптивно-пошукової ідентифікації систем з хаотичною   динамікою}
\newcommand{\booknameRu}{Модели  и  методы адаптивно-поисковой идентификации систем с хаотической динамикой}
\newcommand{\booknameEn}{Models and methods for adaptive-search identification systems with chaotic dynamics}
\newcommand{\bookname}{\booknameRu}

\newcommand{\bookyear}{2017}
\newcommand{\dissauthorUa}{Гуда~А.І.}
\newcommand{\dissauthorRu}{Гуда~А.И.}
\newcommand{\dissauthorEn}{Guda~A.I.}
\newcommand{\dissauthor}{\dissauthorRu}
\newcommand{\dissSpecUa}{математичне    моделювання  та обчислювальні методи}
\newcommand{\dissSpecRu}{математическое моделирование и вычислительные методы}
\newcommand{\dissSpecEn}{Mathematical Modelling and Computational Methods}
\newcommand{\dissSpecMain}{\dissSpecRu}
\newcommand{\dissSpecAref}{\dissSpecUa}
\newcommand{\dissSpecId}{01.05.02}
\newcommand{\dissScopeRu}{технических наук}
\newcommand{\dissScopeUa}{техничних наук}
\newcommand{\dissScopeMain}{\dissScopeRu}
\newcommand{\dissScopeAref}{\dissScopeUa}
\newcommand{\UDC}{681.5.015}
\newcommand{\dissRada}{Д.~08.084.01}
\renewcommand{\Rada}{Д.~08.084.01}
\renewcommand{\SekrRadi}{Селівьорстова~Т.В.}
\newcommand{\institutionRu}{Национальная металлургическая академия Украины}
\newcommand{\institutionUa}{Національна  металургійна     академія України}
\newcommand{\institutionEn}{National Metallurgical academy of Ukraine}
\newcommand{\institutionMain}{\institutionRu}
\newcommand{\institutionAref}{\institutionUa}
\newcommand{\belongRu}{Министерство образования и науки Украины}
\newcommand{\belongUa}{Міністерство освіти і науки      України}
\newcommand{\belongEn}{Ministry of Education and Science of Ukraine}
\newcommand{\belongMain}{\belongRu}
\newcommand{\belongAref}{\belongUa}
\newcommand{\cityRu}{Днипро}
\newcommand{\cityUa}{Дніпро}
\newcommand{\cityEn}{Dnipro}
\newcommand{\cityMain}{\cityRu}
\newcommand{\cityAref}{\cityUa}
\newcommand{\superRu}{Михалёв Александр Ильич}
\newcommand{\superUa}{Михальов Олександр Ілліч}
\newcommand{\superMain}{\superRu}
\newcommand{\superAref}{\superUa}

\title{\booknameRu}
\author{Гуда Антон Игоревич}
\supervisor{Михалёв Александр Ильич}{д.т.н., проф.}
%\speciality[\dissSpecRu]{{01.05.02}}[технических наук]
%\udc{681.5.015}
\date{\bookyear}
%\institution{Национальная металлургическая академия Украины, Министерство образования и науки Украины}{Днипро}



\title{\bookname}
\author{Коллективный автор}

\addbibresource{atuworks.bib}
\addbibresource{artrefs.bib}

\hypersetup{
  pdftitle={\thetitle},
  pdfauthor={\theauthor},
  colorlinks=true,
  linkcolor=blue,
  citecolor=brown
}

%\newcommand{\LinkRef}[1]{ \textit{\color{red}#1} }
\newcommand{\LinkRef}[1]{}
% \newcommand{\Cmt}[1]{ {\small\color{red}#1} }
\newcommand{\Cmt}[1]{}


\pagestyle{stheadings}

\begin{document}

\tableofcontents


\MonoArticle{atu}{Идентификация хаотических систем}{А.И.~Гуда, А.И.~Михалёв}{Гуда~А.И., Михалёв~А.И.}

\subsection*{Введение}

Синтез систем иденификации для сложных нелинейных динамических систем
всегда был сложной задачей. Невозможность построения
аналитической зависимости значения идентифицируемого параметра
от выхода объекта сильно ограничивает множество работоспособных
методов идентификации. Наибольшую сложность для идентификации
представляют системы динамического хаоса~\cite{moon_chaotic_vibr,anisch_nonlin_eff,sprott_212}.
Даже лучшие в своем классе -- методы адаптивно-поисковой идентификации -- % TODO: ref
в чистом виде оказываются в данном случае неработоспособными.
Успешность построения системы идентификации для хаотических
систем определяется наличием интегрального критерия,
описывающего динамику системы и зависящего от идентифицируемых параметров.


\subsection*{Критерии идентификации}

Как уже было отмечено, для успешности процесса идентификации необходимо
существование интегрального критерия, зависящего (желательно наиболее простой зависимостью)
от величины идентифицируемого параметра. Конкретный вид определяется самой идентифицируемой
системой. При этом, довольно часто (но не всегда) в качестве основы для такого критерия выступает
какая-либо энергетическая зависимость. С учётом усреднения на требуемом интервале,
один из простых видов критерия может быть задан так:

\begin{equation}
\od{q_{x^2}}{t}
=
\frac{1}{\tau} \left( x^2(t) - q_{x^2}(t) \right)
,
\label{atu:eq:q}
\end{equation}

\noindent
где $\tau$ -- характерное время оценивания, $x(t)$ -- выбранная переменная состояния системы.
Это далеко не единственный способ, но в данной работе будет использоваться именно он,
и обозначения видов критерием будут основываться на (\ref{atu:eq:q}).

\subsection*{Мультимодельные методы идентификации}

Процесс поисковой идентификации заключается в настройке параметров одной
или нескольких моделей, анализу критериев идентификации
(или соответствующих функций качества), и оцениванию по этой информации
значения идентифицируемого параметра. При использовании в целях идентификации нескольких моделей,
появляются общие действия, применяемые к каждой из них. Следовательно, имеет смысл ввести следующее определение:


\textbf{ Агент } -- совокупность модели, критерия идентификации,
алгоритмов локального поиска и, может быть адаптации.

Как правило, один агент управляет одной моделью. При этом
он может использовать информацию, как полученную как от других
агентов, так и полученную в результате обработки данных от других агентов.

В простейших случаях возможно построение системы идентификации
и одной моделью, и соответственно одним агентом. % TODO: cite
Примерами такого похода могут служить
метод синхронного детектора \cite{adopt_cont_sys}
и оригинальный адаптивно-поисковый метод \cite{mich_92}.
Применение одномодельных методов требует определённого механизма
сохранения истории процесса поиска, по крайней мере в пределах
одного поискового ``периода''. Это сильно ограничивает диапазон
применимости данных методов, в первую очередь -- из-за значительных
временных затрат на повторяющийся поиск.

Применение пары моделей (в этом случае имеет смысл говорить об одном агенте)
позволяет с меньшими временными затратами оценить градиент функции качества,
и, как следствие -- определить значение параметра. Также значительным плюсом
такого подхода является меньшая скорость изменения параметров моделей.

Тем не менее, при использовании пары моделей неоправданно много времени
тратится на перемещение поисковой пары, особенно при резких изменениях
идентифицируемого параметра. Поэтому, применения множества
поисковых агентов может кардинально уменьшить время идентификации.

Рассмотрим одномерный случай.
Пусть задан объект $ \mathbf{O} $, и $N$ моделей:
$ \mathbf{M}_0 \ldots \mathbf{M}_{N-1} $.
На вход как объекта, так и каждой из моделей подаётся (при необходимости)
входной сигнал $u(t)$. Выход моделей $x_{mi}(t)$ измеряется точно,
а объекта $x_o(t)$ -- с определённой помехой $w(t)$. При необходимости помеху
со сходными характеристиками можно подмешивать и выходу моделей.
Обозначим $q_o(t)$ -- текущее значение критерия для
объекта, $q_{mi}(t)$ -- для $i$-той модели.

Каждый агент, определяя величину $q_{mi}(t)$ , и получая $q_o(t)$,
вычисляет безразмерную функцию качества идентификации.

С учётом обозначения

\[
  q_r = \frac{q_o - q_m}{q_\gamma},
\]

\noindent
представлены наиболее распространённые виды таких функций:

\begin{equation}
  F_{\mathrm{gauss}} = \exp( - q_r^2 ),
\label{atu:eq:F_gauss}
\end{equation}

\begin{equation}
  F_{\mathrm{parabolic}} = 1 - q_r^2 \left( 1 - \frac{1}{e} \right),
\label{atu:eq:F_parabolic}
\end{equation}

\begin{equation}
  F_{\mathrm{triangle}} = 1 - |q_r| \left( 1 - \frac{1}{e} \right),
\label{atu:eq:F_triangle}
\end{equation}

\begin{equation}
  F_{\mathrm{hyper}} = \frac{1}{ 1 + |q_r| \left( 1 - \frac{1}{e} \right)},
\label{atu:eq:F_hyper}
\end{equation}

\begin{equation}
  F_{\mathrm{log}} = 1 - \ln \left( 1 + |q_r| \right) \frac{1-1/e}{\ln(2)}.
\label{atu:eq:F_log}
\end{equation}

Для всех рассмотренных функций $q_\gamma$ -- величина, обратная чувствительности
функции качества -- задаёт масштаб и рабочий диапазон функции качества (рис.~\ref{atu:f:F_types}).
При этом, при необходимости, значения этих функций могут быть искусственно ограничены диапазоном $[0;1]$.
Каждая имеет свой набор преимуществ и недостатков~\cite{atu_ISDMCI2016}. В данной работе будет использоваться
функция (\ref{atu:eq:F_gauss}).

\begin{figure}[htb!]
  \centerline{\includegraphics[width=45\TW]{p/F_types.png} }
  \caption{Функции качества идентификации (\ref{atu:eq:F_gauss})--(\ref{atu:eq:F_log})}
  \label{atu:f:F_types}
\end{figure}

Рассмотрим случай, когда каждый агент взаимодействует с двумя своими ближайшими соседями,
запрашивая у них величины $p$ -- текущее значение параметра и $F$.
При анализе конкретного агента, для упрощения записи, его индекс $i$ обозначим как ``c'' (current),
предыдущий получает индекс ``l'' (left), а последующий -- ``r'' (right).
Для единообразия дополним множество моделей двумя неподвижными псевдомоделями (fake models),
обозначив и индексами ``ll'' и ``rr''. Для псевдомоделей считаем $  F_{ll} = F_{rr} = 0$,
а координаты выбираются за пределами рабочего диапазона поиска.


Расмотрим 3 подхода к определению
точки максимума функции качества, а следовательно -- значения идентифицируемого параметра \cite{atu_st99}.
Первый -- реализация
метода COG (Center of gravity, Такаги-Сугено) \cite{atu_asau25},
используемого при дефаззификации систем нечёткой логики:

\begin{equation}
  p_{ge}
  =
  \frac{\sum\limits_{i=0}^{n-1} F_{mi} p_{mi}}
       {\sum\limits_{i=0}^{n-1} F_{mi} }
  .
  \label{atu:eq:p_ge}
\end{equation}

Второй подход призван уменьшить зависимость первого
от влияния локальных экстремумов и границ. В этом
случае определяется модель $M_{i_{m}}$ с максимальным значением
$F$, и в оценке используется только ближайшая окрестность этой модели:

\begin{equation}
  p_{le}
  =
  \frac{ F_{i_m-1} p_{i_m-1} + F_{i_m} p_{i_m} + F_{i_m+1} p_{i_m+1} }
       { F_{i_m-1} + F_{i_m} + F_{i_m+1} }
  .
  \label{atu:eq:p_le}
\end{equation}

Третий подход отличается от второго тем, что по трём точкам вблизи  $M_{i_{m}}$
функция $F(p)$ аппроксимируется параболой, и её абсцисса вершины задаёт искомое
значение параметра. Сместим начало координат в точку
$ ( p_c, F_c ) $. Тогда

\[
  \tilde{p}_c = 0, \,
  \tilde{p}_l = p_l - p_c, \,
  \tilde{p}_r = p_r - p_c.
\]

\[
  \tilde{F}_c = 0, \,
  \tilde{F}_l = F_l - F_c, \,
  \tilde{F}_r = F_r - F_c.
\]

\[
  \left\{
    \begin{array}{l}
      a_2 \tilde{p}_l^2 + a_1 \tilde{p}_l  = \tilde{F}_l
      \\
      a_2 \tilde{p}_r^2 + a_1 \tilde{p}_r  = \tilde{F}_r
    \end{array}
  \right. .
\]

\[
  a_1 = \frac{\tilde{F}_r \tilde{p}_l^2 - \tilde{F}_l \tilde{p}_r^2 }
             { \tilde{p}_l^2 \tilde{p}_r  + \tilde{p}_l \tilde{p}_r^2 }.
\]

\[
  a_2 = \frac{\tilde{F}_r \tilde{p}_l - \tilde{F}_l \tilde{p}_r }
             { \tilde{p}_l^2 \tilde{p}_r  + \tilde{p}_l \tilde{p}_r^2 }.
\]

\begin{equation}
  \tilde{p}_e = - \frac{a_1}{2 a_2};
  \;
  p_e = p_c -- \frac{a_1}{2 a_2}.
  \label{atu:eq:p_e}
\end{equation}


При этом, если
$ p_e \notin [ p_l, p_r ] $, или $ a_2 \ge 0 $, то значение $p_e$ искусственно
ограничивается этим диапазоном. Значение $p_e$ для модели с индексом
$i_m$ обозначим как $p_{ee}$ и будем считать
текущим значением идентифицируемого параметра, полученным с помощью
третьего подхода. Ошибки идентификации в пространстве параметров
для рассмотренных трёх подходов обозначим соответственно:

\begin{equation}
  e_{ge} = p_{ge} - p_o, \;
  e_{le} = p_{le} - p_o, \;
  e_{ee} = p_{ee} - p_o.
  \label{atu:eq:e_xx}
\end{equation}


Динамика изменения параметров моделей (поисковых агентов) задаётся следующим образом:

\begin{equation}
  \frac{dp_c}{dt} = v_f f_t(t),
  \label{atu:eq:dp_dt}
\end{equation}

\noindent
где $f_t$ -- сумма всех действующих ``сил'', $v_f$ -- коэффициент
пропорциональности. Рассматриваем 3 действующие силы
($ f_t = f_c + f_n + f_e $):

\begin{enumerate}
  \item
    $f_c = -k_c (p_c - p_{c,0}) $ -- ``сила притяжения'' к начальному значению
    параметра
    для данной модели. Наличие этой силы не даёт всем моделям принять одно
    и то же значение параметра вблизи экстремума, и, следовательно,
    прекратить процесс поиска. Это также позволяет быстро переключиться
    на другую модель в случае быстрого изменения параметра объекта.

  \item
    $f_n = k_n ( p_r - 2 p_c + p_l ) $ -- ``сила взаимодействия''
    с соседями. Обеспечивает более равномерное распределение
    параметров моделей вблизи экстремума.

  \item
    $f_e = - k_e ( p_c - p_e ) $ -- ``сила притяжения'' к локальной
    оценке экстремума, определённого по выражению~(\ref{atu:eq:p_e}).

\end{enumerate}

Результаты моделирования показали, что в некоторых случаях
имеет смысл введение дополнительных сил ``барьерного'' вида,
для исключения пересечения траекторий поисковых агентов, или же ухода из рабочего диапазона.



Для определения как точности, так и скорости работы системы
идентификации, при моделировании процесса идентификации
значения идентифицируемого параметра будут задаваться такими способами:

\begin{equation}
  p_o(t) = p_0 +  U_{p} \sin( \omega_{p} t ),
  \label{atu:eq:p_sin}
\end{equation}

\begin{equation}
  p_o(t) = p_0 + U_{p} \sign \sin( \omega_{p} t ).
  \label{atu:eq:p_sign}
\end{equation}

При этом будут измеряться среднеквадратические ошибки идентификации,
что позволит исследовать применимость систем идентификации
при различной динамике изменения параметра, а также
настроить параметры самой системы идентификации.


\subsection*{Хаотические тестовые системы }

% 
\FloatBarrier
\section{Система Лоренца} %  % {{{1 _LOR_
\label{atu:sect:lor}

\LinkRef{
  lor: ASAU-22, 23, 24, 25, 26.
  % ~/doc/tex/asau/asau22/atu/atu.tex
  % ~/doc/tex/asau/asau23/atu/atu.tex
  % ~/doc/tex/asau/asau24/atu/atu.tex
  APIR-2012. CSIT-2015. ISDMCI-2014, ISDMCI-2015.
  ITMM-2012, ITMM-2014, ITMM-2015, DSMP-2016
}

\subsection{Определение системы и анализ её динамики} %  % {{{2 _LOR_task

В качестве первой идентифицируемой хаотической системы рассмотрим классическую
систему Лоренца, динамика которой описывается системой
уравнений~\cite{moon_chaotic_vibr,anisch_nonlin_eff,chulichkcov_mm_ml_dyn}:
%
\begin{equation}
\begin{cases}
  \dot{x} = \sigma (y-x ) , \\
  \dot{y} = x (r-z) - y , \\
  \dot{z} = x y - b z .
\end{cases}
\label{atu:eq:lor}
\end{equation}

Наиболее ценным с точки зрения идентификации является параметр
$r$, определяющий как энергетическое состояние системы,
так и вид динамики системы.
Это подтверждают
рассмотренные в дальнейшем физические обоснования.
Для определённости зададим остальные параметры следующим классическим образом:
$b = 2.6666667$, $\sigma = 10$, если не будет явно указано обратное.


При малых значениях параметра $r$ система демонстрирует
затухающие колебания (рис.~\ref{atu:f:lor_attractor_fading}).

\begin{figure}[ht!]
\begin{center}
  \includegraphics[width=0.49\textwidth]{p/cha/lor/lor0-p_xyz_r=022.png}
  \hfill
  \includegraphics[width=0.49\textwidth]{p/cha/lor/lor0-p_t_r=022.png}
\end{center}
  \caption{Аттрактор и поведение переменных состояния системы Лоренца (\ref{atu:eq:lor}) в режиме затухающих колебаний ($r=22$)}
\label{atu:f:lor_attractor_fading}
\end{figure}

Далее,  в широком диапазоне значения параметра $r$
система проявляет хаотическую динамику. Помимо
этого, спектр данной системы в хаотическом режиме довольно широк
(рис.~\ref{atu:f:lor_attractor_phase_chaos28})
и не имеет доминирующих частот, что не характерно для многих систем
динамического хаоса.

\begin{figure}[ht!]
\begin{center}
  \includegraphics[width=0.49\textwidth]{p/cha/lor/lor0-p_xyz_r=028.png}
  \hfill
  \includegraphics[width=0.49\textwidth]{p/cha/lor/lor0_fft-p_f_r=028.png}
\end{center}
  \caption{Аттрактор и спектр системы Лоренца (\ref{atu:eq:lor}) в хаотическом режиме ($r=28$)}
\label{atu:f:lor_attractor_phase_chaos28}
\end{figure}

При дальнейшем росте параметра $r$ динамика системы становится
сложно-периодической, с явно выраженным линейчатым спектром
(рис.~\ref{atu:f:lor_attractor_phase_200})

\begin{figure}[ht!]
\begin{center}
  \includegraphics[width=0.49\textwidth]{p/cha/lor/lor0-p_xyz_r=200.png}
  \hfill
  \includegraphics[width=0.49\textwidth]{p/cha/lor/lor0_fft-p_f_r=200.png}
\end{center}
  \caption{Аттрактор и спектр системы Лоренца (\ref{atu:eq:lor}) в сложно-периодическом режиме ($r=200$)}
\label{atu:f:lor_attractor_phase_200}
\end{figure}



Динамическая система Лоренца является одной из наиболее изученных
хаотических систем~\cite{neimark_stoch_chaos_vibro}. % TODO: more
При этом, существует множество физических систем, для
описания которых применима модель Лоренца. Это дает определённые основания
предполагать, что синтез критерия идентификации, основанного на физических
принципах, для данной системы будет успешным.


Для синтеза критерия идентификации параметра $r$ системы (\ref{atu:eq:lor}), рассмотрим
набор физических систем, для моделирования которых применяется система
Лоренца.

Исторически первой такой системой, рассмотренной самим Лоренцом, является
задача о тепловой конвекции жидкости в плоском слое. Исходная система
уравнений гидродинамики имеет вид:
%
%
\begin{equation}
\begin{cases}
  \pd{\vec{v}}{t} + ( \vec{v} \nabla ) \vec{v} = - \frac{\nabla p}{\rho} + \nu \Delta \vec{v} + \vec{g}, \\
  \pd{\rho}{t} + \nabla ( \rho \vec{v} ) = 0 , \\
  \pd{T}{t} +\nabla ( T \vec{v} ) = \chi \Delta T , \\
  \rho = \rho_0 \left( 1 - \gamma (T - T_0) \right) .
\end{cases}
\label{atu:eq:lor_gidro}
\end{equation}
%
где
$\vec{v} $   -- поле скоростей,
$T$ -- поле температуры,
$T_0$ и $T_0+\Delta T$   -- температуры на верхней и нижней границе соответственно,
$\rho$ и $p$ -- поля плотности и давления,
$g$ -- ускорение свободного падения,
$\nu$, $\chi$, $\gamma$  -- коэффициенты кинематической вязкости, температуропроводности и
теплового расширения соответственно.

При приближении системы (\ref{atu:eq:lor_gidro}) к виду (\ref{atu:eq:lor}),
переменные и параметры системы
Лоренца определяются следующим образом: $x$ задаёт скорость вращения валов
течения, $y$, $z$ -- соответствуют распределению температуры по горизонтали и
вертикали.
$\sigma$   -- число Прандтля (отношение коэффициентов кинематической вязкости и
температуропроводности). Параметр $b$ определяет отношения размеров ячейки.
$r$ -- (идентифицируемый параметр) -- приведённое число Релея, определяющее
энергетические параметры конвекционного течения.

Из трёх переменных состояния проще всего наблюдению поддаётся переменная
$x$. С другой стороны, так как параметр $r$ определяет энергетические
соотношения в системе, то и критерий качества должен представлять собой
квадратичную форму от $x$, причём усреднённую на интервале времени,
существенно большем, чем характерное время оборота жидкостного вала.

Другой системой, для моделирования которой применяется система Лоренца --
это модель одномодового лазера. В этой модели переменной $x$ соответствует
амплитуда поля в резонаторе, $y$ -- поляризации, $z$ -- инверсии заселённости
квантовых уровней активной среды. Параметры $\sigma$ и $b$
определяются отношениями коэффициентов релаксации, а искомый параметр $r$
определяется удельной мощностью накачки.

Как и в случае гидродинамической системы, наиболее просто наблюдаемым
параметром является $x$ -- именно он определяет выходную интенсивность. И
опять же, по аналогии -- идентифицируемый параметр $r$ определяет энергетику
системы. При переходе от амплитуды к мощности совершенно аналогично
следует использовать квадратичную зависимость.

Также система (\ref{atu:eq:lor}) применима для
моделирования конвекции жидкости в замкнутом подогреваемом петле,
динамики водяного колеса, осцилляторе с трением и других~\cite{kuznetsov_dyn_chaos}.

% }}}2


\subsection{Анализ и выбор критериев}  % {{{2

Исходя из вышеизложенного, в первую очередь следует проверить применимость
критерия вида $q_{x^2}$. Тем не менее, проверим все
критерии данного вида, применимые к данной системе.
На рис.~\ref{atu:f:lor_q} приведены исследуемые зависимости
$q_{*}(r)$, полученные путём моделирования динамики
системы (\ref{atu:eq:lor}) для различных значений параметра~$r$,
при усреднении на значительном временном интервале $\tau_q=500$.


\begin{figure}[ht!]
  \centerline{\includegraphics[width=0.7\textwidth]{p/cha/lor/lor_q-p_q_r.png} }
  \caption{Рассматриваемые критерии для системы Лоренца}
  \label{atu:f:lor_q}
\end{figure}

Анализ графиков позволяет сделать вывод, что практически все
рассмотренные виды критериев должны позволять построить
работоспособную систему идентификации. При этом,
большая часть графиков, в том числе и изначально
предложенный $q_{x^2}$, теряют монотонность
при переходе от режима затухающих колебаний к хаотическому,
что может помешать процессу идентификации вблизи этой точки.
Тем не менее, режим затухающих колебаний не представляет
практического интереса, и этим недостатком можно пренебречь.
Этого недостатка лишён критерий $q_{y^2}$, однако,
в рассматриваемых физических задачах значение
$y(t)$ наблюдать сложнее. К тому же, в первом приближении
зависимость $q_{x^2}(r)$ -- линейная, а
$q_{y^2}(r) \sim \sqrt{r}$.
Спектры же сигналов $x(t)$, $y(t)$ и $z(t)$
имеют практически одинаковую структуру.
Поэтому, в дальнейших исследованиях ограничимся
критериями
$q_{x^2}$ и
$q_{y^2}$.

Следующая зависимость, необходимая для синтеза
системы идентификации -- $ \sigma_q(\tau_q) $
или же  $ \sigma_q(a_q) $ -- соотношение между
временем оценивания $\tau_q$ и среднеквадратичной
ошибкой измерения критерия.
Для моделирования непосредственных погрешностей измерения величин
$x(t)$ и $y(t)$ использовался шум с нормальным распределением
и параметрами $\sigma_w=0.5$ и $\tau_w=0.05$.
Для оценивания требуемой зависимости, для каждого
значения $\tau_q$ из заданного диапазона
проводилось $N=200$ процессов моделирования динамики системы,
и в случайный момент (достаточно далеко отстоящий от точки $t=0$ для исключения краевых эффектов)
проводилось измерение и запоминание выбранного критерия.
При этом, для усреднения величины $q$ использовалось 2 метода:
экспненциальное сглаживание, вида (\ref{atu:eq:qlin}), и скользящее среднее~(\ref{atu:eq:moving_avarage}).
Полученные зависимости, обозначенные соответственно
$\sigma_{ql}$ и $\sigma_{qa}$, представлены на
рис.~\ref{atu:f:lor_qy2_tau} и~\ref{atu:f:lor_qx2_tau}.


\begin{figure}[ht!]
\begin{center}
  \includegraphics[width=0.49\textwidth]{p/cha/lor/lor_q_tau-p_aq_sd.png}
  \hfill
  \includegraphics[width=0.49\textwidth]{p/cha/lor/lor_q_tau-p_tau_sd.png}
\end{center}
  \caption{Зависимости $\sigma_{q}(a_q)$ и $\sigma_{q}(\tau_q)$ для системы Лоренца, критерий  $q_{y^2}$}
\label{atu:f:lor_qy2_tau}
\end{figure}


\begin{figure}[ht!]
\begin{center}
  \includegraphics[width=0.49\textwidth]{p/cha/lor/lor_qx2_tau-p_aq_sd.png}
  \hfill
  \includegraphics[width=0.49\textwidth]{p/cha/lor/lor_qx2_tau-p_tau_sd.png}
\end{center}
  \caption{Зависимости $\sigma_{q}(a_q)$ и $\sigma_{q}(\tau_q)$ для системы Лоренца, критерий $q_{x^2}$}
\label{atu:f:lor_qx2_tau}
\end{figure}

Анализ полученных зависимостей позволяет сделать
несколько выводов. Прежде всего, для рассматриваемой системы
результат усреднения с помощью значительно более затратного
в реализации метода скользящего среднего практически везде
уступает более простому методу. Таким образом,
при реализации методов идентификации в условиях с ограниченным ресурсами,
например на микроконтроллерной платформе в реальном времени,
нет смысла реализовывать ресурсоёмкоё скользящее среднее.
Далее, сам вид зависимости оказался достаточно простым:
$ \sigma_q \sim q a_q $ или же
$ \frac{\sigma_q \tau_q}{q} \approx \mathrm{const}$.

% }}}2


\subsection{Тестовая задача идентификации для системы Лоренца}  % {{{2

В соответствии с полученными данными, и используя
предложения~(\ref{atu:eq:po_t_sign}) и~(\ref{atu:eq:po_t_sin}),
определим тестовую задачу следующим образом:
\[
  p(t) \in [20, 60],
\]
%
\begin{equation}
  r_o(t) = p_o(t) = p_0 +  U_{p} \sign \sin( \omega_{p} t ),
  \label{atu:eq:lor_po_t_sign}
\end{equation}
%
%
\begin{equation}
  r_o(t) = p_o(t) = p_0 +  U_{p} \sin( \omega_{p} t ),
  \label{atu:eq:lor_po_t_sin}
\end{equation}
%
где:
$p_0 = 37$, $U_p=12$, $\omega_p=0.09$.

В качестве точки отсчёта рассмотрим применение уже рассмотрено в работе [atu-phd]
метода с одним агентом, управляющим двумя моделями,
и использующий два УГПК и интегратором для определения динамики агента.
Для возможности применения данного метода к системе динамического хаоса
как объект, так и модели были дополнены блоками оценивания критерия $q_{x^2}$.
Таким образом, согласно ведённой классификации,
метод получает обозначение
``Fl2nlosdlcA.$q_{x^2}$''.

Прежде всего, рассмотрим процессы идентификации параметра $r$
системы Лоренца данным методом. Типичные результаты моделирования
приведены на рис.~(\ref{atu:f:lor_id_Fl2nlosdlcA_wp009}).

\begin{figure}[ht!]
  \centerline{
    \includegraphics[width=0.49\textwidth]{p/cha/lor/Fl2nlosdlcA/Fl2nlosdlcA-p_xz_1_wp009.png}
    \hfill
    \includegraphics[width=0.49\textwidth]{p/cha/lor/Fl2nlosdlcA/Fl2nlosdlcA-p_xz_0_wp009.png}
  }
  \caption{Процесс идентификации параметра ``$r$'' системы Лоренца методом Fl2nlosdlcA.$q_{x^2}$ при условиях~(\ref{atu:eq:lor_po_t_sign}) и (\ref{atu:eq:lor_po_t_sin})}
  \label{atu:f:lor_id_Fl2nlosdlcA_wp009}
\end{figure}

Как видно из графиков процессов идентификации,
при заданных условиях и
при достаточно плавном изменении параметров
(\ref{atu:eq:lor_po_t_sin}) процесс идентификации приводит
к положительному результату. Если же значения параметра изменяются
скачкообразно~(\ref{atu:eq:lor_po_t_sign}),
то процесс поиска нарушается -- система просто не успевает
отреагировать на такие изменения. Попытки уменьшить
время реакции за счёт изменения соответствующих
параметров ($a_q$, $k_\omega$, $k_i$) приводят к полному нарушению
процесса поиска. С другой стороны, уменьшение чувствительности
(увеличение $q_\gamma$) даёт возможность восстановить
работоспособность процесса идентификации ценой
значительного увеличения ошибки.

Для проверки тезиса о том, что в данном случае играет роль
именно динамика изменения $p_o(t)$, снизим частоту $\omega_p$ до $0.03$
и проведём моделирование при тех же значениях
всех остальных параметров. Результаты приведены на рис.~\ref{atu:f:lor_id_Fl2nlosdlcA_003}.

\begin{figure}[ht!]
  \centerline{
    \includegraphics[width=0.49\textwidth]{p/cha/lor/Fl2nlosdlcA/Fl2nlosdlcA-p_xz_1_wp003.png}
    \hfill
    \includegraphics[width=0.49\textwidth]{p/cha/lor/Fl2nlosdlcA/Fl2nlosdlcA-p_xz_0_wp003.png}
  }
  \caption{Процесс идентификации параметра ``$r$'' системы Лоренца методом Fl2nlosdlcA.$q_{x^2}$ при $\omega_p=0.03$}
  \label{atu:f:lor_id_Fl2nlosdlcA_003}
\end{figure}

Как и следовало ожидать, при меньшем значении $\omega_p$
оба графика подтверждают работоспособность системы идентификации.

Для оценивания границ применимости метода с одним агентом
и двумя моделями построим зависимости
$\overline{e}_{rge}(\omega_p)$~(рис.~\ref{atu:f:lor_Fl2nlosdlcA_e_omega_p}).

\begin{figure}[ht!]
  \centerline{
    \includegraphics[width=0.49\textwidth]{p/cha/lor/Fl2nlosdlcA/Fl2nlosdlcA-p_omega_p_e_1.png}
    \hfill
    \includegraphics[width=0.49\textwidth]{p/cha/lor/Fl2nlosdlcA/Fl2nlosdlcA-p_omega_p_e_0.png}
  }
  \caption{Зависимости $\overline{e}_{rge}(\omega_p)$ при идентификации системы Лоренца методом Fl2nlosdlcA.$q_{x^2}$}
  \label{atu:f:lor_Fl2nlosdlcA_e_omega_p}
\end{figure}

Совершенно очевидно, что рассматриваемая система идентификации
имеет очень ограниченный диапазон применимости при условии~(\ref{atu:eq:lor_po_t_sign}).
Условие~(\ref{atu:eq:lor_po_t_sin}) не налагает таких жёстких ограничений.
Однако, рассмотрев эту же зависимость в другом масштабе~(рис.~\ref{atu:f:lor_Fl2nlosdlcA_e_omega_p_wide}),
мы увидим, что ограничение всё же присутствует.

\begin{figure}[ht!]
  \centerline{
    \includegraphics[width=0.49\textwidth]{p/cha/lor/Fl2nlosdlcA/Fl2nlosdlcA-p_omega_p_e_0_wide.png}
  }
  \caption{Зависимость $\overline{e}_{rge}(\omega_p)$ при идентификации системы Лоренца методом Fl2nlosdlcA.$q_{x^2}$}
  \label{atu:f:lor_Fl2nlosdlcA_e_omega_p_wide}
\end{figure}

Так как основным недостатком рассмотренного метода является
ограниченная реакция ка резкое изменение параметра,
то имеет смысл выдвинуть предположение,
что использование в системе идентификации нескольких агентов,
распределённых на пространстве параметров, может
скомпенсировать данный недостаток.


Для сравнения были выбраны три группы мультимодельных методов идентификации:
qAuv5.3r.$q_{x^2}$,
qAuv5.3r.$q_{y^2}$ и
FAlv5.3z.$q_{x^2}$. Символ ``A'' на второй позиции каждого из
обозначений~(см.~стр.~\pageref{atu:id_classification})
обозначает, что проверяются
все возможные способы определения $p_\mathrm{id}$,
а количество агентов и способ их группировки (5.3) были выбраны
одинаковыми для корректного сравнения различных походов.


На рис.~\ref{atu:f:lor_id_qAuv5.3r.q_x2_sign} представлены результаты идентификации
методом qAuv5.3r.$q_{x^2}$, при этом на левом графике представлена
динамика перемещения каждой из подвижных моделей,
а на правом -- 4 способа определения $p_{id}(t)$:
$p_{ge}$, $p_{le}$, $p_{ee}$ и $p_{xe}$.
В первую очередь, следует отметить общую работоспособность
методов, и правильную динамику каждой из моделей.


Сравнивая результаты различных способов определения
$p_{id}$ как визуально, так и численно, можно сделать
вывод, что худшие результаты демонстрирует $p_{ee}$.
Этого и следовало ожидать, так как это подход, в первую очередь,
был предназначен для методов идентификации, использующих
функцию качества для определения $p_e$.
Также, совершенно ожидаемо, лучшие результаты продемонстрировал
подход $p_{xe}$. Также, при искусственных ограничения
$v_f=0$ и $q_\gamma=0.1$ были получены величины,
характеризующие качество идентификации для неподвижных агентов:
$\overline{e}_{bc}=9.85$
и
$\overline{e}_{be}=7.09$, что свидетельствует
о оправданности как перемещения агентов,
так и использовании величин $p_e$ для определения $p_{id}$.



\begin{figure}[ht!]
  \centerline{
    \includegraphics[width=0.49\textwidth]{p/cha/lor/qAuv5.3r/lor_qAuv5_3r_qy2-p_t_pi_sign.png}
    \hfill
    \includegraphics[width=0.49\textwidth]{p/cha/lor/qAuv5.3r/lor_qAuv5_3r_qy2-p_t_pz_sign.png}
  }
  \caption{Процесс идентификации параметра ``$r$'' системы Лоренца методом qAuv5.3r.$q_{x^2}$ при условии~(\ref{atu:eq:lor_po_t_sign})}
  \label{atu:f:lor_id_qAuv5.3r.q_x2_sign}
\end{figure}


На рис.~\ref{atu:f:lor_id_qAuv5.3r.q_x2_sin} представлены аналогичные результаты,
но при условии плавного изменение значения параметра объекта. Как и следовало ожидать,
как абсолютные, так и относительные значения ошибок идентификации в данном случае заметно
меньше, при сохранении общей картины.
При этом
$\overline{e}_{bc}=7.80$
и
$\overline{e}_{be}=5.24$.


\begin{figure}[ht!]
  \centerline{
    \includegraphics[width=0.49\textwidth]{p/cha/lor/qAuv5.3r/lor_qAuv5_3r_qy2-p_t_pi_sin.png}
    \hfill
    \includegraphics[width=0.49\textwidth]{p/cha/lor/qAuv5.3r/lor_qAuv5_3r_qy2-p_t_pz_sin.png}
  }
  \caption{Процесс идентификации параметра ``$r$'' системы Лоренца методом qAuv5.3r.$q_{x^2}$ при условии~(\ref{atu:eq:lor_po_t_sin})}
  \label{atu:f:lor_id_qAuv5.3r.q_x2_sin}
\end{figure}


На рис.~\ref{atu:f:lor_id_qAuv5.3r.q_y2_sign} представлены результаты
полученные методом qAuv5.3r.$q_{y^2}$,
отличающего от предыдущего только видом использованного критерия.
При этом
$\overline{e}_{bc}=10.85$
и
$\overline{e}_{be}=7.73$.

\begin{figure}[ht!]
  \centerline{
    \includegraphics[width=0.49\textwidth]{p/cha/lor/qAuv5.3r/lor_qAuv5_3r_qy2-p_t_pi_sign.png}
    \hfill
    \includegraphics[width=0.49\textwidth]{p/cha/lor/qAuv5.3r/lor_qAuv5_3r_qy2-p_t_pz_sign.png}
  }
  \caption{Процесс идентификации параметра ``$r$'' системы Лоренца методом qAuv5.3r.$q_{y^2}$ при условии~(\ref{atu:eq:lor_po_t_sign})}
  \label{atu:f:lor_id_qAuv5.3r.q_y2_sign}
\end{figure}


На рис.~\ref{atu:f:lor_id_qAuv5.3r.q_y2_sin} представлены аналогичные результаты,
только при условии (\ref{atu:eq:lor_po_t_sin}).
При этом
$\overline{e}_{bc}=8.09$
и
$\overline{e}_{be}=5.46$.


\begin{figure}[ht!]
  \centerline{
    \includegraphics[width=0.49\textwidth]{p/cha/lor/qAuv5.3r/lor_qAuv5_3r_qy2-p_t_pi_sin.png}
    \hfill
    \includegraphics[width=0.49\textwidth]{p/cha/lor/qAuv5.3r/lor_qAuv5_3r_qy2-p_t_pz_sin.png}
  }
  \caption{Процесс идентификации параметра ``$r$'' системы Лоренца методом qAuv5.3r.$q_{x^2}$ при условии~(\ref{atu:eq:lor_po_t_sin})}
  \label{atu:f:lor_id_qAuv5.3r.q_y2_sin}
\end{figure}

Также важным результатом является тот факт, что, по сравнению с одноагентным методом,
система идентификации не теряет работоспособности
при резких изменениях параметра. Зависимости $\overline{e}_{rge}(\omega_p)$
для данного семейства методов приведены на рис.~\ref{atu:f:lor_qAuv5.3r_e_omega_p}.


\begin{figure}[ht!]
  \centerline{
    \includegraphics[width=0.49\textwidth]{p/cha/lor/qAuv5.3r/lor_qAuv5_3r_qx2-p_omega_p_e_sign.png}
    \hfill
    \includegraphics[width=0.49\textwidth]{p/cha/lor/qAuv5.3r/lor_qAuv5_3r_qx2-p_omega_p_e_sin.png}
  }
  \caption{Зависимости $\overline{e}_{rge}(\omega_p)$ при идентификации системы Лоренца методом qAuv5.3r.$q_{x^2}$}
  \label{atu:f:lor_qAuv5.3r_e_omega_p}
\end{figure}

Сравнивавшая получившиеся зависимости с аналогичными для одноагентого метода (рис.~\ref{atu:f:lor_Fl2nlosdlcA_e_omega_p}),
можно с уверенностью констатировать,
что применение мультиагентного подхода здесь полностью оправданно.
Во-первых, допустимый диапазон $\omega_p$ расширяется как минимум на порядок. % TODO: limit?
С другой стороны, в этом случае нет принципиальной разницы между
условиями~(\ref{atu:eq:lor_po_t_sign}) и (\ref{atu:eq:lor_po_t_sin}),
что также является несомненным преимуществом.




Сравнивая результаты, представленные на рис.~\ref{atu:f:lor_id_qAuv5.3r.q_x2_sign}--\ref{atu:f:lor_id_qAuv5.3r.q_y2_sin}
можно сделать следующие выводы:

\begin{itemize}

  \item
    Динамика систем идентификации при использовании критериев $q_{x^2}$ и $q_{y^2}$
    практически совпадает.

  \item
    Применение критерия  $q_{y^2}$ в данном конкретном случае позволяет
    меньшую ошибку идентификации, но на настолько,
    чтобы можно было говорить о существенной разнице.

  \item
    Во всех рассмотренных случаях оправданными были и применение нескольких агентов,
    и их смещение в процессе поиска, и определение с помощью $p_e$
    каждого из агентов.

\end{itemize}

Рассмотрим процесс идентификации этой же системы, при тех же условиях,
но семейством методов, основанных на применении функции качества,
а именно FAlv5.3z.$q_{x^2}$

Процесс идентификации при условии~(\ref{atu:eq:lor_po_t_sign})
представлен на рис.~\ref{atu:f:lor_id_FAlv5.3z.q_x2_sign}.
В первую очередь следует отметить большую подвижность
агентов, вплоть до искусственного ограничения подвижности
большей части моделей. С одной стороны, это несколько уменьшает
быстродействие системы при резких изменениях параметра,
с другой -- обеспечивает достаточное смещение ``дальних'' агентов,
что, в какой-то мере, компенсирует возможные ошибки в настройке системы.
Величина $\overline{e}_{bc}=10.84$ имеет примерно такое же значение,
как и предыдущих случаях,
а $\overline{e}_{be}$ в данных условиях не применима.

\begin{figure}[ht!]
  \centerline{
    \includegraphics[width=0.49\textwidth]{p/cha/lor/FAlv5.3z/lor_FAlv5_3z_qx2-pl_n_sign.png}
    \hfill
    \includegraphics[width=0.49\textwidth]{p/cha/lor/FAlv5.3z/lor_FAlv5_3z_qx2-p_p_sign.png}
  }
  \caption{Процесс идентификации параметра ``$r$'' системы Лоренца методом FAlv5.3z.$q_{x^2}$ при условии~(\ref{atu:eq:lor_po_t_sign})}
  \label{atu:f:lor_id_FAlv5.3z.q_x2_sign}
\end{figure}

На рис.~\ref{atu:f:lor_id_FAlv5.3z.q_x2_sin}
процесс идентификации при условии~(\ref{atu:eq:lor_po_t_sin}).
Величина $\overline{e}_{bc}=7.95$ не выбивается из общего ряда,
и общая картина имеет ожидаемый вид.

\begin{figure}[ht!]
  \centerline{
    \includegraphics[width=0.49\textwidth]{p/cha/lor/FAlv5.3z/lor_FAlv5_3z_qx2-pl_n_sin.png}
    \hfill
    \includegraphics[width=0.49\textwidth]{p/cha/lor/FAlv5.3z/lor_FAlv5_3z_qx2-p_p_sin.png}
  }
  \caption{Процесс идентификации параметра ``$r$'' системы Лоренца методом FAlv5.3z.$q_{x^2}$ при условии~(\ref{atu:eq:lor_po_t_sin})}
  \label{atu:f:lor_id_FAlv5.3z.q_x2_sin}
\end{figure}

Также следует отметить, что данный метод, аналогично предыдущему,
проявляет работоспособность при резких изменениях параметра.
Построим зависимости
$\overline{e}_{rge}(\omega_p)$ (рис.~\ref{atu:f:lor_FAlv5.3r_e_omega_p}).


\begin{figure}[ht!]
  \centerline{
    \includegraphics[width=0.49\textwidth]{p/cha/lor/FAlv5.3z/lor_FAlv5_3z_qx2-p_omega_p_e_sign.png}
    \hfill
    \includegraphics[width=0.49\textwidth]{p/cha/lor/FAlv5.3z/lor_FAlv5_3z_qx2-p_omega_p_e_sin.png}
  }
  \caption{Зависимости $\overline{e}_{rge}(\omega_p)$ при идентификации системы Лоренца методом Fblvd1.2.$q_{x^2}$}
  \label{atu:f:lor_FAlv5.3r_e_omega_p}
\end{figure}

Вид этих зависимостей аналогичен таковым для предыдущего метода,
хотя рабочий диапазон по $\omega_p$ незначительно уже.
Таким образом, применение мультимодельного подхода
оправданно, по крайней мере с этой точки трения,
вне зависимости от того, какую величину, $q$ или $F$
используют агенты для определения $p_e$.

% }}}2


% ----------------------------------------- id_params -------------------------------

\subsection{Влияние параметров системы идентификации на ошибку идентификации для системы Лоренца}  % {{{2

Рассмотрим влияние параметров системы идентификации на
процесс идентификации и, следовательно, на её качество.
Для этого будем варьировать каждый из существенных
параметров и строить зависимости $\overline{e}_{r*}$
для каждого из рассмотренных методов.

В первую очередь рассмотрим влияние параметра
$a_q$, определяющего характерное время усреднения.

На рис.~\ref{atu:f:lor_a_q_qAuv5.3r.q_x2} представлены зависимости
усреднённых ошибок идентификации системы Лоренца от $a_q$ при использовании метода qAuv5.3r.$q_{x^2}$.
Достаточно очевидно, что форма кривых с явным экстремумом обусловлена
влияниями противоборствующих факторов. При больших значениях $a_q$,
и, следовательно, малом времени усреднения $\tau_q$,
слишком сильно влияние хаотической динамики, что бы можно было бы
проводить успешную идентификацию. С другой стороны,
при слишком малых значениях $a_q$, время оценивание критерия настолько большое,
что система идентификации не успевает отслеживать изменение параметра.
Этот тезис подтверждается тем фактом, что при более плавном изменении параметра
(\ref{atu:eq:lor_po_t_sin})
минимум ошибок достигается при меньших значениях $a_q$.

\begin{figure}[ht!]
  \centerline{
    \includegraphics[width=0.49\textwidth]{p/cha/lor/qAuv5.3r/lor_qAuv5_3r_qx2-p_a_q_e_sign.png}
    \hfill
    \includegraphics[width=0.49\textwidth]{p/cha/lor/qAuv5.3r/lor_qAuv5_3r_qx2-p_a_q_e_sin.png}
  }
  \caption{Зависимости $\overline{e}_{r*}(a_q)$ при идентификации системы Лоренца методом qAuv5.3r.$q_{x^2}$
   при~(\ref{atu:eq:lor_po_t_sign}) и (\ref{atu:eq:lor_po_t_sin})}
  \label{atu:f:lor_a_q_qAuv5.3r.q_x2}
\end{figure}


На рис.~\ref{atu:f:lor_a_q_qAuv5.3r.q_y2} представлены зависимости
усреднённых ошибок идентификации системы Лоренца при использовании метода qAuv5.3r.$q_{y^2}$.
Здесь ситуация существенно изменяется но сравнению с предыдущим случаем.
Положения экстремумов практически не изменились, но
при больших значениях $a_q$ начинает наблюдаться полное нарушение
процесса поиска. Таким образом, несмотря на то,
что в наилучших условиях критерий $q_{y^2}$ обеспечивает меньшую
ошибку идентификации, диапазон его применимости меньше. С учётом
того, момент нарушения поиска зависит и от формы $p_o(t)$,
а она в реальных задачах заранее не известна, использование этого критерия может
не быть оправданным.

\begin{figure}[ht!]
  \centerline{
    \includegraphics[width=0.49\textwidth]{p/cha/lor/qAuv5.3r/lor_qAuv5_3r_qy2-p_a_q_e_sign.png}
    \hfill
    \includegraphics[width=0.49\textwidth]{p/cha/lor/qAuv5.3r/lor_qAuv5_3r_qy2-p_a_q_e_sin.png}
  }
  \caption{Зависимости $\overline{e}_{r*}(a_q)$ при идентификации системы Лоренца методом qAuv5.3r.$q_{y^2}$
   при~(\ref{atu:eq:lor_po_t_sign}) и (\ref{atu:eq:lor_po_t_sin})}
  \label{atu:f:lor_a_q_qAuv5.3r.q_y2}
\end{figure}


На рис.~\ref{atu:f:lor_a_q_FAlv5.3z.q_x2} представлены зависимости
усреднённых ошибок идентификации системы Лоренца при использовании метода FAlv5.3z.$q_{x^2}$.
Как характер зависимостей, так и положение экстремумов аналогичны
случаю при использовании метода  qAuv5.3r.$q_{x^2}$,
хотя определённые различия наблюдаются. Тем не менее,
можно сделать вывод, что как оптимальные значения величины $a_q$,
так и диапазон применимости в первую очередь определяется
критерием и динамикой изменения параметра, и уже во вторую очередь --
конкретным методом.


\begin{figure}[ht!]
  \centerline{
    \includegraphics[width=0.49\textwidth]{p/cha/lor/FAlv5.3z/lor_FAlv5_3z_qx2-p_a_q_e_sign.png}
    \hfill
    \includegraphics[width=0.49\textwidth]{p/cha/lor/FAlv5.3z/lor_FAlv5_3z_qx2-p_a_q_e_sin.png}
  }
  \caption{Зависимости $\overline{e}_{r*}(a_q)$ при идентификации системы Лоренца методом FAlv5.3z.$q_{x^2}$
   при~(\ref{atu:eq:lor_po_t_sign}) и (\ref{atu:eq:lor_po_t_sin})}
  \label{atu:f:lor_a_q_FAlv5.3z.q_x2}
\end{figure}


Следующий важные параметр -- масштаб функции качества $q_\gamma$.
В первую очередь он должен влиять на методы,
которые используют величину $F$ в процессе поиска.
С другой стороны, методы, использующие только критерий для
определения $p_e$, тоже используют  $q_\gamma$ в процессе
определения $p_{id}$.

На рис.~\ref{atu:f:lor_qg_qAuv5.3r.q_x2} представлены зависимости
усреднённых ошибок идентификации системы Лоренца от $q_\gamma$ при использовании метода qAuv5.3r.$q_{x^2}$.
Как и предполагалось, зависимости достаточно слабые, за исключением $\overline{e}_{rge}$,
для которой большие величины масштаба обозначают малую чувствительность,
и следовательно, избыточное влияние агентов, расположенных
вдали от искомого значения параметра.
Самые устойчиво хорошие результаты демонстрирует $p_{xe}$.

\begin{figure}[ht!]
  \centerline{
    \includegraphics[width=0.49\textwidth]{p/cha/lor/qAuv5.3r/lor_qAuv5_3r_qx2-p_qgamma_e_sign.png}
    \hfill
    \includegraphics[width=0.49\textwidth]{p/cha/lor/qAuv5.3r/lor_qAuv5_3r_qx2-p_qgamma_e_sin.png}
  }
  \caption{Зависимости $\overline{e}_{r*}(q_\gamma)$ при идентификации системы Лоренца методом qAuv5.3r.$q_{x^2}$
   при~(\ref{atu:eq:lor_po_t_sign}) и (\ref{atu:eq:lor_po_t_sin})}
  \label{atu:f:lor_qg_qAuv5.3r.q_x2}
\end{figure}



На рис.~\ref{atu:f:lor_qg_qAuv5.3r.q_y2} представлены зависимости
усреднённых ошибок идентификации системы Лоренца от $q_\gamma$ при использовании метода qAuv5.3r.$q_{y^2}$.
Здесь результаты аналогичны предыдущему случаю. Однако,
данный подход оказался более чувствительным к заниженным значением $q_\gamma$.

\begin{figure}[ht!]
  \centerline{
    \includegraphics[width=0.49\textwidth]{p/cha/lor/qAuv5.3r/lor_qAuv5_3r_qy2-p_qgamma_e_sign.png}
    \hfill
    \includegraphics[width=0.49\textwidth]{p/cha/lor/qAuv5.3r/lor_qAuv5_3r_qy2-p_qgamma_e_sin.png}
  }
  \caption{Зависимости $\overline{e}_{r*}(q_\gamma)$ при идентификации системы Лоренца методом qAuv5.3r.$q_{y^2}$
   при~(\ref{atu:eq:lor_po_t_sign}) и (\ref{atu:eq:lor_po_t_sin})}
  \label{atu:f:lor_qg_qAuv5.3r.q_y2}
\end{figure}


На рис.~\ref{atu:f:lor_qg_FAlv5.3z.q_x2} представлены зависимости
усреднённых ошибок идентификации системы Лоренца от $q_\gamma$ при использовании метода FAlv.3z.$q_{x^2}$.
В этом случае зависимости сильно отличаются от двух предыдущих случаев.
Так как этот метод непосредственно используют функцию
качества для определения $p_e$ каждым агентом, то зависимость
имеет явный экстремальный характер.
При малых значениях $q_\gamma$ чувствительность избыточна,
и процесс поиска нарушается.
При больших -- недостаточна, и помимо избыточного влияния ``дальних'' агентов также снижается скорость и точность поиска.
Влияние  ``дальних'' агентов игнорируется при использовании
$p_{le}$ и $p_{ee}$, однако,
в эти методы теряются точность при переключении
между лучшими агентами.


\begin{figure}[ht!]
  \centerline{
    \includegraphics[width=0.49\textwidth]{p/cha/lor/FAlv5.3z/lor_FAlv5_3z_qx2-p_qg_e_sign.png}
    \hfill
    \includegraphics[width=0.49\textwidth]{p/cha/lor/FAlv5.3z/lor_FAlv5_3z_qx2-p_qg_e_sin.png}
  }
  \caption{Зависимости $\overline{e}_{r*}(q_\gamma)$ при идентификации системы Лоренца методом FAlv5.3z.$q_{x^2}$
   при~(\ref{atu:eq:lor_po_t_sign}) и (\ref{atu:eq:lor_po_t_sin})}
  \label{atu:f:lor_qg_FAlv5.3z.q_x2}
\end{figure}


Ещё один важный параметр, влияющий на динамический свойства
системы идентификации -- $v_f$, определяющий скорость
смещения агента к выбранному значению при постоянной силе.
При нулевом значении этого коэффициента агенты неподвижны,
и можно определить величину $\overline{e}_{be}$,
а если ещё предельно уменьшить $q_\gamma$ --  $\overline{e}_{bb}$.



На рис.~\ref{atu:f:lor_vf_qAuv5.3r.q_x2} представлены зависимости
усреднённых ошибок идентификации системы Лоренца от $v_f$ при использовании метода qAuv5.3r.$q_{x^2}$.
За исключением начального участка, где подвижность агентов
минимальна, зависимость от $v_f$ достаточно слабая.
По мере увеличения $v_f$ немного улучшаются динамические свойства,
но возрастающая скорость перемещения агентов
несколько нивелирует этот эффект. При плавном изменении параметра,
негативный эффект от избыточной подвижности агентов преобладает.

\begin{figure}[ht!]
  \centerline{
    \includegraphics[width=0.49\textwidth]{p/cha/lor/qAuv5.3r/lor_qAuv5_3r_qx2-p_v_f_e_sign.png}
    \hfill
    \includegraphics[width=0.49\textwidth]{p/cha/lor/qAuv5.3r/lor_qAuv5_3r_qx2-p_v_f_e_sin.png}
  }
  \caption{Зависимости $\overline{e}_{r*}(v_f)$ при идентификации системы Лоренца методом qAuv5.3r.$q_{x^2}$
   при~(\ref{atu:eq:lor_po_t_sign}) и (\ref{atu:eq:lor_po_t_sin})}
  \label{atu:f:lor_vf_qAuv5.3r.q_x2}
\end{figure}

Совершенно аналогичная картина
(рис.~\ref{atu:f:lor_vf_qAuv5.3r.q_y2})
наблюдается при использовании критерия $q_{y^2}$.
При этом следует отметить, что при избыточной подвижности
меньшее значение ошибки демонстрируют
подход, основанные на глобальном оценивании $p_{id}$.


\begin{figure}[ht!]
  \centerline{
    \includegraphics[width=0.49\textwidth]{p/cha/lor/qAuv5.3r/lor_qAuv5_3r_qy2-p_v_f_e_sign.png}
    \hfill
    \includegraphics[width=0.49\textwidth]{p/cha/lor/qAuv5.3r/lor_qAuv5_3r_qy2-p_v_f_e_sin.png}
  }
  \caption{Зависимости $\overline{e}_{r*}(v_f)$ при идентификации системы Лоренца методом qAuv5.3r.$q_{y^2}$
   при~(\ref{atu:eq:lor_po_t_sign}) и (\ref{atu:eq:lor_po_t_sin})}
  \label{atu:f:lor_vf_qAuv5.3r.q_y2}
\end{figure}

Несколько другая картина наблюдается при использовании метода  FAlv5.3z.$q_{x^2}$
(рис.~\ref{atu:f:lor_vf_FAlv5.3z.q_x2}).
В достаточно широком диапазоне зависимость достаточно слабая,
а большая скорость приводит к нарушению процесса поиска,
что наблюдается в правой части обеих графиков.

\begin{figure}[ht!]
  \centerline{
    \includegraphics[width=0.49\textwidth]{p/cha/lor/FAlv5.3z/lor_FAlv5_3z_qx2-p_v_f_e_sign.png}
    \hfill
    \includegraphics[width=0.49\textwidth]{p/cha/lor/FAlv5.3z/lor_FAlv5_3z_qx2-p_v_f_e_sin.png}
  }
  \caption{Зависимости $\overline{e}_{r*}(v_f)$ при идентификации системы Лоренца методом FAlv5.3z.$q_{x^2}$
   при~(\ref{atu:eq:lor_po_t_sign}) и (\ref{atu:eq:lor_po_t_sin})}
  \label{atu:f:lor_vf_FAlv5.3z.q_x2}
\end{figure}

Последний из заслуживающих пристального внимания параметр -- $k_e$,
определяющий баланс сил при смещении агентов.
Следует отметить, что в рассматриваемых примерах его влияние различное --
в qAuv5.3r.$q_{x^2}$ и qAuv5.3r.$q_{y^2}$
зависимость $f_e(p_c-p_e)$ с насыщением,
а в FAlv5.3z.$q_{x^2}$ эта зависимость линейная.

На рис.~\ref{atu:f:lor_ke_qAuv5.3r.q_x2} представлены зависимости
усреднённых ошибок идентификации системы Лоренца от $k_e$ при использовании метода qAuv5.3r.$q_{x^2}$.
Малые значения этого коэффициента приводят к там же последствиям, что и малые
значения $v_f$ -- агенты становятся практически неподвижными.

\begin{figure}[ht!]
  \centerline{
    \includegraphics[width=0.49\textwidth]{p/cha/lor/qAuv5.3r/lor_qAuv5_3r_qx2-p_k_e_e_sign.png}
    \hfill
    \includegraphics[width=0.49\textwidth]{p/cha/lor/qAuv5.3r/lor_qAuv5_3r_qx2-p_k_e_e_sin.png}
  }
  \caption{Зависимости $\overline{e}_{r*}(k_e)$ при идентификации системы Лоренца методом qAuv5.3r.$q_{x^2}$
   при~(\ref{atu:eq:lor_po_t_sign}) и (\ref{atu:eq:lor_po_t_sin})}
  \label{atu:f:lor_ke_qAuv5.3r.q_x2}
\end{figure}

Нарушение процесса поиска при использовании критерия $q_{y^2}$~рис.~\ref{atu:f:lor_ke_qAuv5.3r.q_y2}
происходит заметно раньше, чем при использовании $q_y^2$. Это
ещё раз подтверждает меньший диапазон пригодности $q_{y^2}$.

\begin{figure}[ht!]
  \centerline{
    \includegraphics[width=0.49\textwidth]{p/cha/lor/qAuv5.3r/lor_qAuv5_3r_qy2-p_k_e_e_sign.png}
    \hfill
    \includegraphics[width=0.49\textwidth]{p/cha/lor/qAuv5.3r/lor_qAuv5_3r_qy2-p_k_e_e_sin.png}
  }
  \caption{Зависимости $\overline{e}_{r*}(k_e)$ при идентификации системы Лоренца методом qAuv5.3r.$q_{y^2}$
   при~(\ref{atu:eq:lor_po_t_sign}) и (\ref{atu:eq:lor_po_t_sin})}
  \label{atu:f:lor_ke_qAuv5.3r.q_y2}
\end{figure}

На рис.~\ref{atu:f:lor_ke_FAlv5.3z.q_x2} представлены аналогичные зависимости
для метода  FAlv5.3z.$q_{x^2}$. При общей подобной структуре зависимости
не наблюдается нарушение поиска и при вдвое большем значении данного коэффициента.

\begin{figure}[ht!]
  \centerline{
    \includegraphics[width=0.49\textwidth]{p/cha/lor/FAlv5.3z/lor_FAlv5_3z_qx2-p_ke_e_sign.png}
    \hfill
    \includegraphics[width=0.49\textwidth]{p/cha/lor/FAlv5.3z/lor_FAlv5_3z_qx2-p_ke_e_sin.png}
  }
  \caption{Зависимости $\overline{e}_{r*}(k_e)$ при идентификации системы Лоренца методом FAlv5.3z.$q_{x^2}$
   при~(\ref{atu:eq:lor_po_t_sign}) и (\ref{atu:eq:lor_po_t_sin})}
  \label{atu:f:lor_ke_FAlv5.3z.q_x2}
\end{figure}

% }}}2


% ----------------------------------------- F_type -------------------------------

\subsection{Влияние вида функции качества на процесс идентификации}  % {{{2

Рассмотрим влияние типа функции качества на свойства системы идентификации.
Для этого рассмотрим семейство зависимостей
$\overline{e}(q_\gamma)$ для различных видов $F(q)$
(\ref{atu:eq:F_gauss})--(\ref{atu:eq:F_log}).
Для данного исследования выбран метод FAlv5.3z.$q_{x^2}$,
как напрямую использующий функцию качества при
аппроксимации $p_e$.
Из всех видов зависимостей выбранны именно эти,
так как $q_\gamma$ непосредственно входит в определение функции качества.


На рис.~\ref{atu:f:lor_ftype_rge} представлены $\overline{e}_{rge}(q_\gamma)$ 
при использовании метода  FAlv5.3z.$q_{x^2}$.
Следует отметить, что в этом случае функция качества используется два раза:
первый раз в каждом агенте при определении $p_e$,
второй -- при определении глобального $p_{ge}$.
Именно это влияние объясняет плавный прост ошибки идентификации
при росте $q_\gamma$, и следовательно, уменьшению чувствительности в правой части графиков.
В левой части графиков наблюдаются существенные различия
при использовании различных видов функций качества.
Применение параболической, треугольной и логарифмической зависимостей
приводит к резкому росту ошибки при увеличении чувствительности.
Гиперболическая и Гауссовая зависимости остаются применимы и при
завышенной чувствительности функции качества.

\begin{figure}[ht!]
  \centerline{
    \includegraphics[width=0.49\textwidth]{p/cha/lor/FAlv5.3z/f_type/lor_FAlv5_3z_qx2_Ft-p_qg_e_all_sign_rge.png}
    \hfill
    \includegraphics[width=0.49\textwidth]{p/cha/lor/FAlv5.3z/f_type/lor_FAlv5_3z_qx2_Ft-p_qg_e_all_sin_rge.png}
  }
  \caption{Семейства зависимостей $\overline{e}_{rge}(q_\gamma)$ для различных видов функций качества при идентификации системы Лоренца методом FAlv5.3z.$q_{x^2}$
   при~(\ref{atu:eq:lor_po_t_sign}) и (\ref{atu:eq:lor_po_t_sin})}
  \label{atu:f:lor_ftype_rge}
\end{figure}

На рис.~\ref{atu:f:lor_ftype_ree} представлены $\overline{e}_{rge}(q_\gamma)$
при использовании метода  FAlv5.3z.$q_{x^2}$.
Результаты в целом схожи с предыдущими, список видов функций качества, обеспечивающих
более широкий диапазон применимости не изменился. Тем не менее, ввиду того,
что в данном случае используется только локальная оценка, рост
$q_\gamma$ в разумных пределах не приводит к увеличению ошибки идентификации.

\begin{figure}[ht!]
  \centerline{
    \includegraphics[width=0.49\textwidth]{p/cha/lor/FAlv5.3z/f_type/lor_FAlv5_3z_qx2_Ft-p_qg_e_all_sign_ree.png}
    \hfill
    \includegraphics[width=0.49\textwidth]{p/cha/lor/FAlv5.3z/f_type/lor_FAlv5_3z_qx2_Ft-p_qg_e_all_sin_ree.png}
  }
  \caption{Семейства зависимостей $\overline{e}_{ree}(q_\gamma)$ для различных видов функций качества при идентификации системы Лоренца методом FAlv5.3z.$q_{x^2}$
   при~(\ref{atu:eq:lor_po_t_sign}) и (\ref{atu:eq:lor_po_t_sin})}
  \label{atu:f:lor_ftype_ree}
\end{figure}

Таким образом, результаты моделирования при использовании
функций качества видов
(\ref{atu:eq:F_gauss})--(\ref{atu:eq:F_log}) показывают,
что при сравнительно одинаковых результатах в широком
диапазоне $q_\gamma$, лучшие результаты в целом
демонстрируют функции (\ref{atu:eq:F_hyper}) и (\ref{atu:eq:F_gauss}),
за счёт лучшей работы в области высокой чувствительности.
А так как в целом применение гиперболической зависимости
приводит пусть к несущественно, но большей ошибке идентификации в целом,
то в дальнейшем будем применять функцию качества в виде~(\ref{atu:eq:F_gauss}).

% }}}2


% ----------------------------------------- N -------------------------------

\subsection{Влияние количества поисковых агентов на процесс идентификации}  % {{{2

Рассмотрим влияние количества поисковых агентов
на качество идентификации на примере семейства методов FAlvN.3r.$q_{x^2}$.
Структура связей поисковых агентов определяет их минимальное количество $N=3$,
не считая двух неподвижных моделей на границах.
Рассмотрим случаи $N=3,5,7,9$.

В первую очередь отметим,
что из всех параметров рассматриваемой системы идентификации
наиболее тесную и непосредственную связь с количеством
агентов имеет параметр $q_\gamma$. Что должно объясняется тем, что чем больше
агентов на $\mathcal{P}$, тем меньше между ними расстояние в пространстве параметров,
и следовательно, значения критериев идентификации различаются меньше,
что, в свою очередь, требует большей чувствительности.

Рассмотрим получившиеся зависимости.
На рис.~\ref{atu:f:lor_N_rge} представлены $\overline{e}_{rge}(q_\gamma)$
для различных $N$. Здесь получены, на первый взгляд, парадоксальные результаты.
В обеих случаях, при относительно больших значениях $q_\gamma$,
система идентификации с тремя моделями показывает лучшие результаты.
На самом деле, никакого парадокса в полученных данных нет.
При определении $p_{ge}$, а следовательно, а $\overline{e}_{rge}$
учитывается вклад всех моделей, в том числе и расположенных далеко от $p_o$.
При больших значениях $q_\gamma$, и следовательно, при низкой чувствительности,
вклад ``дальних'' моделей возрастает, и, чем больше моделей,
тем больше этот вклад. При нормальной чувствительности (левая часть графиков),
этот эффект практически пропадает. Тем не менее, для данных условий,
использование количества моделей $N>5$ нецелесообразно.



\begin{figure}[ht!]
  \centerline{
    \includegraphics[width=0.49\textwidth]{p/cha/lor/FAlv5.3z/N/lor_FAlvN_3z_qx2_p_qg_e_rge_sign.png}
    \hfill
    \includegraphics[width=0.49\textwidth]{p/cha/lor/FAlv5.3z/N/lor_FAlvN_3z_qx2_p_qg_e_rge_sin.png}
  }
  \caption{Семейства зависимостей $\overline{e}_{rge}(q_\gamma)$ для различных значений $N$ при идентификации системы Лоренца методом FAlv5.3r.$q_{x^2}$
   при~(\ref{atu:eq:lor_po_t_sign}) и (\ref{atu:eq:lor_po_t_sin})}
  \label{atu:f:lor_N_rge}
\end{figure}


Применение локальных методов оценивания $p_{id}$,
с ограниченным количеством используемых моделей, не должно быть подвержено
этому явлению. Для проверки этого тезиса рассмотрим зависимости
$\overline{e}_{ree}(q_\gamma)$
представленные на рис.~\ref{atu:f:lor_N_rge}
для того же набора $N$.

\begin{figure}[ht!]
  \centerline{
    \includegraphics[width=0.49\textwidth]{p/cha/lor/FAlv5.3z/N/lor_FAlvN_3z_qx2_p_qg_e_ree_sign.png}
    \hfill
    \includegraphics[width=0.49\textwidth]{p/cha/lor/FAlv5.3z/N/lor_FAlvN_3z_qx2_p_qg_e_ree_sin.png}
  }
  \caption{Семейства зависимостей $\overline{e}_{ree}(q_\gamma)$ для различных значений $N$ при идентификации системы Лоренца методом FAlv5.3r.$q_{x^2}$
   при~(\ref{atu:eq:lor_po_t_sign}) и (\ref{atu:eq:lor_po_t_sin})}
  \label{atu:f:lor_N_ree}
\end{figure}

В этом случае при больших значениях $q_\gamma$ все рассматриваемые ошибки идентификации
практически совпадают, и проявляют достаточно слабую зависимость.
В левой части графиков, система с $N=3$ заметно
проигрывает системе с $N=5$, а остальные не превосходят последнюю.

\Cmt{Обосновать такое ограничение}



% }}}2


% ----------------------------------------- multi param -------------------------------

\subsection{Зависимости значений критериев идентификации при изменении двух параметров системы Лоренца}  % {{{2

Для оценивания возможности одновременной идентификации нескольких параметров
рассмотрим графики зависимостей критериев, но при условии
изменения двух параметров попарно: ($r$,$\sigma$) и ($r$,$b$).
При этом воспользуется структурой системы ~(\ref{atu:eq:lor})
и введём ещё один дополнительный критерий: $q_{(x-y)^2}$.

На рис.~\ref{atu:f:lor_qx2_r_b} приведена зависимость
$q_{x^2}(r,b)$.
Анализ вида этой поверхности позволяет сделать вывод,
что одного этого критерия недостаточно для
одновременной идентификации параметров $r$ и $b$.
Явно выраженный ``овраг'' на графике соответствует
переходу из режима затухания в хаотический,
следовательно, не имеет принципиального значения.

\begin{figure}[ht!]
  \centerline{  \includegraphics[width=0.60\textwidth]{p/cha/lor/q2d/lor_qx2_r_b.png}  }
  \caption{Зависимость $q_{x^2}(r,b)$ для системы Лоренца}
  \label{atu:f:lor_qx2_r_b}
\end{figure}


На рис.~\ref{atu:f:lor_qy2_r_b} приведена зависимость
$q_{y^2}(r,b)$. Вид этой поверхности практически не отличается от предыдущей.
При этом совершенно не наблюдается никаких очевидных причин обнаруженному
в предыдущих исследованиях
более узкому диапазону применимости данного критерия.

\begin{figure}[ht!]
  \centerline{  \includegraphics[width=0.60\textwidth]{p/cha/lor/q2d/lor_qy2_r_b.png}  }
  \caption{Зависимость $q_{y^2}(r,b)$ для системы Лоренца}
  \label{atu:f:lor_qy2_r_b}
\end{figure}

На рис.~\ref{atu:f:lor_qz2_r_b} приведена зависимость
$q_{z^2}(r,b)$.
Вид данная зависимости существенно отличается от двух предыдущих.
Близкая к квадратичной зависимость от параметра $r$
не представляет особых проблем. Полезным является тот факт,
что для данного критерия зависимость от параметра $b$
достаточно мала. Следовательно, при двупараметрической идентификации
данный критерий имеет смысл  использовать для (оценочной)
идентификации параметра $r$, и совокупность данного критерия
вместе с, например, $q_{x^2}$ позволит идентифицировать
оба параметра без избыточного количества моделей.

\begin{figure}[ht!]
  \centerline{  \includegraphics[width=0.60\textwidth]{p/cha/lor/q2d/lor_qz2_r_b.png}  }
  \caption{Зависимость $q_{z^2}(r,b)$ для системы Лоренца}
  \label{atu:f:lor_qz2_r_b}
\end{figure}

На рис.~\ref{atu:f:lor_qxmy2_r_b} приведена зависимость
$q_{(x-y)^2}(r,b)$.
Очевидно, что для идентификации данный критерий
практически непригоден, тем не менее,
он отличен от нуля только там, где системы не
находится в режиме затухания, что может быть полезно
для быстрого определения режима работы.

\begin{figure}[ht!]
  \centerline{  \includegraphics[width=0.60\textwidth]{p/cha/lor/q2d/lor_qxmy2_r_b.png}  }
  \caption{Зависимость $q_{(x-y)^2}(r,b)$ для системы Лоренца}
  \label{atu:f:lor_qxmy2_r_b}
\end{figure}


На рис.~\ref{atu:f:lor_qx2_r_sigma} приведена зависимость
$q_{x^2}(r,\sigma)$.
Поведение этого критерия на паре параметров $(r,\sigma)$
существенно отличается от его же поведения на паре $(r,b)$.
За исключением узкого ``оврага'', зависимость от параметра $\sigma$
пренебрежимо мала, что даёт основания для
независимой идентификации параметра~$r$.

\begin{figure}[ht!]
  \centerline{  \includegraphics[width=0.60\textwidth]{p/cha/lor/q2d/lor_qx2_r_sigma.png}  }
  \caption{Зависимость $q_{x^2}(r,\sigma)$ для системы Лоренца}
  \label{atu:f:lor_qx2_r_sigma}
\end{figure}


На рис.~\ref{atu:f:lor_qy2_r_sigma} приведена зависимость
$q_{y^2}(r,\sigma)$.
Эта зависимость в целом похожа на предыдущую,
однако, на месте ``оврага'' наблюдается ``хребет''.

\begin{figure}[ht!]
  \centerline{  \includegraphics[width=0.60\textwidth]{p/cha/lor/q2d/lor_qy2_r_sigma.png}  }
  \caption{Зависимость $q_{y^2}(r,\sigma)$ для системы Лоренца}
  \label{atu:f:lor_qy2_r_sigma}
\end{figure}

На рис.~\ref{atu:f:lor_qz2_r_sigma} приведена зависимость
$q_{z^2}(r,\sigma)$. В отличие от применения данного
критерия на паре $(r,b)$, не наблюдается принципиальной
разницы между данным критерием, и критерием $q_{x^2}$.
Отсутствие существенного различия не даёт достаточных
оснований для одновременной идентификации
пары параметров $(r,\sigma)$ с применением только этих
двух критериев.

\begin{figure}[ht!]
  \centerline{  \includegraphics[width=0.60\textwidth]{p/cha/lor/q2d/lor_qz2_r_sigma.png}  }
  \caption{Зависимость $q_{z^2}(r,\sigma)$ для системы Лоренца}
  \label{atu:f:lor_qz2_r_sigma}
\end{figure}

На рис.~\ref{atu:f:lor_qxmy2_r_sigma} приведена зависимость
$q_{(x-y)^2}(r,\sigma)$. Эта зависимость принципиально не отличается
от представленной на рис.~(\ref{atu:f:lor_qxmy2_r_b}),
за исключением того, что присутствует ``хребет''.

\begin{figure}[ht!]
  \centerline{  \includegraphics[width=0.60\textwidth]{p/cha/lor/q2d/lor_qxmy2_r_sigma.png}  }
  \caption{Зависимость $q_{(x-y)^2}(r,\sigma)$ для системы Лоренца}
  \label{atu:f:lor_qxmy2_r_sigma}
\end{figure}

% }}}2


\subsection{Выводы}  % {{{2

Результаты моделирования как собственно динамики
системы Лоренца,
так и процессов идентификации параметра ``$r$''
позволяют в целом сделать следующие выводы:

\begin{itemize}

  \item
    Для синтеза работоспособной системы идентификации параметра  ``$r$''
    системы Лоренца могут применяться практически всё рассмотренные критерии.
    При этом, больший диапазон применимости -- у критерия~$q_{x^2}$.

  \item
    Система мультиагентной идентификации может быть
    реализована как с агентами, осуществляющими поиск
    как с использованием критерия, так и с использованием
    функции качества. При этом, ошибка идентификации
    определяется в основном не конкретным методом,
    а динамическими свойствами самого идентифицируемого
    объекта и используемым временем усреднения критерия.

  \item
    Для рассматриваемого типа объектов большинство параметров
    системы идентификации допускают изменение в достаточно
    широком диапазоне без потери работоспособности,
    что упрощает применение системы идентификации
    в случаях, когда нет возможности проводить
    широкий набор предварительных измерений.
    Исключением является параметр $a_q$, при этом его допустимый диапазон
    определяется динамическим свойствами объекта.

  \item
    Применение функции качества в виде~(\ref{atu:eq:F_gauss})
    позволяет в целом получить лучшие результаты идентификации.

  \item
    Существует определённое количество агентов,
    превышение которого не увеличивает качество идентификации,
    а в некоторых случаях, например, при использовании глобальных
    методов оценивания $p_{id}$ и низкой чувствительности,
    приводит к ухудшению свойств системы идентификации.

  \item
    Возможна одновременная идентификация параметров $r$ и $b$,
    при условии совместного применения критериев $q_{x^2}$ и $q_{z^2}$.

\end{itemize}

% }}}2


% }}}1




% habr: 
% 1. Конвекция в тороидальной трубе (Ланда П.С. Нелинейные колебания и волны. — М: Либроком, 2010, с. 454-455)
% 2. Одномодовый лазер (Покровский Л.А. Решение системы уравнений Лоренца в
%  асимптотическом пределе большого числа Релея. I. Система Лоренца в простейшей
%  квантовой модели лазера и приложение к ней метода усреднения // Теоретическая
%  и математическая физика, 1985, т. 62, №2, с. 272-290);
% 3. Осциллятор с инерционным возбуждением (Неймарк Ю.И., Ланда П.С.
% Стохастические и хаотические колебания. — М: Либроком, 2009, с. 288-295).

% vim: fdm=marker ft=tex
 % unfilled, but is near to good

% 
\FloatBarrier
\subsection{Система Дуффинга} % _DUFF_

\LinkRef{
 duff: ASAU-12, 15. APIR-2009. DSMP-2016
}

\begin{equation}
 \ddot{x} + c_0 \dot{x} + \alpha x + \beta x^3 = u(t) ,
\label{atu:eq:duff}
\end{equation}
%
\begin{equation}
 m \ddot{x} + \nu \dot{x} + k_1 x + k_3 x^3 = F(t) ,
\label{atu:eq:duff_phys}
\end{equation}

Здесь $m$ -- масса объекта,
$x(t)$ -- координата (выходной сигнал),
$u(t) = U_{in} \sin( \omega_{in} t ) $ -- внешняя возмущающая сила,
$ k_1 $ -- коэффициент линейной компоненты возвращающей силы,
$ k_3 $ -- коэффициент при нелинейной части,
$ \alpha $ -- безразмерный коэффициент линейной компоненты возвращающей силы,
 при $ \alpha >0 $ и отсутствии нелинейности пределяет собственную частоту: ($\Omega_0^2 = \alpha $),
$ \nu $ и $ c_0$ -- размерный и безразмерный коэффициенты демпфирования,
$ \beta $ -- безразмерный коэффициент нелинейной части.

Идентифицируемый параметр:
$ \beta \approx 2 $.

Остальные параметры:
\(U_{in}=1\), \(\omega_{in}=1\),
\(c_0 = 0.05\), \( \Omega_0 = 1 \).

\begin{figure}[htb!]
\centerline{\includegraphics[width=0.5\textwidth]{p/cha/duff_phase.pdf} }
\caption{Фазовый портрет системы Дуффинга (\ref{atu:eq:duff})}
\label{atu:f:duff_phase}
\end{figure}

Критерий
$\overline{x^2}$




\FloatBarrier

\section{Система Чуа} %  % {{{1 _CHUA_
\label{atu:sect:chua}

\LinkRef{
 chua: ASAU-18, MKMM-2014, APIR-2011
}

\subsection{Определение системы и анализ её динамики} %  % {{{2 _chua_task

Одной из известных хаотических систем, легко реализуемых как модельно (\ref{atu:eq:chua}),
так и схемотехнически (рис.~\ref{atu:f:chuascheme}),
является нелинейная система Чуа~\cite{moon_chaotic_vibr,buga_chua,Kennedy92robustop}:

\begin{figure}[htb!]
\begin{center}
% vi:syntax=tex

\begin{circuitikz}[line width=0.7]
  \ctikzset{bipoles/thickness=2}
  \def\Top{3.0}
  \draw (0.0,0.0) to[L,l=$L$,i=$I_L$] (0,\Top)
   to[R=R] (6.0,\Top)
   to[ageneric,l=$R_c$] (6.0,0.0) -- (0.0,0.0);
  \draw(1.5,0.0) to[C,l=$C_2$,v=$V_2$] (1.5,\Top );
  \draw(4.5,0.0) to[C,l=$C_1$,v=$V_1$] (4.5,\Top );
\end{circuitikz}

% \begin{tikzpicture}[circuit ee IEC,very thick,circuit symbol unit=3.5mm]
%   \node (L1) at (0,1.5) [point up,elelem,inductor={info = $L$}] {};
%   \node      at (0.2,2.4) {$I_L$};
%   \node (C2) at (1.5,1.5) [point up,elelem,capacitor={info = $C_2$}] {};
%   \node      at (1.8,1.8) {$V_2$};
%   \node (pc2d) at (1.5,0) [contact] {};
%   \node (pc2u) at (1.5,3) [contact] {};
%   \node (C1) at (5.0,1.5) [point up,elelem,capacitor={info = $C_1$}] {};
%   \node      at (5.3,1.8) {$V_1$};
%   \node (pc1d) at (5,0) [contact] {};
%   \node (pc1u) at (5,3) [contact] {};
%   \node (R) at (3,3) [elelem,resistor={info = $R$}] {};
%   \node (Rc) at (7,1.5) [elelem,point up,resistor={info = $R_c$}] {};
%   \draw (Rc) ++(-0.15,-0.7) rectangle+(0.3,0.2);
%   \draw (L1) |- (pc2d) -- (pc1d) -| (Rc) [wire];
%   \draw (L1) |- (pc2u) -- (R) -- (pc1u) -| (Rc) [wire];
%   \draw (pc2u) -- (C2) [wire]; \draw (pc2d) -- (C2) [wire];
%   \draw (pc1u) -- (C1) [wire]; \draw (pc1d) -- (C1) [wire];
%   \node (Gr) at (5,-0.3) [elelem,point down,ground] {};
%   \draw (pc1d) -- (Gr) [wire];
% \end{tikzpicture}

\end{center}
\caption{Условная электрическая цепь, реализующая хаотическую систему Чуа}
\label{atu:f:chuascheme}
\end{figure}


\begin{equation}
\begin{cases}
  C_1 \dot{V_1}  = \frac{1}{R} ( V_2 - V_1 ) - g(V_1), \\
  C_2 \dot{V_2}  = \frac{1}{R} ( V_1 - V_2 ) + I_L, \\
  \dot{I_L}      = - \frac{1}{L} V_2 .
\end{cases}
\label{atu:eq:chua}
\end{equation}

Единственным нелинейным элементом в данной системе является ``диод Чуа''
(обозначен на схеме как $R_c$) с
характеристикой $g(V)$~(рис.~\ref{atu:f:diodchua}),
обладающей различным отрицательными наклонами
($m_0$ и $m_1$) на разных участках,
и тем самым являющийся управляемым источником энергии.
%
%
\begin{equation}
g(V) =
\begin{cases}
  m_1 V = ( m_0 + m_2 ) V , & |V| <   U_0, \\
  m_0 V ,                   & |V| \ge U_0.
\end{cases}
\label{atu:eq:diodchua}
\end{equation}

\begin{figure}[htb!]
\begin{center}
% vi:syntax=tex
\begin{tikzpicture}
  \coordinate (XMIN) at (-4.5,0.0);
  \coordinate (XMAX) at ( 4.5,0.0);
  \coordinate (YMIN) at (0,-2.5);
  \coordinate (YMAX) at (0, 2.5);
  \draw (XMIN) -- (XMAX) [medline,->] node[below] {$U$};
  \draw (YMIN) -- (YMAX) [medline,->] node[left]  {$I$};
  \draw (-4,2) -- (-1,1) -- (1,-1) -- (4,-2) [boldline];
  \draw (1,-1) -- (1,0) [dashed,medline];
  \draw (1,-1) -- (4,-1) [dashed,medline];
  \draw (1.41,0) arc [medline,->,start angle=0,end angle=-45,radius=1.41];
  \draw (1,-1) ++(2,0) arc [medline,->,start angle=0,end angle=-18,radius=2.0];
  \filldraw (1,0) circle[radius=0.05,fill=black] node[above] {$U_0$};
  \node[right] at (1.2,-0.6) {$\alpha_1: \tan(\alpha_1)=m_1$};
  \node[right] at (3.0,-1.3) {$\alpha_0$};
\end{tikzpicture}

\end{center}
\caption{Характеристика \(I=g(V)\) диода Чуа}
\label{atu:f:diodchua}
\end{figure}


При этом параметр \(m_0\) определяет поступление энергии в систему
при больших амплитудах \(V_1\), и, в целом, характеризует
энергетические возможности источника.
Аналогично, параметр \(m_1\) характеризует поступление энергии
при малых колебаниях, в частности, определяет, будет ли
система переходить в колебательное состояние при малых начальных
возмущениях, и какой будет режим этих колебаний.
С другой стороны, поскольку параметр \(m_1\) является суммой
``глобального параметра'' \(m_0\) и ``довеска'',
определяющего дополнительный вклад при малых амплитудах,
то имеет смысл перейти от параметра \(m_1\) к параметру
\( m_2 = m_1 - m_0 \), который
и определяет в целом нелинейность
системы. При \( m_2 = 0 \) система становится линейной
и не представляет особого интереса. Поэтому
в данной работе в качестве
идентифицируемого параметра рассматривается именно параметр \(m_2\).

Важно, что в зависимости от величины параметра $m_2$,
система может переходить в режимы затухания,
периодического и сложно-периодического движения, а также в режим
хаотических колебаний~\cite{anisch_nonlin_eff,magni_new_meth}. При этом сложно-периодическое и хаотическое
движения чередуются с изменением величины \(m_2\).


Классически параметры системы Чуа задаются следующим образом~\cite{buga_chua}:
$C_1 = 1/9$, $C_2 = 1$, $L= 1/7$, $R = 1/0.7$, $m_0=-0.5$, $ m_2 \in [ -0.15; -0.7 ] $.

Введём обозначения:
\[
  a_{11} = \frac{1}{R C_1}; \;
  a_{13} = \frac{1}{C_1}; \;
  a_{21} = \frac{1}{R C_2}; \;
  a_{23} = \frac{1}{C_2}; \;
  a_{31} = -\frac{1}{L}; \;
  a_g = - \frac{m_0}{C_1}; \;
  \mu = - \frac{m_2}{C_1}.
\]
%
%\noindent
Тогда:
%
\begin{equation}
\begin{cases}
  \dot{V}_1  = -a_{11} V_1 + a_{11}  V_2  + g_1(V_1) , \\
  \dot{V}_2  = +a_{21} V_1 - a_{21}  V_2  + a_{23} I_L    , \\
  \dot{I}_L  =  a_{31} V_2.
\end{cases}
\label{atu:eq:chua2}
\end{equation}
%
%
\begin{equation}
g_1(V) =
\begin{cases}
  ( a_g + \mu ) V , & |V| <   U_0, \\
  a_g V           , & |V| \ge U_0.
\end{cases}
\label{atu:eq:diodchua2}
\end{equation}

При этом пересчитанные классические значения параметров будут представлены следующим образом:
$ a_{11} = 6.5 $, $a_{21} = 0.7$, $ a_{23} = -7 $, $ a_g = 4.5 $,
$ \mu \in [ 1.29 ; 5.6 ] $.
Соответственно, в этих обозначениях
идентифицируемым параметров является $\mu$.



\begin{figure}[htb!]
\centerline{
  \includegraphics[width=0.49\textwidth]{p/cha/chua/chua_1-p_xyz_mu=2x74.png}
  \includegraphics[width=0.49\textwidth]{p/cha/chua/chua_1-p_xyz_mu=4x50.png}
}
\caption{Аттрактор системы Чуа (\ref{atu:eq:chua2}) при различных значениях $\mu$}
\label{atu:f:chua_phase}
\end{figure}


% }}}2

\subsection{Анализ и выбор критериев}  % {{{2


Для определения вида критерия рассмотрим зависимости
$q_{*}(\mu) $ (индексом ``*'' будем обозначать применение какого-либо из индексов,
обозначающих исходный сигнал для критерия и способ усреднения),
полученные путём моделирования
для системы Чуа (рис.~\ref{atu:f:chua_q}):

\begin{figure}[htb!]
\centerline{
  \includegraphics[width=0.49\textwidth]{p/cha/chua/chua_q-p_mu2.png}
  \includegraphics[width=0.49\textwidth]{p/cha/chua/chua_q-p_mu1.png}
}
  \caption{Зависимости $q_{*}(\mu) $ для системы Чуа (\ref{atu:eq:chua2})}
\label{atu:f:chua_q}
\end{figure}

Из графиков очевидно, что величина $ q_{V_1}(\mu) $
является наилучшим кандидатом в критерии, ввиду близкой к линейной зависимости
в рабочем диапазоне.

Следующим важным параметром, необходимым для эффективной работы системы идентификации, является
характерное время время оценивания $\tau$ (\ref{atu:eq:qlin}), или же обратная ему величина $a_q$.

Для предварительного оценивания величины $a_q$ рассмотрим спектры системы при различных
$\mu$ (рис.~\ref{atu:f:chua_spectrum}). Как следует из графиков, спектр системы достаточно
ограничен сверху, однако, в хаотическом режиме является сплошным практически до нуля.
Это не даёт возможности непосредственно определить $a_q$ исходя из спектра,
однако, первые существенные пики наблюдаются при $ \omega \approx 0.3 $, следовательно,
первоначальное значение $a_q$ можно оценить как $ a_q \approx 0.3 / \pi \approx 0.1 $.


\begin{figure}[htb!]
\centerline{
  \includegraphics[width=0.32\textwidth]{p/cha/chua/chua_f-p_f_mu=2x00.png}
  \includegraphics[width=0.32\textwidth]{p/cha/chua/chua_f-p_f_mu=2x74.png}
  \includegraphics[width=0.32\textwidth]{p/cha/chua/chua_f-p_f_mu=4x50.png}
}
\caption{Спектры системы Чуа (\ref{atu:eq:chua2}) при различных значениях $\mu$}
\label{atu:f:chua_spectrum}
\end{figure}

Следующий способ оценить $a_q$ -- исследовать полученную в результате моделирования
зависимость среднеквадратичного отклонения $e_q$ оценки величины $q$, нормированной
на саму величину $q$ в стационарном случае. Полученная зависимость представлена
на рис.~\ref{atu:f:chua_tau}.

\begin{figure}[htb!]
\centerline{
  \includegraphics[width=0.4\textwidth]{p/cha/chua/chua_tau-p_e_a.png}
}
\caption{Типичная зависимость $e_q/q(a_q)$ для системы (\ref{atu:eq:chua2})}
\label{atu:f:chua_tau}
\end{figure}

Из этого графика можно сделать вывод, что первоначальная оценка $a_q$
была сделана корректно.

% }}}2


В качестве системы идентификации использовалась система с 5 поисковыми агентами и
двумя неподвижными моделями. Для исследования динамических свойств системы идентификации
параметр $\mu_o$ для объекта задавался двумя способами:
%
\begin{equation}
 \mu_o(t) = p_0 + U_p \sign \sin( \omega_p t )
  \label{atu:eq:chua_mu_sign}
\end{equation}
%
\begin{equation}
 \mu_o(t) = p_0 + U_p \sin( \omega_p t ).
  \label{atu:eq:chua_mu_sin}
\end{equation}

Динамика процессов идентификации для системы Чуа представлена на рис.~\ref{atu:f:chua_id}.


\begin{figure}[htb!]
\centerline{
  \includegraphics[width=0.49\textwidth]{p/cha/chua/chua_m5p-pl_n_sign.png}
  \includegraphics[width=0.49\textwidth]{p/cha/chua/chua_m5p-pl_n_sin.png}
}
\caption{Процесс идентификации параметра $\mu$ системы (\ref{atu:eq:chua2})
  при условиях (\ref{atu:eq:chua_mu_sign}) и (\ref{atu:eq:chua_mu_sin})
}
\label{atu:f:chua_id}
\end{figure}



\subsection{Влияние параметров системы идентификации на ошибку идентификации для системы Дуффинга}  % {{{2



Для более точной настройки параметров самой системы идентификации
рассмотрим зависимости среднеквадратических ошибок идентификации
от основных параметров системы идентификации.

Одним из важнейших параметров является $q_\gamma$ -- масштаб функции качества
((\ref{atu:eq:F_gauss})--(\ref{atu:eq:F_log})).
Зависимости для этого параметра приведены на рис.~\ref{atu:f:chua_e_qgamma}.

\begin{figure}[htb!]
\centerline{
  \includegraphics[width=0.49\textwidth]{p/cha/chua/chua_m5p-p_qg_e_sign.png}
  \includegraphics[width=0.49\textwidth]{p/cha/chua/chua_m5p-p_qg_e_sin.png}
}
  \caption{Зависимости  $\overline{e}(q_\gamma)$ для системы (\ref{atu:eq:chua2})
  при условиях (\ref{atu:eq:chua_mu_sign}) и (\ref{atu:eq:chua_mu_sin})
}
\label{atu:f:chua_e_qgamma}
\end{figure}

Явно выраженного экстремума не наблюдается, что свидетельствует
о сильной робастности метода.

Для проверки корректности выбора величины $a_q$ была построены зависимости
среднеквадратических ошибок идентификации (рис.\ref{atu:f:chua_e_a_q}).


\begin{figure}[htb!]
\centerline{
  \includegraphics[width=0.49\textwidth]{p/cha/chua/chua_m5p-p_a_q_e_sign.png}
  \includegraphics[width=0.49\textwidth]{p/cha/chua/chua_m5p-p_a_q_e_sin.png}
}
  \caption{Зависимости  $\overline{e}(a_q)$ для системы (\ref{atu:eq:chua2})
  при условиях (\ref{atu:eq:chua_mu_sign}) и (\ref{atu:eq:chua_mu_sin})
}
\label{atu:f:chua_e_a_q}
\end{figure}

Как видно, первоначальная оценка ``правильного'' значения величины $a_q$
была сделана достаточно точно. При этом, при синусоидальном изменении параметра объекта
меньшая ошибка идентификации наблюдается при меньших значениях $a_q$, что связано
с тем, в данном случае нет необходимости в слежении за скачкообразно изменяющимся параметром,
и, следовательно, допустимо большее время оценивания. Напротив, в случае (\ref{atu:eq:chua_mu_sign})
увеличение времени оценивания приводит к заметному снижению интегральной точности идентификации.

% }}}2


\subsection{Выводы}  % {{{2

Выводы

% }}}2


% }}}1

% vim: fdm=marker ft=tex
 % good

% 
\FloatBarrier
\section{Система Рёсслера} %  % {{{1 _ROSS_
\label{atu:sect:ross}

\LinkRef{
  ross: ASAU-14. ISDMCI-2011, ISDMCI-2012
  % ~/doc/tex/asau/asau14/atu/atu.tex
}

\subsection{Определение системы и анализ её динамики} %  % {{{2 _ross_task

In~\cite{neimark_stoch_chaos_vibro,koltsova_nl_dyn_chem,berje_order_in_chaos,chulichkcov_mm_ml_dyn}

\begin{equation}
\begin{cases}
  \dot{x}  = -y - z  ,  \\
  \dot{y}  = x + a y ,\\
  \dot{z}  = b + z \cdot ( x-c ) .
\end{cases}
\label{atu:eq:rossler}
\end{equation}

Здесь \(x\), \(y\), \(z\) -- переменные состояния системы,
которые соответствуют концентрациям основных реагентов
в моделируемой химической системе.
Соответственно \(a\), \(b\), \(c\) --
параметры, определяющие динамику системы
(в моделируемой системе определяются константами химического равновесия
и концентрациями вспомогательных реагентов).

При моделировании данной системы положим
\(a=0.25\), \(b=1\).
В этом случае параметр \(c\) определяет
тип динамики системы.
Определение значения данного параметра и будет
целью задачи идентификации.

В данной системе нет внешнего входного сигнала \( u(t) \).
Это объясняется тем, что за счет
поддержания постоянных концентраций вспомогательных
компонент, постоянного пополнения исходных веществ
и удаления продуктов реакции система обладает
собственным источником энергии, который обеспечивает
динамику системы и при отсутствии
внешнего воздействия.

Как и другие системы хаотической динамики, система Ресслера
не позволяет построить систему идентификации, основанную
на формировании критерия качества идентификации
как меры близости непосредственных значений выходных сигналов
объекта \( x_0(t) \) и модели \( x_m(t) \).
Более того, сам вид поведения данной системы может значительно изменяться
при малых изменениях параметров, совершая переход от
хаотического к сложно-периодическому и обратно.

При малых значениях параметра (\(c \approx 3 \))
система проявляет регулярную динамику,
совершая колебательное движение вокруг точки
неустойчивого равновесия~(рис.~\ref{atu:f:ross_attractor_0300}).

\begin{figure}[ht!]
\begin{center}
  \includegraphics[width=0.49\textwidth]{p/cha/ross/ross0-p_xyz_c=03x50.png}
  \hfill
  \includegraphics[width=0.49\textwidth]{p/cha/ross/ross_f-p_f_c=03x50.png}
\end{center}
  \caption{Аттрактор и спектр системы Рёсслера (\ref{atu:eq:rossler}) в режиме регулярных колебаний ($c=3.5$)}
\label{atu:f:ross_attractor_0300}
\end{figure}

При увеличении значения параметра \(c\) происходит удвоение периода,
поведение системы становится все более сложным, и в определенном
диапазоне значений параметра система демонстрирует
хаотическую динамику(рис.~\ref{atu:f:ross_attractor_0588}).

\begin{figure}[ht!]
\begin{center}
  \includegraphics[width=0.49\textwidth]{p/cha/ross/ross0-p_xyz_c=05x88.png}
  \hfill
  \includegraphics[width=0.49\textwidth]{p/cha/ross/ross_f-p_f_c=05x88.png}
\end{center}
  \caption{Аттрактор и спектр системы Рёсслера (\ref{atu:eq:rossler}) в режиме хаотических колебаний ($c=5.88$)}
\label{atu:f:ross_attractor_0588}
\end{figure}

Прослеживается отличие спектра системы Рёсслера в хаотическом режиме
от аналогичных условий для систем Лоренца. Отличие заключается в том, что
существует ярко выраженный пик, соответствующий базовой частоте,
а области сплошного спектра характеризуются небольшой амплитудой.
При некоторых больших значениях параметра $c$
область сплошного спектра становится более заметной,
однако общая структура спектра остаётся такой же.


При дальнейшем увеличении значения параметра \(c\)
наблюдаются переходы от хаотического к сложно-периодическому
(рис.~\ref{atu:f:ross_attractor_2500})
и  обратно.

\begin{figure}[ht!]
\begin{center}
  \includegraphics[width=0.49\textwidth]{p/cha/ross/ross0-p_xyz_c=25x00.png}
  \hfill
  \includegraphics[width=0.49\textwidth]{p/cha/ross/ross_f-p_f_c=25x00.png}
\end{center}
  \caption{Аттрактор и спектр системы Рёсслера (\ref{atu:eq:rossler}) в режиме сложно-периодических колебаний ($c=25.0$)}
\label{atu:f:ross_attractor_2500}
\end{figure}

При этом наблюдается линейчатый спектр. Также при
таких периодах может скачкообразно изменятся
размер области, в которую вписан аттрактор.
В этом смысле система Рёсслера потенциально является более сложной
для идентификации, чем системы Лоренца и ``Sprott A''.

% Идентифицируемый параметр:
% $ c \in [2; 50] $, $c_0=5.88$.
%
% Остальные параметры:
% \( a \in (0, 0.35 ) \), $a_0=0.25$,
% \(b \in[0;4] \), $b_0=1$.


% }}}2

\subsection{Анализ и выбор критериев}  % {{{2

Критерий
$ z_{\max}$, $ \overline{z} $.

\begin{figure}[ht!]
\begin{center}
  \includegraphics[width=0.49\textwidth]{p/cha/ross/ross_q-p_q.png}
  \hfill
  \includegraphics[width=0.49\textwidth]{p/cha/ross/ross_q-p_q1.png}
\end{center}
  \caption{Рассматриваемые критерии для системы Рёсслера}
\label{atu:f:ross_q}
\end{figure}

\begin{figure}[ht!]
\begin{center}
  \includegraphics[width=0.49\textwidth]{p/cha/ross/ross_pwr-x_a_c.png}
  \hfill
  \includegraphics[width=0.49\textwidth]{p/cha/ross/ross_pwr-x_b_c.png}
\end{center}
  \caption{Зависимости $q_{x^2}(a,c)$ и  $q_{x^2}(b,c) $ для системы Рёсслера}
\label{atu:f:ross_q_x2_ac_bc}
\end{figure}

\begin{figure}[ht!]
\begin{center}
  \includegraphics[width=0.49\textwidth]{p/cha/ross/ross_zmax_a_c.png}
  \hfill
  \includegraphics[width=0.49\textwidth]{p/cha/ross/ross_zmax_b_c.png}
\end{center}
  \caption{Зависимости $q_{zmax}(a,c)$ и  $q_{zmax}(b,c) $ для системы Рёсслера}
\label{atu:f:ross_q_zmax_ac_bc}
\end{figure}



% }}}2

\subsection{Тестовая задача идентификации для системы Рёсслера}  % {{{2


% }}}2

\subsection{Влияние параметров системы идентификации на ошибку идентификации для системы Рёсслера}  % {{{2

% }}}2

\subsection{Зависимости значений критериев идентификации при изменении двух параметров системы Рёсслера}  % {{{2

% }}}2



\subsection{Выводы}  % {{{2

Выводы

% }}}2


% }}}1

% vim: fdm=marker foldlevelstart=3 fdc=4 ft=tex
 % ??

% 
\FloatBarrier

\section{Система Ван-дер-Поля} %  % {{{1 _VDP_
\label{atu:sect:vdp}

\LinkRef{
  vdp: ASAU-16, 17(alt), ITMM-2011
}

\subsection{Визначення системи та аналіз її динаміки} %{{{2 vdp_task

Модель нелінійної автоколивальній системи
Ван-Дер-Поля~(\ref{atu:eq:vdp}) широко використовується при дослідженні
динаміки коливальних систем, в яких відбувається поповнення
енергії системи з зовнішнього джерела~\cite{Ginoux2012VanDP,anisch_nonlin_eff,magni_theory_dyn_chaos,atu_asau16,atu_st75,chulichkcov_mm_ml_dyn}:
%
\begin{equation}
 \ddot{x} - \varepsilon (1-x^2)  \dot{x} + \Omega_0^2 x  = u(t) .
\label{atu:eq:vdp}
\end{equation}
\noindent
де
\(x(t)\) --- координата коливань,
\(\varepsilon \) --- параметр, що визначає отримання енергії системою,
\(\Omega_0 \) --- власна частота при \(\varepsilon= 0 \),
\( u (t) \) --- зовнішнє вплив, що збурює.

При \(\varepsilon> 0 \) і \(u (t) = 0 \) розглянута система реалізує режим
автоколивань з постійною частотою, яка залежить від \(\varepsilon \):
%
\begin{equation}
\Omega \approx \Omega_0 \sqrt{ 1 - \left( \frac{\varepsilon}{2 \Omega_0} \right)^2 }.
\label{atu:eq:vdp_Omega}
\end{equation}

При цьому амплітуда коливань залишається майже незмінною, а
сама форма коливань приймає все більш нелінійний характер.

Під впливом вхідного гармонійного сигналу
\( u(t) = U_{in} \sin ( \omega_{in} t ) \)
система~(\ref{atu:eq:vdp}) може проявляти регулярну, складно-періодичну
і хаотичну динаміку~\cite{atu_itcs2011,atu_ISDMCI2011,gang_chaos_on_phase_noise,baranov_chaos_vdp,kuznetsov_phenomen_vdp,math5040070}.

Ідентифікований параметр \(\varepsilon \) характеризує надходження
енергії у систему. При
$ x \approx 0 $ членом з
$ x ^ 2 $ можна знехтувати, і нелінійність вироджується в ``негативне тертя''.
При великих амплітудах коливань член з
$x^2$ починає переважати, там самим обмежуючи зростання
амплітуди. Існує також система, яка має назву
Ван-дер-Поля-Дуффінга~\cite{landa_nonlin_vivro_waves}, та об'єднує властивості
даної системи і системи Дуффінга.

При аналізі чисельного моделювання системи~(\ref{atu:eq:vdp})
слід обережно ставиться до висновків про тип динаміки, які
можна зробити, виходячи з фазового або розширеного фазового
портрета.
Наприклад, на рис.~\ref{atu:f:vdp_phase_f_reg}, за умов
$ \varepsilon = 1.50 $,
$ U_{in} = 0.3 $,
$ \omega_{in} = 0.27 $ на лівому графіку фазова траєкторія виходить
щільною, однак спектр свідчить про регулярну динаміку з
обмеженим набором частот.

\begin{figure}[ht!]
  \PicDouble{p/cha/vdp/vdp_0-p_ph2d_1x50_0x30_0x27.png}{p/cha/vdp/vdp_fft-p_f_1x50_0x30_0x27.png}
  \caption{Розширений фазовий портрет~(a) і спектр~(b) системи Ван-дер-Поля (\ref{atu:eq:vdp}) в режимі регулярних коливань}
\label{atu:f:vdp_phase_f_reg}
\end{figure}

Безпосередньо хаотичні коливання цієї системи часто
характеризуються менш вираженою щільністю аттрактора. Проте,
ділянки складного спектра (рис.~\ref{atu:f:vdp_phase_f_chaos}) свідчать на
користь хаосу. Дана ілюстрація відповідає умовам
$ \varepsilon = 2.65 $,
$ U_{in} = 1.2 $,
$ \omega_{in} = 0.27 $. При цьому малі зміни параметрів, наприклад зниження
величини
$ \varepsilon $ до
$ 2.5 $ призводить до реалізації режиму простих регулярних
коливань.

\begin{figure}[ht!]
  \PicDouble{p/cha/vdp/vdp_0-p_ph2d_2x65_1x20_0x27.png}{p/cha/vdp/vdp_fft-p_f_2x65_1x20_0x27.png}
  \caption{Розширений фазовий портрет~(a) і спектр~(b) системи Ван-дер-Поля (\ref{atu:eq:vdp}) в режимі хаотичних коливань}
\label{atu:f:vdp_phase_f_chaos}
\end{figure}

Аналіз спектру системи Ван-дер-Поля, як і багатьох інших систем
динамічного хаосу, слід проводити, враховуючи спектральну
роздільну здатність. Наприклад, на рис.~\ref{atu:f:vdp_phase_f_complex} представлені
результати моделювання системи при
$ \varepsilon = 4.8 $,
$ U_{in} = 0.7 $,
$ \omega_{in} = 0.7 $. В спектрі системи спостерігаються ділянки з дуже
близько розташованими піками. При недостатній роздільної
здатності цю ділянку буде прийнято за зону суцільного
спектру. Це може привести до помилкового висновку про хаотичність системи,
особливо з врахуванням структури аттрактора.

\begin{figure}[ht!]
  \PicDouble{p/cha/vdp/vdp_0-p_ph2d_4x80_0x70_0x70.png}{p/cha/vdp/vdp_fft-p_f_4x80_0x70_0x70.png}
  \caption{Розширений фазовий портрет~(a) і спектр~(b) системи Ван-дер-Поля (\ref{atu:eq:vdp}) в режимі складних регулярних коливань}
\label{atu:f:vdp_phase_f_complex}
\end{figure}

При аналізі фізичних систем немає можливості довільним чином
задавати час вимірювання для отримання спектра з необхідною
роздільною здатністю. Більш того, з урахуванням обмеженої точності вимірів
немає можливості точно визначити показники Ляпунова. Таким
чином, цілком можливе існування систем, які не відрізняються
від хаотичних при проведенні вимірювань, але не є хаотичними в
строгому розумінні. Проте, з точки зору задачі ідентифікації,
цей випадок практично не відрізняється від реального хаосу,
і вимагає застосування відповідних методів і критеріїв.

% }}}2


\subsection{Аналіз і вибір критеріїв}%{{{2

На відміну від систем Лоренца, Ресслера та їм подібних, у системи
Ван-дер-Поля є тільки одна величина, що спостерігається: $x(t)$. Це
відразу сильно обмежує коло критеріїв, які потребують аналізу.

На перший погляд, якщо доступний сигнал
$x(t)$, то можна обчислити сигнали
$\dot{x}(t) $ і
$\ddot{x}(t)$. Після цього, підставивши отримані залежності
в (\ref{atu:eq:vdp}), можна отримати значення ідентифікованого
параметра. Насправді, це можливо тільки в разі практично
повної відсутності перешкод. Навіть невеликий (0.001\%) рівень
помилок вимірювання при обчисленні похідних, особливо другої,
призводить до абсолютно довільним результатам. Таким чином,
для цієї системи без застосування інтегральних критеріїв
немає можливості створити працездатну систему ідентифікації.

Критерії
$q_{x^2}$,
$q_{rx}$ і
$q_{|x|}$ відрізняються для цього завдання не
принципово. Незважаючи на те, що ці величини в даному конкретному
разі не відображають будь-якої закон збереження, їх застосування
може бути виправданим~\cite{atu_asau17}. При зростанні параметра
$\varepsilon$ система все більше виявляє нелінійні властивості. Це
виражається в тому, що при однаковій амплітуді сигналу
залежність
$x(t)$ більшу частину часу проводить у районі амплітудних
значень, що призводить до зростання цих критеріїв. Для
визначеності в подальшому будемо використовувати
$q_{x^2}$.

Також має сенс розглянути досить очевидний для коливальних
систем критерій
$q_T$, що полягає у вимірюванні ``періоду''~\cite{atu_asau16}. Очевидно, що
поняття періоду для хаотичних систем не можна застосовувати. Однак,
можна прийняти за поточне значення періоду інтервал часу між
двома послідовними спрацьовуваннями коректно налаштованого
тригера Шмідта, на вхід якого подається сигнал
$x(t)$. Цей підхід використовується при побудові вхідних
ланцюгів частотометрів. При цьому автоматично відбувається
усереднення на інтервалах порядку цього ``періоду''. Однак,
для систем ідентифікації таке усереднення швидше за все
буде недостатнім, особливо в хаотичних і складно-періодичних
режимах. Отже, має сенс використовувати додаткове усереднення,
наприклад~(\ref{atu:eq:qlin}).

Розглянемо характерні залежності розглянутих критеріїв,
отримані в результаті моделювання динаміки системи
(\ref{atu:eq:vdp}). На рис.~\ref{atu:f:vdp_q1} представлений вид цих залежностей
для різних умов.


\begin{figure}[ht!]
  \PicDouble{p/cha/vdp/vdp_q-p_q_0x30_0x27.png}{p/cha/vdp/vdp_q-p_q_1x20_0x6283185.png}
  \caption{Залежності $ q_{x^2} (\varepsilon) $ і $ q_T (\varepsilon) $ системи Ван-дер-Поля за умов $ U_{in}=0.3, \omega_{in}=0.27$~(a) і $U_{in}=1.2, \omega_{in}=0.628$~(b)}
\label{atu:f:vdp_q1}
\end{figure}

Лівий графік відповідає такому набору параметрів, при якому
спостерігається періодичний і складно-періодичний рух. При цьому
обидва критерії підходять для синтезу системи ідентифікації,
а критерій $ q_T $ виявляє меншу нелінійність. Правий графік відображає
випадок, коли зміна параметра
$ \varepsilon $ призводить до істотної зміни поведінки. При
$ \varepsilon <7 $ відбувається ``захоплення частоти'', тобто
сигнал
$u(t)$ ``нав'язує'' свою частоту системі, що призводить до простої
динаміці і відсутності залежності
$q_T$ від
$\varepsilon$. Ідентифікація на даній ділянці неможлива, зважаючи на
відсутність впливу параметра на ``період''. Права частина цього
графіка відображає постійні переходи від складних коливань
до хаосу і назад. На залежності
$ q_T (\varepsilon) $ з'являються злами, що потенційно має привести
до зниження точності ідентифікації на тлі загальної
працездатності. Графік залежності
$q_{x^2}(\varepsilon) $ має екстремальний характер, що дозволяє
використовувати цей критерій тільки на тих діапазонах
$\varepsilon$, на яких спостерігається монотонність.

Розглянемо залежності
$q_T (\varepsilon, U_{in}) $ і
$q_{x^2}(\varepsilon, U_{in})$ для різних значень
$\omega_{in} $.
На рис.~\ref{atu:f:vdp_q2_027} представлені ці залежності при
$ \omega_{in} = 0.27 $.


\begin{figure}[ht!]
  \PicDouble{p/cha/vdp/vdp_q_2d-p_qT_ome_0x27.png}{p/cha/vdp/vdp_q_2d-p_qx2_ome_0x27.png}
  \caption{Залежності $q_T(\varepsilon,U_{in})$~(a) та $q_{x^2}(\varepsilon,U_{in})$~(b)  при $\omega_{in}=0.27$}
\label{atu:f:vdp_q2_027}
\end{figure}

При високих значеннях
$U_{in} $ відбувається захоплення частоти, графік залежності
$q_T (\varepsilon, U_{in})$ в цій області є плоску поверхню, що свідчить
про непридатність критерію в цих умовах. При малих значення
$U_{in}$, навпаки, спостерігається досить близька до лінійної
залежність. У проміжку між двома цими областями можливості
ідентифікації з використанням даного критерію існують, але
обмежені. Критерій
$q_{x^2}$ при цих же умовах має дещо інші обмеження. Ідентифікація
утруднена в області ``яру'', який відповідає перехідному режиму
на попередньому графіку. Як в області низьких, так і в області
високих значень
$U_{in}$ ідентифікація можлива.

На рис.~\ref{atu:f:vdp_q2_120} представлені поверхні критеріїв при
$\omega_{in} = 1.2$.

\begin{figure}[ht!]
  \PicDouble{p/cha/vdp/vdp_q_2d-p_qT_ome_1x20.png}{p/cha/vdp/vdp_q_2d-p_qx2_ome_1x20.png}
  \caption{Залежності $q_T(\varepsilon,U_{in})$~(a) та $q_{x^2}(\varepsilon,U_{in})$~(b)  при $\omega_{in}=1.2$}
\label{atu:f:vdp_q2_120}
\end{figure}

На залежності
$q_T(\varepsilon, U_{in}) $ в цьому випадку також видно плоску ділянку,
що відповідає режиму захоплення частоти. Але більша частина
цієї залежності свідчить про те, що ідентифікація можлива,
і при цьому швидше за все буде спостерігатися збільшення
похибки ідентифікації через нерівномірність графіка. Навпаки,
залежність
$q_T (\varepsilon, U_{in})$ проявляє мультимодальний характер, що дозволяє
проводити ідентифікацію тільки у вузьких діапазонах.

Таким чином, встановлено, що жоден з розглянутих критеріїв не
забезпечує можливість ідентифікації при будь-яких значеннях
параметрів. Однак, найбільший діапазон працездатності
продемонстрував критерій
$q_{T}$. Він і буде використаний в подальших дослідженнях. При
цьому при постановці завдання будемо уникати режимів, на яких
відбувається захоплення частоти.

% }}}2

\subsection{Тестова задача ідентифікації для системи Ван-дер-Поля} % {{{2

При синтезу системи ідентифікації використовувалася група
методів ``ql3rlWvnAAW''. З урахуванням вибору критерію
$q_{T}$ при постановці задачі ідентифікації параметри
вибиралися з урахуванням діапазону працездатності цього
критерію. Перша група значень параметрів:
$U_{in} = 0.3$,
$\omega_{in} = 0.27$,
$\varepsilon \in [1, 4]$.

Розглянемо процес ідентифікації в квазістаціонарному випадку,
при повільної зміні параметра~(\ref{atu:eq:po_t_ramp}),
$p_0 = 1$,
$U_p = 2$. Динаміка агентів і різних способів завдання
$p_\mathrm{id} $ представлена на рис.~\ref{atu:f:vdp_id1_ramp}.

\begin{figure}[ht!]
  \PicDouble{p/cha/vdp/vdp_id-p_t_pi_ql3rlWvnAAW_ramp.png}{p/cha/vdp/vdp_id-p_t_p_ql3rlWvnAAW_ramp.png}
  \caption{Динаміка агентів~(a) і ідентифікованого значення~(b) для системи Ван-дер-Поля за умови (\ref{atu:eq:po_t_ramp})}
\label{atu:f:vdp_id1_ramp}
\end{figure}

За винятком початкової ділянки, на якої відбувається
стабілізація стану самої системи ідентифікації, агенти
демонструють коректну поведінку, супроводжуючи значення параметра
$\varepsilon$, яке повільно змінюється.
Як наслідок, всі методи визначення
$p_\mathrm{id}$ дозволяють досить точно провести
ідентифікацію. Невеликі коливання обумовлені тим, що насправді
система не в повній мірі може бути названа квазістаціонарною,
через великий часу усереднення критерію. Ділянок, на яких
порушувався б процес ідентифікації, або ж істотно зросла
похибка, не спостерігається.

Розглянемо динамку процесу ідентифікації (рис.~\ref{atu:f:vdp_id1_sign})
при умовах (\ref{atu:eq:po_t_sign}),
$p_0 = 2$,
$U_p = 0.8$,
$\omega_{in} = 0.01047$.

\begin{figure}[ht!]
  \PicDouble{p/cha/vdp/vdp_id-p_t_pi_ql3rlWvnAAW_sign.png}{p/cha/vdp/vdp_id-p_t_p_ql3rlWvnAAW_sign.png}
  \caption{Динаміка агентів~(a) і ідентифікованого значення~(b) для системи Ван-дер-Поля за умови (\ref{atu:eq:po_t_sign})}
\label{atu:f:vdp_id1_sign}
\end{figure}

Як вже спостерігалося, усі методи визначення
$p_\mathrm{id}$ на рівні координатора пошуку дають схожі
результати. Слід зазначити помітні коливання на першій половині
періоду зміни параметра, в порівнянні з третім. Надмірні
коливанням на початковому етапі можна зменшити шляхом
попередньої настройки елементів усереднення критерію. Однак,
для такого налаштування необхідна додаткова інформація, яка
невідома a priori.

На рис.~\ref{atu:f:vdp_id1_sin} представлені аналогічні результати, але
в разі плавної зміни параметра~(\ref{atu:eq:po_t_sin}).

\begin{figure}[ht!]
  \PicDouble{p/cha/vdp/vdp_id-p_t_pi_ql3rlWvnAAW_sin.png}{p/cha/vdp/vdp_id-p_t_p_ql3rlWvnAAW_sin.png}
  \caption{Динаміка агентів~(a) і ідентифікованого значення~(b) для системи Ван-дер-Поля за умови (\ref{atu:eq:po_t_sin})}
\label{atu:f:vdp_id1_sin}
\end{figure}

Помилка ідентифікації в цьому випадку істотно менше, також
зберігаються коливання на початковому етапі.

% }}}2


\subsection{Вплив параметрів системи ідентифікації на похибку ідентифікації для системи Ван-дер-Поля} % {{{2

При ідентифікації системи Ван-дер-Поля з використанням критерію
$q_T$ вплив параметра
$a_q$ представляє особливий інтерес, зважаючи на наявність двох
рівнів усереднення. Залежності усереднених помилок ідентифікації
від цього параметра представлені на рис.~\ref{atu:f:vdp_e_a_q}.

\begin{figure}[ht!]
  \PicDouble{p/cha/vdp/vdp_id-p_a_q_sign.png}{p/cha/vdp/vdp_id-p_a_q_sin.png}
  \caption{Залежності $\overline{e}(a_q)$ для системи Ван-дер-Поля за умов (\ref{atu:eq:po_t_sign})~(a) і (\ref{atu:eq:po_t_sin})~(b)}
\label{atu:f:vdp_e_a_q}
\end{figure}

Виходячи з характерних значень ``періоду'' можна було б очікувати
зміну характеру цих залежностей при
$a_q \approx 0.1 $. Однак, ніяких істотних змін на графіках не
виявлено. Сам вигляд цих залежностей практично нічим, крім
масштабу, не відрізняється від таких для вже розглянутих
систем. Таким чином отримала підтвердження теза про необхідність
додаткового усереднення на цьому етапі.

Розглянемо вплив ще одного параметра системи ідентифікації ---
$q_\gamma$. Відповідні залежності наведені на рис.~\ref{atu:f:vdp_e_q_gamma}.

\begin{figure}[ht!]
  \PicDouble{p/cha/vdp/vdp_id-p_q_gamma_sign.png}{p/cha/vdp/vdp_id-p_q_gamma_sin.png}
  \caption{Залежності $\overline{e}(q_\gamma)$ для системи Ван-дер-Поля за умов (\ref{atu:eq:po_t_sign})~(a) і (\ref{atu:eq:po_t_sin})~(b)}
\label{atu:f:vdp_e_q_gamma}
\end{figure}

Використовувана група методів ``ql3rlWvnAAW'' характеризується
працездатністю для широкого діапазону величини
$q_\gamma$, за винятком області надмірної чутливості, у
якої пошук все ж можливий, але його діапазон обмежується
малим околом агентів. Представлені графіки повністю це
підтверджують. Застосування різних методів визначення
$p_\mathrm{id} $ по різному себе проявляє в області заниженої
чутливості, але ця різниця мала.


Залежності
$\overline{e} (v_f) $ отримані при ідентифікації системи Ван-дер-Поля
(рис.~\ref{atu:f:vdp_e_v_f}), демонструють істотну різницю при різних
способах завдання нестаціонарності параметра.

\begin{figure}[ht!]
  \PicDouble{p/cha/vdp/vdp_id-p_v_f_sign.png}{p/cha/vdp/vdp_id-p_v_f_sin.png}
  \caption{Залежності $\overline{e}(v_f)$ для системи Ван-дер-Поля за умов (\ref{atu:eq:po_t_sign})~(a) і (\ref{atu:eq:po_t_sin})~(b)}
\label{atu:f:vdp_e_v_f}
\end{figure}

Якщо в динаміці ідентифікованого параметра немає різких змін
(правий графік), то залежності мають класичний вигляд, тобто існує таке
значення $v_f$, при якому спостерігається мінімум похибки. У тому
ж випадку, коли залежність являє собою залежність з
розривами (\ref{atu:eq:po_t_sign}), то вплив стає слабким
і неоднозначним. Це пов'язано з тим, що перебудова положень
агентів хоч і відбувається, але не призводить до зменшення
похибки з огляду на те, що час реакції критерію стає одного
порядку з часом реакції на зміну параметра.


Вплив коефіцієнта
$k_e$ (рис.~\ref{atu:f:vdp_e_k_e}) для даної системи також істотно залежить
від динаміки параметра, і з тієї ж причини. Якщо динаміка
параметра задана як (\ref{atu:eq:po_t_sin}), то існує екстремум, що
відповідає оптимальному розподілу агентів. При стрибкоподібних
змінах параметра екстремум практично непомітний на загальному рівні.

\begin{figure}[ht!]
  \PicDouble{p/cha/vdp/vdp_id-p_k_e_sign.png}{p/cha/vdp/vdp_id-p_k_e_sin.png}
  \caption{Залежності $\overline{e}(k_e)$ для системи Ван-дер-Поля за умов (\ref{atu:eq:po_t_sign})~(a) і (\ref{atu:eq:po_t_sin})~(b)}
\label{atu:f:vdp_e_k_e}
\end{figure}

Вплив параметра
$k_{nl} $ (рис.~\ref{atu:f:vdp_e_k_nl}) відображає властивості множини агентів
як ансамблю. На відміну від двох попередніх залежностей, мінімум
похибки ідентифікації спостерігається для двох способів
визначення динаміки параметрів, тобто коректна взаємодія
між агентами необхідна для отримання результату навіть в тих
випадках, коли власне переміщення агентів не виправдано.

\begin{figure}[ht!]
  \PicDouble{p/cha/vdp/vdp_id-p_k_nl_sign.png}{p/cha/vdp/vdp_id-p_k_nl_sin.png}
  \caption{Залежності $\overline{e}(k_{nl})$ для системи Ван-дер-Поля за умов (\ref{atu:eq:po_t_sign})~(a) і (\ref{atu:eq:po_t_sin})~(b)}
\label{atu:f:vdp_e_k_nl}
\end{figure}

Як і в попередньому випадку, штучні обмеження на переміщення
агентів практично нівелює вплив коефіцієнта
$k_{cl}$ (рис.~\ref{atu:f:vdp_e_k_cl}). У цих умовах силу
$f_c$ має сенс взагалі не визначати.

\begin{figure}[ht!]
  \PicDouble{p/cha/vdp/vdp_id-p_k_cl_sign.png}{p/cha/vdp/vdp_id-p_k_cl_sin.png}
  \caption{Залежності $\overline{e}(k_{cl})$ для системи Ван-дер-Поля за умов (\ref{atu:eq:po_t_sign})~(a) і (\ref{atu:eq:po_t_sin})~(b)}
\label{atu:f:vdp_e_k_cl}
\end{figure}

На рис.~\ref{atu:f:vdp_e_varepsilon_2} представлена залежність похибки
ідентифікації від величини самого параметра
$\varepsilon$ в умовах, коли постійно чергуються хаотичні і
складно-періодичні режими. У цих умовах похибка ідентифікації
зростає, але сама система ідентифікації залишається
працездатною.


\begin{figure}[ht!]
\begin{center}
  \includegraphics[width=0.60\textwidth]{p/cha/vdp/vdp_id2-p_p_e_ql3rlWvnAAW_scan.png}
\end{center}
  \caption{Залежності $\overline{e}(\varepsilon)$ для системи Ван-дер-Поля за умов чергування режимів коливань}
\label{atu:f:vdp_e_varepsilon_2}
\end{figure}

% }}}2

\subsection{Висновки}%{{{2

Результати моделювання процесів ідентифікації параметра ``$\varepsilon$''
системи Ван-дер-Поля дозволяють зробити наступні
висновки:

\begin{itemize}

  \item
    Найбільший діапазон зразковості для цієї системи демонструє
    критерій $ q_T $ з наступним усередненням.

  \item
    Не виявлено критерію, який би був працездатний у всій області
    визначення параметрів, в зв'язку з різноманітністю динаміки,
    яку проявляє система.

  \item
    Група методів ql3rlWvnAAW показала свою хорошу працездатність,
    навіть в умовах чергування режимів коливань.

  \item
    Параметри самої системи ідентифікації можуть змінюватися
    в досить широких межах без істотного збільшення похибки
    ідентифікації.

\end{itemize}

% }}}2



% }}}1

% vim: fdm=marker foldlevel=1 foldignore="%#" fdc=4 ft=tex
 % ??

\FloatBarrier
\subsubsection{Генератор Колпитца} % _COLP_

\LinkRef{
  colp: ASAU-21, APIR-2013
}

В  практике   создания   радиоэлектронных   устройств   часто   используются
генераторы сигналов, способные генерировать разнообразные виды  сигналов, в том числе
сложно-периодические и хаотические.
В частности, генератор  Колпитца~\cite{kennedy_chaos_colpitts,atu_asau21}, в
зависимости от условий,   может
генерировать  колебания,  как  близкие  к  гармоническим,  так  и  проявлять
хаотическую  динамику  в  широком  спектральном   диапазоне.   Идентификация
параметров рассматриваемого генератора  необходима,  с  одной  стороны,  для
обеспечения  требуемого  режима  работы.  С  другой  стороны,  информация  о
параметрах системы необходима при проведении  контроля  работоспособности  в
процессе эксплуатации~\cite{atu_apir2013}.

На рис.~\ref{atu:f:colp_schem} приставлена одна из электрических схем,
реализующих генератор Колпитца на биполярном транзисторе.
Из множества схем данная была выбрана из-за наличия
одного источника напряжения, и простоты схемотехнической реализации.


\begin{figure}[htb!]
\begin{center}
% vi:syntax=tex

\begin{circuitikz}[line width=0.7]
  \ctikzset{bipoles/thickness=2}
  \def\Top{8.0}
  \def\Rig{7.0}
  \def\LinC{5.5}
  \def\LinQ{3.5}
  % transistor
  \draw (\LinQ,3.0) node[npn](npn) {}
        (npn.center) ++(-0.2,0) circle[radius=0.5]
        (npn.C) node[left=3mm, above=-3.6mm]{$Q_1$}
        (npn.B) node[left=2mm, above=0.2mm]{$V_b$};
  \draw (npn.C) ++(0.4,-0.4) -- ++(0.0,-0.8) [->] node[right] {$I_{ce}$};
  % border
  \draw (0.0,0.0) -- (\Rig,0.0)
   to[battery,l=$V_{cc}$]  (\Rig,\Top)
   to[short]         (0,\Top)
   to[R,mirror,l=$R_1$]     (0.0,3.0)
   to[R,mirror,l=$R_2$,*-]  (0.0,0.0)
   -- (0.0,0.0);
  % base part
  \draw (1.5,0.0) to[C,l=$C_0$,*-*] (1.5,3.0);
  \draw (0.0,3.0) -- (npn.base);
  % emitter part
  \draw (\LinQ,0.0) to[R,l=$R_e$,*-*] (\LinQ,2.0)
   to[short] (npn.E);
   % collector part
  \draw (npn.collector)
   to[L,mirror,l=$L$,i<=$I_L$] (\LinQ,6.0)
   to[vR,l=$R_c$] (\LinQ,\Top);
  %
  \draw (\LinC,0.0) to[C,l=$C_2$,v=$V_2$,*-*] (\LinC,2.0);
  \draw (\LinC,2.0) -- (\LinQ,2.0);
  \draw (\LinC,2.0) to[C,l=$C_1$,v=$V_1$]     (\LinC,4.0);
  \draw (\LinC,4.0) -- (\LinQ,4.0);
  \filldraw (\LinQ,4.0) circle[radius=0.05];
\end{circuitikz}




\end{center}
\caption{Электрическая схема генератора Колпитца на биполярном транзисторе}
\label{atu:f:colp_schem}
\end{figure}

При создании модели генератора Колпитца систему уравнений можно
заранее упростить, если заметить, что
делитель на резисторах
$\mathrm{R}_1$, $\mathrm{R}_2$,
вместе с конденсатором
$\mathrm{C}_0$ обеспечивают
постоянство потенциала базы
$V_b = V_{CC} \frac{R_1}{R_1+R_2}$,
поэтому из дальнейшего рассмотрения данные элементы следует
исключить.

Рассмотрев процессы заряда конденсаторов и изменение тока через
катушку идуктивности, получим следующую систему уравнений:

\begin{equation}
\label{atu:eq:colp_phys}
\begin{dcases}
  C_1 \od{V_{1}}{t}  = I_L - I_{CE} , \\
  L   \od{I_L}{t}    = V_{CC} - V_{1} - V_{2} - I_L R_C , \\
  C_2 \od{V_{2}}{t}  = I_L - \frac{V_{2}}{R_e}.
\end{dcases}
\end{equation}


\noindent
где
$V_{CC} $ -- напряжение питания,
$V_1,$ $V_2$ -- разность потенциалов между выводами конденсаторов
$\mathrm{C}_1$ и $\mathrm{C}_2$ соответственно,
$I_L$, $I_{CE}$ -- токи катушки индуктивности и транзистора (коллектор-эмиттер).

Перейдём к безразмерным величинам.
При переходе к безразмерному виду следует определить,
какие физические параметры определяют безразмерные величины.
Это потребуется для синтеза критерия идентификации.
Для упрощения рассмотрения, не снижая общности,
будем считать $C_1 = C_2 = C$.

Прежде всего, воспользуемся тем, что система содержит только один
активный нелинейный компонент -- транзистор.
Следовательно, именно этот элемент определяет
масштаб по напряжению. В простейшей модели транзистора
такой масштабной величиной может служить
$V_{je}$ -- падение напряжения на переходе база-эмиттер
в активном режиме. Следовательно, все разности потенциалов в схеме можно нормировать
на эту величину.

Динамические свойства (в т.ч. условия начала генерации и перехода в хаотический режим) определяются
соотношением активных и реактивных свойств системы. При этом величина
$ \rho = \sqrt{L/C} $ имеет размерность сопротивления
и определяет реактивное сопротивление. Эту величину можно использовать
для обезразмеривания активных сопротивлений.

Для приведения токов к безразмерному виду, с учётом уже выбранных величин,
следует использовать величину $ V_{je} / \rho$.


Исходя из всего вышеперечисленного, обозначим:

\[
  x = \frac{V_{1}}{V_{je}} ; \quad
  y = \frac{\rho I_L}{V_{je}} ; \quad
  z = \frac{V_{2}}{V_{je}}, \quad
  i_{ce} = \frac{\rho I_{ce}}{V_{je}}, \quad
  c = \frac{V_{CC}}{V_{je}}, \quad
  e = \frac{V_{b}}{V_{je}}.
\]
\[
  b = \frac{R_c}{\rho}; \quad
  d = \frac{\rho}{R_e}. % sic!
\]

Система уравнений принимает вид:


\begin{equation}
\label{atu:eq:colp_phys2}
\begin{dcases}
  \od{x}{t}  = \dfrac{1}{\rho C}  y - \dfrac{1}{\rho C} i_{ce} , \\
  \od{y}{t}  = \dfrac{\rho}{L} c    - \dfrac{\rho}{L} r_c y - \dfrac{\rho}{L} x- \dfrac{\rho}{L} z, \\
  \od{z}{t}  = \dfrac{1}{\rho C}  y - \dfrac{1}{\rho C} \dfrac{1}{r_e} z.
\end{dcases}
\end{equation}

Общий множитель $ \frac{1}{\rho C} = \frac{\rho}{L} = \sqrt{\frac{1}{LC}} $ в правых частях уравнений
естественным образом задаёт масштаб по времени.
Это подчёркивает, что частотные характеристики рассматриваемого генератора,
в отличие, например, от релаксационного,
определяются ёмкостью и индуктивностью,
поэтому масштаб времени задаём так:
$ T_s = \sqrt{L C} $.
Тогда безразмерное время $t_s$
и соответствующие производные
будут определены таким образом:

\[
  t_s = \frac{t}{T_s}; \quad
  \mathrm{d}\, t = T_s \mathrm{d}\, t_s; \quad
  \od{}{t}  = \frac{1}{T_s} \od{}{t_s}; \quad
  \od{x}{t_s} \equiv \dot{x} = T_s \od{x}{t} .
\]

Поведение величины $I_{ce}$ достаточно хорошо описывает модель
Эберса-Молла~\cite{horowitz}:

\begin{equation}
  I_c = I_s \left( \exp\frac{V_{be}}U_t{} - 1 \right),
  \label{atu:eq:ebers-moll}
\end{equation}

\noindent
где
$I_s$ -- ток насыщения (паспортная или определяемая экспериментально величина),
$U_t=kT/q$,
$q = \SI{1.6e-19}{\coulomb}$ -- заряд электрона,
$k = \SI{1.38e-24}{\joule/\kelvin}$ -- постоянная Больцмана.
Необходимо учесть, что в режиме отсечки ($V_b < V_e$) ток коллектора пренебрежимо мал,
а в режиме насыщения определяется другими элементами схемы.
Существуют и более сложные модели, например,
программы для моделирования электронных схем часто используют так называемую SPICE модель.

К сожалению, при моделировании генератора Колпитца в литературе,
посвящённой хаотической динамике, используют
простейшую модель транзистора, считая, что переход
база-эмиттер открывается при $V_{BE} = V_{je}$, $ I_c \gg I_b$,
а ток коллектора

\begin{equation}
I_c =
  \begin{cases}
    \alpha ( V_b - V_e - V_{je} ), & V_b - V_e > V_{je} \\
    0                              & \text{otherwise}.
  \end{cases}
  \label{atu:eq:bjt_libear_model}
\end{equation}



С учётом всего вышеизложенного получаем следующую систему уравнений:

\begin{equation}
\label{atu:eq:colp}
\begin{cases}
  \dot{x} = y - a F(z), \\
  \dot{y} = c - x - by - z, \\
  \dot{z} = y - d z.
\end{cases}
\end{equation}

При этом параметр $b$ характеризует соотношение
активного и реактивного сопротивления,
и, следовательно, режима работы генератора.
Величиной этого параметра проще всего управлять,
изменяя $R_c$.
Поставленной задачей будем считать идентификацию
данного параметра.

При физическом моделировании в рамках данной работы использовались следующие
элементы с соответствующими параметрами:

\[
  V_{cc} = \SI{12.06}{\volt},          \;
  R_1 = R_2 = \SI{2.2}{\kilo\ohm},     \;
  R_e = \SI{430}{\ohm},
\]

\[
  C_1 = C_2 = \SI{1.03}{\micro\farad}, \;
  L = \SI{6.22}{\milli\henry},         \;
  T = \SI{305}{\kelvin},
\]

\[
  \text{Q: 2N2222A}, \quad
  h_{fe}=285, \;
  V_f = \SI{0.677}{\volt}, \;
  I_s = \SI{9.61e-14}{\ampere}, \;
  \alpha \approx 1.
\]

Тогда безразмерные коэффициенты:
\[
 a = 77,     \quad
 c = 18.08,  \quad
 d = 0.19,   \quad
 e = 9.07.
\]

\[
F(z) =
\begin{cases}{l}
  e-1-z, & z \le e-1  \\
  0,     & z  >  e-1
\end{cases}.
\]


Диапазон изменения идентифицируемого параметра
$b \in [ 0.02; 4.2 ]$
определяется, с одной стороны, собственным сопротивлением катушки индуктивности,
с другой -- срывом генерации.


На рис.~\ref{atu:f:colp_real_xzz}--\ref{atu:f:colp_model_f} предствлены как результаты реального эксперимента,
так и данные, полученные в результате численного моделирования динамики системы (\ref{atu:eq:colp}).
Представлены проекции аттракторов но плоскость $(x+z,z)$ (естественный вид для оциллографа),
трёхмерный вид аттракторов, и спектры.
На каждом рисунке представлено три режима: обычный, момент первого удвоения периода и хаотический режим.


\begin{figure}[htb!]
 \centerline{
   \includegraphics[width=0.32\textwidth]{p/cha/colp/colp_m1_vv.png}
   \includegraphics[width=0.32\textwidth]{p/cha/colp/colp_m2_vv.png}
   \includegraphics[width=0.32\textwidth]{p/cha/colp/colp_m3_vv_ac.png}
 }
  \caption{Проекции аттракторов реальной системы Колпитца на плоскость $(x+z,z)$
  для трёх режимов}
  \label{atu:f:colp_real_xzz}
\end{figure}

\begin{figure}[htb!]
 \centerline{
   \includegraphics[width=0.32\textwidth]{p/cha/colp/colp_0-p_z_xpz_b=1x70.png}
   \includegraphics[width=0.32\textwidth]{p/cha/colp/colp_0-p_z_xpz_b=1x37.png}
   \includegraphics[width=0.32\textwidth]{p/cha/colp/colp_0-p_z_xpz_b=0x99.png}
 }
  \caption{Проекции аттракторов модели (\ref{atu:eq:colp}) системы Колпитца на плоскость $(x+z,z)$
  для трёх режимов}
  \label{atu:f:colp_model_xzz}
\end{figure}


\begin{figure}[htb!]
 \centerline{
   \includegraphics[width=0.32\textwidth]{p/cha/colp/colp_0-p_xyz_b=1x70.png}
   \includegraphics[width=0.32\textwidth]{p/cha/colp/colp_0-p_xyz_b=1x37.png}
   \includegraphics[width=0.32\textwidth]{p/cha/colp/colp_0-p_xyz_b=0x99.png}
 }
  \caption{Аттрактороы модели (\ref{atu:eq:colp}) системы Колпитца для трёх режимов}
  \label{atu:f:colp_model_xyz}
\end{figure}

\begin{figure}[htb!]
 \centerline{
   \includegraphics[width=0.32\textwidth]{p/cha/colp/colp_m1_f.png}
   \includegraphics[width=0.32\textwidth]{p/cha/colp/colp_m2_f.png}
   \includegraphics[width=0.32\textwidth]{p/cha/colp/colp_m3_f.png}
 }
  \caption{Спектры реальной системы Колпитца  для трёх режимов}
  \label{atu:f:colp_real_f}
\end{figure}

\begin{figure}[htb!]
 \centerline{
   \includegraphics[width=0.32\textwidth]{p/cha/colp/colp_f-p_f_b=1x70.png}
   \includegraphics[width=0.32\textwidth]{p/cha/colp/colp_f-p_f_b=1x37.png}
   \includegraphics[width=0.32\textwidth]{p/cha/colp/colp_f-p_f_b=0x99.png}
 }
  \caption{Спектры модели (\ref{atu:eq:colp}) системы Колпитца для трёх режимов}
  \label{atu:f:colp_model_f}
\end{figure}

Сравнение результатов физического и численного моделирования позволяет сделать вывод
о качественном подобии поведения реальной системы и модели.
Тем не менее, величины параметра $b$, при которых получены
рассматриваемые режимы, совпадают недостаточно точно.
Для реальной системы значения параметра: $b = 1.06, \; 0.94, \; 0.90 $,
а для модели: $b = 1.70, \; 1.34, \; 0.99 $.
Скорее всего, это связано с грубостью модели (\ref{atu:eq:bjt_libear_model}) транзистора,
и требует дальнейшего исследования.
К сожалению, ограниченный набор данных, получаемых с осциллографа (8192 отсчёта) не позволяют
получить достаточно подробный спектр реальной системы.

Для определения критерия рассмотрим зависимости
$q_{*}(\mu) $, полученные путём моделирования
для системы Колпитца (рис.~\ref{atu:f:colp_q}).
При этом следует учесть, что наиболее просто измерямыми величинами являются $x$ и $z$,
Соответствующие напряжениям $V_1$ и $V_2$.
Первый набор зависимостей даёт два основных кандидата -- $q_{x^2}$ и $q_{z^2}$.
При этом первый из них показывает более равномерную зависимость.
С другой стороны, при большинство из рассморенных зависимостей имею явно
выраженный гиперболический характер, особенно при малых значениях $b$.
Следовательно, в список кандидатов следует добавить $q_{x^{-2}} $ и $q_{z^{-2}}$.

\begin{figure}[htb!]
\centerline{
  \includegraphics[width=0.49\textwidth]{p/cha/colp/colp_p-p_b_e.png}
  \includegraphics[width=0.49\textwidth]{p/cha/colp/colp_p-p_b_1ex2.png}
}
  \caption{Зависимости $q_{*}(b) $ для системы Колпитца (\ref{atu:eq:colp})}
\label{atu:f:colp_q}
\end{figure}

Из графиков очевидно, что
обратные зависимости не дают заметного выигрыша, поэтому выберем
величину $ q_{x^2}(b) $ в качестве критерия.

В качестве системы идентификации использовалась система с 5 поисковыми агентами и
двумя неподвижными моделями. Аналогично предыдущим системам,
для исследования динамических свойств системы идентификации
параметр $b_o$ так:

\begin{equation}
 b_o(t) = p_0 + U_p \sign \sin( \omega_p t ),
  \label{atu:eq:colp_b_sign}
\end{equation}

\begin{equation}
 b_o(t) = p_0 + U_p \sin( \omega_p t ).
  \label{atu:eq:colp_b_sin}
\end{equation}

Динамика процессов идентификации для системы Колпитца представлена на рис.~\ref{atu:f:colp_id}.

\begin{figure}[htb!]
\centerline{
  \includegraphics[width=0.49\textwidth]{p/cha/colp/colp_m5p-pl_n_sign.png}
  \includegraphics[width=0.49\textwidth]{p/cha/colp/colp_m5p-pl_n_sin.png}
}
\caption{Процесс идентификации параметра $b$ системы (\ref{atu:eq:colp})
  при условиях (\ref{atu:eq:colp_b_sign}) и (\ref{atu:eq:colp_b_sin})
}
\label{atu:f:colp_id}
\end{figure}

Зависимость среднеквадратических ошибок идентификации от величины $q_\gamma$ (рис.~\ref{atu:f:colp_e_qgamma})
даёт информацию о правильной настройке этого параметра системы идентификации.

\begin{figure}[htb!]
\centerline{
  \includegraphics[width=0.49\textwidth]{p/cha/colp/colp_m5p-p_qg_e_sign.png}
  \includegraphics[width=0.49\textwidth]{p/cha/colp/colp_m5p-p_qg_e_sin.png}
}
  \caption{Зависимости  $\bar{e}(q_\gamma)$ для системы (\ref{atu:eq:colp})
  при условиях (\ref{atu:eq:colp_b_sign}) и (\ref{atu:eq:colp_b_sin})
}
\label{atu:f:colp_e_qgamma}
\end{figure}



Аналогично, зависимости $\bar{e_*}(a_q)$ (рис.~\ref{atu:f:colp_e_a_q})
позволяют корректно определить время усреденения.
Полученные результаты хорошо согласуются со спектрами системы.

\begin{figure}[htb!]
\centerline{
  \includegraphics[width=0.49\textwidth]{p/cha/colp/colp_m5p-p_a_q_e_sign.png}
  \includegraphics[width=0.49\textwidth]{p/cha/colp/colp_m5p-p_a_q_e_sin.png}
}
  \caption{Зависимости  $\bar{e}(a_q)$ для системы (\ref{atu:eq:colp})
  при условиях (\ref{atu:eq:colp_b_sign}) и (\ref{atu:eq:colp_b_sin})
}
\label{atu:f:colp_e_a_q}
\end{figure}

В целом синтез критерия идентификации, и построение работоспособной системы идентификации для
системы Колпитца не потребовало никаких специальных подходов.

 % good

% 
\FloatBarrier

\section{Колебательная с зоной нечувствительности в возвращающей силе} %  % {{{1 _DEADVI_
\label{atu:sect:deadvi}


\subsection{Определение системы и анализ её динамики} %  % {{{2 _deadvi_task

\LinkRef{
  deadvi: ASAU-20, ISDMCI-2013
}

\begin{equation}
\ddot{x} + c_0 \dot{x} + a \cdot x + b \cdot \mathrm{db}(x,x_0) = u(t),
\label{atu:eq:deadvi}
\end{equation}

$ u(t) = U_0 \sin( \omega_{in} t ) $.

Идентифицируемый параметр:
$ x_0 \in [2;2.5] $ -- ширина зоны нечувствительности.

Остальные параметры:
$U_0 = 1.4$, $\omega_{in} = 1.4$, $c_0=0.1$, $a=-0.3$, $b=1.0$.


\begin{figure}[htb!]
\centerline{\includegraphics[width=0.5\textwidth]{p/cha/deadvi_phase.pdf} }
\caption{Фазовый портрет системы ``deadvi'' (\ref{atu:eq:deadvi})}
\label{atu:f:deadvi_phase}
\end{figure}

% }}}2

\subsection{Анализ критериев}  % {{{2

Критерий
$\overline{x^2(t)}$

% }}}2

\subsection{Выводы}  % {{{2

Выводы

% }}}2



% }}}1


% vim: fdm=marker ft=tex
 %


\FloatBarrier
\section{Система Sprott A}

\LinkRef{
  spr\_a: MKMM-2016
}

В своих работах J.C.~Sprott рассмотрел целое семейство динамических
систем, реализующих хаотическое поведение, обозначив их латинскими буквами
от ``A'' до ``S''~\cite{sprott_212,sprott_strange_attr}. Особое место среди них
занимает система, обозначаемая как ``Sprott A''. Отличительной особенностью
этой системы является отсутствие положений равновесия, что делает
невозможным применение многих известных методов анализа, основанных на
каком-либо разложении в окрестностях точек равновесия. Соответствующая ей
система уравнений имеет вид:
%
\begin{equation}
  \begin{cases}
    \dot{x} =  y, \\
    \dot{y} = -x + yz, \\
    \dot{z} =  1 - y^2.
  \end{cases}
  \label{atu:eq:spr_a_orig}
\end{equation}


В исходном виде система (\ref{atu:eq:spr_a_orig}) имеет фиксированные значения параметров.
Не изменяя структуры системы, можно ввести 5 параметров, влияющие на её динамку.
Поскольку целью данной работы является определение критериев идентификации параметров
хаотических объектов, для данной системы
работе рассмотрим только один параметр -- $c_{x_y} $. Система принимает следующий вид:
%
\begin{equation}
  \begin{cases}
    \dot{x} =  c_{x_y} y, \\
    \dot{y} = -x + yz, \\
    \dot{z} =  1 - y^2.
  \end{cases}
  \label{atu:eq:spr_a}
\end{equation}

В таком виде система, при изменении $c_{x_y} $
в достаточно широком диапазоне может демонстрировать как
сложно-периодическое (рис.~\ref{atu:f:spr_a_p_0372}), так и преимущественно, хаотическое
поведение (рис.~\ref{atu:f:spr_a_p_0610}). 

\begin{figure}[htb!]
\centerline{
  \includegraphics[width=0.49\textwidth]{p/cha/spr_a/sprott_a-p_xyz_cx_y=0x372.png}
  \includegraphics[width=0.49\textwidth]{p/cha/spr_a/sprott_a_f-p_f_cx_y=0x372.png}
}
\caption{Аттрактор и спектр системы (\ref{atu:eq:spr_a}) при $ c_{x_y} =0.372 $.
  Сложно-периодический режим.
}
\label{atu:f:spr_a_p_0372}
\end{figure}

При этом, в диапазоне $c_{x_y} \in [0.1 ; 0.7] $
наблюдаются перестройки структуры аттрактора, а при относительно больших
значениях данного параметра аттрактор представляет собой полый тор.

\begin{figure}[htb!]
\centerline{
  \includegraphics[width=0.49\textwidth]{p/cha/spr_a/sprott_a-p_xyz_cx_y=0x610.png}
  \includegraphics[width=0.49\textwidth]{p/cha/spr_a/sprott_a_f-p_f_cx_y=0x610.png}
}
\caption{Аттрактор и спектр системы (\ref{atu:eq:spr_a}) при $ c_{x_y} =0.610 $.
  Хаотический режим
}
\label{atu:f:spr_a_p_0610}
\end{figure}




Важной особенностью поведения этой системы является то, что при $ c_{x_y} \ge 1 $
в спектре системы имеются очень ограниченные участки сплошного спектра~(рис.~\ref{atu:f:spr_a_p_1000}).
При этом, как и для получения корректного спектра, так и для обнаружения ``разбегания'' траекторий
необходимо моделирование системы на протяжении достаточно длительного
(по сравнению с многими схожими системами) модельного времени.

\begin{figure}[htb!]
\centerline{
  \includegraphics[width=0.49\textwidth]{p/cha/spr_a/sprott_a-p_xyz_cx_y=1x000.png}
  \includegraphics[width=0.49\textwidth]{p/cha/spr_a/sprott_a_f-p_f_cx_y=1x000.png}
}
\caption{Аттрактор и спектр системы (\ref{atu:eq:spr_a}) при $ c_{x_y} =1.0 $.
}
\label{atu:f:spr_a_p_1000}
\end{figure}


Рассмотрим зависимости $q_{*}(c_{x_y}) $ (рис.~\ref{atu:f:spr_a_q})
для системы (\ref{atu:eq:spr_a}). Анализ этих зависимостей
даёт практически однозначный ответ о возможном виде критерия -- $q_{x^2}$.
Второй кандидат -- $q_{x}$ обладает меньшей линейностью,
значения критериев, в которые не входит $x$, практически не зависят от $c_{c_y}$,
а вид критериев $q_{xy}$ и $q_{xz}$ ничем не лучше, чем у $q_{x}$.
Таким образом, систему идентификации будем строить, используя критерий  $q_{x^2}$.

\begin{figure}[htb!]
\centerline{
  \includegraphics[width=0.60\textwidth]{p/cha/spr_a/sprott_a_q-p_c_x_y.png}
}
\caption{Зависимости $q_{*}(c_{x_y})$ для системы (\ref{atu:eq:spr_a}) }
\label{atu:f:spr_a_q}
\end{figure}

Нельзя не отметить, что в начальной области
$q_{x^2}(c_{x_y}) $, а именно там, где происходят постоянные
перестройки структуры аттрактора, ни один из рассматриваемых критериев
не имеет достаточной монотонности, что не даёт возможности
построить работоспособную систему идентификации.
Для работы в этой области, скорее всего, требуется синтез специальных критериев.
Также, в окрестности точки $c_{x_y}=1.775$ аттрактор также резко изменяет свою структуру.
Так как это достаточно узкая окрестность,
то можно предположить, что мультиагентная система идентификации
не окажется неработоспособной в этой области, просто вырастет
ошибка идентификации.



Динамика процессов идентификации для системы Sprott A представлена на рис.~\ref{atu:f:spr_a_id}.
Общий вид динамики поиска свидетельствует о работоспособности системы.

\begin{figure}[htb!]
\centerline{
  \includegraphics[width=0.49\textwidth]{p/cha/spr_a/sprott_a_m5p-pl_n_sign.png}
  \includegraphics[width=0.49\textwidth]{p/cha/spr_a/sprott_a_m5p-pl_n_sin.png}
}
\caption{Процесс идентификации параметра $c_{x_y} $ системы (\ref{atu:eq:spr_a})
  при различных видах нестационарности этого параметра
}
\label{atu:f:spr_a_id}
\end{figure}

Зависимость $\overline{e}( q_\gamma )$ (рис.~\ref{atu:f:spr_a_e_qgamma})
свидетельствует о довольно слабом влиянии этого параметра
на динамику системы идентификации.
Проявляются робастные свойства поисковых агентов.

\begin{figure}[htb!]
\centerline{
  \includegraphics[width=0.49\textwidth]{p/cha/spr_a/sprott_a_m5p-p_qg_e_sign.png}
  \includegraphics[width=0.49\textwidth]{p/cha/spr_a/sprott_a_m5p-p_qg_e_sin.png}
}
  \caption{Зависимости  $\overline{e}(q_\gamma)$ для системы (\ref{atu:eq:spr_a})
  при различных видах нестационарности этого параметра
}
\label{atu:f:spr_a_e_qgamma}
\end{figure}


Зависимости $\overline{e_*}(a_q)$ (рис.~\ref{atu:f:spr_a_e_a_q})
с одной стороны, позволяют корректно определить время усреднения,
с другой -- подтверждают тезис о том, что более ярко выраженное изменение
параметра требует большего времени наблюдения за системой
для проведения идентификации.

\begin{figure}[htb!]
\centerline{
  \includegraphics[width=0.49\textwidth]{p/cha/spr_a/sprott_a_m5p-p_a_q_e_sign.png}
  \includegraphics[width=0.49\textwidth]{p/cha/spr_a/sprott_a_m5p-p_a_q_e_sin.png}
}
  \caption{Зависимости  $\overline{e}(a_q)$ для системы (\ref{atu:eq:spr_a})
  при различных видах нестационарности этого параметра
}
\label{atu:f:spr_a_e_a_q}
\end{figure}

Таким образом, для хаотической системы ``Sprott A''
удалось создать и критерий идентификации и работоспособную систему идентификации
на основании энергетического критерия вида $q_{x^2}$.


 % good


\FloatBarrier
\section{Система с сухим трением} %  % {{{1  __FRIC__
\label{atu:sect:fric}

\LinkRef{
  fric: ASAU-11, DSMP-2016
}

\subsection{Определение системы и анализ её динамики} %  % {{{2 _fric_task


Существуют динамические системы, не обладающие хаотическим поведением,
которые, тем не менее, проявляют сходные свойства с точки зрения идентификации.
А именно, непосредственное сравнение выходов системы и модели не позволяет
сделать никаких выводов о соотношениях между параметрами модели и объекта.
Одним из примеров таких систем является динамическая система (\ref{atu:eq:dryfric_sys}),
моделирующая поведение тела заданной массы по действием внешней вынуждающей силы
и силы сухого трения
\cite{berger_friction,atu_asau11}:
%
\begin{equation}
  m \ddot{x} + f_{df}( x, \dot{x}, \ldots)  = u(t).
\label{atu:eq:dryfric_sys}
\end{equation}
%
%\noindent
где
$m$ -- масса тела,
$u(t)$ -- вынуждающая сила,
$ f_{df}( x, \dot{x}, \ldots)  $ -- сила сухого трения.

Важной особенностью при моделировании силы сухого трения является тот факт,
что это силу невозможно корректно выразить аналитически. Более того,
её алгоритмическое представления неизбежно потребует учёта всех других сил,
действующих на тело, то крайней мере, для корректного представления
силы трения покоя. Эта особенность не даёт возможности
аналитического анализа таких систем, за исключением ограниченного набора
вырожденных случаев. При этом, такое поведение существенно затрудняет идентификацию,
и, в определённых случая, делает её принципиально невозможной.

Основным параметром, в простейшем случае определяющим силу сухого
трения, является $f_{dm}$ -- максимальная значение её модуля.
Определение этой величины и будем считать целью задачи идентификации
системы с сухим трением.

В свою очередь, существенным свойством данной системы является то, что в каждой точке \(x\)
система может находится в покое, даже если на неё действует
внешняя сила (по модулю не превышающая $f_{dm}$).
Если же рассмотреть пару или более подобных систем,
то получившаяся система имеет общие свойства с системами
хаотической динамики, а именно: малые возмущения входного сигнала
или коэффициентов модели приводят к значительным изменениям
выходного, причем реакция на возмущения может быть
не ограничена во времени.
Более того, даже если после какого-то момента времени
параметры объекта и коэффициенты модели полностью совпадут,
величина ошибки по выходнам не устремится к нулю, а останется на прежнем уровне.

На рис.~\ref{atu:f:fric_outs} представлен сравнительный пример динамики трёх
моделей, при одинаковом входном сигнале
и различных значениях $f_{dm}$.

\begin{figure}[htb!]
  \centerline{
    \includegraphics[width=0.5\textwidth]{p/cha/fric/fric_outs1.png}
  }
  \caption{Динамика трёх моделей вида (\ref{atu:eq:dryfric_sys})}
  \label{atu:f:fric_outs}
\end{figure}

При дальнейшем моделировании, в качестве сигнала $u(t)$ использовался кусочно-линейный периодический сигнал,
в котором чередуются ``плато'' и резкие изменения. Выбор такой формы обусловлен
характерными режимами работы систем позиционирования электромеханических
устройств, для которых и характерно влияние сухого трения.

% }}}2

\subsection{Анализ и выбор критериев}  % {{{2

Для обеспечения возможности применения методов идентификации,
необходимо существование критерия
\( q(x(t)) \),
удовлетворяющего следующим требованиям:

\begin{itemize}

\item
чувствительность к \textit{динамике} модели и объекта;

\item
свойство астатизма, то есть
независимость
от смещения выхода объекта или модели:
\( q(x(t)+a ) \approx q( x(t) ) \);

\item
достаточная устойчивость к шумам измерения;

\item
физическая реализуемость.

\end{itemize}

Первые два требования
могут быть достигнуты путём вычисления производной --
скорости изменения выходных сигналов
\(v = \mathrm{d}\,x(t)/ \mathrm{d}\,t \),
и формирования критерия идентификации на её основе.

Однако, при этом система становится исключительно чувствительной
к шумам измерения. Даже в том случае, когда
оценка производной производится физически реализуемыми методами,
создать работоспособную систему идентификации на основе критериев
подобного вида практически невозможно.

В работе \cite{atu_asau11} был предложен метод синтеза критерия идентификации
на основе гистерезисной фильтрации выходных сигналов, с последующим
вычислением производной. Метод показал свою работоспособность, однако,
он требует достаточно точных сведений об уровне и виде шумов -- для
настройки гистерезисного фильтра. В данной работе сделаем предположение,
что уровень шумов позволяет создать фильтр, позволяющий отсеивать шумы
за (как максимум) характерное время реакции системы.
После фильтра действует реальное дифференцирующее звено.
Соответствующий вид критерия обозначим как $ q_{dx} $ (рис.~\ref{atu:f:fric_q}).



\begin{figure}[htb!]
\centerline{
  \includegraphics[width=0.60\textwidth]{p/cha/fric/fric_q-p_f_dm_q.png}
}
  \caption{Зависимость $q_{dx}(f_{dm})$ для системы (\ref{atu:eq:dryfric_sys}) }
\label{atu:f:fric_q}
\end{figure}

Несмотря на выраженную нелинейность в начальной части графика,
данный критерий представляется вполне применимым для задачи идентификации,
ввиду монотонности ограниченным диапазоном производной.
При больших значениях параметра $f_{dm}$ смысл этого критерия теряется,
так как отсутствует динамика системы. При малых значениях возможности идентификации
тоже сильно ограничены, так как динамика системы больше определяется
массой объекта, чем пренебрежимо малой силой трения.

% }}}2


\subsection{Тестовая задача идентификации }   % {{{2

Аналогично уже рассмотренным система
определим тестовую задачу следующим образом:
\[
  f_{dm}(t) \equiv p_o(t) \in (0.1, 1.5),
\]
%
\begin{equation}
  p_o(t) = p_0 +  U_{p} \sign \sin( \omega_{p} t ),
  \label{atu:eq:fric_po_t_sign}
\end{equation}
%
%
\begin{equation}
  p_o(t) = p_0 +  U_{p} \sin( \omega_{p} t ),
  \label{atu:eq:fric_po_t_sin}
\end{equation}
%
где:
$p_0 = 0.7$, $U_p=0.4$, $\omega_p=0.0104719755$.

Динамика процессов идентификации для системы (\ref{atu:eq:dryfric_sys})
с использованием группы методов ql3rlWvnAAW.$q_{dx}$
при условии (\ref{atu:eq:fric_po_t_sign})
представлена на рис.~\ref{atu:f:fric_id_ql3rlWvnAAW_q_dx_sign}.


\begin{figure}[htb!]
  \centerline{
    \includegraphics[width=0.49\textwidth]{p/cha/fric/ql3rlWvnAAW/fric_id-p_t_pi_ql3rlWvnAAW_sign.png}
    \hfill
    \includegraphics[width=0.49\textwidth]{p/cha/fric/ql3rlWvnAAW/fric_id-p_t_p_ql3rlWvnAAW_sign.png}
  }
  \caption{Процесс идентификации параметра ``$f_{dm}$'' системы (\ref{atu:eq:dryfric_sys}) группой методов ql3rlWvnAAW.$q_{dx}$ при условии~(\ref{atu:eq:fric_po_t_sign})}
  \label{atu:f:fric_id_ql3rlWvnAAW_q_dx_sign}
\end{figure}

Для рассматриваемых условий наблюдается малый уровень возмущений
идентифицируемого параметра. Однако, при малых значениях $f_{dm}$
наблюдается заметная статическая ошибка идентификации. Это связано с тем,
что в этой области влияние трения становится мало,
и критерий теряет свою адекватность.

Следует отметить, что при
больших значениях $\omega_p$ наблюдается
сильное влияние входного сигнала, В силу того, что когда системы неподвижны
на одном из ``плато'', в эти моменты нет возможности различить их параметры,
какой бы критерий при этом не использовался.

Результаты моделирования при аналогичных условиях, но
при динамике параметра, заданной (\ref{atu:eq:fric_po_t_sin}),
представлены на рис.~\ref{atu:f:fric_id_ql3rlWvnAAW_q_dx_sin}.

\begin{figure}[htb!]
  \centerline{
    \includegraphics[width=0.49\textwidth]{p/cha/fric/ql3rlWvnAAW/fric_id-p_t_pi_ql3rlWvnAAW_sin.png}
    \hfill
    \includegraphics[width=0.49\textwidth]{p/cha/fric/ql3rlWvnAAW/fric_id-p_t_p_ql3rlWvnAAW_sin.png}
  }
  \caption{Процесс идентификации параметра ``$f_{dm}$'' системы (\ref{atu:eq:dryfric_sys}) группой методов ql3rlWvnAAW.$q_{dx}$ при условии~(\ref{atu:eq:fric_po_t_sin})}
  \label{atu:f:fric_id_ql3rlWvnAAW_q_dx_sin}
\end{figure}

За исключением уже упомянутой области, в которой критерий перестаёт работать,
наблюдается весьма низкий уровень ошибок идентификации.
В первую очередь это обусловлено тем, что данная система не является
хаотической, и проявляет только некоторые подобные свойства.
И как следствие, критерий, основанный на физических свойствах системы
позволяет создать систему идентификации с низким уровнем ошибок.


% }}}2




\subsection{Влияние параметров системы идентификации на ошибку идентификации для системы }  % {{{2

Относительно низкий достигнутый уровень ошибки идентификации
для данной системы даёт возможность выпукло проявить влияние
параметров системы идентификации.


Зависимости $\overline{e_*}(a_q)$ (рис.~\ref{atu:f:fric_a_q_ql3rlWvnAAW_q_dx})
 позволяют определить корректное время усреденения критерия.

\begin{figure}[htb!]
  \centerline{
    \includegraphics[width=0.49\textwidth]{p/cha/fric/ql3rlWvnAAW/fric_id-p_a_q_sign.png}
    \hfill
    \includegraphics[width=0.49\textwidth]{p/cha/fric/ql3rlWvnAAW/fric_id-p_a_q_sin.png}
  }
  \caption{Зависимости $\overline{e}(a_q)$ при идентификации системы (\ref{atu:eq:dryfric_sys}) группой методов ql3rlWvnAAW.$q_{dx}$
   при~(\ref{atu:eq:fric_po_t_sign}) и (\ref{atu:eq:fric_po_t_sin})}
  \label{atu:f:fric_a_q_ql3rlWvnAAW_q_dx}
\end{figure}


Зависимость ошибок идентификации от величины коэффициента масштаба
$q_\gamma$
(рис.~\ref{atu:f:fric_q_gamma_ql3rlWvnAAW_q_dx})
достаточно сильно отличатся от аналогичной, например, для системы ``Sprott A''.
% На ней явно выражены
%экстремумы, дающие возможность тонкой настройки системы идентификации.
Следовательно,
дополнительная адаптация может заметно улучшить качество идентификации.

\begin{figure}[htb!]
  \centerline{
    \includegraphics[width=0.49\textwidth]{p/cha/fric/ql3rlWvnAAW/fric_id-p_q_gamma_sign.png}
    \hfill
    \includegraphics[width=0.49\textwidth]{p/cha/fric/ql3rlWvnAAW/fric_id-p_q_gamma_sin.png}
  }
  \caption{Зависимости $\overline{e}(q_\gamma)$ при идентификации системы (\ref{atu:eq:dryfric_sys}) группой методов ql3rlWvnAAW.$q_{dx}$
   при~(\ref{atu:eq:fric_po_t_sign}) и (\ref{atu:eq:fric_po_t_sin})}
  \label{atu:f:fric_q_gamma_ql3rlWvnAAW_q_dx}
\end{figure}


(рис.~\ref{atu:f:fric_v_f_ql3rlWvnAAW_q_dx})

\begin{figure}[htb!]
  \centerline{
    \includegraphics[width=0.49\textwidth]{p/cha/fric/ql3rlWvnAAW/fric_id-p_v_f_sign.png}
    \hfill
    \includegraphics[width=0.49\textwidth]{p/cha/fric/ql3rlWvnAAW/fric_id-p_v_f_sin.png}
  }
  \caption{Зависимости $\overline{e}(v_f)$ при идентификации системы (\ref{atu:eq:dryfric_sys}) группой методов ql3rlWvnAAW.$q_{dx}$
   при~(\ref{atu:eq:fric_po_t_sign}) и (\ref{atu:eq:fric_po_t_sin})}
  \label{atu:f:fric_v_f_ql3rlWvnAAW_q_dx}
\end{figure}

(рис.~\ref{atu:f:fric_k_e_ql3rlWvnAAW_q_dx})

\begin{figure}[htb!]
  \centerline{
    \includegraphics[width=0.49\textwidth]{p/cha/fric/ql3rlWvnAAW/fric_id-p_k_e_sign.png}
    \hfill
    \includegraphics[width=0.49\textwidth]{p/cha/fric/ql3rlWvnAAW/fric_id-p_k_e_sin.png}
  }
  \caption{Зависимости $\overline{e}(k_e)$ при идентификации системы (\ref{atu:eq:dryfric_sys}) группой методов ql3rlWvnAAW.$q_{dx}$
   при~(\ref{atu:eq:fric_po_t_sign}) и (\ref{atu:eq:fric_po_t_sin})}
  \label{atu:f:fric_k_e_ql3rlWvnAAW_q_dx}
\end{figure}

(рис.~\ref{atu:f:fric_k_nl_ql3rlWvnAAW_q_dx})

\begin{figure}[htb!]
  \centerline{
    \includegraphics[width=0.49\textwidth]{p/cha/fric/ql3rlWvnAAW/fric_id-p_k_nl_sign.png}
    \hfill
    \includegraphics[width=0.49\textwidth]{p/cha/fric/ql3rlWvnAAW/fric_id-p_k_nl_sin.png}
  }
  \caption{Зависимости $\overline{e}(k_{nl})$ при идентификации системы (\ref{atu:eq:dryfric_sys}) группой методов ql3rlWvnAAW.$q_{dx}$
   при~(\ref{atu:eq:fric_po_t_sign}) и (\ref{atu:eq:fric_po_t_sin})}
  \label{atu:f:fric_k_nl_ql3rlWvnAAW_q_dx}
\end{figure}


% }}}2

\subsection{Выводы}  % {{{2

Для данной системы энергетические критерии, аналогичные применённым в предыдущих случаях,
оказались неприменимы. Критерий, хоть и основанный на измерении производной (после фильтрации),
оказался работоспособным. В первую очередь это связано с тем, что в основе были положены
физические принципы, которые для системы с сухим трением, вполне очевидны.

% }}}2



% }}}1

% vim: fdm=marker foldlevel=1 foldignore="%#" fdc=4 ft=tex
  % good

% 
\FloatBarrier
\section{Релаксационные генераторы}  % {{{1
\label{atu:sect:relax}

\LinkRef{
  relax: ASAU-18 (no id?)
}

\begin{figure}[htb!]
\centerline{\includegraphics[width=0.5\textwidth]{p/cha/relax_phase3_0500.pdf} }
\caption{Аттрактор системы с релаксационными генераторами}
\label{atu:f:relax_phase3}
\end{figure}

% }}}1

% vim: fdm=marker ft=tex
 % no id

% 
\FloatBarrier
\section{Колебательная система с гистерезисом в возвращающей силе} %  % {{{1 _VGLASS_
\label{atu:sect:vglass}

\LinkRef{
  vglass: ITMM-2013
}

\subsection{Определение системы и анализ её динамики} %  % {{{2 _vglass_task

Колебательные системы с различными видами нелинейностями
как в возвращающей силе, так и члене при первой производной,
часто рассматриваются как потенциальные кандидаты в хаотические системы.
Особый интерес представляют системы, нелинейные компоненты которых
отображают реальные элементы систем управления~\cite{atu_st85,atu_ISDMCI2013,atu_asau20}.
Одним из таких существенно нелинейных элементов
является гистерезис~\cite{sys_hyst,in_theory_mech_vibro,ivanov_alg_id_dyn_hyst,andronn_id_ns_hyst,mai_iss_dyn_isp}.
Рассмотрим изначально неустойчивую  систему,
которая стабилизируется релейно-пропорциональным
стабилизирующим воздействием.
Такая система стабилизации активируется при заданном отклонении наблюдаемой
величины, а выключается при меньшем. Такая неоднозначность в поведении
систем стабилизации приводит к разнообразным колебательным явлениям вблизи
точки стабилизации, а все попытки линеаризации рассматриваемых
нелинейностей, в данном случае, приводят к принципиальным ошибкам в
моделировании.

Наличие точки неустойчивого равновесия, требующего стабилизирующего
воздействия в большем масштабе, явной, а зачастую и скрытой нелинейности
элементов системы, априорная параметрическая неопределенность внешних
воздействий, --- всё это создаёт предпосылки для появления у систем сложной
колебательной динамики, в том числе и хаотической.
Рассмотрим уравнение, задающее динамику этой системы:
%
\begin{equation}
  \ddot{x} + c_o \dot{x} + r( x, \ldots ) = u(t),
  \label{atu:eq:vglass}
\end{equation}
%
где
$x(t)$ -- координата (выходной сигнал),
$ c_0$ -- безразмерный коэффициент демпфирования,
$u(t) = U_{in} \sin( \omega_{in} t ) $ -- внешняя возмущающая сила,
$r( x, \ldots) $ --- возвращающая сила,
включающая линейную составляющую ($ax$, $a<0$),
обуславливающую неустойчивость при малых отклонениях,
и гистерезисную компоненту, которая задаётся алгоритмически.
Характерный вид возвращающий силы приведён на  рис.~\ref{atu:f:vg_rf}.

\begin{figure}[htb!]
\centerline{\includegraphics[width=0.5\textwidth]{p/cha/vg/vg_rf-p_rf.png} }
\caption{Гистерезисная возвращающая сила}
\label{atu:f:vg_rf}
\end{figure}

Параметрами этой зависимости, помимо уже упомянутого коэффициента $a$,
являются $s_x$ --- центральная точка включения и выключения
стабилизирующего воздействия,
$x_0$ --- полуширина гистерезиса.
Ширина петли гистерезиса для данной системы определяет
энергию, которую получает система от системы стабилизации в процессе работы.
Именно это величина и будет идентифицируемым параметром.


В отличие от системы Дуффинга, рассматриваемая система имеет собственную
динамику при отсутствии возмущающего воздействия.
При малых значениях $x_0$ происходят нелинейные,
но достаточно простые колебания по одну сторону от
точки $x=0$~(рис.~\ref{atu:f:vglass_phase_f_u00}),
причём сторона определяется начальными условиями.
Спектр колебаний весьма бедный.

\begin{figure}[ht!]
\begin{center}
  \includegraphics[width=0.49\textwidth]{p/cha/vg/vg_0-p_phe_0x00_0x70_0x12.png}
  \hfill
  \includegraphics[width=0.49\textwidth]{p/cha/vg/vg_fft-p_f_0x00_0x70_0x12.png}
\end{center}
  \caption{Расширенный фазовый портрет и спектр системы (\ref{atu:eq:vglass}) при $U_{in}=0$ и $x_0=0.12$}
\label{atu:f:vglass_phase_f_u00}
\end{figure}

При росте параметра $x_0$ динамика системы становится более сложной
(рис.~\ref{atu:f:vglass_phase_f_u01}),
однако спектр системы остаётся простым линейчатым.
Такое поведение сохраняется вплоть до естественного ограничения $x_o < s_x$,
при этом фазовый портрет приближается к окружности.

\begin{figure}[ht!]
\begin{center}
  \includegraphics[width=0.49\textwidth]{p/cha/vg/vg_0-p_phe_0x00_0x70_0x15.png}
  \hfill
  \includegraphics[width=0.49\textwidth]{p/cha/vg/vg_fft-p_f_0x00_0x70_0x15.png}
\end{center}
  \caption{Расширенный фазовый портрет и спектр системы (\ref{atu:eq:vglass}) при $U_{in}=0$ и $x_0=0.15$}
\label{atu:f:vglass_phase_f_u01}
\end{figure}

При наличии ненулевого внешнего воздействия $u(t)$
картина заметно усложняется.
Наблюдаются такие значения параметра $x_0$,
при которых наблюдается сложный аттрактор, характерный для
хаотической динамики~(рис.~\ref{atu:f:vglass_phase_f_u10})
и участки сплошного спектра.

\begin{figure}[ht!]
\begin{center}
  \includegraphics[width=0.49\textwidth]{p/cha/vg/vg_0-p_phe_0x20_0x70_0x05.png}
  \hfill
  \includegraphics[width=0.49\textwidth]{p/cha/vg/vg_fft-p_f_0x20_0x70_0x05.png}
\end{center}
  \caption{Расширенный фазовый портрет и спектр системы (\ref{atu:eq:vglass}) при $U_{in}=0.2$ и $x_0=0.05$}
\label{atu:f:vglass_phase_f_u10}
\end{figure}

Общим свойством этой системы и системы Дуффинга является то,
что участок сплошного спектра примыкает к точке нулевой частоты,
что затрудняет процесс усреднения критерия.

Также наблюдаются такие диапазоны $x_0$,
в которых наблюдаются регулярные колебания
(рис.~\ref{atu:f:vglass_phase_f_u11}).

\begin{figure}[ht!]
\begin{center}
  \includegraphics[width=0.49\textwidth]{p/cha/vg/vg_0-p_phe_0x20_0x70_0x40.png}
  \hfill
  \includegraphics[width=0.49\textwidth]{p/cha/vg/vg_fft-p_f_0x20_0x70_0x40.png}
\end{center}
  \caption{Расширенный фазовый портрет и спектр системы (\ref{atu:eq:vglass}) при $U_{in}=0.2$ и $x_0=0.40$}
\label{atu:f:vglass_phase_f_u11}
\end{figure}

Помимо регулярных колебаний, наблюдаются
режимы сложно-периодической динамики,
с сложным аттрактором и спектром,
в котором наблюдаются ряд близко расположенных пиков
(рис.~\ref{atu:f:vglass_phase_f_u12}).

\begin{figure}[ht!]
\begin{center}
  \includegraphics[width=0.49\textwidth]{p/cha/vg/vg_0-p_phe_0x20_0x70_0x70.png}
  \hfill
  \includegraphics[width=0.49\textwidth]{p/cha/vg/vg_fft-p_f_0x20_0x70_0x70.png}
\end{center}
  \caption{Расширенный фазовый портрет и спектр системы (\ref{atu:eq:vglass}) при $U_{in}=0.2$ и $x_0=0.70$}
\label{atu:f:vglass_phase_f_u12}
\end{figure}

Такое поведение сложно, а в случае нестационарного значения параметра
может быть и невозможно отличить от хаотического поведения.

Таким образом, так как система в рабочем диапазоне параметра
$x_0$ демонстрирует различные виды динамик, в том числе хаотическую,
то для идентификации имеет смысл рассмотреть набор методов,
рассматриваемых в данной работе.

% }}}2

\subsection{Анализ и выбор критерия}  % {{{2

При синтезе критерия для нелинейной колебательной системы,
если нет возможности использовать
какой-либо явный и очевидный критерий,
в первую очередь следует рассмотреть
критерии вида $q_{x^2}$, $q_{rx}$ и $q_T$.
Так как гистерезисная возвращающая сила
не даёт возможности использовать аналитический
подход, то рассмотрим
зависимости $q_{rx}$ и $q_T$
(рис.~\ref{atu:f:vglass_q}),
полученные путём численного моделирования.

\begin{figure}[ht!]
\begin{center}
  \includegraphics[width=0.49\textwidth]{p/cha/vg/vg_q1-p_q.png}
\end{center}
  \caption{Зависимости $q_{rx}(x_0)$ и $q_T(x_0)$ для системы (\ref{atu:eq:vglass})}
\label{atu:f:vglass_q}
\end{figure}

Анализ зависимостей позволяет сделать вывод о том, что критерий $q_T$
не пригоден для идентификации параметра $x_0$
рассматриваемой системы, ввиду того,
что зависимости от параметра, если не принимать во внимание колебания,
практически отсутствует.

Применений критерия $q_{rx}$ достаточно противоречиво.
С одной стороны, в целом наблюдается близкая к линейной зависимость,
что снижает требования к тонкой настройке системы идентификации.
С другой стороны, на тех участках,
где наблюдаются переходы от сложно-периодических колебаний
к хаотическим, монотонность зависимости нарушается.
Однако, амплитуда этих возмущений невелика,
что даёт возможность построения работоспособной системы идентификации,
но с существенными ограничениями на достижимую точность.
Таким образом, на неимением лучшего, будем использовать этот критерий.



% }}}2

\subsection{Тестовая задача идентификации}  % {{{2

Для идентификации параметра $x_0$
колебательной системы с гистерезисной возвращающей силой
была использована группа методов  ``ql3rlWvnAAW''.
Использование критерия $q_{rx}$ для данной системы,
с учетом наличия участков отсутствия монотонности
требует отдельной проверки работоспособности,
так как используемый ранее подход с
медленно изменяющимся параметром (\ref{atu:eq:po_t_ramp}) не позволяет
отделить ошибку, связанную с установлением режима системы идентификации
от ошибки, вызванной не вполне адекватным критерием.

Для определения ошибки идентификации без влияния переходных процессов,
была проведена серия вычислительных экспериментов.
К каждом из них значения параметра $x_0$ было фиксировано.
Диапазон изменения этого параметра был выбран $[0.2 ; 0.6]$.
Предварительно была сделана проверка, что за полное время идентификации
$T= 1000$ все переходные процессы действительно завершились.
Зависимость полученной ошибки идентификации от значения
параметра $x_0$ для различных способов оценивания $p_\mathrm{id}$
представлена на рис.~\ref{atu:f:vg_id_scan}.

\begin{figure}[ht!]
\begin{center}
  \includegraphics[width=0.60\textwidth]{p/cha/vg/vg_id-p_p_e_ql3rlWvnAAW_scan.png}
\end{center}
  \caption{Зависимости $e(x_0) $ для колебательной системы с гистерезисной возвращающей силой}
\label{atu:f:vg_id_scan}
\end{figure}

Анализ полученных зависимостей показывает, что в тех областях, где нарушается монотонность
критерия, ошибка идентификации значительно возрастает.
Однако, уровень этой ошибки, за исключением одной узкой области,
не превышает исходного расстояния между агентами,
что позволяет сделать вывод том, что система идентификации оказывается работоспособной
и этом неблагоприятном случае.

Рассмотрим динамику агентов и идентифицируемого значения (рис.~\ref{atu:f:vg_id_sign}) в случае
существенно нестационарного параметра.
При этом динамика величины $x_0(t)$ задаётся (\ref{atu:eq:po_t_sign}),
$p_0=0.4$, $U_p=0.14$, $\omega_p= 0.01048$.

\begin{figure}[ht!]
\begin{center}
  \includegraphics[width=0.49\textwidth]{p/cha/vg/vg_id-p_t_pi_ql3rlWvnAAW_sign.png}
  \hfill
  \includegraphics[width=0.49\textwidth]{p/cha/vg/vg_id-p_t_p_ql3rlWvnAAW_sign.png}
\end{center}
  \caption{Динамика агентов и идентифицируемого значения для системы колебательной системы с гистерезисной возвращающей силой при условии (\ref{atu:eq:po_t_sign})}
\label{atu:f:vg_id_sign}
\end{figure}

В рассматриваемых условиях на величину ошибки идентификации
негативно влияют два фактора: примыкающий к нулевой частоте
участок сплошного спектра, а также отсутствие монотонности критерия
в области минимального значения параметра.
Это приводит к значительным ошибкам в процессе идентификации,
при сохранении работоспособности метода.


На рис.~\ref{atu:f:vg_id_sin}
приведены аналогичные зависимости, то при условии~(\ref{atu:eq:po_t_sign}).

\begin{figure}[ht!]
\begin{center}
  \includegraphics[width=0.49\textwidth]{p/cha/vg/vg_id-p_t_pi_ql3rlWvnAAW_sin.png}
  \hfill
  \includegraphics[width=0.49\textwidth]{p/cha/vg/vg_id-p_t_p_ql3rlWvnAAW_sin.png}
\end{center}
  \caption{Динамика агентов и идентифицируемого значения для системы колебательной системы с гистерезисной возвращающей силой при условии (\ref{atu:eq:po_t_sin})}
\label{atu:f:vg_id_sin}
\end{figure}

Более плавное изменение параметра,
как и в предыдущих случаях,
приводит к уменьшению ошибки идентификации.
Однако, вид критерия не позволяет получить
достаточно малую ошибку идентификации.



% }}}2

\subsection{Влияние параметров системы идентификации на ошибку идентификации}  % {{{2

Влияние параметра $a_q$ на ошибку идентификации
для колебательной системы с гистерезисной возвращающей силой
имеет вид, типичный для систем со спектром,
примыкающим к нулевой частоте~(рис.~\ref{atu:f:vg_e_a_q}).
Нет явно выраженного минимума, что свидетельствует
об ограниченности фильтрующих свойств подхода к усреднению критерия.

\begin{figure}[ht!]
\begin{center}
  \includegraphics[width=0.49\textwidth]{p/cha/vg/vg_id-p_a_q_sign.png}
  \hfill
  \includegraphics[width=0.49\textwidth]{p/cha/vg/vg_id-p_a_q_sin.png}
\end{center}
  \caption{Зависимости $\bar{e}(a_q)$ для колебательной системы с гистерезисной возвращающей силой}
\label{atu:f:vg_e_a_q}
\end{figure}

На фоне относительно высокой ошибки идентификации,
определяемой свойствами критерия,
зависимость от масштаба функции качества
(рис.~\ref{atu:f:vg_e_q_gamma})
пренебрежимо мала, если, конечно,
параметр $q_\gamma$ системы идентификации
находится в разумных пределах.

\begin{figure}[ht!]
\begin{center}
  \includegraphics[width=0.49\textwidth]{p/cha/vg/vg_id-p_q_gamma_sign.png}
  \hfill
  \includegraphics[width=0.49\textwidth]{p/cha/vg/vg_id-p_q_gamma_sin.png}
\end{center}
  \caption{Зависимости $\bar{e}(q_\gamma)$ для колебательной системы с гистерезисной возвращающей силой}
\label{atu:f:vg_e_q_gamma}
\end{figure}

Свойства критерия $q_{rx}$ для данной системы
также обуславливают слабую зависимость ошибки идентификации
от параметра $v_f$ (рис.~\ref{atu:f:vg_e_v_f}).
Для быстро изменяющегося параметра перемещение агентов
может быть не порванным.

\begin{figure}[ht!]
\begin{center}
  \includegraphics[width=0.49\textwidth]{p/cha/vg/vg_id-p_v_f_sign.png}
  \hfill
  \includegraphics[width=0.49\textwidth]{p/cha/vg/vg_id-p_v_f_sin.png}
\end{center}
  \caption{Зависимости $\bar{e}(v_f)$ для колебательной системы с гистерезисной возвращающей силой}
\label{atu:f:vg_e_v_f}
\end{figure}

Совершенно аналогичная картина наблюдается для параметра $k_e$~(рис.~\ref{atu:f:vg_e_k_e}).

\begin{figure}[ht!]
\begin{center}
  \includegraphics[width=0.49\textwidth]{p/cha/vg/vg_id-p_k_e_sign.png}
  \hfill
  \includegraphics[width=0.49\textwidth]{p/cha/vg/vg_id-p_k_e_sin.png}
\end{center}
  \caption{Зависимости $\bar{e}(k_e)$ для колебательной системы с гистерезисной возвращающей силой}
\label{atu:f:vg_e_k_e}
\end{figure}

Вид зависимости $\bar{e}(k_{nl})$ для рассматриваемой системы имеет определённые
отличия~(рис.~\ref{atu:f:vg_e_k_nl}).
Слабо выраженный, но всё же заметный минимум наблюдается
при значениях $k_{nl}$, практически совпадающих с используемым значением $k_e$.
Это можно объяснить тем, что большее квазиравновесное расстояние
между агентами позволяет в какой-то мере нивелировать
возмущения критерия небольшой амплитуды.

\begin{figure}[ht!]
\begin{center}
  \includegraphics[width=0.49\textwidth]{p/cha/vg/vg_id-p_k_nl_sign.png}
  \hfill
  \includegraphics[width=0.49\textwidth]{p/cha/vg/vg_id-p_k_nl_sin.png}
\end{center}
  \caption{Зависимости $\bar{e}(k_{nl})$ для колебательной системы с гистерезисной возвращающей силой}
\label{atu:f:vg_e_k_nl}
\end{figure}

Использование неоднозначного критерия,
в совокупности с ограничениями на перемещения агентов
делает зависимость $\bar{e}(k_{cl})$ (рис.~\ref{atu:f:vdp_e_k_cl})
практически не наблюдаемой.


\begin{figure}[ht!]
\begin{center}
  \includegraphics[width=0.49\textwidth]{p/cha/vg/vg_id-p_k_cl_sign.png}
  \hfill
  \includegraphics[width=0.49\textwidth]{p/cha/vg/vg_id-p_k_cl_sin.png}
\end{center}
  \caption{Зависимости $\bar{e}(k_{cl})$ для колебательной системы с гистерезисной возвращающей силой}
\label{atu:f:vg_e_k_cl}
\end{figure}



% }}}2


\subsection{Выводы}  % {{{2

Результаты моделирования
как динамики системы~(\ref{atu:eq:vglass})
так и процессов идентификации её параметра $x_0$
позволяют в сделать следующие выводы:

\begin{itemize}

  \item
    Колебательная система с гистерезисной возвращающей силой
    под воздействием внешней гармонического возмущения
    может демонстрировать различные виды динамик, в
    том числе сложно-периодическую и хаотическую.


  \item
    Критерий $q_{rx}$ хоть и подходит для идентификации параметра ``$x_0$''
    колебательной системы с гистерезисной возвращающей силой,
    но обладает нарушениями монотонности,
    что существенно увеличивает ошибку идентификации.

  \item
    Наличие сплошного спектра системы, примыкающего к нулевой частоте,
    усложняет задачу корректного усреднения критерия.

  \item
    Группа методов ql3rlWvnAAW показала свою хорошую работоспособность
    даже в условиях использования критерия с ограниченной применимостью.


\end{itemize}
% }}}2

% }}}1

% vim: fdm=marker foldlevel=1 foldignore="%#" fdc=4 ft=tex
 %



\subsection*{Выводы}

В целом, применение рассмотренных видов критериев, в совокупности
с мультимодельными (мультиагентными) методами идентификации
работоспособность для определённого класса сложных динамических систем.
Полученные зависимости ошибок идентификации от параметров
самой системы идентификации, с одной стороны, позволяют
правильно настроить эти параметры для получения максимально быстрого
и точного процесса идентификации. В другой стороны, эти зависимости
подчёркивают связь свойств идентифицируемого объекта с
требуемыми качествами системы идентификации.


\printbibliography
\label{e:atu}{~}

\FloatBarrier


\end{document}

