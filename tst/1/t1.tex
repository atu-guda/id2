\documentclass[a4paper,12pt]{article}

\usepackage[left=2cm,right=2cm,top=2cm,bottom=2cm,bindingoffset=0cm]{geometry}

\usepackage{cmap}

\usepackage[TS1,T1,T2A]{fontenc}
\usepackage[utf8]{inputenc}
\usepackage[unicode]{hyperref}

\usepackage{amssymb}
\usepackage{amsmath}
\usepackage{fouriernc}


%\usepackage{dejavu}
\usepackage{paratype}
%\usepackage{droid}

% \usepackage{atu_csb}
% \usepackage{atu_prg}
% \usepackage{atu_nim}

\usepackage{indentfirst}
\usepackage{graphicx}
\usepackage[nodisplayskipstretch]{setspace} %setstretch
\usepackage{siunitx}
\usepackage{mathtools}
\usepackage[english,ukraineb,russian]{babel}
\usepackage{tikz}

\newcommand{\Mc}[1]{{\text{\it #1}}}

\usepackage{blox}
\usepackage[europeanresistors,americaninductors,siunitx,fulldiodes]{circuitikz}

\usetikzlibrary{calc}
\usetikzlibrary{arrows}
\usetikzlibrary{patterns}
%\usepgflibrary{shapes.geometric}
\usetikzlibrary{external}


\definecolor{haircolor}{rgb}{0.7,0.7,1.0}
\newcommand{\TikzAddPadding}{\path (current bounding box.north east) ++(+0.1,+0.1); \path (current bounding box.south west) ++(-0.1,-0.1);}

\tikzset{
  >=stealth,
  %semiRed/.style={fill=red,opacity=0.3,draw=black,thin},
  hair/.style={draw,color=haircolor,line width=0.1pt},
  medline/.style={draw=black,line width=0.6pt},
  medlinep/.style={draw=black,line width=0.6pt,->},
  semiboldline/.style={draw=black,line width=1.2pt},
  semiboldlinep/.style={draw=black,line width=1.2pt,->},
  infoline/.style={draw=gray,line width=1.4pt},
  boldline/.style={draw=black,line width=2.0pt},
  boldlinep/.style={draw=black,line width=2.0pt,->},
  wire/.style={draw=black,line width=1.0pt},
  elelem/.style={draw=black,line width=1.5pt},
  subelem/.style={draw=black,dashed,line width=0.6pt}
}



\usepackage[english,ukraineb,russian]{babel}

\title{Tikz figures test}
\author{А.И.~Гуда}

\begin{document}

\begin{center}
\Large{Tikz figures test}
\end{center}

Проверка tikz рисунков.

\begin{figure}[h!]
\begin{center}
% vi:syntax=tex
\begin{tikzpicture}
  %\draw[hair,step=1.0em] (0,-3) grid (12.0,3.0);
  \bXStyleBloc{semiboldline,inner sep=2pt};
  \bXLineStyle{medline};
  % --- U
  \bXInput{U};
  \path (U.center) ++(0.9em,0.0em) coordinate (UxM);
  \fill (UxM) circle[radius=0.05];
  %\bXLinkName[0.5]{U}{$u(t)$};
  % --- M1
  \bXBlocL[3.0]{M1}{$\mathbf{M}_1$}{U};
  \bXLink[$u(t)$]{U}{M1}; %% node 'U-M1' is here
  \path (M1.east) ++(0.0,-1.0em) coordinate (Mxm1);
  % --- A1
  \bXBloc[10.0]{A1}{$A_{1}$}{M1};
  \path (A1.west) ++(0.0,-1.0em) coordinate (Axm1);
  \path (A1.west) ++(0.0,+1.0em) coordinate (Aqo1);
  \path (Aqo1) ++(-1.0em,0.0em) coordinate (Aqoi1) {};  % external input
  \draw[medlinep] (Aqoi1) -- (Aqo1);
  \fill (Aqoi1) circle[radius=0.05];
  \bXLink[$x_{1}(t)$]{Mxm1}{Axm1};
  \draw[infoline,<->] (A1.40) -- ++(3.0em,0.0em);
  \draw[medlinep] (A1.east) -- ++(3.0em,0.0em);
  \node[above right] at(A1.east) {$p_1(t)$};
  \draw[medlinep] (A1.east) -- ++(0.5em,0.0em) -- ++(0.0em,-2.0em) -| (M1.south);
  \draw[infoline,->] (A1.south) -- ++(0.0em,-0.8em);
  \path (A1.east)  ++(6.0em,-1.0em) coordinate (Pout);
  \draw[medlinep] (Pout) -- ++(2.0em,0.0em);
  \node[above right] at (Pout) {$p_{\mathrm{id}}(t)$};
  % --- M0
  \bXBranchy[-4]{UxM}{U0};
  \bXBloc[2.1]{M0}{$\mathbf{M}_0$}{U0};
  \path (M0.east) ++(0.0,-1.0em) coordinate (Mxm0);
  \bXLinkyx{UxM}{M0};
  % --- A0
  \bXBloc[10.0]{A0}{$A_{0}$}{M0};
  \path (A0.west) ++(0.0,-1.0em) coordinate (Axm0);
  \path (A0.west) ++(0.0,+1.0em) coordinate (Aqo0);
  \path (Aqo0) ++(-1.0em,0.0em)  coordinate (Aqoi0) {};  % external input
  \fill (Aqoi0) circle[radius=0.05];
  \draw[medlinep] (Aqoi0) -- (Aqo0);
  \bXLink[$x_{0}(t)$]{Mxm0}{Axm0};
  \path (A0.north east)        ++(3.0em,0.0em) coordinate (AAlt);
  \draw[infoline,<->] (A0.40) -- ++(3.0em,0.0em);
  \draw[medlinep] (A0.east) -- ++(3.0em,0.0em);
  \node[above right] at(A0.east) {$p_0(t)$};
  \draw[medlinep] (A0.east) -- ++(0.5em,0.0em) -- ++(0.0em,-2.0em) -| (M0.south);
  \draw[infoline,<->] (A0.south) -- (A1.north);
  % --- Obj
  \bXBranchy[-8]{UxM}{UO};
  \bXBloc[2.1]{Obj}{$\mathbf{O}$}{UO};
  \bXLinkyx{UxM}{Obj};
  \bXCompSum[3.0]{W}{Obj}{}{}{}{};
  \bXLink{Obj}{W};
  \draw[medline,<-] (W.north) -- ++(0.0em,1.0em) node[right] {$w(t)$};
  \bXBloc[2.5]{Qo}{$q$}{W};
  \bXLink[$x_o(t)$]{W}{Qo};
  \node[above right] at (Qo.east) {$q_o(t)$};
  %
  % --- Mn
  \bXBranchy[6]{UxM}{Un};
  \bXBloc[2.1]{Mn}{$\mathbf{M}_{n-1}$}{Un};
  \path (Mn.east) ++(0.0,-1.0em) coordinate (Mxmn);
  \bXLinkyx{UxM}{Mn};
  \draw[dotted,boldline] (M1.south) ++(0.0em,-1.0em) -- (Mn.north);
  % --- An
  \bXBloc[10.0]{An}{$A_{n-1}$}{Mn};
  \path (An.west) ++(0.0,-1.0em) coordinate (Axmn);
  \path (An.west) ++(0.0,+1.0em) coordinate (Aqon);
  \path (Aqon) ++(-1.0em,+0.0em) coordinate (Aqoin) {};  % external input
  \draw[medlinep] (Aqoin) -- (Aqon);
  \draw[medline] (Qo) -| (Aqoin);
  \bXLink[$x_{n-1}(t)$]{Mxmn}{Axmn};
  \path (An.south east) ++(6.0em,0.0em) coordinate (AArb);
  \draw[infoline,<->] (An.40) -- ++(3.0em,0.0em);
  \draw[medlinep] (An.east) -- ++(3.0em,0.0em);
  \node[above right] at(An.east) {$p_{n-1}$};
  \draw[medlinep] (An.east) -- ++(0.5em,0.0em) -- ++(0.0em,-2.0em) -| (Mn.south);
  \draw[infoline,->] (An.north) -- ++(0.0em,0.5em);
  %
  \draw[semiboldline] (AAlt) rectangle (AArb);
  %
  \draw[white,dotted,line width=3.0pt] (UxM) ++(0.0em,-1.0em) -- ++(0.0em,-4.0em);
  %
  \bXStyleBlocDefault;
  \bXDefaultLineStyle;
  %
  \TikzAddPadding
  %
\end{tikzpicture}

\end{center}
\caption{Нулевая картинка}
\label{atu:f:f0}
\end{figure}

И ещё.

\begin{figure}[htb!]
\begin{center}
% vi:syntax=tex
\begin{tikzpicture}
  \bXStyleBloc{semiboldline,inner sep=2pt};
  \bXLineStyle{medline};
  % --- U
  \bXInput{U};
  % --- M
  \bXBlocL[2.0]{M}{$\mathbf{M}_i$}{U};
  \bXLink[$u(t)$]{U}{M};
  % --- Q
  \bXBloc[3.5]{Q}{$q$}{M};
  \path (Q.east) ++(0.0,-1.0em) coordinate (Qqm);
  \path (Q.south west) ++(-0.3,-0.4) coordinate (BLKlb);
  \bXLink[$x_i(t)$]{M}{Q};
  % --- F
  \bXBloc[2.5]{F}{$F(q_o,q_{mi})$}{Q};
  \path (F.west) ++(0.0,-1.0em) coordinate (Fqm);
  \path (F.west) ++(0.0,+1.0em) coordinate (Fqo);
  \path (Fqo) ++(-1.6em,+2.8em) coordinate (Fqoi) {};  % external input
  \draw[medlinep] (Fqoi) |- (Fqo);
  \node[below right] at (Fqoi) {$q_o(t) \qquad A_i$};
  \bXLink[$q_i(t)$]{Qqm}{Fqm};
  % --- P
  \bXBloc[2]{P}{$P$}{F};
  \draw[infoline,<->] (P.north) -- +(0,0.8);
  \path (P.north east) ++(0.1,+0.4) node (BLKrt) {};
  \bXLink[$F_i(t)$]{F}{P};
  % -- output
  \bXOutput[2.8]{Po}{P};
  \bXLink[$p_i(t)$]{P}{Po};
  \bXOutput[1.0]{Por}{P};
  \fill(Por) circle[radius=0.05];
  \bXLineStyle{semiboldline};
  \bXReturn{Por}{M}{$p_i(t)$};
  % -- block
  \draw[subelem] (BLKlb) |- (BLKrt) |- (BLKlb);
  \bXStyleBlocDefault;
  \bXDefaultLineStyle;
  %
  \TikzAddPadding
  %
\end{tikzpicture}

\end{center}
\caption{Поисковый агент, настраивающий параметр одной модели}
\label{atu:f:agent1}
\end{figure}


\[
  \int\limits_{a}^{b} \varepsilon(t)
  =
  \sum\limits_{a_w=-1}^{n} \sin\left(
    \frac{ 1 + z_{a_w}^T + \aleph \hbar \Re - \nabla \mathbf{x} + \mathfrak{PZ} }
         {\eta \cdot \theta \gamma \varepsilon \omega}
  \right) \mathrm{d} x.
\]


\end{document}
