\clearpage
\phantomsection
\chapter*{Висновки}


У роботі вирішена науково-технічна проблема ідентифікації параметрів складних технічних
систем у режимі хаотичної динаміки
з метою забезпечення їх керованої поведінки. При цьому:

\begin{itemize}

  \item
    створено нові критерії ідентифікації, які, на відміну від тих, що
    існують, придатні для аналізу стану та динаміки
    хаотичних систем, що створить фізично зумовлене обґрунтування працездатності систем
    ідентифікації;

  \item
    створено новий клас систем ідентифікації у межах
    адаптивно-пошуковой парадигми,
    які за рахунок використання колективної динаміки
    ансамблю пошукових агентів забезпечують
    кращу швидкість ідентифікації без істотного впливу на похибку ідентифікації;

  \item
    проведена перевірка працездатності запропонованих методів
    на прикладах як відомих систем хаотичної динаміки,
    так і на декількох інших динамічних систем, які проявляють
    складну та хаотичну динаміку;

  \item
   визначено, що системи з сухим тертям з точки зору задачі ідентифікації
   при певних  умовах функціонування
   мають властивості, що поєднують їх з системами хаотичної динаміки;

 \item
  створено програмне забезпечення, придатне для моделювання як систем
  хаотичної динаміки, так і систем мультиагентної ідентифікації;

  \item
  проведено як комп'ютерне моделювання процесів ідентифікації систем
  хаотичної динаміки, так і фізичне моделювання таких, що підтверджує адекватність
  побудованих моделей.

\end{itemize}

% В целом, применение рассмотренных видов критериев, в совокупности
% с мультимодельными (мультиагентными) методами поисковой идентификации
% оказалось продуктивным и показало их
% работоспособность для рассмотренных хаотических, и в какой-то мере эквивалентным им
% сложных динамических систем.
%
% Для большинства из рассмотренных систем вид критерия
% был получен практически автоматически, из анализа полученных
% в результате моделирования зависимостей, имеющих прямое или косвенное отношение
% к энергетическим параметрам. Для системы с сухим трением,
% синтез критерия был основан на явных физических зависимостях.
% Созданные критерии позволяют реализовать процесс идентификации для случаев
% сложной или же хаотической динамики,
% когда полностью отсутствуют хоть какая-нибудь информация о
% трендовом поведении объекта, и невозможно получить аналитическую
% или же статистическую связь между параметрами и выходом системы.
%
% Одновременная настройка параметров нескольких моделей ансамблем поисковых агентов
% позволила максимально преодолеть противоречие между скоростью и точностью
% идентификации. При таком подходе время идентификации определяется
% временем реакции идентифицируемой системы на изменение параметра.
% Для систем со сложной, а тем более хаотической динамикой это время достигает
% значительных величин, по сравнению с характерными временами самой динамики системы.
%
%
% Полученные зависимости ошибок идентификации от параметров
% самой системы идентификации, с одной стороны, позволяют
% правильно настроить эти параметры для получения максимально быстрой
% и точной идентификации. С другой стороны, эти зависимости
% характеризуют связь свойств идентифицируемого объекта с
% требуемыми качествами самой системы идентификации.

