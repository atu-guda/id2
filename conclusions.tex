\chapter*{Выводы}

В целом, применение рассмотренных видов критериев, в совокупности
с мультимодельными (мультиагентными) методами поисковой идентификации
оказалось продуктивным и показало их
работоспособность для рассмотренных хаотических, и в какой-то мере эквивалентным им
сложных динамических систем.



Для большинства из рассмотренных систем вид критерия
был получен практически автоматически, из анализа полученных
в результате моделирования зависимостей, имеющих прямое или косвенное отношение
к энергетическим параметрам. Для системы с сухим трением,
синтез критерия был основан на явных физических зависимостях.
Созданные критерии позволяют реализовать процесс идентификации для случаев
сложной или же хаотической динамики,
когда полностью отсутствуют хоть какая-нибудь информация о
трендовом поведении объекта, и невозможно получить аналитическую
или же статистическую связь между параметрами и выходом системы.

Одновременная настройка параметров нескольких моделей ансамблем поисковых агентов
позволила максимально преодолеть противоречие между скоростью и точностью
идентификации. При таком подходе время идентификации определяется
временем реакции идентифицируемой системы на изменение параметра.
Для систем со сложной, а тем более хаотической динамикой это время достигает
значительных величин, по сравнению с характерными временами самой динамики системы.


Полученные зависимости ошибок идентификации от параметров
самой системы идентификации, с одной стороны, позволяют
правильно настроить эти параметры для получения максимально быстрой
и точной идентификации. С другой стороны, эти зависимости
характеризуют связь свойств идентифицируемого объекта с
требуемыми качествами самой системы идентификации.
