\documentclass[10pt,utf8]{beamer}
\setbeamersize{text margin left=15pt,text margin right=15pt}
\usepackage{cmap}
\usepackage[T1,T2A]{fontenc}
\usefonttheme[onlymath]{serif}
\usepackage{paratype}
\usepackage {latexsym}
%\usepackage {fancybox}
\usepackage {fouriernc}

\usepackage{caption}
\setbeamertemplate{caption}[numbered]
%\addtobeamertemplate{navigation symbols}{}{ \hspace{1em}  \usebeamerfont{footline}  {\normalsize \insertframenumber / \inserttotalframenumber}}
\addtobeamertemplate{navigation symbols}{}{ \hspace{1em}  \usebeamerfont{footline}  {\normalsize \color{black} \insertframenumber }}

\usepackage{tikz}
% \usetikzlibrary{tikzmark,calc}
\usepackage[english,russian]{babel}
\usepackage{bropd} % od, pd

\usetheme{Warsaw}

\DeclareMathOperator*{\sign}{sign}


\newcommand{\booknameUa}{Ансамблеві пошукові моделі і методи параметричної ідентифікації систем з хаотичною поведінкою}
\newcommand{\booknameRu}{Ансамблевые поисковые модели и методы параметрической идентификации систем с хаотическим поведением}
\newcommand{\booknameEn}{Ensemble search models and methods for parametric identification of systems with chaotic behavior}
\newcommand{\bookname}{\booknameRu}

\newcommand{\bookyear}{2018}
\newcommand{\dissauthorUa}{Гуда~А.І.}
\newcommand{\dissauthorRu}{Гуда~А.И.}
\newcommand{\dissauthorEn}{Guda~A.I.}
\newcommand{\dissauthorFullRu}{Гуда Антон Игоревич}
\newcommand{\dissauthorFullUa}{Гуда Антон Ігорович}
\newcommand{\dissauthorMain}{\dissauthorRu}
\newcommand{\dissauthorAref}{\dissauthorUa}
\newcommand{\dissauthorFullMain}{\dissauthorFullRu}
\newcommand{\dissauthorFullAref}{\dissauthorFullUa}

\newcommand{\dissSpecUa}{математичне    моделювання  та обчислювальні методи}
\newcommand{\dissSpecRu}{математическое моделирование и вычислительные методы}
\newcommand{\dissSpecEn}{Mathematical Modelling and Computational Methods}
\newcommand{\dissSpecMain}{\dissSpecRu}
\newcommand{\dissSpecAref}{\dissSpecUa}
\newcommand{\dissSpecId}{01.05.02}
\newcommand{\dissScopeRu}{технических наук}
\newcommand{\dissScopeUa}{техничних наук}
\newcommand{\dissScopeMain}{\dissScopeRu}
\newcommand{\dissScopeAref}{\dissScopeUa}
\newcommand{\UDC}{004: 681.5.015}
\newcommand{\dissRada}{Д.~08.084.01}
\newcommand{\dissSekrRadi}{Селівьорстова~Т.В.}
\newcommand{\institutionRu}{Национальная металлургическая академия Украины}
\newcommand{\institutionUa}{Національна  металургійна     академія України}
\newcommand{\institutionEn}{National Metallurgical academy of Ukraine}
\newcommand{\institutionMain}{\institutionRu}
\newcommand{\institutionAref}{\institutionUa}
\newcommand{\belongRu}{Министерство образования и науки Украины}
\newcommand{\belongUa}{Міністерство освіти і науки      України}
\newcommand{\belongEn}{Ministry of Education and Science of Ukraine}
\newcommand{\belongMain}{\belongRu}
\newcommand{\belongAref}{\belongUa}
\newcommand{\cityRu}{Днепр}
\newcommand{\cityUa}{Дніпро}
\newcommand{\cityEn}{Dnipro}
\newcommand{\cityMain}{\cityRu}
\newcommand{\cityAref}{\cityUa}
\newcommand{\superRu}{Михалёв Александр Ильич}
\newcommand{\superUa}{Михальов Олександр Ілліч}
\newcommand{\superMain}{\superRu}
\newcommand{\superAref}{\superUa}



\author{\dissauthorRu}

\title[Семинар -- 2017]{\booknameRu}

\begin{document}

\begin{frame}
  \frametitle{}
  \begin{center}
    {\Large \color{blue} \booknameRu}

    \vfill

    {\large \dissauthorMain}

    \vfill

    Научный консультант -- д.т.н., проф. \superRu

    \vfill

    Днепр -- 2017
  \end{center}
\end{frame}

% -----------------------------------------------------------------------

\begin{frame}
  \frametitle{Актуальность}
  %\framesubtitle{}

  Актуальность обусловлена следующими фактами:

  \begin{itemize}

    \item
      Нелинейные динамические системы, представленные в современных
      технологических процессах, природных явлениях, зачастую
      демонстрируют хаотическое поведение.
      Основная причина -- малые возмущения параметров или входных сигналов
      приводят к существенным изменениям выходного сигнала.

    \item
      Существующие методы идентификации или принципиально непригодны для
      работы с хаотическими системами, или же требуют выполнения
      достаточно жёстких условий.

    \item
      Существуют динамические системы, не обладающие строгими свойствами хаотичности,
      но имеющие с ними общие свойства с точки зрения идентификации.

    \item
      Свойства существующих методов идентификации нелинейных динамических систем
      ограничивают достижимое качество идентификации,
      особенно применительно к хаотическим системам.

  \end{itemize}


\end{frame}


% -----------------------------------------------------------------------

\begin{frame}
  \frametitle{Объект, предмет, методы}
  %\framesubtitle{}

\textbf{Объект исследования} ---
технические системы, которые в процессе функционирования
могут входить в хаотический режим.

\medskip

\textbf{Предмет исследования} ---
математические модели процессов и методы
адаптивно-поисковой идентификации технических систем с хаотической динамикой.

\medskip

\textbf{Методы исследования} ---
математический аппарат теории управления и идентификации,
динамического хаоса,
нечёткой логики,
теории информации,
вычислительные методы
\ldots

\end{frame}



% -----------------------------------------------------------------------

\begin{frame}
  \frametitle{Задачи исследования}
  %\framesubtitle{}

  \begin{itemize}

    \item
      Разработать новые критерии идентификации, которые, в отличие от существующих,
      при моделировании были бы пригодны для анализа состояния и динамики хаотических
      систем, что создаст основу для создания работоспособных систем идентификации.

    \item
      Развить существующие и разработать новые методы поиска, которые в полной мере
      использовали преимущества использования ансамбля поисковых агентов.

    \item
      Разработать новые и развить существующие и  методы настройки параметров системы
      идентификации, способные приспособиться к смене режимов работы системы.

    \item
      Разработать программное обеспечение, пригодное для моделирования как систем
      хаотической динамики, так и систем идентификации.

    \item
      Провести компьютерное моделирование процессов идентификации систем хаотической
      динамики и исследовать их работоспособность, возможности и характеристики.

  \end{itemize}

\end{frame}



% -----------------------------------------------------------------------

\begin{frame}
  \frametitle{Прототипы}
  %\framesubtitle{}

  Кто занимался и какие методы.


\end{frame}







% -------------------------- P2 ---------------------------------------------


% -----------------------------------------------------------------------

\begin{frame}
  \frametitle{Необходимость новых критериев}
  \framesubtitle{Влияние малых возмущений}


\end{frame}



% -----------------------------------------------------------------------

\begin{frame}
  \frametitle{Определения идентификации}
  %\framesubtitle{}

  Заде -- Райбман, \ldots


\end{frame}

% -----------------------------------------------------------------------


\begin{frame}
  \frametitle{Критерии идентификации: требования}
  %\framesubtitle{}


\end{frame}



% -----------------------------------------------------------------------

\begin{frame}
  \frametitle{Критерии идентификации: энергетические основы}
  %\framesubtitle{}


\end{frame}



% -----------------------------------------------------------------------

\begin{frame}
  \frametitle{Критерии идентификации: предлагаемый набор величин}
  %\framesubtitle{}


\end{frame}



% -----------------------------------------------------------------------

\begin{frame}
  \frametitle{Отличие задачи идентификации}
  %\framesubtitle{}


\end{frame}



% -----------------------------------------------------------------------

\begin{frame}
  \frametitle{Динамические характеристики критериев}
  %\framesubtitle{}


\end{frame}



% -----------------------------------------------------------------------

\begin{frame}
  \frametitle{Выводы?}
  %\framesubtitle{}


\end{frame}



% --------------------------- P3 --------------------------------------------



\begin{frame}
  \frametitle{Безразмерный вид критерия и функции качества}
  %\framesubtitle{}


\end{frame}

% -----------------------------------------------------------------------


\begin{frame}
  \frametitle{Элементы системы идентификации: определения}
  %\framesubtitle{}


\end{frame}



% -----------------------------------------------------------------------

\begin{frame}
  \frametitle{Плоская структура мультиагентной системы}
  %\framesubtitle{}


\end{frame}



% -----------------------------------------------------------------------

\begin{frame}
  \frametitle{Виды мультиагентных систем идентификации}
  %\framesubtitle{}


\end{frame}



% -----------------------------------------------------------------------

\begin{frame}
  \frametitle{Виды и задачи поисковых агентов}
  %\framesubtitle{}


\end{frame}



% -----------------------------------------------------------------------

\begin{frame}
  \frametitle{Методы агентов, использующих значения критерия}
  %\framesubtitle{}


\end{frame}


% -----------------------------------------------------------------------

\begin{frame}
  \frametitle{Методы агентов, использующих значения функции качества}
  %\framesubtitle{}


\end{frame}




% -----------------------------------------------------------------------

\begin{frame}
  \frametitle{Методы, параметры и алгоритмы работы поисковых координаторов }
  %\framesubtitle{}


\end{frame}



% -----------------------------------------------------------------------

\begin{frame}
  \frametitle{Ошибки и качество идентификации}
  %\framesubtitle{}


\end{frame}





% -----------------------------------------------------------------------

\begin{frame}
  \frametitle{Классификация и обозначения систем идентификации}
  %\framesubtitle{}


\end{frame}





% ------------------------------- P4 ----------------------------------------

\begin{frame}
  \frametitle{Система моделирования ``qontrol''}
  %\framesubtitle{}

  Задачи системы


\end{frame}



% -----------------------------------------------------------------------

\begin{frame}
  \frametitle{Основные особенности программы ``qontrol''}
  %\framesubtitle{}


\end{frame}



% -----------------------------------------------------------------------

\begin{frame}
  \frametitle{Интерфейс программы ``qontrol''}
  %\framesubtitle{}


\end{frame}



% -----------------------------------------------------------------------

\begin{frame}
  \frametitle{Программно-аппаратная реализация взаимодействия с физическими объектами}
  %\framesubtitle{}


\end{frame}



% -----------------------------------------------------------------------

\begin{frame}
  \frametitle{Программно-аппаратная реализация }
  \framesubtitle{Оборудование}


\end{frame}


% -----------------------------------------------------------------------

\begin{frame}
  \frametitle{Программно-аппаратная реализация }
  \framesubtitle{Интерфейс}


\end{frame}



% ----------------------------- P5 ------------------------------------------

% --- lor ---
\begin{frame}
  \frametitle{Система Лоренца}
  \framesubtitle{Определение}


\end{frame}



% -----------------------------------------------------------------------

\begin{frame}
  \frametitle{Система Лоренца}
  \frametitle{Свойства}


\end{frame}



% -----------------------------------------------------------------------

\begin{frame}
  \frametitle{Система Лоренца}
  \framesubtitle{Анализ критериев}


\end{frame}



% -----------------------------------------------------------------------

\begin{frame}
  \frametitle{Система Лоренца}
  \framesubtitle{Тестовая задача}


\end{frame}



% -----------------------------------------------------------------------

\begin{frame}
  \frametitle{Система Лоренца}
  \framesubtitle{Идентификация методом  Fblvd1.2.$q_{x2}$ }


\end{frame}



% -----------------------------------------------------------------------

\begin{frame}
  \frametitle{Система Лоренца}
  \framesubtitle{Идентификация методом qAuv5.3r.$q_{x2}$}


\end{frame}



% -----------------------------------------------------------------------

\begin{frame}
  \frametitle{Система Лоренца}
  \framesubtitle{Идентификация методом FAlv5.3z.$q_{x2}$}


\end{frame}



% -----------------------------------------------------------------------

\begin{frame}
  \frametitle{Система Лоренца}
  \framesubtitle{Зависимости $\overline{e}(a_q)$}


\end{frame}



% -----------------------------------------------------------------------

\begin{frame}
  \frametitle{Система Лоренца}
  \framesubtitle{Зависимости $\overline{e}(q_\gamma)$}


\end{frame}

% -----------------------------------------------------------------------

\begin{frame}
  \frametitle{Система Лоренца}
  \framesubtitle{Зависимости $\overline{e}(v_f)$}


\end{frame}


% -----------------------------------------------------------------------

\begin{frame}
  \frametitle{Система Лоренца}
  \framesubtitle{Влияние вида функции качества на процесс идентификации}


\end{frame}


% -----------------------------------------------------------------------

\begin{frame}
  \frametitle{Система Лоренца}
  \framesubtitle{Влияние количества поисковых агентов на процесс идентификации}


\end{frame}



% -----------------------------------------------------------------------

% ---- spr_a ----
\begin{frame}
  \frametitle{Система Sprott A}
  \framesubtitle{Определение}


\end{frame}

% -----------------------------------------------------------------------

\begin{frame}
  \frametitle{Система Sprott A}
  \framesubtitle{Свойства}


\end{frame}


% -----------------------------------------------------------------------

\begin{frame}
  \frametitle{Система Sprott A}
  \framesubtitle{Идентификация методом qAuv5.3r.$q_{x2}$}


\end{frame}


% -----------------------------------------------------------------------

\begin{frame}
  \frametitle{Система Sprott A}
  \framesubtitle{Идентификация методом FAlv5.3z.$q_{x2}$}


\end{frame}



% -----------------------------------------------------------------------

\begin{frame}
  \frametitle{Система Sprott A}
  \framesubtitle{Зависимости $\overline{e}(a_q)$}


\end{frame}


% -----------------------------------------------------------------------

\begin{frame}
  \frametitle{Система Sprott A}
  \framesubtitle{Зависимости $\overline{e}(q_\gamma)$}


\end{frame}

% --- chua ---
\begin{frame}
  \frametitle{Система Чуа}
  \framesubtitle{Определение}


\end{frame}



% -----------------------------------------------------------------------

\begin{frame}
  \frametitle{Система Чуа}
  \frametitle{Свойства}


\end{frame}



% -----------------------------------------------------------------------

\begin{frame}
  \frametitle{Система Чуа}
  \framesubtitle{Анализ критериев}


\end{frame}



% -----------------------------------------------------------------------

\begin{frame}
  \frametitle{Система Чуа}
  \framesubtitle{Тестовая задача}


\end{frame}



% -----------------------------------------------------------------------

\begin{frame}
  \frametitle{Система Чуа}
  \framesubtitle{Идентификация методом  Fblvd1.2.$q_{x2}$ }


\end{frame}



% -----------------------------------------------------------------------

\begin{frame}
  \frametitle{Система Чуа}
  \framesubtitle{Идентификация методом qAuv5.3r.$q_{x2}$}


\end{frame}



% -----------------------------------------------------------------------

\begin{frame}
  \frametitle{Система Чуа}
  \framesubtitle{Идентификация методом FAlv5.3z.$q_{x2}$}


\end{frame}



% -----------------------------------------------------------------------

\begin{frame}
  \frametitle{Система Чуа}
  \framesubtitle{Зависимости $\overline{e}(a_q)$}


\end{frame}



% -----------------------------------------------------------------------

\begin{frame}
  \frametitle{Система Чуа}
  \framesubtitle{Зависимости $\overline{e}(q_\gamma)$}


\end{frame}

% -----------------------------------------------------------------------

\begin{frame}
  \frametitle{Система Чуа}
  \framesubtitle{Зависимости $\overline{e}(v_f)$}


\end{frame}


% -- fric ---

% -----------------------------------------------------------------------

\begin{frame}
  \frametitle{Система с сухим трением}
  \framesubtitle{Определение}

\end{frame}


% -----------------------------------------------------------------------

\begin{frame}
  \frametitle{Система с сухим трением}
  \framesubtitle{Свойства}

\end{frame}

% -----------------------------------------------------------------------

\begin{frame}
  \frametitle{Система с сухим трением}
  \framesubtitle{Определение критерия}

\end{frame}


% -----------------------------------------------------------------------

\begin{frame}
  \frametitle{Система с сухим трением}
  \framesubtitle{Процесс идентификации}

\end{frame}





% -----------------------------------------------------------------------

\begin{frame}
  \frametitle{Другие рассмотренные хаотические системы}
  %\framesubtitle{}

  Были расмотрены системы Дуффинга, Рёсслера, Ван-дер-Поля,
  еолебательная система с нечувствительностью в возвращающей силе.

  Для всех этих систем на основе предложенных критериев были
  созданы системы идентификации, показана их работоспособность
  и определены зависимости ошибок идентификации от параметров поисковой системы.


\end{frame}




% ----------------------------- P6 ------------------------------------------

\begin{frame}
  \frametitle{Система Колпитца}
  %\framesubtitle{}


\end{frame}



% -----------------------------------------------------------------------

\begin{frame}
  \frametitle{Система Колпитца}
  %\framesubtitle{}


\end{frame}



% -----------------------------------------------------------------------

\begin{frame}
  \frametitle{Система Колпитца}
  %\framesubtitle{}


\end{frame}



% -----------------------------------------------------------------------

\begin{frame}
  \frametitle{Система Колпитца}
  %\framesubtitle{}


\end{frame}



% -----------------------------------------------------------------------

\begin{frame}
  \frametitle{Система Колпитца}
  %\framesubtitle{}


\end{frame}



% ------------------ P7 -----------------------------------------------------

\begin{frame}
  \frametitle{Система связанных релаксационных генераторов}
  %\framesubtitle{}


\end{frame}



% -----------------------------------------------------------------------

\begin{frame}
  \frametitle{Система связанных релаксационных генераторов}
  %\framesubtitle{}


\end{frame}



% -----------------------------------------------------------------------

\begin{frame}
  \frametitle{Система связанных релаксационных генераторов}
  %\framesubtitle{}


\end{frame}



% -----------------------------------------------------------------------

\begin{frame}
  \frametitle{Система связанных релаксационных генераторов}
  %\framesubtitle{}


\end{frame}



% -----------------------------------------------------------------------

\begin{frame}
  \frametitle{Система связанных релаксационных генераторов}
  %\framesubtitle{}


\end{frame}



% ----------------------------- FINAL ------------------------------------------

\begin{frame}
  \frametitle{Научная новизна}
  %\framesubtitle{}

  \noindent
  Впервые:

  \begin{itemize}

    \item
      Предложены \textbf{критерии} идентификации нелинейных динамических систем, которые, в
      отличие от существующих, позволяют оценить их состояние и хаотическую динамику,
      и дают основания для создания эффективных алгоритмов настройки параметров
      моделей систем идентификации.

    \item
      Созданы \textbf{методы} адаптивно-поисковой идентификации на основании
      адаптивно-поисковой парадигмы с использованием \textbf{ансамбля взаимодействующих поисковых агентов},
      и, в отличие от методов, которые
      используют одну модель или пару моделей, значительно повышают скорость поиска и
      способны за минимальное время перестреливаться при резкой смене параметра, а в
      отличие от роевых алгоритмов, новые методы требуют значительно меньшего
      количества моделей и обеспечивают определенные гарантии поиска.

    \item
      Создана новая \textbf{классификация} систем идентификации динамических систем, которая
      как включает в себя как существующие методы, так и позволяет создавать новые методы
      идентификации за счет комбинирования их составных частей.

  \end{itemize}


\end{frame}




% -----------------------------------------------------------------------

\begin{frame}
  \frametitle{Научная новизна (2)}
  %\framesubtitle{}

  \begin{itemize}

    \item
      Установлено, что системы с \textbf{сухим трением} с точки зрения задачи идентификации при
      определенных условиях функционирования обладают свойствами, которые объединяют их с
      системами хаотической динамики, а именно:
      существенная зависимость от начальных условий и вид аттракторов, что
      также требует использования новых методов идентификации.

    \item
      Предложена модель системы хаотической динамики системы
      \textbf{связанных релаксационных генераторов},
      которая отличаешься от существующих отсутствием индуктивных
      компонентов, работоспособностью при малых напряжениях и возможностью управления
      частотным диапазоном в широком диапазоне, что способствует процессу анализа
      хаотической динамики физического объекта, проверки адекватности математической
      модели и свойств системы идентификации на этой системе.

  \end{itemize}

\end{frame}



% -----------------------------------------------------------------------

\begin{frame}
  \frametitle{Научная новизна (3)}
  %\framesubtitle{}

\noindent
Получили дальнейшее развитие:

  \begin{itemize}

    \item
      \textbf{Методы оценки качества идентификации},
      которые в отличие от существующих,
      учитывают использование множества агентов.

    \item
      \textbf{Подходы к адаптации параметров}
      систем адаптивно-поисковой идентификации,
      которые пригодны использовать информацию от ансамбля синергированих моделей, и
      корректировать глобальные параметры поиска.

    \item
      Модель \textbf{генератора Копитца}, учитывающей большее количество нелинейных эффектов,
      что обеспечивает более адекватные результаты процесса идентификации ее
      параметров новыми методами.

  \end{itemize}


\end{frame}



% -----------------------------------------------------------------------

\begin{frame}
  \frametitle{Практическая ценность}
  %\framesubtitle{}

  \begin{itemize}

    \item
      Разработанные методы идентификации были использованы при проектировании,
      создании, настройке параметров стенда исследования вибрационного и
      акустического воздействия. Анализ результатов данных по этому стенду дал
      возможность указать нужные нелинейные свойства системы, и диапазон параметров,
      в совокупности обеспечивают широкополосный спектр колебаний.

    \item
      Созданное программное среда для моделирования нелинейных динамических систем
      используется при проведении практических работ по дисциплинам
      ``Моделирование систем'', ``Современные системы управления''
      на кафедре информационных технологий
      и систем Национальной металлургической академии Украины.

  \end{itemize}


\end{frame}



% -----------------------------------------------------------------------

\begin{frame}
  \frametitle{Печатные работы и апробация}
  %\framesubtitle{}

{\scriptsize
По теме дисертации опубликовано
49 печатных работ,
из них
24 входят в международные наукометрические базы,
13 статей опубликовано в материалах конференций.

Основные положения дисертационной работи докладывались на
научно-практических коференциях:
``Інформатика та системні науки'' (ІСН-2011) Полтава--2011,
``Интеллектуальные системы принятия решений и проблемы вычислительного интеллекта'' (ISDMCI) Херсон--2011,
``Информационные технологии в управлении сложными системами'' Днепропетровск--2011,
``Автоматизация: проблемы, идеи, решения'' Севастополь-2011,
``Интеллектуальные системы принятия решений и проблемы вычислительного интеллекта'' (ISDMCI) Херсон--2012,
``Автоматизация: проблемы, идеи, решения'' Севастополь-2012,
``Интеллектуальные системы принятия решений и проблемы вычислительного интеллекта'' (ISDMCI) Херсон--2013,
``Автоматизация: проблемы, идеи, решения'' Севастополь-2013,
``Интеллектуальные системы принятия решений и проблемы вычислительного интеллекта'' (ISDMCI) Херсон--2014,
``Интеллектуальные системы принятия решений и проблемы вычислительного интеллекта'' (ISDMCI) Херсон--2015,
``Computer Sciences and Information Technologies'' (CSIT) Lviv--2015 (Scopus),
``Интеллектуальные системы принятия решений и проблемы вычислительного интеллекта'' (ISDMCI) Херсон--2016,
``Data Stream Mining and Processing'' DSMP Lviv-2016 (Scopus,Web of Science).
}

\end{frame}




% -----------------------------------------------------------------------

\begin{frame}
  \frametitle{Выводы 1}
  %\framesubtitle{}

  \begin{itemize}

    \item
      Созданы новые критерии идентификации,
      пригодные для анализа состояния и динамики хаотических систем, что создаёт
      физическое  обоснование для создания работоспособных систем идентификации.

    \item
      Создан новый класс систем идентификации в рамках адаптивно-поисковом парадигмы,
      которые за счет использования коллективной динамики ансамбля поисковых
      агентов обеспечивают лучшее качество идентифицации.

    \item
      Создано программное обеспечение, пригодное для моделирования как систем
      хаотической динамики, так и систем мультиагентной идентификации.

    \item
      Проведено компьютерное моделирование процессов идентификации систем хаотической
      динамики, подтверждено их работоспособность.

  \end{itemize}


\end{frame}



% -----------------------------------------------------------------------

\begin{frame}
  \frametitle{Выводы 2}
  %\framesubtitle{}


\end{frame}



\end{document}

