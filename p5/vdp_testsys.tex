
\FloatBarrier

\section{Система Ван-дер-Поля} %  % {{{1 _VDP_
\label{atu:sect:vdp}

\LinkRef{
  vdp: ASAU-16, 17(alt), ITMM-2011
}

\subsection{Определение системы и анализ её динамики} %  % {{{2 _vdp_task

% TODO: dyn_sys_chaos, chaos_in_phase_dyn_VDP_modul_dobr.pdf buffer_phemon_vdp

Модель нелинейной автоколебательной системы Ван-Дер-Поля~(\ref{atu:eq:vdp})
широко используется при исследовании динамики
колебательных систем, в которых происходит восполнение
энергии системы из внешнего источника~\cite{anisch_nonlin_eff,magni_theory_dyn_chaos,atu_asau16}:
%
\begin{equation}
 \ddot{x} - \varepsilon (1-x^2)  \dot{x} + \Omega_0^2 x  = u(t) .
\label{atu:eq:vdp}
\end{equation}
\noindent
где
\(x(t)\) -- координата колебаний,
\( \varepsilon \) -- параметр, определяющий получение
  энергии системой,
\( \Omega_0 \) -- собственная частота при \( \varepsilon = 0 \),
\(u(t)\) -- внешнее возмущающее воздействие.

При \( \varepsilon > 0 \) и \( u(t) = 0 \)
рассматриваемая система реализует режим автоколебаний
с постоянной частотой, зависящей от \( \varepsilon \):
%
\begin{equation}
\Omega \approx \Omega_0 \sqrt{ 1 - \left( \frac{\varepsilon}{2 \Omega_0} \right)^2 }.
\label{atu:eq:vdp_Omega}
\end{equation}

При этом амплитуда колебаний остаётся практически постоянной,
а сама форма колебаний принимает всё более нелинейный характер.

Под воздействием входного гармонического сигнала
\( u(t) = U_{in} \sin ( \omega_{in} t ) \)
система~(\ref{atu:eq:vdp}) может проявлять регулярную, сложно-периодическую
и хаотическую динамику.

Идентифицируемый параметр \( \varepsilon \)
характеризует поступление энергии в систему.
При $x \approx 0$ членом с $x^2$ можно пренебречь,
и нелинейность вырождается в ``отрицательное трение''.
При больших амплитудах колебаний член с $x^2$
начинает преобладать, там самым ограничивая рост амплитуды.

При анализе численного моделирования системы~(\ref{atu:eq:vdp})
следует осторожно относится к выводам о типе динамики,
которые можно сделать, исходя из фазового или расширенного фазового портрета.
Например, на рис.~\ref{atu:f:vdp_phase_f_reg},
при условиях
$ \varepsilon=1.50$, $U_{in}=0.3$, $\omega_{in}=0.27$
на левом графике
фазовая траектория получается плотной,
однако спектр свидетельствует о регулярной динамике
с ограниченным набором частот.

\begin{figure}[ht!]
\begin{center}
  \includegraphics[width=0.49\textwidth]{p/cha/vdp/vdp_0-p_ph2d_1x50_0x30_0x27.png}
  \hfill
  \includegraphics[width=0.49\textwidth]{p/cha/vdp/vdp_fft-p_f_1x50_0x30_0x27.png}
\end{center}
  \caption{Расширенный фазовый портрет и спектр системы Ван-дер-Поля (\ref{atu:eq:vdp}) в режиме регулярных колебаний}
\label{atu:f:vdp_phase_f_reg}
\end{figure}

Непосредственно хаотические колебания этой системы часто характеризуются
менее выраженной плотностью аттрактора. Тем не менее,
участки сложного спектра (рис.~\ref{atu:f:vdp_phase_f_chaos})
свидетельствуют в пользу хаоса.
Данная иллюстрация соответствует условиям
$ \varepsilon=2.65$, $U_{in}=1.2$, $\omega_{in}=0.27$.
При этом малые изменения параметров, например
снижение величины $\varepsilon$ до $2.5$
приводит к реализации режима простых регулярных колебаний.

\begin{figure}[ht!]
\begin{center}
  \includegraphics[width=0.49\textwidth]{p/cha/vdp/vdp_0-p_ph2d_2x65_1x20_0x27.png}
  \hfill
  \includegraphics[width=0.49\textwidth]{p/cha/vdp/vdp_fft-p_f_2x65_1x20_0x27.png}
\end{center}
  \caption{Расширенный фазовый портрет и спектр системы Ван-дер-Поля (\ref{atu:eq:vdp}) в режиме хаотических колебаний}
\label{atu:f:vdp_phase_f_chaos}
\end{figure}

Анализ спектра системы Ван-дер-Поля, как и многих других систем
динамического хаоса, следует проводить, учитывая
спектральное разрешение.
Например, на  рис.~\ref{atu:f:vdp_phase_f_complex}
представлены результаты моделирования системы при
$ \varepsilon=4.8$, $U_{in}=0.7$, $\omega_{in}=0.7$.
В спектре системы наблюдаются участки с очень близко
расположенными пиками. При недостаточной разрешающей способности
этот участок будет принят за зону сплошного спектра.
Это может привести к выводу о хаотичности системы,
тем более, если учесть вид аттрактора.

\begin{figure}[ht!]
\begin{center}
  \includegraphics[width=0.49\textwidth]{p/cha/vdp/vdp_0-p_ph2d_4x80_0x70_0x70.png}
  \hfill
  \includegraphics[width=0.49\textwidth]{p/cha/vdp/vdp_fft-p_f_4x80_0x70_0x70.png}
\end{center}
  \caption{Расширенный фазовый портрет и спектр системы Ван-дер-Поля (\ref{atu:eq:vdp}) в режиме сложных регулярных колебаний}
\label{atu:f:vdp_phase_f_complex}
\end{figure}


% \begin{figure}[htb!]
% \centerline{\includegraphics[width=0.5\textwidth]{p/cha/vdp_phase.pdf} }
% \caption{Фазовый портрет системы Ван-дер-Поля (\ref{atu:eq:vdp})}
% \label{atu:f:vdp_phase}
% \end{figure}

При анализе физических систем нет возможности произвольным образом
задавать время измерения для получения спектра с требуемым разрешением.
Более того, с учётом ограниченной точности измерений нет возможности
точно определить показатели Ляпунова. Таким образом,
вполне возможно существование систем, неотличимых от хаотических
при проведении измерений, но не являющимися хаотическими в строгом понимании.
Тем не менее, с точки зрения задачи идентификации,
этот случай практически неотличим от реального хаоса, и требует
применения соответствующих методов и критериев.

% }}}2


\subsection{Анализ и выбор критериев}  % {{{2

В отличие от систем Лоренца, Рёсслера и им подобных,
у системы Ван-дер-Поля есть только одна наблюдаемая величина: $x$.
Это сразу сильно ограничивает круг критериев, требующих анализа.

На первый взгляд, если доступен сигнал $x(t)$,
то можно вычислить сигналы $\dot{x}(t)$ и $\ddot{x}(t)$.
После этого, подставив получившиеся зависимости в (\ref{atu:eq:vdp}),
можно получить значение идентифицируемого параметра.
На самом деле, это возможно только в случае практически полного отсутствия помех.
Даже небольшой  (0.001\%) уровень ошибок измерения
при вычислении производных, особенно второй,
приводит к совершенно произвольным результатам.
Таким образом, и для этой системы без применения
интегральных критериев нет возможности создать работоспособную систему идентификации.

Критерии $q_{x^2}$, $q_{rx}$ и $q_{|x|}$ отличаются
для этой задачи непринципиально.
Несмотря на то, что эти величины в данном конкретном случае не
отображают какой-либо закон сохранения, их применение может быть оправданным~\cite{atu_asau17}.
При росте параметра $\varepsilon$ система всё больше проявляет нелинейные свойства.
Это выражается в том, что при одинаковой амплитуде сигнала
зависимость $x(t)$ большую часть времени проводит вблизи
амплитудных значений, что приводит к росту этих критериев.
Для определённости в дальнейшем будем использовать $q_{x^2}$.

Также имеет смысл рассмотреть достаточно очевидный для
колебательных систем критерий $q_T$,
заключающийся в измерении ``периода''~\cite{atu_asau16}.
Очевидно, что понятие периода для хаотических систем неприменимо.
Однако, можно принять за текущее значение периода
интервал времени между двумя последовательными срабатываниями
правильно настроенного триггера Шмидта, на вход которого подаётся сигнал
$x(t)$. Этот подход используется при построении входных
цепей частотомеров. При этом автоматически происходит усреднение
на интервалах порядка этого ``периода''. Однако,
для систем идентификации такое усреднение скорее всего будет
недостаточным, особенно в хаотических и сложно-периодических
режимах. Следовательно, имеет смысл использовать дополнительное
усреднение, например~(\ref{atu:eq:qlin}).

Рассмотрим характерные зависимости рассматриваемых критериев,
полученные в результате моделирования динамики системы (\ref{atu:eq:vdp}).
На  рис.~\ref{atu:f:vdp_q1} представлен вид этих зависимостей
для различных условий.


\begin{figure}[ht!]
\begin{center}
  \includegraphics[width=0.49\textwidth]{p/cha/vdp/vdp_q-p_q_0x30_0x27.png}
  \hfill
  \includegraphics[width=0.49\textwidth]{p/cha/vdp/vdp_q-p_q_1x20_0x6283185.png}
\end{center}
  \caption{Зависимости $q_{x^2}(\varepsilon)$ и $q_T(\varepsilon) $ системы Ван-дер-Поля}
\label{atu:f:vdp_q1}
\end{figure}

Левый график соответствует такому набору параметров,
при котором наблюдается периодическое и сложно-периодическое движение.
При этом оба критерия подходят для синтеза системы идентификации, а
критерий $q_T$ проявляет меньшую нелинейность.
Правый график отображает случай, когда изменение параметра $\varepsilon$
приводит к существенному изменению поведения.
При $\varepsilon < 7 $ происходит ``захват частоты'',
то есть возмущающий сигнал $u(t)$ ``навязывает'' свою частоту системе,
что приводит к простой динамике и отсутствию зависимости $q_T$ от $\varepsilon$.
Идентификация на данном участке невозможна, ввиду
отсутствия влияния параметра на ``период''.
Правая часть этого графика отображает постоянные переходы от
сложных колебаний к хаосу и обратно. На зависимости $q_T(\varepsilon)$
появляются изломы, что потенциально должно привести к снижению
точности идентификации на фоне общей работоспособности.
График зависимости $q_{x^2}(\varepsilon)$ имеет экстремальный характер,
что позволяет использовать этот критерий только на тех диапазонах $\varepsilon$,
на которых наблюдается монотонность.

Рассмотрим зависимости $q_T(\varepsilon,U_{in})$ и $q_{x^2}(\varepsilon,U_{in})$
для различных значений $\omega_{in}$.
На  рис.~\ref{atu:f:vdp_q2_027}
представлены эти зависимости при $\omega_{in}=0.27$.


\begin{figure}[ht!]
\begin{center}
  \includegraphics[width=0.49\textwidth]{p/cha/vdp/vdp_q_2d-p_qT_ome_0x27.png}
  \hfill
  \includegraphics[width=0.49\textwidth]{p/cha/vdp/vdp_q_2d-p_qx2_ome_0x27.png}
\end{center}
  \caption{Зависимости $q_T(\varepsilon,U_{in})$ и $q_{x^2}(\varepsilon,U_{in})$  при $\omega_{in}=0.27$}
\label{atu:f:vdp_q2_027}
\end{figure}

При высоких значениях $U_{in}$ происходит захват частоты,
график зависимости $q_T(\varepsilon,U_{in})$ в этой области
представляет собой плоскую поверхность, что свидетельствует
о неприменимости критерия в этих условиях. При малых значения $U_{in}$,
наоборот, наблюдается достаточно близкая к линейной зависимость.
В промежутке между двумя этими областями возможности
идентификации с использованием данного критерия существуют, но ограничены.
Критерий $q_{x^2}$ при этих же условиях имеет несколько другие ограничения.
Идентификация затруднена в области ``оврага'', который соответствует
переходному режиму на предыдущем графике.
Как в области низких, так и в области высоких значений
$U_{in}$ идентификация возможна.

На  рис.~\ref{atu:f:vdp_q2_120}
представлены поверхности критериев при $\omega_{in}=1.2$.

\begin{figure}[ht!]
\begin{center}
  \includegraphics[width=0.49\textwidth]{p/cha/vdp/vdp_q_2d-p_qT_ome_1x20.png}
  \hfill
  \includegraphics[width=0.49\textwidth]{p/cha/vdp/vdp_q_2d-p_qx2_ome_1x20.png}
\end{center}
  \caption{Зависимости $q_T(\varepsilon,U_{in})$ и $q_{x^2}(\varepsilon,U_{in})$  при $\omega_{in}=1.2$}
\label{atu:f:vdp_q2_120}
\end{figure}

На зависимости $q_T(\varepsilon,U_{in})$ в этом случае также виден
плоский участок, соответствующий режиму захвата частоты.
Но большая часть этой зависимости свидетельствует о том, что
идентификация возможна, и при этом скорее всего
будет наблюдаться увеличение ошибки идентификации из-за неравномерности графика.
Напротив, зависимость $q_T(\varepsilon,U_{in})$ проявляет
мультимодальный характер, что позволяет проводить идентификацию только в узких диапазонах.

Таким образом, установлено, что ни один из рассматриваемых критериев не
обеспечивает возможность идентификации при любых значениях параметров.
Однако, наибольший диапазон работоспособности продемонстрировал критерий $q_{T}$.
Он и будет использован в дальнейших исследованиях. При этом при постановке задачи
будем избегать режимов, на которых происходит захват частоты.









% }}}2

\subsection{Тестовая задача идентификации для системы Ван-дер-Поля}  % {{{2

% }}}2


\subsection{Влияние параметров системы идентификации на ошибку идентификации для системы Ван-дер-Поля}  % {{{2


Зависимости здесь

% }}}2

\subsection{Выводы}  % {{{2

Выводы

% }}}2



% }}}1

% vim: fdm=marker foldlevel=1 foldignore="%#" fdc=4 ft=tex
