
\FloatBarrier
\section{Система Sprott A}  % {{{1
\label{atu:sect:spr_a}

\LinkRef{
  spr\_a: MKMM-2016
}

\subsection{Определение системы и анализ её динамики} %  % {{{2 _SPR_task

В своих работах J.C.~Sprott рассмотрел целое семейство динамических
систем, реализующих хаотическое поведение, обозначив их латинскими буквами
от ``A'' до ``S''~\cite{sprott_212,sprott_strange_attr}. Особое место среди них
занимает система, обозначаемая как ``Sprott A''. Отличительной особенностью
этой системы является отсутствие положений равновесия, что делает
невозможным применение многих известных методов анализа, основанных на
каком-либо разложении в окрестностях точек равновесия. Соответствующая ей
система уравнений имеет вид:
%
\begin{equation}
  \begin{cases}
    \dot{x} =  y, \\
    \dot{y} = -x + yz, \\
    \dot{z} =  1 - y^2.
  \end{cases}
  \label{atu:eq:spr_a_orig}
\end{equation}


В исходном виде система (\ref{atu:eq:spr_a_orig}) имеет фиксированные значения параметров.
Не изменяя структуры системы, можно ввести 5 параметров, влияющие на её динамку.
Поскольку целью данной работы является определение возможности применения
различных критериев идентификации параметров
хаотических объектов, а также свойств методов идентификации, то для данной системы
рассмотрим только один параметр -- $c_{x_y} $. Система принимает следующий вид:
%
\begin{equation}
  \begin{cases}
    \dot{x} =  c_{x_y} y, \\
    \dot{y} = -x + yz, \\
    \dot{z} =  1 - y^2.
  \end{cases}
  \label{atu:eq:spr_a}
\end{equation}

В таком виде система, при изменении $c_{x_y} $
в достаточно широком диапазоне может демонстрировать как
сложно-периодическое (рис.~\ref{atu:f:spr_a_p_0372}), так и преимущественно, хаотическое
поведение (рис.~\ref{atu:f:spr_a_p_0610}).

\begin{figure}[htb!]
\centerline{
  \includegraphics[width=0.49\textwidth]{p/cha/spr_a/sprott_a-p_xyz_cx_y=0x372.png}
  \includegraphics[width=0.49\textwidth]{p/cha/spr_a/sprott_a_f-p_f_cx_y=0x372.png}
}
\caption{Аттрактор и спектр системы (\ref{atu:eq:spr_a}) при $ c_{x_y} =0.372 $.
  Сложно-периодический режим.
}
\label{atu:f:spr_a_p_0372}
\end{figure}

При этом, в диапазоне $c_{x_y} \in [0.1 ; 0.7] $
наблюдаются перестройки структуры аттрактора, а при относительно больших
значениях данного параметра аттрактор представляет собой полый тор.

\begin{figure}[htb!]
\centerline{
  \includegraphics[width=0.49\textwidth]{p/cha/spr_a/sprott_a-p_xyz_cx_y=0x610.png}
  \includegraphics[width=0.49\textwidth]{p/cha/spr_a/sprott_a_f-p_f_cx_y=0x610.png}
}
\caption{Аттрактор и спектр системы (\ref{atu:eq:spr_a}) при $ c_{x_y} =0.610 $.
  Хаотический режим
}
\label{atu:f:spr_a_p_0610}
\end{figure}




Важной особенностью поведения этой системы является то, что при $ c_{x_y} \ge 1 $
в спектре системы имеются очень ограниченные участки сплошного спектра~(рис.~\ref{atu:f:spr_a_p_1000}).
При этом, как и для получения корректного спектра, так и для обнаружения ``разбегания'' траекторий
необходимо моделирование системы на протяжении достаточно длительного
(по сравнению с многими схожими системами) модельного времени.

\begin{figure}[htb!]
\centerline{
  \includegraphics[width=0.49\textwidth]{p/cha/spr_a/sprott_a-p_xyz_cx_y=1x000.png}
  \includegraphics[width=0.49\textwidth]{p/cha/spr_a/sprott_a_f-p_f_cx_y=1x000.png}
}
\caption{Аттрактор и спектр системы (\ref{atu:eq:spr_a}) при $ c_{x_y} =1.0 $.
}
\label{atu:f:spr_a_p_1000}
\end{figure}

% }}}2


\subsection{Анализ и выбор критериев}  % {{{2

В первую очередь, на аналогии с системой Лоренца,
проверим возможность применения энергетических критериев,
так как существуют реальные физические системы,
для моделирования которых применяется система~~(\ref{atu:eq:spr_a_orig}).
Рассмотрим зависимости $q_{*}(c_{x_y}) $ (рис.~\ref{atu:f:spr_a_q})
для системы (\ref{atu:eq:spr_a}). Анализ этих зависимостей
даёт практически однозначный ответ о возможном виде критерия -- $q_{x^2}$.
Второй кандидат -- $q_{x}$ -- обладает меньшей линейностью.
Также следует отметить, что
значения критериев, в которые не входит $x$, практически не зависят от $c_{x_y}$,
а вид критериев $q_{xy}$ и $q_{xz}$ ничем не лучше, чем у $q_{x}$.
Таким образом, систему идентификации будем строить, используя критерий  $q_{x^2}$.

\begin{figure}[htb!]
\centerline{
  \includegraphics[width=0.60\textwidth]{p/cha/spr_a/sprott_a_q-p_c_x_y.png}
}
\caption{Зависимости $q_{*}(c_{x_y})$ для системы (\ref{atu:eq:spr_a}) }
\label{atu:f:spr_a_q}
\end{figure}

Нельзя не отметить, что в начальной области
$q_{x^2}(c_{x_y}) $, а именно там, где происходят постоянные
перестройки структуры аттрактора, ни один из рассматриваемых критериев
не имеет достаточной монотонности, что не даёт гарантии
построения работоспособной системы идентификации.
Для работы в этой области, скорее всего, требуется синтез специальных критериев.
Также, в окрестности точки $c_{x_y}=1.775$ аттрактор также резко изменяет свою структуру.
Так как это достаточно узкая окрестность,
то можно предположить, что мультиагентная система идентификации
не окажется неработоспособной в этой области, просто вырастет
ошибка идентификации.

Для проведения идентификации было выбрано два семейства методов:
qAuv5x3r.$q_{x^2}$ и
FAlv5.3r.$q_{x^2}$.

Рассмотрим зависимости, определяющие
требуемые интервалы усреднения критериев идентификации
-- $ \sigma_q(\tau_q) $
или же  $ \sigma_q(a_q) $ -- соотношение между
временем оценивания $\tau_q$ и среднеквадратичной
ошибкой измерения критерия.

Для моделирования непосредственных погрешностей измерения величины
$x(t)$ использовался шум с нормальным распределением
и параметрами $\sigma_w=0.1$ и $\tau_w=0.05$.
Для оценивания требуемой зависимости, для каждого
значения $\tau_q$ из заданного диапазона
проводилось $N=200$ процессов моделирования динамики системы,
и в случайный момент (достаточно далеко отстоящий от точки $t=0$ для исключения краевых эффектов)
проводилось измерение и запоминание выбранного критерия.
Так же, как для системы Лоренца,
для усреднения величины $q$ использовалось 2 метода:
простейший линейный, вида (\ref{atu:eq:qlin}), и скользящее среднее.
Полученные зависимости,  представлены на
рис.~\ref{atu:f:spr_a_qx2_tau}.


\begin{figure}[h!]
\begin{center}
  \includegraphics[width=0.49\textwidth]{p/cha/spr_a/sprott_a_qx2_tau-p_aq_sd.png}
  \hfill
  \includegraphics[width=0.49\textwidth]{p/cha/spr_a/sprott_a_qx2_tau-p_tau_sd.png}
\end{center}
  \caption{Зависимости $\sigma_{q}(a_q)$ и $\sigma_{q}(\tau_q)$ для системы Sprott A, критерий  $q_{x^2}$}
\label{atu:f:spr_a_qx2_tau}
\end{figure}

Вид этих зависимостей практически отличается от аналогичных для системы Лоренца (рис.\ref{atu:f:lor_qx2_tau}).
Совершенно аналогично, скользящее среднее в данных условиях
не даёт никаких преимуществ и не будет использваться далее.

% }}}2


\subsection{Тестовая задача идентификации для системы Sprott A}   % {{{2

В соответствии с полученными данными, и используя
предложения~(\ref{atu:eq:po_t_sign}) и~(\ref{atu:eq:po_t_sin}),
определим тестовую задачу следующим образом:
\[
  c_{x_y}(t) \equiv p_o(t) \in (0, 5],
\]
%
\begin{equation}
  p_o(t) = p_0 +  U_{p} \sign \sin( \omega_{p} t ),
  \label{atu:eq:spr_a_po_t_sign}
\end{equation}
%
%
\begin{equation}
  p_o(t) = p_0 +  U_{p} \sin( \omega_{p} t ),
  \label{atu:eq:spr_a_po_t_sin}
\end{equation}
%
где:
$p_0 = 2.8$, $U_p=1.9$, $\omega_p=0.04$.

На рис.~\ref{atu:f:spr_a_id_qAuv5.3r.q_x2_sign} представлены результаты
моделирования процесса идентификации
методом qAuv5.3r.$q_{x^2}$. Аналогично системе Лоренца, на левом графике представлена
динамика перемещения каждой из подвижных моделей,
а на правом -- 4 способа определения $p_{id}(t)$:
$p_{ge}$, $p_{le}$, $p_{ee}$ и $p_{xe}$.
В первую очередь, следует отметить различие между процессами
идентификации этим методом данной системы и система Лоренца.
Зависимость $p_{xe}(t)$, в данном случае демонстрирует
высокий уровень колебаний, и следовательно,
худшие результаты поиска. При этом,
остальные подходы к определению $p_\mathrm{id}$
демонстрируют схожие между собой результаты.
$\overline{e}_{bm}=1.41$
$\overline{e}_{ba}=1.49$.


\begin{figure}[h!]
  \centerline{
    \includegraphics[width=0.49\textwidth]{p/cha/spr_a/qAuv5.3r/sprott_a_qAuv5_3r_qx2-p_t_pi_sign.png}
    \hfill
    \includegraphics[width=0.49\textwidth]{p/cha/spr_a/qAuv5.3r/sprott_a_qAuv5_3r_qx2-p_t_pz_sign.png}
  }
  \caption{Процесс идентификации параметра ``$c_{x_y}$'' системы Sprott A методом qAuv5.3r.$q_{x^2}$ при условии~(\ref{atu:eq:spr_a_po_t_sign})}
  \label{atu:f:spr_a_id_qAuv5.3r.q_x2_sign}
\end{figure}

На рис.~\ref{atu:f:spr_a_id_qAuv5.3r.q_x2_sin} представлены аналогичные результаты,
но при условии плавного изменение значения $p_o(t)$. Также очевидно, что
как абсолютные, так и относительные значения ошибок идентификации в данном случае заметно
меньше, при сохранении общей картины.
$\overline{e}_{bm}=0.91$
и
$\overline{e}_{ba}=0.84$.

\begin{figure}[h!]
  \centerline{
    \includegraphics[width=0.49\textwidth]{p/cha/spr_a/qAuv5.3r/sprott_a_qAuv5_3r_qx2-p_t_pi_sin.png}
    \hfill
    \includegraphics[width=0.49\textwidth]{p/cha/spr_a/qAuv5.3r/sprott_a_qAuv5_3r_qx2-p_t_pz_sin.png}
  }
  \caption{Процесс идентификации параметра ``$c_{x_y}$'' системы Sprott A методом qAuv5.3r.$q_{x^2}$ при условии~(\ref{atu:eq:spr_a_po_t_sin})}
  \label{atu:f:spr_a_id_qAuv5.3r.q_x2_sin}
\end{figure}


На рис.~\ref{atu:f:spr_a_id_FAlv5.3r.q_x2_sign} представлены результаты
методом FAlv5.3r.$q_{x^2}$.
$\overline{e}_{bm}=1.71$


\begin{figure}[h!]
  \centerline{
    \includegraphics[width=0.49\textwidth]{p/cha/spr_a/FAlv5.3A/sprott_a_FAlv5x3r-pl_n_sign.png}
    \hfill
    \includegraphics[width=0.49\textwidth]{p/cha/spr_a/FAlv5.3A/sprott_a_FAlv5x3r-p_p_sign.png}
  }
  \caption{Процесс идентификации параметра ``$c_{x_y}$'' системы Sprott A методом FAlv5.3r.$q_{x^2}$ при условии~(\ref{atu:eq:spr_a_po_t_sign})}
  \label{atu:f:spr_a_id_FAlv5.3r.q_x2_sign}
\end{figure}

На рис.~\ref{atu:f:spr_a_id_FAlv5.3r.q_x2_sin} представлены аналогичные результаты,
но при условии плавного изменение значения параметра объекта. Очевидно, что
как абсолютные, так и относительные значения ошибок идентификации в данном случае заметно
меньше, при сохранении общей картины.
$\overline{e}_{bm}=0.87$

\begin{figure}[h!]
  \centerline{
    \includegraphics[width=0.49\textwidth]{p/cha/spr_a/FAlv5.3A/sprott_a_FAlv5x3r-pl_n_sin.png}
    \hfill
    \includegraphics[width=0.49\textwidth]{p/cha/spr_a/FAlv5.3A/sprott_a_FAlv5x3r-p_p_sin.png}
  }
  \caption{Процесс идентификации параметра ``$c_{x_y}$'' системы Sprott A методом FAlv5.3r.$q_{x^2}$ при условии~(\ref{atu:eq:spr_a_po_t_sin})}
  \label{atu:f:spr_a_id_FAlv5.3r.q_x2_sin}
\end{figure}


% }}}2


\subsection{Влияние параметров системы идентификации на ошибку идентификации для системы Sprott A}  % {{{2


Рассмотрим влияние параметров системы идентификации на
ошибки идентификации.

Первый параметр --
$a_q$, определяющий характерное время усреднения критерия.

На рис.~\ref{atu:f:spr_a_a_q_qAuv5.3r.q_x2} представлены зависимости
$\overline{e}_{r*}(a_q)$ при применении метода qAuv5.3r.$q_{x^2}$.
Вид зависимостей аналогичен таковым, полученным для идентификации
системы Лоренца -- и определяется он теми же факторами,
то есть положение экстремума определяется балансом между
динамическими свойствами самой системы и динамикой
изменения параметра.


\begin{figure}[h!]
  \centerline{
    \includegraphics[width=0.49\textwidth]{p/cha/spr_a/qAuv5.3r/sprott_a_qAuv5_3r_qx2-p_a_q_e_sign.png}
    \hfill
    \includegraphics[width=0.49\textwidth]{p/cha/spr_a/qAuv5.3r/sprott_a_qAuv5_3r_qx2-p_a_q_e_sin.png}
  }
  \caption{Зависимости $\overline{e}_{r*}(a_q)$ при идентификации системы Sprott A методом qAuv5.3r.$q_{x^2}$
   при~(\ref{atu:eq:spr_a_po_t_sign}) и (\ref{atu:eq:spr_a_po_t_sin})}
  \label{atu:f:spr_a_a_q_qAuv5.3r.q_x2}
\end{figure}


На рис.~\ref{atu:f:spr_a_a_q_FAlv5.3r.q_x2} представлены зависимости
$\overline{e}_{r*}(a_q)$ при применении метода FAlv5.3r.$q_{x^2}$.
И в этом случае получены результаты, не выбивающиеся из общего ряда.
Минимальные значения ошибок идентификации практически совпадают
с таковыми при применении предыдущего метода, а требуемое
время усреднения критерия немого больше. Несмотря на это,
при прочих равных данный метод обеспечивает меньше колебания $p_{id}$
в процессе поиска.


\begin{figure}[h!]
  \centerline{
    \includegraphics[width=0.49\textwidth]{p/cha/spr_a/FAlv5.3A/sprott_a_FAlv5x3r-p_a_q_e_sign.png}
    \hfill
    \includegraphics[width=0.49\textwidth]{p/cha/spr_a/FAlv5.3A/sprott_a_FAlv5x3r-p_a_q_e_sin.png}
  }
  \caption{Зависимости $\overline{e}_{r*}(a_q)$ при идентификации системы Sprott A методом FAlv5.3r.$q_{x^2}$
   при~(\ref{atu:eq:spr_a_po_t_sign}) и (\ref{atu:eq:spr_a_po_t_sin})}
  \label{atu:f:spr_a_a_q_FAlv5.3r.q_x2}
\end{figure}

Следующий параметр -- $q_\gamma$ должен больше влиять на процессы идентификации,
использующие для оценки $p_e$ функцию качества.

На рис.~\ref{atu:f:spr_a_qg_qAuv5.3r.q_x2} представлены зависимости
$\overline{e}_{r*}(q_\gamma)$ при применении метода qAuv5.3r.$q_{x^2}$.
Рост ошибок идентификации $\overline{e}_{ge}$ при росте $q_\gamma$
прогнозируем, а остальные способы оценивания $p_{id}$
дают практически неразличимые результаты.

\begin{figure}[h!]
  \centerline{
    \includegraphics[width=0.49\textwidth]{p/cha/spr_a/qAuv5.3r/sprott_a_qAuv5_3r_qx2-p_qgamma_e_sign.png}
    \hfill
    \includegraphics[width=0.49\textwidth]{p/cha/spr_a/qAuv5.3r/sprott_a_qAuv5_3r_qx2-p_qgamma_e_sin.png}
  }
  \caption{Зависимости $\overline{e}_{r*}(q_\gamma)$ при идентификации системы Sprott A методом qAuv5.3r.$q_{x^2}$
   при~(\ref{atu:eq:spr_a_po_t_sign}) и (\ref{atu:eq:spr_a_po_t_sin})}
  \label{atu:f:spr_a_qg_qAuv5.3r.q_x2}
\end{figure}

На рис.~\ref{atu:f:spr_a_qg_FAlv5.3r.q_x2} представлены зависимости
$\overline{e}_{r*}(q_\gamma)$ при применении метода FAlv5.3r.$q_{x^2}$.
Опять же, наблюдается полная аналогия. Отличие от системы Лоренса наблюдается только
в левой части графика, где не наблюдается роста ошибок при
малых значениях $q_\gamma$. Для данной системы
рассматриваемый диапазон недостаточен для наблюдения этого эффекта,
и рост ошибки идентификации наблюдается только при достаточно малых его значениях.

\begin{figure}[h!]
  \centerline{
    \includegraphics[width=0.49\textwidth]{p/cha/spr_a/FAlv5.3A/sprott_a_FAlv5x3r-p_qg_e_sign.png}
    \hfill
    \includegraphics[width=0.49\textwidth]{p/cha/spr_a/FAlv5.3A/sprott_a_FAlv5x3r-p_qg_e_sin.png}
  }
  \caption{Зависимости $\overline{e}_{r*}(q_\gamma)$ при идентификации системы Sprott A методом FAlv5.3r.$q_{x^2}$
   при~(\ref{atu:eq:spr_a_po_t_sign}) и (\ref{atu:eq:spr_a_po_t_sin})}
  \label{atu:f:spr_a_qg_FAlv5.3r.q_x2}
\end{figure}

В целом, для данной системы
зависимости $\overline{e}( q_\gamma )$ % (рис.~\ref{atu:f:spr_a_e_qgamma})
свидетельствует о довольно слабом влиянии этого параметра
на динамику системы идентификации.
Проявляются робастные свойства поисковых агентов.


Рассмотрим влияние
параметра $v_f$, определяющий динамику поисковых агентов.


На рис.~\ref{atu:f:spr_a_v_f_qAuv5.3r.q_x2} представлены зависимости
$\overline{e}_{r*}(v_f)$ при применении метода qAuv5.3r.$q_{x^2}$.
Зависимости практически нет. Это свидетельствует о том,
что смещение поисковых агентов в процессе поиска
для данной системы не имеет практически никакого смысла.
Скорее всего, это связано с тем, что в рассматриваемой постановке
время реакции системы на скачкообразные изменения параметра
имеет тот же порядок, что и требуемое время оценивания параметра,
что делает поиск практически бесполезным.
Рост ошибок идентификации на правом графике обусловлен
избыточной динамикой поисковых агентов.



\begin{figure}[h!]
  \centerline{
    \includegraphics[width=0.49\textwidth]{p/cha/spr_a/qAuv5.3r/sprott_a_qAuv5_3r_qx2-p_v_f_e_sign.png}
    \hfill
    \includegraphics[width=0.49\textwidth]{p/cha/spr_a/qAuv5.3r/sprott_a_qAuv5_3r_qx2-p_v_f_e_sin.png}
  }
  \caption{Зависимости $\overline{e}_{r*}(v_f)$ при идентификации системы Sprott A методом qAuv5.3r.$q_{x^2}$
   при~(\ref{atu:eq:spr_a_po_t_sign}) и (\ref{atu:eq:spr_a_po_t_sin})}
  \label{atu:f:spr_a_v_f_qAuv5.3r.q_x2}
\end{figure}

На рис.~\ref{atu:f:spr_a_v_f_FAlv5.3r.q_x2} представлены зависимости
$\overline{e}_{r*}(v_f)$ при применении метода FAlv5.3r.$q_{x^2}$.
Данные графики проявляют пусть незначительное,
но всё же заметное отличие от предыдущих.
Для этого метода существует слабо выраженный минимум,
причём для него $v_f \ne 0 $, что позволяет
оправдать применение поискового метода.

\begin{figure}[h!]
  \centerline{
    \includegraphics[width=0.49\textwidth]{p/cha/spr_a/FAlv5.3A/sprott_a_FAlv5x3r-p_v_f_e_sign.png}
    \hfill
    \includegraphics[width=0.49\textwidth]{p/cha/spr_a/FAlv5.3A/sprott_a_FAlv5x3r-p_v_f_e_sin.png}
  }
  \caption{Зависимости $\overline{e}_{r*}(v_f)$ при идентификации системы Sprott A методом FAlv5.3r.$q_{x^2}$
   при~(\ref{atu:eq:spr_a_po_t_sign}) и (\ref{atu:eq:spr_a_po_t_sin})}
  \label{atu:f:spr_a_v_f_FAlv5.3r.q_x2}
\end{figure}

Параметр $k_e$, определяющий соотношение сил, действующих
на поисковый агент, для рассматриваемой системы не должен
иметь существенного влияния на процесс поиска.
Слабая зависимость от $v_f$ практически автоматически обозначает
и слабую зависимость от $k_e$, так как оба эти параметра
входят в определение одной и той же ``силы'',
определяющей смещение поисковых агентов, а из
анализа влияния предыдущего параметра было установлено,
что в данных конкретных условиях поиск практически не улучшает результаты.


На рис.~\ref{atu:f:spr_a_k_e_qAuv5.3r.q_x2} представлены зависимости
$\overline{e}_{r*}(k_e)$ при применении метода qAuv5.3r.$q_{x^2}$.

\begin{figure}[h!]
  \centerline{
    \includegraphics[width=0.49\textwidth]{p/cha/spr_a/qAuv5.3r/sprott_a_qAuv5_3r_qx2-p_k_e_e_sign.png}
    \hfill
    \includegraphics[width=0.49\textwidth]{p/cha/spr_a/qAuv5.3r/sprott_a_qAuv5_3r_qx2-p_k_e_e_sin.png}
  }
  \caption{Зависимости $\overline{e}_{r*}(k_e)$ при идентификации системы Sprott A методом qAuv5.3r.$q_{x^2}$
   при~(\ref{atu:eq:spr_a_po_t_sign}) и (\ref{atu:eq:spr_a_po_t_sin})}
  \label{atu:f:spr_a_k_e_qAuv5.3r.q_x2}
\end{figure}

На рис.~\ref{atu:f:spr_a_k_e_FAlv5.3r.q_x2} представлены зависимости
$\overline{e}_{r*}(k_e)$ при применении метода FAlv5.3r.$q_{x^2}$.

\begin{figure}[h!]
  \centerline{
    \includegraphics[width=0.49\textwidth]{p/cha/spr_a/FAlv5.3A/sprott_a_FAlv5x3r-p_ke_e_sign.png}
    \hfill
    \includegraphics[width=0.49\textwidth]{p/cha/spr_a/FAlv5.3A/sprott_a_FAlv5x3r-p_ke_e_sin.png}
  }
  \caption{Зависимости $\overline{e}_{r*}(k_e)$ при идентификации системы Sprott A методом FAlv5.3r.$q_{x^2}$
   при~(\ref{atu:eq:spr_a_po_t_sign}) и (\ref{atu:eq:spr_a_po_t_sin})}
  \label{atu:f:spr_a_k_e_FAlv5.3r.q_x2}
\end{figure}

Как и ожидалось, зависимости $\overline{e}_{r*}(k_e)$
довольно слабые, и аналогично зависимостям  $\overline{e}_{r*}(v_f)$,
только для метода FAlv5.3r.$q_{x^2}$ можно
утверждать о необходимости поиска.

% }}}2


\subsection{Выводы}  % {{{2

В целом, результаты моделирования процессов идентификации системы ``Sprott A'',
и сравнение этих результатов, с данными, полученными
доя системы Лоренца, позволяют сделать следующие выводы:

\begin{itemize}

  \item
    Попытка создать и критерий идентификации и работоспособную систему идентификации
    на основании энергетического критерия вида $q_{x^2}$ оказалась успешной. В отличие
    от системы Лоренца, для которой выбор возможных критериев был шире,
    здесь критерий определён однозначно.

  \item
    Для рассматриваемой постановки задачи идентификации,
    смещение агентов слабо влияет на качество идентификации.

  \item
    В целом, использование метода  FAlv5.3r.$q_{x^2}$ для данной системы
    и начальных условий более оправданно.

\end{itemize}

% }}}2


% }}}1

% vim: fdm=marker foldlevelstart=2 fdc=4 ft=tex
