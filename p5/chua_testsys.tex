
\FloatBarrier

\section{Система Чуа} %  % {{{1 _CHUA_
\label{atu:sect:chua}

\LinkRef{
 chua: ASAU-18, MKMM-2014, APIR-2011
}

\subsection{Определение системы и анализ её динамики} %  % {{{2 _chua_task

Одной из известных хаотических систем, легко реализуемых как модельно (\ref{atu:eq:chua}),
так и схемотехнически (рис.~\ref{atu:f:chuascheme}),
является нелинейная система Чуа~\cite{moon_chaotic_vibr,buga_chua,Kennedy92robustop}:

\begin{figure}[htb!]
\begin{center}
% vi:syntax=tex

\begin{circuitikz}[line width=0.7]
  \ctikzset{bipoles/thickness=2}
  \def\Top{3.0}
  \draw (0.0,0.0) to[L,l=$L$,i=$I_L$] (0,\Top)
   to[R=R] (6.0,\Top)
   to[ageneric,l=$R_c$] (6.0,0.0) -- (0.0,0.0);
  \draw(1.5,0.0) to[C,l=$C_2$,v=$V_2$] (1.5,\Top );
  \draw(4.5,0.0) to[C,l=$C_1$,v=$V_1$] (4.5,\Top );
\end{circuitikz}

% \begin{tikzpicture}[circuit ee IEC,very thick,circuit symbol unit=3.5mm]
%   \node (L1) at (0,1.5) [point up,elelem,inductor={info = $L$}] {};
%   \node      at (0.2,2.4) {$I_L$};
%   \node (C2) at (1.5,1.5) [point up,elelem,capacitor={info = $C_2$}] {};
%   \node      at (1.8,1.8) {$V_2$};
%   \node (pc2d) at (1.5,0) [contact] {};
%   \node (pc2u) at (1.5,3) [contact] {};
%   \node (C1) at (5.0,1.5) [point up,elelem,capacitor={info = $C_1$}] {};
%   \node      at (5.3,1.8) {$V_1$};
%   \node (pc1d) at (5,0) [contact] {};
%   \node (pc1u) at (5,3) [contact] {};
%   \node (R) at (3,3) [elelem,resistor={info = $R$}] {};
%   \node (Rc) at (7,1.5) [elelem,point up,resistor={info = $R_c$}] {};
%   \draw (Rc) ++(-0.15,-0.7) rectangle+(0.3,0.2);
%   \draw (L1) |- (pc2d) -- (pc1d) -| (Rc) [wire];
%   \draw (L1) |- (pc2u) -- (R) -- (pc1u) -| (Rc) [wire];
%   \draw (pc2u) -- (C2) [wire]; \draw (pc2d) -- (C2) [wire];
%   \draw (pc1u) -- (C1) [wire]; \draw (pc1d) -- (C1) [wire];
%   \node (Gr) at (5,-0.3) [elelem,point down,ground] {};
%   \draw (pc1d) -- (Gr) [wire];
% \end{tikzpicture}

\end{center}
\caption{Условная электрическая цепь, реализующая хаотическую систему Чуа}
\label{atu:f:chuascheme}
\end{figure}


\begin{equation}
\begin{cases}
  C_1 \dot{V_1}  = \frac{1}{R} ( V_2 - V_1 ) - g(V_1), \\
  C_2 \dot{V_2}  = \frac{1}{R} ( V_1 - V_2 ) + I_L, \\
  \dot{I_L}      = - \frac{1}{L} V_2 .
\end{cases}
\label{atu:eq:chua}
\end{equation}

Единственным нелинейным элементом в данной системе является ``диод Чуа''
(обозначен на схеме как $R_c$) с
характеристикой $g(V)$~(рис.~\ref{atu:f:diodchua}),
обладающей различным отрицательными наклонами
($m_0$ и $m_1$) на разных участках,
и тем самым являющийся управляемым источником энергии.
%
%
\begin{equation}
g(V) =
\begin{cases}
  m_1 V = ( m_0 + m_2 ) V , & |V| <   U_0, \\
  m_0 V ,                   & |V| \ge U_0.
\end{cases}
\label{atu:eq:diodchua}
\end{equation}

\begin{figure}[htb!]
\begin{center}
% vi:syntax=tex
\begin{tikzpicture}
  \coordinate (XMIN) at (-4.5,0.0);
  \coordinate (XMAX) at ( 4.5,0.0);
  \coordinate (YMIN) at (0,-2.5);
  \coordinate (YMAX) at (0, 2.5);
  \draw (XMIN) -- (XMAX) [medline,->] node[below] {$U$};
  \draw (YMIN) -- (YMAX) [medline,->] node[left]  {$I$};
  \draw (-4,2) -- (-1,1) -- (1,-1) -- (4,-2) [boldline];
  \draw (1,-1) -- (1,0) [dashed,medline];
  \draw (1,-1) -- (4,-1) [dashed,medline];
  \draw (1.41,0) arc [medline,->,start angle=0,end angle=-45,radius=1.41];
  \draw (1,-1) ++(2,0) arc [medline,->,start angle=0,end angle=-18,radius=2.0];
  \filldraw (1,0) circle[radius=0.05,fill=black] node[above] {$U_0$};
  \node[right] at (1.2,-0.6) {$\alpha_1: \tan(\alpha_1)=m_1$};
  \node[right] at (3.0,-1.3) {$\alpha_0$};
\end{tikzpicture}

\end{center}
\caption{Характеристика \(I=g(V)\) диода Чуа}
\label{atu:f:diodchua}
\end{figure}


При этом параметр \(m_0\) определяет поступление энергии в систему
при больших амплитудах \(V_1\), и, в целом, характеризует
энергетические возможности источника.
Аналогично, параметр \(m_1\) характеризует поступление энергии
при малых колебаниях, в частности, определяет, будет ли
система переходить в колебательное состояние при малых начальных
возмущениях, и какой будет режим этих колебаний.
С другой стороны, поскольку параметр \(m_1\) является суммой
``глобального параметра'' \(m_0\) и ``довеска'',
определяющего дополнительный вклад при малых амплитудах,
то имеет смысл перейти от параметра \(m_1\) к параметру
\( m_2 = m_1 - m_0 \), который
и определяет в целом нелинейность
системы. При \( m_2 = 0 \) система становится линейной
и не представляет особого интереса. Поэтому
в данной работе в качестве
идентифицируемого параметра рассматривается именно параметр \(m_2\).

Важно, что в зависимости от величины параметра $m_2$,
система может переходить в режимы затухания,
периодического и сложно-периодического движения, а также в режим
хаотических колебаний~\cite{anisch_nonlin_eff,magni_new_meth}. При этом сложно-периодическое и хаотическое
движения чередуются с изменением величины \(m_2\).


Классически параметры системы Чуа задаются следующим образом~\cite{buga_chua}:
$C_1 = 1/9$, $C_2 = 1$, $L= 1/7$, $R = 1/0.7$, $m_0=-0.5$, $ m_2 \in [ -0.15; -0.7 ] $.

Введём обозначения:
\[
  a_{11} = \frac{1}{R C_1}; \;
  a_{13} = \frac{1}{C_1}; \;
  a_{21} = \frac{1}{R C_2}; \;
  a_{23} = \frac{1}{C_2}; \;
  a_{31} = -\frac{1}{L}; \;
  a_g = - \frac{m_0}{C_1}; \;
  \mu = - \frac{m_2}{C_1}.
\]
%
%\noindent
Тогда:
%
\begin{equation}
\begin{cases}
  \dot{V}_1  = -a_{11} V_1 + a_{11}  V_2  + g_1(V_1) , \\
  \dot{V}_2  = +a_{21} V_1 - a_{21}  V_2  + a_{23} I_L    , \\
  \dot{I}_L  =  a_{31} V_2.
\end{cases}
\label{atu:eq:chua2}
\end{equation}
%
%
\begin{equation}
g_1(V) =
\begin{cases}
  ( a_g + \mu ) V , & |V| <   U_0, \\
  a_g V           , & |V| \ge U_0.
\end{cases}
\label{atu:eq:diodchua2}
\end{equation}

При этом пересчитанные классические значения параметров будут представлены следующим образом:
$ a_{11} = 6.5 $, $a_{21} = 0.7$, $ a_{23} = -7 $, $ a_g = 4.5 $,
$ \mu \in [ 1.29 ; 5.6 ] $.
Соответственно, в этих обозначениях
идентифицируемым параметров является $\mu$.



\begin{figure}[htb!]
\centerline{
  \includegraphics[width=0.49\textwidth]{p/cha/chua/chua_1-p_xyz_mu=2x74.png}
  \includegraphics[width=0.49\textwidth]{p/cha/chua/chua_1-p_xyz_mu=4x50.png}
}
\caption{Аттрактор системы Чуа (\ref{atu:eq:chua2}) при различных значениях $\mu$}
\label{atu:f:chua_phase}
\end{figure}


% }}}2

\subsection{Анализ и выбор критериев}  % {{{2


Для определения вида критерия рассмотрим зависимости
$q_{*}(\mu) $ (индексом ``*'' будем обозначать применение какого-либо из индексов,
обозначающих исходный сигнал для критерия и способ усреднения),
полученные путём моделирования
для системы Чуа (рис.~\ref{atu:f:chua_q}):

\begin{figure}[htb!]
\centerline{
  \includegraphics[width=0.49\textwidth]{p/cha/chua/chua_q-p_mu2.png}
  \includegraphics[width=0.49\textwidth]{p/cha/chua/chua_q-p_mu1.png}
}
  \caption{Зависимости $q_{*}(\mu) $ для системы Чуа (\ref{atu:eq:chua2})}
\label{atu:f:chua_q}
\end{figure}

Из графиков очевидно, что величина $ q_{V_1}(\mu) $
является наилучшим кандидатом в критерии, ввиду близкой к линейной зависимости
в рабочем диапазоне.

Следующим важным параметром, необходимым для эффективной работы системы идентификации, является
характерное время время оценивания $\tau$ (\ref{atu:eq:qlin}), или же обратная ему величина $a_q$.

Для предварительного оценивания величины $a_q$ рассмотрим спектры системы при различных
$\mu$ (рис.~\ref{atu:f:chua_spectrum}). Как следует из графиков, спектр системы достаточно
ограничен сверху, однако, в хаотическом режиме является сплошным практически до нуля.
Это не даёт возможности непосредственно определить $a_q$ исходя из спектра,
однако, первые существенные пики наблюдаются при $ \omega \approx 0.3 $, следовательно,
первоначальное значение $a_q$ можно оценить как $ a_q \approx 0.3 / \pi \approx 0.1 $.


\begin{figure}[htb!]
\centerline{
  \includegraphics[width=0.32\textwidth]{p/cha/chua/chua_f-p_f_mu=2x00.png}
  \includegraphics[width=0.32\textwidth]{p/cha/chua/chua_f-p_f_mu=2x74.png}
  \includegraphics[width=0.32\textwidth]{p/cha/chua/chua_f-p_f_mu=4x50.png}
}
\caption{Спектры системы Чуа (\ref{atu:eq:chua2}) при различных значениях $\mu$}
\label{atu:f:chua_spectrum}
\end{figure}

Следующий способ оценить $a_q$ -- исследовать полученную в результате моделирования
зависимость среднеквадратичного отклонения $e_q$ оценки величины $q$, нормированной
на саму величину $q$ в стационарном случае. Полученная зависимость представлена
на рис.~\ref{atu:f:chua_tau}.

\begin{figure}[htb!]
\centerline{
  \includegraphics[width=0.4\textwidth]{p/cha/chua/chua_tau-p_e_a.png}
}
\caption{Типичная зависимость $e_q/q(a_q)$ для системы (\ref{atu:eq:chua2})}
\label{atu:f:chua_tau}
\end{figure}

Из этого графика можно сделать вывод, что первоначальная оценка $a_q$
была сделана корректно.

% }}}2

\subsection{Тестовая задача идентификации для системы Чуа}  % {{{2

В соответствии с полученными данными, и используя
предложения~(\ref{atu:eq:po_t_sign}) и~(\ref{atu:eq:po_t_sin}),
определим тестовую задачу следующим образом:
%
\begin{equation}
 \mu_o(t) = p_0 + U_p \sign \sin( \omega_p t )
  \label{atu:eq:chua_mu_sign}
\end{equation}
%
\begin{equation}
 \mu_o(t) = p_0 + U_p \sin( \omega_p t ).
  \label{atu:eq:chua_mu_sin}
\end{equation}
%
где:
$p_0 = 3.25$, $U_p=0.6$, $\omega_p=0.0104719755$.
При моделировании процессов идентификации было обнаружено
следующее явление: если динамика параметра $\mu$ задана (\ref{atu:eq:chua_mu_sign}),
то существуют такие значения $p_0$ и $U_p$,
при которых система полностью теряет устойчивость.
Для схемотехнических реализаций системы Чуа таких явлений
не наблюдается, так как реальные системы имеют ограниченные возможности
по поддержанию отрицательного сопротивления. Для таких случаев
представление диода Чуа в виде (\ref{atu:eq:diodchua})
не является адекватным. Точнее, учитывая тот факт,
что схемотехнические реализации появились после аналитичного представления,
более правильным будет утверждение о том, что реализации недостаточно адекватно
реализуют аналитическую зависимость для диода Чуа.
Более адекватная реализация требует наличия неограниченного
источника энергии. С учётом вышеупомянутого,
множество тестовых систем для рассматриваемой задачи
является более ограниченным.

На рис.~\ref{atu:f:chua_id_ql3rlWvnAAW_sign} представлены результаты
моделирования процесса идентификации
группой методов ql3rlWvnAAW при условии~(\ref{atu:eq:chua_mu_sign}).

\begin{figure}[htb!]
  \centerline{
    \includegraphics[width=0.49\textwidth]{p/cha/chua/ql3rlWvnAAW/chua_id-p_t_pi_ql3rlWvnAAW_sign.png}
    \hfill
    \includegraphics[width=0.49\textwidth]{p/cha/chua/ql3rlWvnAAW/chua_id-p_t_p_ql3rlWvnAAW_sign.png}
  }
  \caption{Процесс идентификации параметра ``$\mu$'' системы Чуа группой методов ql3rlWvnAAW при условии~(\ref{atu:eq:chua_mu_sign})}
  \label{atu:f:chua_id_ql3rlWvnAAW_sign}
\end{figure}

При заданных условиях система вполне работоспособна,
уровень колебаний идентифицируемого параметра вполне адекватный задаче.

На рис.~\ref{atu:f:chua_id_ql3rlWvnAAW_sin} представлены результаты
моделирования процесса идентификации
группой методов ql3rlWvnAAW при условии~(\ref{atu:eq:chua_mu_sin}).

\begin{figure}[htb!]
  \centerline{
    \includegraphics[width=0.49\textwidth]{p/cha/chua/ql3rlWvnAAW/chua_id-p_t_pi_ql3rlWvnAAW_sin.png}
    \hfill
    \includegraphics[width=0.49\textwidth]{p/cha/chua/ql3rlWvnAAW/chua_id-p_t_p_ql3rlWvnAAW_sin.png}
  }
  \caption{Процесс идентификации параметра ``$\mu$'' системы Чуа группой методов ql3rlWvnAAW при условии~(\ref{atu:eq:chua_mu_sin})}
  \label{atu:f:chua_id_ql3rlWvnAAW_sin}
\end{figure}

Как общий уровень ошибки, так и размах колебаний значения параметра
в этом случае заметно меньше, аналогично процессам идентификации
других рассмотренных хаотических систем.

Процесс идентификации в условиях, когда параметр
$\mu$ медленно и линейно ``пробегает'' весь рассматриваемый рабочий диапазон,
проиллюстрирован на  рис.~\ref{atu:f:chua_id_ql3rlWvnAAW_ramp}.

\begin{figure}[htb!]
  \centerline{
    \includegraphics[width=0.49\textwidth]{p/cha/chua/ql3rlWvnAAW/chua_id-p_t_pi_ql3rlWvnAAW_ramp.png}
    \hfill
    \includegraphics[width=0.49\textwidth]{p/cha/chua/ql3rlWvnAAW/chua_id-p_t_p_ql3rlWvnAAW_ramp.png}
  }
  \caption{Процесс идентификации параметра ``$\mu$'' системы Чуа группой методов ql3rlWvnAAW при условии~(\ref{atu:eq:po_t_ramp})}
  \label{atu:f:chua_id_ql3rlWvnAAW_ramp}
\end{figure}

Зависимости на графиках свидетельствуют о том, что с учётом уровня
ошибки, идентификация на всём диапазоне происходит без каких-либо
нарушений.


% }}}2


\subsection{Влияние параметров системы идентификации на ошибку идентификации для системы Чуа}  % {{{2

Для более точной настройки параметров самой системы идентификации
рассмотрим зависимости ошибок идентификации
от основных параметров системы идентификации.

Для проверки корректности выбора величины $a_q$ была построены зависимости
ошибок идентификации (рис.~\ref{atu:f:chua_e_a_q}) от этого параметра.


\begin{figure}[htb!]
  \centerline{
    \includegraphics[width=0.49\textwidth]{p/cha/chua/ql3rlWvnAAW/chua_id-p_a_q_sign.png}
    \hfill
    \includegraphics[width=0.49\textwidth]{p/cha/chua/ql3rlWvnAAW/chua_id-p_a_q_sin.png}
  }
  \caption{Зависимости  $\overline{e}(a_q)$ для системы (\ref{atu:eq:chua2})
  при условиях (\ref{atu:eq:chua_mu_sign}) и (\ref{atu:eq:chua_mu_sin})
}
\label{atu:f:chua_e_a_q}
\end{figure}

Как видно, первоначальная оценка ``правильного'' значения величины $a_q$
была сделана достаточно точно. При этом, при синусоидальном изменении параметра объекта
меньшая ошибка идентификации наблюдается при меньших значениях $a_q$, что связано
с тем, в данном случае нет необходимости в слежении за скачкообразно изменяющимся параметром,
и, следовательно, допустимо большее время оценивания. Напротив, в случае (\ref{atu:eq:chua_mu_sign})
увеличение времени оценивания приводит к заметному снижению интегральной точности идентификации.



Одним из важнейших параметров является $q_\gamma$ -- масштаб функции качества
((\ref{atu:eq:F_gauss})--(\ref{atu:eq:F_log})).
Зависимости для этого параметра приведены на рис.~\ref{atu:f:chua_e_qgamma}.

\begin{figure}[htb!]
  \centerline{
    \includegraphics[width=0.49\textwidth]{p/cha/chua/ql3rlWvnAAW/chua_id-p_q_gamma_sign.png}
    \hfill
    \includegraphics[width=0.49\textwidth]{p/cha/chua/ql3rlWvnAAW/chua_id-p_q_gamma_sin.png}
  }
  \caption{Зависимости  $\overline{e}(q_\gamma)$ для системы (\ref{atu:eq:chua2})
  при условиях (\ref{atu:eq:chua_mu_sign}) и (\ref{atu:eq:chua_mu_sin})
}
\label{atu:f:chua_e_qgamma}
\end{figure}

Явно выраженного экстремума не наблюдается, что свидетельствует
о сильной робастности метода. При условиях скачкообразной
динамики параметра, ошибки, связанные с динамикой системы
маскируют эту зависимость. При условиях (\ref{atu:eq:chua_mu_sin})
проявляется увеличение уровня ошибок при избыточной чувствительности функции
качества, и и близкое к равномерному ``плато'' при её снижении.

Зависимости ошибки идентификации от коэффициента скорости поиска $v_f$
приведены на рис.~\ref{atu:f:chua_e_v_f}.

\begin{figure}[htb!]
  \centerline{
    \includegraphics[width=0.49\textwidth]{p/cha/chua/ql3rlWvnAAW/chua_id-p_v_f_sign.png}
    \hfill
    \includegraphics[width=0.49\textwidth]{p/cha/chua/ql3rlWvnAAW/chua_id-p_v_f_sin.png}
  }
  \caption{Зависимости  $\overline{e}(v_f)$ для системы (\ref{atu:eq:chua2})
  при условиях (\ref{atu:eq:chua_mu_sign}) и (\ref{atu:eq:chua_mu_sin})
}
\label{atu:f:chua_e_v_f}
\end{figure}

Снижение ошибки идентификации за счёт динамики агентов
в случае плавного изменения параметра $\mu$ достигает 250\%.
Если же динамика агентов задана (\ref{atu:eq:chua_mu_sign}),
то выигрыш существенно ниже.

Зависимости
$\overline{e}(k_e)$,
$\overline{e}(k_{nl})$ и
$\overline{e}(k_{cl})$ имеют вид, совершенно аналогичный тому,
что был получен для других систем хаотической динамики.



% }}}2


\subsection{Выводы}  % {{{2

Результаты моделирования процессов идентификации параметра ``$\mu$''системы Чуа,
и сравнение этих результатов, с данными, полученными
для других система, позволяют сделать следующие выводы:

\begin{itemize}

  \item
    Для идентификации параметра $\mu$ системы Чуа
    можно использовать несколько критериев.
    При этом критерий $q_{V1}$ позволяет достичь лучших результатов.

  \item
    Использование метода  ql3rlWvnleW для данной системы
    оправданно.

  \item
    Принципиальных отличий процесса идентификации от других систем не обнаружено.

\end{itemize}


% }}}2


% }}}1

% vim: fdm=marker foldlevel=1 foldignore="%#" fdc=4 ft=tex
