\clearpage

\phantomsection
\pdfbookmark[0]{АНОТАЦІЯ}{annotauabookmark}

\begin{center}
\textbf{АНОТАЦІЯ}
\end{center}

\medskip

\textit{\dissauthorUa} \booknameUa.
--- Кваліфікаційна робота на правах рукопису.

Дисертація на здобутті наукового ступеня доктора технічних наук за спеціальністю
\dissSpecId ``\dissSpecUa''. --- \institutionUa, \belongUa, \bookyear.


Робота присвячена актуальній науково-технічній проблемі
ідентифікації параметрів складних технічних систем у режимі хаотичної динаміки.
Нелінійні динамічні системи, розповсюджені в сучасних технологічних і
природних процесах, можуть проявляти хаотичні властивості в своїй динаміці.
Існуючи системи ідентифікації нелінійних динамічних об'єктів практично не
здатні функціонувати у таких умовах.
Проаналізовано існуючи методи ідентифікації та моделі динаміки пошуку.
Визначено спільні риси систем ідентифікації,
досліджено чинники, які не дозволяють використовувати їх
для ідентифікації параметрів хаотичних систем.
Обґрунтована необхідність створення нових систем ідентифікації.

Об'єктом дослідження є 
технічні системи, які в процесі їх функціювання можуть
входити в хаотичні режими.

Предметом дослідження є
математичні моделі процесів та методи
ансамблевої ідентифікації технічних систем з хаотичною динамікою.

Методи дослідження:
Для вирішення поставлених задач використовувався математичний апарат
теорії управління та ідентифікації нелінійних систем, динамічного хаосу,
обчислювальних методів, нечіткої логіки, теорії інформації тощо.

Практичне значення отриманих результатів полягає у тому,
що озроблені методи ідентифікації було використано
при проектуванні, створенні, налаштовуванні параметрів
стенду дослідження вібраційного та акустичного впливу.
Аналіз результатів даних з цього стенду
дав можливість вказати необхідні нелінійні властивості системи
та діапазон параметрів, які у сукупності
забезпечують широкосмуговий спектр коливань.

Створене програмне середовище для моделювання нелінійних динамічних систем
використовується при проведенні практичних робіт по дисциплінам
``Моделювання систем'',
``Сучасні системи управління'' на кафедрі інформаційних технологій
і систем Національної металургійної академії України.

В дисертаційній роботі
  створено критерії ідентифікації нелінійних динамічних систем,
  які, на відміну від тих, що існують, дозволяють оцінити їх стан та
  хаотичну динаміку, а також дають підстави для створення ефективних алгоритмів
  настроювання параметрів моделей систем ідентифікації.

Створено методи ідентифікації на підставі
  адаптивно-пошукової парадигми з використанням ансамблю пошукових агентів,
  які взаємодіють проміж собою, які на відміну від методів, що використовують
  одну модель або пару моделей, значно підвищують швидкість пошуку та
  здатні за мінімальний час  перелаштовуватися при різкій зміні параметрів, а на
  відміну від ройових алгоритмів нові методи потребують значно меншої
  кількості моделей та забезпечують певні гарантії пошуку.

Створено нову класифікацію систем ідентифікації динамічних систем,
  яка, як вбирає у себе методи, що існують, так і дозволяє
  створювати нові методи ідентифікації за рахунок
  комбінування їх складових частин;

Запропоновано модель системи хаотичної динаміки
   з використанням зв'язаних релаксаційних генераторів,
   яка відрізняється від існуючих відсутністю індуктивних компонентів,
   працездатністю при малих напругах та можливістю
   керування частотним діапазоном у широкому інтервалі,
   що сприяє процесу аналізу хаотичної динаміки
   фізичного об'єкту, перевірці адекватності математичної моделі
   та властивостей системи ідентифікації.

Вдосконалено
  методи оцінювання якості ідентифікації,
  які на відміну від тих, що існують,
  враховують використання множини агентів.

Набуло подальший розвиток
  підходи до адаптації параметрів систем
  ансамблевої ідентифікації, які придатні використовувати поточну
  інформацію від ансамблю синергірованих моделей та коригувати глобальні
  параметри пошуку в умовах апріорної та поточної невизначеності.

Вдосконалено
модель генератора Колпітца, яка враховує
    більшу кількість нелінійних ефектів,
    що забезпечує більш адекватні результати процесу
    ідентифікації її параметрів запропонованими методами.

Встновлено, що
для вирішення завдання ідентифікації необхідно існування
критерію, що відображає схожість динаміки об'єкта і моделей.
    У тих випадках, коли утруднено формування критерію на
    підставі чітких фізичних принципів, має сенс вид критерію
    отримати перебором, заснованому на обмеженій множині уявлень
    інтегральних величин.

    Саме поняття інтегрального критерію має на увазі певний
    вид усереднення, що в певній мірі відповідає застосуванню
    низькочастотної фільтрації.

    Застосування функцій якості ідентифікації необхідно для
    визначення досягнення мети ідентифікації, отримання значення
    ідентифікованого параметра координатором пошуку. Деякі методи
    роботи пошукових агентів вимагають значень цих функцій в якості
    вхідного сигналу.

    Встановлено основні вимоги до властивостей функцій якості
    ідентифікації, розглянуті основні способи визначення.

    Визначено набір структур систем ідентифікації, заснованих на
    взаємодії пошукових агентів і координаторів пошуку. Визначено
    підклас системи ідентифікації, що використовують при своїй
    роботі динаміку ансамблю пошукових агентів.

    Визначено методи локальної оцінки значення ідентифікованого
    параметра як для пошукових агентів, які використовують значення
    як критерію, так і функції якості.

    Визначено способи завдання динаміки пошукових агентів.
    Визначено методи отримання значення ідентифікованого параметра
    координатором пошуку.

    Створена класифікація ансамблевих пошукових систем, що дозволяє
    коротким і однозначним чином визначити структуру і динаміку,
    а також створювати нові методи, за умови визначення правил
    роботи елементів пошукових агентів і координаторів пошуку.

    На простих не динамічних тестових завданнях вивчені базові
    властивості як окремих елементів ансамблевих систем
    ідентифікації, так і систем в цілому.

Результати було впроваджено у рамках науково-практичного дослідження
``Оцінка можливості заміни випробувань КА на стійкість до акустичного навантаження
випробуваннями широкосмугової вібрації'', згідно договору \hbox{№~V-105-16-3} від 07.09.2016
(Спеціальне конструкторсько-технологічне бюро ІТМ НАН України, м.~Дніпро),
та при ідентифікації режиму роботи вертикально-осьової вітроустановки з H-ротором Дар'є
з врахуванням складно-періодичної або хаотичної структури коливань лопатей
(Інститут транспортних систем і технологій НАН України, м.~Дніпро).

\textbf{Ключові слова:}
хаотичні системи,
критерій ідентифікації,
пошуковий агент,
координатор пошуку,
ансамблеві методи пошукової ідентифікації.






