\chapter{Анализ и идентификация физической реализации системы Колпитца}
\label{atu:ch:colpreal}

\LinkRef{
  colp: ASAU-21, APIR-2013
}

\section{Определение системы и анализ её динамики} %  % {{{1
\label{atu:s:colp_task}

В ряде радиоэлектронных устройств используются
генераторы сигналов, способные генерировать разнообразные виды  сигналов, в том числе
сложно-периодические и хаотические~\cite{dmitriev_gen_chaos}.
В частности, генератор  Колпитца~\cite{kennedy_chaos_colpitts,atu_asau21}, который, в
зависимости от условий,   может
генерировать  колебания,  как  близкие  к  гармоническим,  так  и  проявлять
хаотическую  динамику  в  широком  спектральном   диапазоне.   Идентификация
параметров рассматриваемого генератора  необходима,  с  одной  стороны,  для
обеспечения  требуемых  режимов  работы.  С  другой  стороны,  информация  о
параметрах системы необходима при проведении  контроля  работоспособности  в
процессе эксплуатации устройства~\cite{atu_apir2013}.

На рис.~\ref{atu:f:colp_schem} приставлена одна из электрических схем,
реализующих генератор Колпитца на биполярном транзисторе.
Из множества схем, данная была выбрана из-за наличия
одного источника напряжения и простоты схемотехнической реализации генератора.


\begin{figure}[htb!]
\begin{center}
% vi:syntax=tex

\begin{circuitikz}[line width=0.7]
  \ctikzset{bipoles/thickness=2}
  \def\Top{8.0}
  \def\Rig{7.0}
  \def\LinC{5.5}
  \def\LinQ{3.5}
  % transistor
  \draw (\LinQ,3.0) node[npn](npn) {}
        (npn.center) ++(-0.2,0) circle[radius=0.5]
        (npn.C) node[left=3mm, above=-3.6mm]{$Q_1$}
        (npn.B) node[left=2mm, above=0.2mm]{$V_b$};
  \draw (npn.C) ++(0.4,-0.4) -- ++(0.0,-0.8) [->] node[right] {$I_{ce}$};
  % border
  \draw (0.0,0.0) -- (\Rig,0.0)
   to[battery,l=$V_{cc}$]  (\Rig,\Top)
   to[short]         (0,\Top)
   to[R,mirror,l=$R_1$]     (0.0,3.0)
   to[R,mirror,l=$R_2$,*-]  (0.0,0.0)
   -- (0.0,0.0);
  % base part
  \draw (1.5,0.0) to[C,l=$C_0$,*-*] (1.5,3.0);
  \draw (0.0,3.0) -- (npn.base);
  % emitter part
  \draw (\LinQ,0.0) to[R,l=$R_e$,*-*] (\LinQ,2.0)
   to[short] (npn.E);
   % collector part
  \draw (npn.collector)
   to[L,mirror,l=$L$,i<=$I_L$] (\LinQ,6.0)
   to[vR,l=$R_c$] (\LinQ,\Top);
  %
  \draw (\LinC,0.0) to[C,l=$C_2$,v=$V_2$,*-*] (\LinC,2.0);
  \draw (\LinC,2.0) -- (\LinQ,2.0);
  \draw (\LinC,2.0) to[C,l=$C_1$,v=$V_1$]     (\LinC,4.0);
  \draw (\LinC,4.0) -- (\LinQ,4.0);
  \filldraw (\LinQ,4.0) circle[radius=0.05];
\end{circuitikz}




\end{center}
\caption{Электрическая схема генератора Колпитца на биполярном транзисторе}
\label{atu:f:colp_schem}
\end{figure}

При создании модели генератора Колпитца систему уравнений можно
заранее упростить, если заметить, что
делитель на резисторах
$\mathrm{R}_1$, $\mathrm{R}_2$,
вместе с конденсатором
$\mathrm{C}_0$ обеспечивают
постоянство потенциала базы
$V_b = V_{CC} \frac{R_1}{R_1+R_2}$,
поэтому из дальнейшего рассмотрения данные элементы можно
исключить.

Рассмотрев процессы заряда конденсаторов и изменение тока через
катушку индуктивности, получим следующую систему уравнений:
%
\begin{equation}
\label{atu:eq:colp_phys}
\begin{dcases}
  C_1 \od{V_{1}}{t}  = I_L - I_{CE} , \\
  L\, \od{I_L}{t} \; = V_{CC} - V_{1} - V_{2} - I_L R_C , \\
  C_2 \od{V_{2}}{t}  = I_L - \frac{V_{2}}{R_e}.
\end{dcases}
\end{equation}
%
%
%\noindent
где
$V_{CC} $ -- напряжение питания,
$V_1,$ $V_2$ -- разность потенциалов между выводами конденсаторов
$\mathrm{C}_1$ и $\mathrm{C}_2$ соответственно,
$I_L$, $I_{CE}$ -- токи катушки индуктивности и транзистора (коллектор-эмиттер).

Перейдём к безразмерным величинам.
При переходе к безразмерному виду следует определить,
какие физические параметры определяют безразмерные величины.
Это потребуется для синтеза критерия идентификации.
Для упрощения рассмотрения, не снижая общности,
будем считать $C_1 = C_2 = C$.

Прежде всего, воспользуемся тем, что система содержит только один
активный нелинейный компонент --- транзистор.
Следовательно, именно этот элемент определяет
масштаб по напряжению. В простейшей модели транзистора
такой масштабной величиной может служить
$V_{je}$ -- падение напряжения на переходе база-эмиттер
в активном режиме. Следовательно, все разности потенциалов в схеме можно нормировать
на эту величину.

Динамические свойства (в т.ч. условия начала генерации и перехода в хаотический режим) определяются
соотношением активных и реактивных свойств системы. При этом величина
$ \rho = \sqrt{L/C} $ имеет размерность сопротивления
и определяет величину реактивного сопротивления. Эту величину можно использовать
для обезразмеривания активных сопротивлений.

Для приведения токов к безразмерному виду, с учётом уже выбранных величин,
следует использовать величину $ V_{je} / \rho$.


Исходя из всего вышеперечисленного, обозначим:
%
\[
  x = \frac{V_{1}}{V_{je}} ; \quad
  y = \frac{\rho I_L}{V_{je}} ; \quad
  z = \frac{V_{2}}{V_{je}}, \quad
  i_{ce} = \frac{\rho I_{ce}}{V_{je}}, \quad
  c = \frac{V_{CC}}{V_{je}}, \quad
  e = \frac{V_{b}}{V_{je}}.
\]
%
\[
  b = \frac{R_c}{\rho}; \quad
  d = \frac{\rho}{R_e}. % sic!
\]

Система уравнений принимает вид:
%
\begin{equation}
\label{atu:eq:colp_phys2}
\begin{dcases}
  \od{x}{t}  = \dfrac{1}{\rho C}  y - \dfrac{1}{\rho C} i_{ce} , \\
  \od{y}{t}  = \dfrac{\rho}{L} c    - \dfrac{\rho}{L} r_c y - \dfrac{\rho}{L} x- \dfrac{\rho}{L} z, \\
  \od{z}{t}  = \dfrac{1}{\rho C}  y - \dfrac{1}{\rho C} \dfrac{1}{r_e} z.
\end{dcases}
\end{equation}

Общий множитель $ \frac{1}{\rho C} = \frac{\rho}{L} = \sqrt{\frac{1}{LC}} $ в правых частях уравнений
естественным образом задаёт масштаб по времени.
Это подчёркивает, что частотные характеристики рассматриваемого генератора,
в отличие, например, от релаксационного,
определяются ёмкостью и индуктивностью,
поэтому масштаб времени задаём так:
$ T_s = \sqrt{L C} $.
Тогда безразмерное время $t_s$
и соответствующие производные
будут определяться как:
%
\[
  t_s = \frac{t}{T_s}; \quad
  \mathrm{d}\, t = T_s \mathrm{d}\, t_s; \quad
  \od{}{t}  = \frac{1}{T_s} \od{}{t_s}; \quad
  \od{x}{t_s} \equiv \dot{x} = T_s \od{x}{t} .
\]

Поведение величины $I_{ce}$ (в дальнейшем просто $I_c$) достаточно хорошо описывает модель
Эберса-Молла~\cite{horowitz}:
%
\begin{equation}
  I_c = I_s \left( \exp\frac{V_{be}}U_t{} - 1 \right),
  \label{atu:eq:ebers-moll}
\end{equation}
%
%\noindent
где
$I_s$ -- ток насыщения (паспортная или определяемая экспериментально величина),
$U_t=kT/q$,
$q = \SI{1.6e-19}{\coulomb}$ -- заряд электрона,
$k = \SI{1.38e-23}{\joule/\kelvin}$ -- постоянная Больцмана.
Необходимо учесть, что в режиме отсечки ($V_b < V_e$) ток коллектора пренебрежимо мал,
а в режиме насыщения определяется другими элементами схемы.
Существуют и более сложные модели, например,
программы для моделирования электронных схем часто используют так называемую SPICE модель.

К сожалению, при моделировании генератора Колпитца в литературе,
посвящённой хаотической динамике, используют
простейшую модель транзистора, считая, что переход
база-эмиттер открывается при $V_{BE} = V_{je}$, $ I_c \gg I_b$,
а ток коллектора
%
\begin{equation}
I_c =
  \begin{cases}
    \alpha ( V_b - V_e - V_{je} ), & V_b - V_e > V_{je} \\
    0                              & \text{otherwise}.
  \end{cases}
  \label{atu:eq:bjt_libear_model}
\end{equation}


Для оценки адекватности моделирования, в также пригодности
такой упрощённой модели, в первую очередь
рассмотрим модель системы с нелинейной частью вида~(\ref{atu:eq:bjt_libear_model}).
С учётом всего вышеизложенного получаем следующую систему уравнений:
%
\begin{equation}
\label{atu:eq:colp}
\begin{cases}
  \dot{x} = y - a F(z), \\
  \dot{y} = c - x - by - z, \\
  \dot{z} = y - d z.
\end{cases}
\end{equation}

При этом параметр $b$ характеризует соотношение
активного и реактивного сопротивления,
и, следовательно, режима работы генератора.
Величиной этого параметра проще всего управлять,
изменяя $R_c$.
Ставится задача идентификации данного параметра.


Для предварительной проверки адекватности данной модели
был создан физический генератор Колпитца.
Более подробная информации об этом будет приведена далее.
При физическом моделировании, проведённом в рамках данной работы использовались следующие
элементы с соответствующими параметрами:
%
\[
  V_{cc} = \SI{12.06}{\volt},          \;
  R_1 = R_2 = \SI{2.2}{\kilo\ohm},     \;
  R_e = \SI{430}{\ohm},
\]
%
\[
  C_1 = C_2 = \SI{1.03}{\micro\farad}, \;
  L = \SI{6.22}{\milli\henry},         \;
  T = \SI{305}{\kelvin},
\]
%
\[
  \text{Q: 2N2222A}, \quad
  h_{fe}=285, \;
  V_f = \SI{0.677}{\volt}, \;
  I_s = \SI{9.61e-14}{\ampere}, \;
  \alpha \approx 1.
\]

Тогда безразмерные коэффициенты:
\[
 a = 77,     \quad
 c = 18.08,  \quad
 d = 0.19,   \quad
 e = 9.07.
\]
%
\[
F(z) =
\begin{cases}{l}
  e-1-z, & z \le e-1  \\
  0,     & z  >  e-1
\end{cases}.
\]


Диапазон изменения идентифицируемого параметра
$b \in [ 0.02; 4.2 ]$
определяется, с одной стороны, собственным сопротивлением катушки индуктивности,
с другой -- срывом генерации.


% }}}2


\subsection{Влияние параметров системы идентификации на ошибку идентификации для системы Колпитца}  % {{{2



На рис.~\ref{atu:f:colp_real_xzz}--\ref{atu:f:colp_model_f} представлены как результаты реального эксперимента,
так и данные, полученные в результате численного моделирования динамики системы (\ref{atu:eq:colp}).
Представлены проекции аттракторов на плоскость $(x+z,z)$ (естественный вид для осциллографа),
трёхмерный вид аттракторов, и спектры.
На каждом рисунке представлено три режима: обычный, момент первого удвоения периода и хаотический режим.


\begin{figure}[htb!]
 \centerline{
   \includegraphics[width=0.32\textwidth]{p/mod/colp_m1_vv.png}
   \includegraphics[width=0.32\textwidth]{p/mod/colp_m2_vv.png}
   \includegraphics[width=0.32\textwidth]{p/mod/colp_m3_vv_ac.png}
 }
  \caption{Проекции аттракторов реальной системы Колпитца на плоскость $(x+z,z)$
  для трёх режимов}
  \label{atu:f:colp_real_xzz}
\end{figure}

\begin{figure}[htb!]
 \centerline{
   \includegraphics[width=0.32\textwidth]{p/mod/colp_0-p_z_xpz_b=1x70.png}
   \includegraphics[width=0.32\textwidth]{p/mod/colp_0-p_z_xpz_b=1x37.png}
   \includegraphics[width=0.32\textwidth]{p/mod/colp_0-p_z_xpz_b=0x99.png}
 }
  \caption{Проекции аттракторов модели (\ref{atu:eq:colp}) системы Колпитца на плоскость $(x+z,z)$
  для трёх режимов}
  \label{atu:f:colp_model_xzz}
\end{figure}


\begin{figure}[htb!]
 \centerline{
   \includegraphics[width=0.32\textwidth]{p/mod/colp_0-p_xyz_b=1x70.png}
   \includegraphics[width=0.32\textwidth]{p/mod/colp_0-p_xyz_b=1x37.png}
   \includegraphics[width=0.32\textwidth]{p/mod/colp_0-p_xyz_b=0x99.png}
 }
  \caption{Аттракторы модели (\ref{atu:eq:colp}) системы Колпитца для трёх режимов}
  \label{atu:f:colp_model_xyz}
\end{figure}

\begin{figure}[htb!]
 \centerline{
   \includegraphics[width=0.32\textwidth]{p/mod/colp_m1_f.png}
   \includegraphics[width=0.32\textwidth]{p/mod/colp_m2_f.png}
   \includegraphics[width=0.32\textwidth]{p/mod/colp_m3_f.png}
 }
  \caption{Спектры реальной системы Колпитца  для трёх режимов}
  \label{atu:f:colp_real_f}
\end{figure}

\begin{figure}[htb!]
 \centerline{
   \includegraphics[width=0.32\textwidth]{p/mod/colp_f-p_f_b=1x70.png}
   \includegraphics[width=0.32\textwidth]{p/mod/colp_f-p_f_b=1x37.png}
   \includegraphics[width=0.32\textwidth]{p/mod/colp_f-p_f_b=0x99.png}
 }
  \caption{Спектры модели (\ref{atu:eq:colp}) системы Колпитца для трёх режимов}
  \label{atu:f:colp_model_f}
\end{figure}

Сравнение результатов физического и численного моделирования позволяет сделать вывод
о качественном подобии поведения реальной системы и модели.
Тем не менее, величины параметра $b$, при которых получены
рассматриваемые режимы, совпадают весьма приближённо.
Для реальной системы значения параметра: $b = 1.06, \; 0.94, \; 0.90 $, в то время как
для модели: $b = 1.70, \; 1.34, \; 0.99 $.
Расхождение значений, скорее всего, связано с грубостью модельного представления (\ref{atu:eq:bjt_libear_model}) транзистора.
Вопрос применимости различных моделей биполярного транзистора в рамках рассматриваемой системы
требует дальнейшего исследования.
С другой стороны, ограниченный набор данных, получаемых с осциллографа Rigol DS1052E (8192 отсчёта) не позволяют
получить достаточно подробный спектр реальной системы.

Для определения критерия идентификации рассмотрим зависимости
$q(\mu) $, полученные путём моделирования
для системы Колпитца (рис.~\ref{atu:f:colp_q}).
При этом следует учесть, что наиболее просто измеряемыми величинами являются $x$ и $z$,
соответствующие напряжениям $V_1$ и $V_2$.
Первый набор зависимостей даёт два основных кандидата -- $q_{x^2}$ и $q_{z^2}$.
При этом первый из них показывает более равномерную зависимость.
В то же время, большинство из рассмотренных зависимостей имеют явно
выраженный гиперболический характер, особенно при малых значениях $b$.
Следовательно, в список кандидатов следует добавить $q_{x^{-2}} $ и $q_{z^{-2}}$.

\begin{figure}[htb!]
\centerline{
  \includegraphics[width=0.49\textwidth]{p/mod/colp_p-p_b_e.png}
  \includegraphics[width=0.49\textwidth]{p/mod/colp_p-p_b_1ex2.png}
}
  \caption{Зависимости $q(b) $ для системы Колпитца (\ref{atu:eq:colp})}
\label{atu:f:colp_q}
\end{figure}

Из графиков очевидно, что
обратные зависимости не дают заметного выигрыша, поэтому выберем
величину $ q_{x^2}(b) $ в качестве критерия.

В качестве системы идентификации использовалась система с пятью поисковыми агентами и
двумя неподвижными моделями. Аналогично предыдущим системам,
для исследования динамических свойств системы идентификации
изменения параметра $b_o$ как:
%
\begin{equation}
 b_o(t) = p_0 + U_p \sign \sin( \omega_p t ),
  \label{atu:eq:colp_b_sign}
\end{equation}
%
\begin{equation}
 b_o(t) = p_0 + U_p \sin( \omega_p t ).
  \label{atu:eq:colp_b_sin}
\end{equation}

Динамика процессов идентификации для системы Колпитца представлена на рис.~\ref{atu:f:colp_id}.

\begin{figure}[htb!]
\centerline{
  \includegraphics[width=0.49\textwidth]{p/mod/colp_m5p-pl_n_sign.png}
  \includegraphics[width=0.49\textwidth]{p/mod/colp_m5p-pl_n_sin.png}
}
\caption{Процесс идентификации параметра $b$ системы (\ref{atu:eq:colp})
  при условиях (\ref{atu:eq:colp_b_sign}) и (\ref{atu:eq:colp_b_sin})
}
\label{atu:f:colp_id}
\end{figure}

Зависимость среднеквадратических ошибок идентификации от величины $q_\gamma$ (рис.~\ref{atu:f:colp_e_qgamma})
даёт информацию о корректной настройке этого параметра системы идентификации.

\begin{figure}[htb!]
\centerline{
  \includegraphics[width=0.49\textwidth]{p/mod/colp_m5p-p_qg_e_sign.png}
  \includegraphics[width=0.49\textwidth]{p/mod/colp_m5p-p_qg_e_sin.png}
}
  \caption{Зависимости  $\overline{e}(q_\gamma)$ для системы (\ref{atu:eq:colp})
  при условиях (\ref{atu:eq:colp_b_sign}) и (\ref{atu:eq:colp_b_sin})
}
\label{atu:f:colp_e_qgamma}
\end{figure}



Аналогично, зависимости $\overline{e_*}(a_q)$ (рис.~\ref{atu:f:colp_e_a_q})
позволяют корректно определить время усреденения.
При этом следует отметить, что
полученные результаты хорошо согласуются со спектрами системы.

\begin{figure}[htb!]
\centerline{
  \includegraphics[width=0.49\textwidth]{p/mod/colp_m5p-p_a_q_e_sign.png}
  \includegraphics[width=0.49\textwidth]{p/mod/colp_m5p-p_a_q_e_sin.png}
}
  \caption{Зависимости  $\overline{e}(a_q)$ для системы (\ref{atu:eq:colp})
  при условиях (\ref{atu:eq:colp_b_sign}) и (\ref{atu:eq:colp_b_sin})
}
\label{atu:f:colp_e_a_q}
\end{figure}

% }}}2




% }}}1

\section{Физическая реализация генератора Колпитца и исследование её динамики} % {{{1


Как уже было отмечено,
рассмотренная  модель
генератора Колпитца, несмотря на её широкое использование
при исследованиях, связанных с хаотической динамикой,
имеет существенные отличия от реального генератора.
Подавляющие большинство элементов достаточно хорошо описываются линейным
приближением. Однако, есть два элемента,
применение линейной модели для которых или же вообще невозможно (транзистор),
или же допустимо при определённых условиях (индуктивность).




На рис.~\ref{atu:f:colp_schem_real} представлена электрическая схема
реализации генератора Колпитца, используемая в данной работе.
Отличия от схемы, используемой в главе \ref{atu:ch:testsys}
и представленной на рис.~\ref{atu:f:colp_schem}
не принципиальны с точки зрения модели,
и предназначены как для упрощения подключения
измерительных устройств, так и для
обеспечения стабильности определённых потенциалов схемы.

\begin{figure}[htb!]
\centerline{\includegraphics[width=0.8\textwidth]{p/colp_schem_real.png} }
\caption{Электрическая схема реального генератора Колпитца, используемого в данной работе}
\label{atu:f:colp_schem_real}
\end{figure}

Для стабилизации потенциалов $V_{cc}$ и $V_b$
в широком диапазоне частот использовались пары конденсаторов, соответственно
$\mathrm{C}_3$, $\mathrm{C}_4$ и
$\mathrm{C}_5$, $\mathrm{C}_6$. Конденсатор большей ёмкости
был электролитический, меньшей --- с керамическим диэлектриком.
Непосредственно в модели такой изменение схемы отражения не находит.
Напротив, эти изменения позволяют с меньшими допущениями считать
в модели вышеупомянутые потенциалы постоянными.

Резистор из исходной схемы $\mathrm{R}_c$,
в реальной схеме представлен двумя переменными резисторами
$\mathrm{R}_{cv1}$ и
$\mathrm{R}_{cv2}$. Это позволяет получить как широкий диапазон значений $\mathrm{R}_c$,
так и возможность тонкой настройки.
Следует отметить, что в полное сопротивление $\mathrm{R}_c$
входит и активное сопротивление катушки индуктивности $\mathrm{L}_{1}$.

Разъёмы $\mathrm{P}_1$, $\mathrm{P}_\mathrm{gnd}$ и $\mathrm{P}_\mathrm{Vcc}$
предназначены как для обеспечения электрическим питанием как самого генератора,
так и других приборов, которые участвуют в измерении.
Разъём $\mathrm{P}_4$ предназначен для получения сигнала $V_e(t)$.
Разъём $\mathrm{P}_2$ позволяет контролировать стабильность
потенциала $V_b$, однако, в процессе измерения этот сигнал не записывался.
Разъём с набором перемычек $\mathrm{JP}_2$, в первую очередь,
используется для получения сигнала $V_c$. Помимо этого,
он позволяет отключить от схемы конденсатор $\mathrm{C}_1$,
тем самым прекратить генерацию и исследовать поведение схемы в стационарном состоянии.

Разъём с набором перемычек $\mathrm{JP}_1$ также выполняет несколько функций.
Убрав с него перемычку, можно измерить полное сопротивление $\mathrm{R}_{c}$,
что даёт возможность сравнить результат идентификации с реальным значением.
Также важным свойством является возможность с помощью этого разъёма
заменить пару $\mathrm{R}_{c}$, $\mathrm{L}_{1}$
на другую, что расширяет возможности эксперимента.

Разъём  $\mathrm{P}_{3}$, с одной стороны, позволяет
ввести параллельное сопротивление к
$\mathrm{R}_{cv1}$ и
$\mathrm{R}_{cv2}$, что необходимо для исследования динамических
свойств системы идентификации. С другой стороны,
измерение падения напряжения на измененном сопротивлении $\mathrm{R}_{cv1} +\mathrm{R}_{cv2}$
позволяет оценить величину $I_c(t)$, эквивалент безразмерной величины $y(t)$.
В этом случае нельзя говорить об идентификации, но данный сигнал может быть полезен
при построении аттрактора по наблюдаемым данным.


Принципиальная схема и печатная плана были спроектированы в
программном комплексе ``kicad''. % TODO: ref
Печатная плата изготовлена с использованием негативного фоторезиста.
Внешний вид собранной схемы на печатной плате представлен на~(рис.~\ref{atu:f:relax3d_board}).

\begin{figure}[htb!]
\centerline{\includegraphics[width=0.5\textwidth]{p/colp_board.jpg} }
\caption{Печатная плата генератора Колпитца}
\label{atu:f:colp_board}
\end{figure}


Как уже отмечалось, использование в качестве регистратора данных
цифрового осциллографа класса Rigol DS1052E,
несмотря на  хорошее разрешение во временной области ($\SI{10}{\nano\second}$),
скорость и простоту получения вида аттрактора,
имеет ряд существенных недостатков.

В первую очередь, за исключением режима предельной частоты дискретизации,
нет возможности получить достаточно длинную выборку, что
сильно ограничивает возможности спектрального анализа.
При анализе сигналов хаотических и близких к ним систем
особенно важным является спектральное разрешение,
определяемое количеством наблюдений при заданной частоте дискретизации.
При недостаточном разрешении нет способа отличить участок сплошного спектра
от линейчатого.

Также, ограниченная разрядность входных АЦП (8 бит)
приводит к относительно высокому уровню шумов квантования.
Также, наличие только двух каналов измерения не позволяет
получить вид аттрактора для системы с тремя измерениями.

Для устранения этих недостатков был создан
измерительный комплекс на основе
микроконтроллера STM32F746.
Ядром системы служила отладочная плата Waveshare STM32F746.
Помимо собственно контроллера, на ней установлена оперативная память
SDRAM 8MB, что позволило сохранять в сумме до 4 миллионов отсчётов.
Использовалось до четырёх каналов 12-ти АЦП с разрешением 12 бит.
С учётом предела $2 \cdot 10^6$ отчётов в секунду
(с использованием DMA для пересылки полученных данных в память и таймера как сигнала для начала измерений),
максимальная достижимая частота дискретизации в этих условиях достигает $\SI{500}{\kilo\hertz}$,
что для исследуемой системы несколько избыточно,
но позволяет убедится в том, что не была потеряна часть спектра.

Для управления микроконтроллером была создана программа
на языке C++. Так как для данной задачи требуется
квази-одновременное, так и полностью одновременное выполнение
различных действий, то была использована миниатюрная операционная
система реального времени для микроконтроллеров FreeRTOS.
Управление программой осуществлялось в режиме терминала
с использование протокола UART. Реализованная
в программе система команд позволяет настраивать параметры измерения,
проводить само измерение, и сохранять результат.

В качестве основного способа сохранения результата
использовалась SDHC карта, подключенная по протоколу SDIO с одной линией данных.
Следует отметить, что при проведении измерения обязательным условием
является отключение SDIO подсистемы контроллера.
В противном случае уровень шумов измерения не позволяет
получить значимые результаты.
Альтернативным способом сохранение данных является
использование того же канала, который используется для передачи команд
и приёма ответа на них --- UART. Для сохранения результатов в файл
применяются средства используемой программы эмуляции терминала.
Однако, относительно низкая скорость такого способа передачи данных
требует длительного времени для сохранения результатов.

Важным условием получения данных является согласование уровней
сигналов в исследуемой системе с допустимыми входными уровнями АЦП.
Амплитудное значение сигналов в рассматриваемой системе достигало
$\SI{15}{\volt}$, в то время как входное напряжение АЦП не должно
превышать $\SI{3}{\volt}$.
Использование для согласования резистивного делителя
не оправданно. Малое полное сопротивление делителя
искажает динамику исследуемой системы, особенно в условиях
хаотических колебаний. Большие значения приводят
к росту шумов измерения и невозможности
использования защитных средств, например, пар диодов,
для защиты входных цепей контроллера.

Поэтому, для согласования использовалась микросхема TL074,
объединяющая в одном корпусе 4 операционных усилителя,
которые характеризуются высоким входным сопротивлением ($\approx \SI{1e12}{\ohm}$),
низким уровнем нелинейных искажений ($\approx 0.003\%$),
подходящим для данной задачи частотным диапазоном
и допустимыми входными напряжениями.
Каждый из усилителей включался в режиме повторителя~\cite{horowitz},
и на его выход был подключён резистивный делитель
с подобранными сопротивлениями и два защитных диода.

Перед каждой серией экспериментов проводилась калибровка
измерительного комплекса на набор заданных значений напряжений.
После измерений проводилась проверка на стабильность
параметров, и в случая отрицательного результата
данные отбрасывались.


Примеры динамики с сравнением.

% }}}1

\section{Численная модель генератора Колпитца}  % {{{1

Методы расчёта~\cite{zaeplnii_radio_calc}.

% }}}1

\section{Анализ и выбор критериев}  % {{{1

Проверяем старые.

Исследуем всякие.

\begin{figure}[htb!]
\centerline{\includegraphics[width=0.7\textwidth]{p/colp_bjt_q-p_Rc_q.png} }
\caption{Зависимости значений критериев идентификации для модели системы Колпитца}
\label{atu:f:colp_bjt_q-p_Rc_q}
\end{figure}


\begin{figure}[htb!]
\centerline{\includegraphics[width=0.7\textwidth]{p/colp_read_q-p_Rc_q.png} }
  \caption{Зависимости значений критериев идентификации $q_{x^2}$, $q_{z^2}$ и $q_{xz}$ для реального генератора Колпитца}
\label{atu:f:colp_read_q-p_Rc_q-p_Rc_q}
\end{figure}

\begin{figure}[htb!]
\centerline{\includegraphics[width=0.7\textwidth]{p/colp_q_cml.png} }
\caption{Сравнение зависимостей значений критерия идентификации $q_{xz}$ реального генератора Колпитца с моделью}
\label{atu:f:colp_q_cml}
\end{figure}

Предлагаем новые.

% }}}1

\section{Синтез и анализ и системы идентификации}  % {{{1

Описание с картинкой.

Для процессов параметрической идентификации генератора Колпитца представлены результаты четырёх экспериментов,
проведённых в аналогичных условиях.

% C_1, C_2 = 1.0 uF, L = 1.0 mH, R_l=24.2 Ohm, R_e = 300 Ohm, R_1 = R_2 = 2.0 kOhm,
% Q: MMBT2222A: h_{FE}=210, V_{je} = 0.667V.
% V_{cc} = 12.0 V, V_{b} = 5.94 V
% d_0.txt: 27.0 - 50.0 Ohm  28.5  +-  11.5
% d_1.txt: 29.5 - 35.0 Ohm  32.25 +-  2.75
% d_2.txt: 32.0 - 40.0 Ohm  36.0  +-  4.0
% d_3.txt: 28.0 - 37.0 Ohm  32.5  +-  4.5
% old:
% d_0.txt: 27.3 - 44.2 Ohm  35.75 +-  8.45
% d_1.txt: 29.0 - 35.0 Ohm  32    +-  3
% d_2.txt: 26.0 - 30.0 Ohm  28    +-  2
% d_3.txt: 31.0 - 52.0 Ohm  41.5  +- 10.5
% d_4.txt: 29.0 - 32.0 Ohm  30.5  +-  1.5

Параметры первого эксперимента для генератора Колпитца
%
\begin{equation}
  \begin{array}{c}
    R_{c,\min} = \SI{27.0}{\ohm};
    \;
    R_{c,\max} = \SI{50.0}{\ohm};
    \;
    T_{R_c} = \SI{200}{\milli\second};
  \\
    R_c(t) = \SI{28.5}{\ohm} + \SI{11.5}{\ohm} \cdot \sign \sin \left(  \frac{\pi t}{T_{R_c}}  \right).
  \end{array}
  \label{atu:eq:colp_test1_cond}
\end{equation}

\begin{figure}[htb!]
  \centerline{
    \includegraphics[width=0.48\textwidth]{p/r/colp_real_id-p_t_pi_ql3rlWvnAAW_real_d_0.png}
    \hfill
    \includegraphics[width=0.48\textwidth]{p/r/colp_real_id-p_t_p_ql3rlWvnAAW_real_d_0.png}
  }
  \caption{Процесс идентификации параметра $R_c$ реального генератора Колпитца при условиях \ref{atu:eq:colp_test1_cond} }
  \label{atu:f:colp_r_id_1}
\end{figure}


Параметры второго эксперимента для генератора Колпитца
%
\begin{equation}
  \begin{array}{c}
    R_{c,\min} = \SI{29.5}{\ohm};
    \;
    R_{c,\max} = \SI{35.0}{\ohm};
    \;
    T_{R_c} = \SI{200}{\milli\second};
  \\
    R_c(t) = \SI{32.25}{\ohm} + \SI{2.75}{\ohm} \cdot \sign \sin \left( \frac{\pi t}{T_{R_c}}  \right).
  \end{array}
  \label{atu:eq:colp_test2_cond}
\end{equation}

\begin{figure}[htb!]
  \centerline{
    \includegraphics[width=0.48\textwidth]{p/r/colp_real_id-p_t_pi_ql3rlWvnAAW_real_d_1.png}
    \hfill
    \includegraphics[width=0.48\textwidth]{p/r/colp_real_id-p_t_p_ql3rlWvnAAW_real_d_1.png}
  }
  \caption{Процесс идентификации параметра $R_c$ реального генератора Колпитца при условиях \ref{atu:eq:colp_test2_cond} }
  \label{atu:f:colp_r_id_2}
\end{figure}


Параметры третьего эксперимента для генератора Колпитца
%
\begin{equation}
  \begin{array}{c}
    R_{c,\min} = \SI{32.0}{\ohm};
    \;
    R_{c,\max} = \SI{40.0}{\ohm};
    \;
    T_{R_c} = \SI{200}{\milli\second};
  \\
    R_c(t) = \SI{36.0}{\ohm} + \SI{4.0}{\ohm} \cdot \sign \sin \left(   \frac{\pi t}{T_{R_c}}   \right).
  \end{array}
  \label{atu:eq:colp_test3_cond}
\end{equation}


\begin{figure}[htb!]
  \centerline{
    \includegraphics[width=0.48\textwidth]{p/r/colp_real_id-p_t_pi_ql3rlWvnAAW_real_d_2.png}
    \hfill
    \includegraphics[width=0.48\textwidth]{p/r/colp_real_id-p_t_p_ql3rlWvnAAW_real_d_2.png}
  }
  \caption{Процесс идентификации параметра $R_c$ реального генератора Колпитца при условиях \ref{atu:eq:colp_test3_cond} }
  \label{atu:f:colp_r_id_3}
\end{figure}



Параметры четвёртого эксперимента для генератора Колпитца
%
\begin{equation}
  \begin{array}{c}
    R_{c,\min} = \SI{28.0}{\ohm};
    \;
    R_{c,\max} = \SI{37.0}{\ohm};
    \;
    T_{R_c} = \SI{200}{\milli\second};
  \\
    R_c(t) = \SI{32.5}{\ohm} + \SI{4.5}{\ohm} \cdot \sign \sin \left( \frac{\pi t}{T_{R_c}}  \right).
  \end{array}
  \label{atu:eq:colp_test4_cond}
\end{equation}

\begin{figure}[htb!]
  \centerline{
    \includegraphics[width=0.48\textwidth]{p/r/colp_real_id-p_t_pi_ql3rlWvnAAW_real_d_3.png}
    \hfill
    \includegraphics[width=0.48\textwidth]{p/r/colp_real_id-p_t_p_ql3rlWvnAAW_real_d_3.png}
  }
  \caption{Процесс идентификации параметра $R_c$ реального генератора Колпитца при условиях \ref{atu:eq:colp_test4_cond} }
  \label{atu:f:colp_r_id_4}
\end{figure}



Рассмотрим зависимости ошибок идентификации от параметров самой системы идентификации.
Параметр $a_q$, задающий
фильтрующие способности критерия идентификации, должен рассматриваться в первую очередь.
На рис.~\ref{atu:f:colp_real_id_p_a_q_d_0} представлены полученные зависимости
для процессов идентификации параметра $R_c$ генератора Колпитца.

\begin{figure}[htb!]
  \centerline{\includegraphics[width=0.6\textwidth]{p/r/colp_real_id-p_a_q_d_0.png} }
  \caption{Зависимости $\overline{e}(a_q)$ при идентификации параметра $R_c$ генератора Колпитца при условиях \ref{atu:eq:colp_test1_cond} }
  \label{atu:f:colp_real_id_p_a_q_d_0}
\end{figure}


\begin{figure}[htb!]
  \centerline{\includegraphics[width=0.6\textwidth]{p/r/colp_real_id-p_q_gamma_d_0.png} }
  \caption{Зависимости $\overline{e}(q_\gamma)$ при идентификации параметра $R_c$ генератора Колпитца при условиях \ref{atu:eq:colp_test1_cond} }
  \label{atu:f:colp_real_id_p_q_gamma_d_0}
\end{figure}

\begin{figure}[htb!]
  \centerline{\includegraphics[width=0.6\textwidth]{p/r/colp_real_id-p_v_f_d_0.png} }
  \caption{Зависимости $\overline{e}(v_f)$ при идентификации параметра $R_c$ генератора Колпитца при условиях \ref{atu:eq:colp_test1_cond} }
  \label{atu:f:colp_real_id_p_v_f_d_0.png}
\end{figure}


\begin{figure}[htb!]
  \centerline{\includegraphics[width=0.6\textwidth]{p/r/colp_real_id-p_k_e_d_0.png} }
  \caption{Зависимости $\overline{e}(k_e)$ при идентификации параметра $R_c$ генератора Колпитца при условиях \ref{atu:eq:colp_test1_cond} }
  \label{atu:f:colp_real_id_p_k_e_d_0.png}
\end{figure}


\begin{figure}[htb!]
  \centerline{\includegraphics[width=0.6\textwidth]{p/r/colp_real_id-p_k_nl_d_0.png} }
  \caption{Зависимости $\overline{e}(k_{nl})$ при идентификации параметра $R_c$ генератора Колпитца \ref{atu:eq:colp_test1_cond} }
  \label{atu:f:colp_real_id_p_k_nl_d_0.png}
\end{figure}
%
%
% \begin{figure}[htb!]
%   \centerline{\includegraphics[width=0.6\textwidth]{p/colp_real_id_qi_fv5_prm_0-p_k_cn.png} }
%   \caption{Зависимости $\overline{e}_{r*}(c_n)$ при идентификации параметра $R_c$ генератора Колпитца}
%   \label{atu:f:colp_real_id_qi_fv5_prm_0-p_k_cn.png}
% \end{figure}

% }}}1


\section{Выводы по разделу \thechapter}  % {{{1


В целом синтез критерия идентификации и построение работоспособной системы идентификации для
системы генератора Колпитца не потребовало никаких специальных подходов и введения дополнительных
мер оценивания взаимосвязей параметров генератора.

% }}}1

% vim: fdm=marker foldlevel=0 foldignore="%#" fdc=4 ft=tex
