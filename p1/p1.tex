\chapter{Обзор состояния проблемы и постановка задачи исследования}

Современный уровень науки и технологий характеризуется
широким использованием нелинейных динамических
систем~\cite{andronov_vitt_haikin,anisch_nonlin_eff,mishenko_du_small_relax,nonlin_vibro}.
При этом нелинейные эффекта присущи как объектам,
так и системам управления~\cite{kubik_nlsc,vukobr_nonadopt}. Динамика нелинейных систем,
характеризуется общим уровнем сложности моделей,
в большинстве случаев не допускающего
точного аналитического представления.
Помимо этого, был обнаружен, и в последнее время представляет
постоянный интерес для исследователей класс систем,
проявляющих хаотическую динамику~\cite{moon_chaotic_vibr,magni_theory_dyn_chaos,kuznetsov_dyn_chaos,neimark_stoch_chaos_vibro}.
Для таких систем, вследствие чувствительности к начальным условиям
и параметрам, характерным свойством является наличие ограниченного горизонта прогноза как
в прошлое, так и в будущее, рост энтропии
и необратимость процессов~\cite{chernavskii_syn_info,prigogine_from_existent,koltsova_nl_dyn_chem}.
Это свойство делает неприменимыми
большинство методов, разработанных для описания, управления и
идентификации динамических систем. Также было замечено,
что некоторые известные динамические системы, в том числе широко
используемые в научной, производственной и технологической практике,
зарекомендовавшие себя как системы регулярной динамики,
при определённых условиях демонстрируют хаотическое поведение.
Хаотически свойства проявляют системы как классической,
так и квантовой механики.
Новые свойства подобных систем требуют создания новых
методов для решения задач моделирования, управления и идентификации.

\section{Свойства систем хаотической динамики применительно к задаче идентификации}  % {{{1

Одним из основных свойств систем хаотической динамики является чувствительность
к малым возмущениям сходных сигналов, состояния системы или же параметров.
В линейном приближении это соответствует положительным значениям
коэффициентов Ляпунова~\cite{magni_theory_dyn_chaos,moon_chaotic_vibr}.


% }}}1

\section{Основные понятия задачи идентификации динамических систем} % {{{1

Одно из первых определений задачи идентификации было дано
Л.~Заде~\cite{zadeh_id_1956}.

Определение Райбмана.

Под идентификацией

Определение Растригина.


% }}}1



\section{Существующие методы идентификации нелинейных динамических систем и их свойства}  % {{{1

Основы идентификации систем управления~\cite{eykhoff_id_base,eykhoff_modern_id,gropp_methods_id,deith_method_id_ds,lung_id_sys,seidg_id_su,leondes_modern_tu,nelles_nlsys_id}

Введение в идентификацию объектов управления~\cite{rastr_intro,rastr_adop_complex_sys}.

Случайный поиск в задачах оптимизации многопараметрических систем~\cite{rastr_rand_search,rastr_rand_search_adopt}.

Статистические методы поиска~\cite{rastr_stat_meth_search}

Беспоисковые самонастраивающиеся системы~\cite{kozlov_nosearch_sns}

Системы экстремального управления~\cite{rastr_seu,kras_dyn_nsn}

Основы информационной теории идентификации~\cite{info_cipkin,straton_inf,karabut}

Адаптация в непрерывных системах автоматического поиска~\cite{adopt_cont_sys}

Карабутов~\cite{karabutov_adapt_id_sys,saliga_id_ctl_black}

Автоматическая оптимизация в задачах пространственного распределения~\cite{auto_optim_intask}

Адаптивно-поисковые методы и алгоритмы оптимизации и идентификации динамических систем~\cite{mich_92}

% }}}1


\section{Постановка задачи идентификации}  % {{{1

Пусть задан динамический объект $ \mathbf{O}$, характеризуемый параметром $p_o(t)$.
Информация об этом параметре может быть представлена
различными способами. Например, может быть задан только диапазон
$[p_{\min}, p_{\max}]$,
возможных значений параметра,
этот же диапазон может иметь зависимость от времени:
$[p_{\min}(t), p_{\max}(t)]$.
Множество допустимых значений параметра обозначим через $\mathcal{P}$.
На основании априорных исследований может
быть задана как плотность вероятности для значений параметра,
так и зависимость этой вероятности от времени. Наличие плотности вероятности
позволяет использовать информационные оценки идентификации~\cite{info_cipkin,atu_asau10},
но не является необходимым условием для синтеза системы идентификации в целом.
Более того, для получения достаточно точных информационных оценок
требуется затратить значительно больше ресурсов на проведение моделирования,
чем на собственно идентификацию.
Также могут быть заданы дополнительные ограничения, такие как максимальна скорость
изменения параметра и т.д.

Пусть также существует конечное множество моделей
\label{atu:d:N}$\mathbf{M}_i$, $i=0 \ldots N-1$.
Структура моделей предполагается известной и одинаковой,
а параметру объекта $p_o(t)$ соответствуют параметры моделей $p_{i}(t)$.

На вход как моделей, так и объекта подаётся входной сигнал \label{atu:d:u}$u(t)$.
Частный случай, когда ни объект, ни модели не требуют входного сигнала
не нарушает общности постановки. В некоторых случаях
не следует пренебрегать ошибкой измерения входного сигнала $w_u(t)$,
однако, при анализе динамики систем идентификации в целом,
этой величиной, как правило, пренебрегают.

В общем случае $u(t)$, $x(t)$, $p(t)$ -- векторные величины.
В тех случаях, когда размерность $x(t)$ не выше трёх,
компоненты этого вектора будем обозначать соответственно $x(t)$, $y(t)$, $z(t)$.

Выходные сигналы моделей
\label{atu:d:x}$x_i(t)$ считаем известными точно, так как ошибки
представления значений при численных вычислениях заведомо
пренебрежимо малы с точностью измерений. Напротив,
выход объекта $x_{op}(t)$ считаем измеренным
с определённой ошибкой измерения \label{atu:d:w}$w(t)$:
%
\[
  x_o(t) = x_{op}(t) + w(t).
\]
%
Сигнал $w(t)$ обычно задаётся как случайный сигнал с
известным видом распределения, а также параметрами этого распределения,
например, нормальное распределения с среднеквадратичным отклонением $\sigma_w$
и характерным временем автокорреляции $\tau_w$.
При моделировании процессов измерения внешних сигналов в помощью АЦП может потребоваться
более сложное представление:
%
\[
  x_o(t) = W( x_{op}(t) + w(t) ).
\]
%
Данный способ определения $x_o(t)$ необходим в тех случаях,
когда надо отобразить при моделировании тот факт,
что данный сигнал принимает фиксированный набор значений, определяемый разрядностью АЦП.

Временные ограничения на работу системы идентификации могут быть заданы
следующими способами. В простейшем случае, значение $p_o$
считается постоянным, но ограничено полное время измерения $T$.
В противном случае задаются ограничения на динамику $p_o(t)$.
Распространённые способы:

a) ограничение производной:
%
\begin{equation}
  \left| \od{p_o(t)}{t} \right| < d_{p,\max}.
  \label{atu:eq:p_lim_diff}
\end{equation}
Самый простой метод, но имеет ограниченное применение в реальных
задачах, так как резкие изменение значений параметра,
связанные с переключениями состояния или режимов работы системы,
выходом из строя элементов системы на дают возможности корректно ограничить
величину  $d_{p,\max}$. С другой стороны, такие изменения
происходят достаточно редко, и этот факт не отражает ограничение (\ref{atu:eq:p_lim_diff}).

b) ограничение длины кривой $p_o(t)$ на любом из интервалов времени $[t,t+\Delta t]$:
\begin{equation}
  \int\limits_{t}^{t+\Delta t} \sqrt{1 + \left( \od{p_o(t)}{t} \right)^2 } \, \mathrm{d}t < l_{p,\max}.
  \label{atu:eq:p_lim_l}
\end{equation}
%
Это ограничение выглядит несколько искусственным,
но оно, с одной стороны, вводит ограничение на возможные изменения $p_o(t)$,
и с другой стороны, не ограничивает резкие изменения параметра в отдельных областях.

c) упрощённая форма варианта b) -- ограничение длины проекции  $p_o(t)$ на ось $p$:
\begin{equation}
  \int\limits_{t}^{t+\Delta t} \left| \od{p_o(t)}{t} \right | \, \mathrm{d}t < l_{pl,\max}.
  \label{atu:eq:p_lim_pl}
\end{equation}

Целью задачи идентификации будем считать нахождение
такого сигнала $p_\mathrm{id}(t)$, для которого
отличие $q_\mathrm{id}(t)$ от $q_o(t)$ минимально.
Понятие ``отличие'', как и вид критерия $q$, определяется исходной задачей.





Козлов~\cite{kozlov_nosearch_sns} -- беспоисковые.

Растригин~\cite{rastr_stat_meth_search,rastr_seu,rastr_intro,rastr_adop_complex_sys,rastr_rand_search}

Борцов~\cite{borcov} -- эле-мех.


% }}}1


\section{Постановка задачи исследования}  % {{{1

Критерии.

Методы.

Средства моделирования и измерения.

Проверка адекватности.


% }}}1


\section{Выводы по разделу \thechapter}  % {{{1


% }}}1


% vim: fdm=marker foldlevel=1 foldignore="%#" fdc=4 ft=tex
