\chapter{Огляд стану проблеми і постановка задачі дослідження}

Сучасний рівень науки і технологій характеризується широким
використанням нелінійних динамічних систем~\cite{andronov_vitt_haikin,
anisch_nonlin_eff, mishenko_du_small_relax, nonlin_vibro, malinetskii_modern_methods_nl_dyn}. При цьому
нелінійні ефекти притаманні як об'єктам, так і системам
управління~\cite{kubik_nlsc, vukobr_nonadopt}. Динаміка нелінійних систем,
характеризується загальним рівнем складності моделей,
та у більшості випадків не допускає точного аналітичного
уявлення.
Крім цього, було виявлено існування
класу систем, які проявляють
хаотичну динаміку~\cite{moon_chaotic_vibr, magni_theory_dyn_chaos, kuznetsov_dyn_chaos,
neimark_stoch_chaos_vibro, anisch_reg_and_chaotic_vibro}.
Ці системи мають незвичні відносно класичних представлень характеристики, та
забезпечують
постійний інтерес для дослідників.
Для таких систем, внаслідок
чутливості до початкових умов і параметрів, характерною
властивістю є наявність обмеженого горизонту прогнозу як в
минуле, так і в майбутнє, зростання ентропії і незворотність
процесів~\cite{chernavskii_syn_info,prigogine_from_existent,koltsova_nl_dyn_chem}.
Ця властивість робить непридатними більшість методів, розроблених
для опису, управління та ідентифікації динамічних систем. Також
було відмічено, що деякі відомі динамічні системи, в тому числі ті, що
поширено використовуються в науковій, виробничій та технологічній
практиці, та зарекомендували себе як системи регулярної динаміки,
при певних умовах демонструють хаотичну поведінку. Хаотичні
властивості проявляють системи як класичної, так і квантової
механіки. Нові властивості подібних систем вимагають створення
нових методів для вирішення задач моделювання, управління та
ідентифікації~\cite{karabutov_adapt_id_sys,dmitriev_trans_chaos_lowpower}.


\section{Основні поняття задачі ідентифікації динамічних систем} % {{{1

Одне з перших визначень задачі ідентифікації було дано
Л.~Заде~\cite{zadeh_id_1956}. Згідно з цим визначенням,
``Ідентифікація полягає у знаходженні по вхідним і вихідних сигналам деякої
системи еквівалентної їй системи з деякого заданого
класу''~\cite{eykhoff_id_base, eykhoff_modern_id, lung_id_sys}.


Н.С.~Райбман в передмові до відомої роботі
П.~Ейкхоффа~\cite{eykhoff_id_base} дав наступне визначення ідентифікації:
``за результатами спостережень над вхідними та вихідними
змінними системи повинна бути побудована оптимальна в
деякому сенсі модель, тобто формалізоване представлення цієї
системи''. Дане визначення відрізняється незвичною широтою
охоплення за рахунок того, що під визначення ``оптимальна в
деякому сенсі'' потрапляє практично будь-яка умова, в тому числі яка не
відноситься до ідентифікації. В одній з наступних
робіт~\cite{raibman_id_obj_ctl} вид цього визначення став більш суворим і
формальним: ``В якості критерію оптимальності використовується
функція вихідних змінних об'єкта
$y(t) $ і моделі
$y^{*}(t) = A_t^{*} x (s) $, і на її математичне очікування накладається
умова
$ M \{\rho [y_t, y_t^{*}] \} \to{\min} $''.
Це визначення
відповідає задачі ідентифікації ціною введення істотних
обмежень: можливості подання динаміки системи у вигляді
оператора, існування і можливості визначення математичного
очікування, і відповідності мінімуму цього очікування
близькості динаміки системи і її моделі.



% Определение Растригина.



%Критерии

При описі і аналізі динаміки хаотичних систем
використовуються спеціалізовані характеристики і
показники. Наприклад, перетин Пуанкаре~\cite{moon_chaotic_vibr,
anisch_complex_vibrations_in_simple_systems, atu_st105} дозволяє візуально визначити
ознаки хаотичної динаміки. Однак, використання візуальних
ознак в якості критерію вельми складно.

Іншою важливою ознакою є фрактальна розмірність. На відміну від
перетину Пуанкаре, вона має скалярний вид, що робить можливим
її використання в якості критерію. Однак, діапазон зміни цієї
величини для більшості сигналів досить малий, і в умовах шумів
вимірювання виділити корисну для ідентифікації складову
вельми скрутно.

% }}}1

\section{Існуючі методи ідентифікації нелінійних динамічних систем}%{{{1

З огляду на те, що задача ідентифікації складних динамічних
систем завжди була актуальною, і привертала увагу багатьох
дослідників, було створено значну кількість методів і систем
ідентифікації~\cite{eykhoff_id_base, leondes_modern_tu, nelles_nlsys_id}. Створення
універсальної системи ідентифікації, яка була би безумовно працездатною
для будь-якої динамічної системи, представляється практично
нездійсненним завданням. Отже, кожен з існуючих методів має
свою область застосування~\cite{rastr_intro}. Найбільш математично
обґрунтованим є методи ідентифікації лінійних систем.

Для дуже вузького класу нелінійних систем є можливим отримання
аналітичного виразу для значення параметра за вимірюваннями
вхідних і вихідних сигналів. Найчастіше, для складних динамічних
систем отримати таку аналітичну залежність неможливо. Тому
широкого поширення набули методи з використанням паралельної
моделі~\cite{lung_id_sys, gropp_methods_id, deith_method_id_ds}, в яких на вхід і моделі,
і об'єкта подається один і той же вхідний сигнал, і параметри
моделі налаштовуються таким чином, щоб вихід моделі був
найбільш близьким до виходу об'єкта. В якійсь мірі саме такі
методи відображають визначення ідентифікації. Між собою ці
методи відрізняються способами настройки параметрів моделі.


\subsection{Метод синхронного детектора} % {{{2

Первісне призначення синхронного детектора полягало у
виділенні корисного сигналу з модульованого високочастотного
несучого сигналу на тлі перешкод.
Дуже вузька смуга пропускання, характерна для цього методу,
дозволяла виділяти корисний сигнал навіть у тому випадку, коли
рівень шуму перевищував рівень корисного сигналу.

При використанні у якості системи ідентифікації
модулюється не сигнал, а значення параметра моделі~(рис.~\ref{atu:f:syncdet})~\cite{eykhoff_id_base}.
Спочатку в якості сигналу, що збурює,
використовувався або гармонійний, або ж прямокутний сигнал $g(t)$.
Для нелінійних систем добре себе зарекомендувало застосування
для цієї мети випадкового або псевдовипадкового  сигналу.

\begin{figure}[htb!]
\begin{center}
% vi:syntax=tex
\begin{tikzpicture}
  % \draw[hair,step=1.0em] (0,-3) grid (12.0,3.0);
  \bXStyleBloc{semiboldline,inner sep=2pt};
  \bXLineStyle{medline};
  % --- U
  \bXInput{U};
  \path (U.center) ++(2.5em,0.0em) coordinate (UxM);
  \draw (U.center) -- (UxM);
  \fill (UxM) circle[radius=0.05];
  \node[above right] at(U) {$u(t)$};
  % --- Obj
  \bXBranchy[-3]{UxM}{U0};
  % \fill[red](U0) circle[radius=0.05];
  \bXBlocL[1.0]{Obj}{$\mathbf{O}$}{U0};
  \bXLinkyx{UxM}{Obj};
  % --- M1
  \bXBranchy[3]{UxM}{U1};
  %\fill[green](U1) circle[radius=0.05];
  \bXBlocL[1.0]{M1}{$\mathbf{M}$}{U1};
  \bXLinkyx{UxM}{M1};
  % --- W
  \bXCompSum[5.0]{W}{Obj}{}{}{}{};
  \bXLink[$x_o(t)$]{Obj}{W};
  \path (W.north) ++(0.0em,2.0em) coordinate (Win);
  \draw[medlinep] (Win) -- (W.north);
  \node[below right] at(Win) {$w(t)$};
  % --- Err
  \bXCompSum[9.5]{Err}{UxM}{+}{-}{}{};
  \bXLinkxy{W}{Err};
  \bXLinkxy{M1}{Err};
  \node[above right] at(M1.east) {$x_m(t)$};
  % --- Mult
  \bXCompSum*[9.0]{Mult}{Err}{}{}{}{};
  \bXLink[$e(t)$]{Err}{Mult};
  %\fill[black](Mult) circle[radius=0.07];
  \node at(Mult) {$\times$};
  % --- LPF
  \bXBloc[2.0]{LPF}{\textbf{LPF}}{Mult};
  \bXLink{Mult}{LPF};
  % --- Mixer
  \bXBranchy[6]{Mult}{MixPoint};
  \bXCompSum[0.0]{Mixer}{MixPoint}{}{}{}{};
  % \bXReturn{LPF}{Mixer}{};
  \bXLinkyx{LPF}{Mixer};
  % --- param
  \draw[semiboldlinep] (Mixer.west) -| (M1.south);
  \node[below right] at(M1.south east) {$p(t)$};
  % --- Gener
  \bXBranchy[3]{Err}{GenPoint};
  \bXBloc[2]{Gener}{\textbf{G}}{GenPoint};
  \bXLinkxy[$g(t)$]{Gener}{Mult};
  \bXLinkxy{Gener}{Mixer};
  \fill (Mixer) ++(0.0em,3.0em) circle[radius=0.05];
  %
  \TikzAddPadding
  %
\end{tikzpicture}

\end{center}
\caption{Система ідентифікації із синхронним детектором}
\label{atu:f:syncdet}
\end{figure}

Після множення сигналу похибки
$ e (t) $ на сигнал, що збурює,
і подальшої низькочастотної фільтрації,
отримуємо оцінку градієнта функції похибки.
Ця оцінка використовується для
зсуву поточного значення параметра моделі $p(t)$
у бік, відповідний меншому середньому значенні похибки.
При квазістаціонарних умовах, та близький до лінійної залежності
усередненої похибки від параметра, швидкість
зсуву пропорційна цьому градієнту.

Істотним обмеженням цього методу є чутливість до фазових
зрушень для параметричного сигналу. Врахувати зрушення фаз
можливо тільки для найпростіших об'єктів, для
яких не потрібні складні методи ідентифікації.

% }}}2

\subsection{Метод ідентифікації зі змінною частотою пробного впливу} % {{{2

Висока стійкість до перешкод синхронного детектора є його
позитивною рисою. Природною платою за цю стійкість є мала
швидкодія методу.
При цьому, потрібен розділ пошукової та робочої частоти, що не
завжди можливо, особливо для нелінійних об'єктів.


Метод ідентифікації зі змінною частотою пробного впливу
був призначений підвищити швидкодію ідентифікації
за умов обмеженості шумів.
Замість безпосереднього використання сигналу похибки формувалася
парна функція якості~\cite{adopt_cont_sys,rastr_adop_complex_sys}.
Потім вимірювалась похідна функції якості по часу,
та генерувався трикутний пошуковий сигнал~(рис.~\ref{atu:f:varfreq}).


\begin{figure}[htb!]
\begin{center}
% vi:syntax=tex
\begin{tikzpicture}
  % \draw[hair,step=1.0em] (0,-3) grid (12.0,3.0);
  \bXStyleBloc{semiboldline,inner sep=2pt};
  \bXLineStyle{medline};
  % --- U
  \bXInput{U};
  \path (U.center) ++(2.5em,0.0em) coordinate (UxM);
  \draw (U.center) -- (UxM);
  \fill (UxM) circle[radius=0.05];
  \node[above right] at(U) {$u(t)$};
  % --- Obj
  \bXBranchy[-3]{UxM}{U0};
  % \fill[red](U0) circle[radius=0.05];
  \bXBlocL[1.0]{Obj}{$\mathbf{O}$}{U0};
  \bXLinkyx{UxM}{Obj};
  % --- M1
  \bXBranchy[3]{UxM}{U1};
  %\fill[green](U1) circle[radius=0.05];
  \bXBlocL[1.0]{M1}{$\mathbf{M}$}{U1};
  \bXLinkyx{UxM}{M1};
  \node[below right] at(M1.south east) {$p(t)$};
  % --- W
  \bXCompSum[5.0]{W}{Obj}{}{}{}{};
  \bXLink[$x_o(t)$]{Obj}{W};
  \path (W.north) ++(0.0em,2.0em) coordinate (Win);
  \draw[medlinep] (Win) -- (W.north);
  \node[below right] at(Win) {$w(t)$};
  % --- Err
  \bXCompSum[9.5]{Err}{UxM}{+}{-}{}{};
  \bXLinkxy{W}{Err};
  \bXLinkxy{M1}{Err};
  \node[above right] at(M1.east) {$x_m(t)$};
  % --- F
  \bXBloc{F}{$F(e)$}{Err};
  \bXLink[$e(t)$]{Err}{F};
  % --- Diff
  \bXBloc{Diff}{$\nabla$}{F};
  \bXLink[$F(t)$]{F}{Diff};
  % --- Omega
  \bXBlocL{Omega}{$\omega\left(\frac{\mathrm{d}F}{\mathrm{d}t}\right)$}{Diff};
  \node[above right] at(Omega.east) {$\omega(t)$};
  % --- Gener
  \bXBranchy[6]{Omega}{GenPoint};
  \bXBlocr[-1.5]{Gener}{\textbf{G}}{GenPoint};
  \draw[medlinep] (Omega.east) -- ++(1.0em,0.0em) |- (Gener.east);
  % --- Trigg
  \bXBlocr{Trigg}{\textbf{T}}{Gener};
  \bXLink{Gener}{Trigg};
  \node[above right] at(Trigg.east){$g(t)$};
  % --- Integr
  \bXBlocrL{Integr}{$k_i \int{}$}{Trigg};
  % % --- param
  \draw[semiboldlinep] (Integr.west) -| (M1.south);
  %
  \TikzAddPadding
  %
\end{tikzpicture}

\end{center}
\caption{Система ідентифікації зі змінною частотою пробного впливу}
\label{atu:f:varfreq}
\end{figure}

Для створення пошукового сигналу використовувалася досить
проста схема,
що складалася з керованого генератора прямокутних коливань,
рахункового тригера і інтегратора.

За умови невеликого рівня шуму вимірювання
цей метод проявляє кращу швидкодію,
ніж метод з використанням синхронного детектора
завдяки скорішої оцінки градієнту елементом,
який вимірює похідну.
Але цей самий елемент
робить метод особливо чутливим до шумів вимірювання, а в
задачах ідентифікації з паралельною моделлю вплив вхідного
сигналу \(u(t)\) на похибку вимірювання \(e (t) \) також
грає роль шуму, що повністю порушує процес ідентифікації.

Діапазон застосування цього методу розширюють різноманітні
методи адаптації~\cite{auto_optim_intask}.

% }}}2

\subsection{Метод випадкового пошуку}%{{{2

У разі великої розмірності ідентифікованого параметра класичні
методи оцінки градієнта похибки ідентифікації мають обмежену
застосовність.
В цьому випадку добре себе зарекомендували методи випадкового
пошуку, запропоновані Л.А.~Растрігиним і послідовниками~\cite{rastr_rand_search,rastr_rand_search_adopt,ivanov_stoh_alg_int,gladkov_optim_nongrad}.

Алгоритм, який використовується цими методами полягає в тому,
оцінка похибки ідентифікації проводиться за протягом певного
кроку за часом,
наприкінці шагу
порівнюється поточне і попереднє значення похибки
ідентифікації, і рішення про змішування робочої точки
приймається відповідно до обраною тактикою випадкового пошуку.

При використанні лінійної тактики в разі зменшення похибки
ідентифікації проводиться крок в тому ж напрямі, що і
попередній. В іншому випадку проводиться крок в випадково
обраному напрямку. При використанні нелінійної тактики
випадковий крок робиться в разі зменшення похибки. В іншому
випадку робиться крок назад.

Метод має широкий діапазон застосовності, але проявляє
свої переваги на складних системах з великою кількістю
ідентифікованих параметрів. У простіших випадках проявляється
такий недолік, як мала швидкість пошуку.

% }}}2

\subsection{Інші методи} % {{{2

У тих випадках, коли немає необхідності отримати значення
параметра об'єкта, а досить просто існування моделі, широкого
поширення набули непараметричні методи~\cite{medved_nepar_alg_id_nds},
у першу чергу методи з використанням нейроних
мереж~\cite{chen_nn_for_nls_mod,chen_nls_id_radial_basis,patra_nds_id_cheb,narendra_id_ctl_ds_nn,bodyan_adapt_viyavl}.

У даній роботі ставиться задача саме параметричної
ідентифікації, тому  непараметричні методи незастосовні по
визначенню.

Також існують умови, за котрими має сенс використовувати
статистичні методи пошуку~\cite{rastr_stat_meth_search},
або ж
безпошукові самоналаштовні системи~\cite{kozlov_nosearch_sns}.


У наукових працях М.М.~Карабутова
запропоновано використання при ідентифікації
поняття ``коефіцієнта структурності системи''
~\cite{karabutov_adapt_id_sys,saliga_id_ctl_black},
який задається або як відношення
вхідного і вихідного сигналів:
%
\[
  k_s(t) = \frac{x(t)}{u(t)},
\]
%
або як відношення евклідових норм сигналів у разі їх багатовимірністі:
%
\[
  k_s(t) = \frac{\|x(t)\|}{\|u(t)\|}.
\]

На підставі цього коефіцієнту робиться висновок у тому числі
про структуру системи.
Але, для складних сигналів, особливо якщо виконується умова
$u(t)=0$ або $\|u(t)\| = 0$,
немає можливості коректно використовувати цей коефіцієнт.


% }}}2

\subsection{Адаптивно-пошукова ідентифікація зі збуренням параметра однієї моделі} % {{{2

Метод адаптивно-пошукової ідентифікації зі збуренням параметрів
однієї моделі спочатку був розроблений як метод пошукової
оптимізації~\cite{ivah_int_meth_direct,rastr_seu,kras_dyn_nsn,borcov}.
Мінімальні вимоги до апріорної інформації, хороша стійкість і
простота схемотехнічної реалізації дозволили застосувати
даний метод не тільки для вирішення задач оптимізації, але
і для задач ідентифікації параметрів складних систем
\cite{mich_92,mai_adopt_meth_direct,mai_iss_adop_alg_etalon}~(рис.~\ref{atu:f:apid1}).

\begin{figure}[htb!]
\begin{center}
% vi:syntax=tex
% use: УГПК =  CRO (controlled rectangular oscillator)
\begin{tikzpicture}
  % \draw[hair,step=1.0em] (0,-3) grid (12.0,3.0);
  \bXStyleBloc{semiboldline,inner sep=2pt};
  \bXLineStyle{medline};
  % --- U
  \bXInput{U};
  \path (U.center) ++(2.5em,0.0em) coordinate (UxM);
  \draw (U.center) -- (UxM);
  \fill (UxM) circle[radius=0.05];
  \node[above right] at(U) {$u(t)$};
  % --- Obj
  \bXBranchy[-3]{UxM}{U0};
  % \fill[red](U0) circle[radius=0.05];
  \bXBlocL[1.0]{Obj}{$\mathbf{O}$}{U0};
  \bXLinkyx{UxM}{Obj};
  % --- M1
  \bXBranchy[3]{UxM}{U1};
  %\fill[green](U1) circle[radius=0.05];
  \bXBlocL[1.0]{M1}{$\mathbf{M}$}{U1};
  \bXLinkyx{UxM}{M1};
  \node[below right] at(M1.south east) {$p(t)$};
  % --- W
  \bXCompSum[5.0]{W}{Obj}{}{}{}{};
  \bXLink[$x_o(t)$]{Obj}{W};
  \path (W.north) ++(0.0em,2.0em) coordinate (Win);
  \draw[medlinep] (Win) -- (W.north);
  \node[below right] at(Win) {$w(t)$};
  % --- Err
  \bXCompSum[9.5]{Err}{UxM}{+}{-}{}{};
  \bXLinkxy{W}{Err};
  \bXLinkxy{M1}{Err};
  \node[above right] at(M1.east) {$x_m(t)$};
  % --- F
  \bXBloc{F}{$F(e)$}{Err};
  \bXLink[$e(t)$]{Err}{F};
  % --- Omega
  \bXBloc{Omega}{$\omega\left(F\right)$}{F};
  \bXLink[$F(t)$]{F}{Omega};
  \node[above right] at(Omega.east) {$\omega(t)$};
  % --- Gener - CRO
  \bXBranchy[6]{Omega}{GenPoint};
  \bXBlocr[-1.5]{Gener}{\textbf{CRO}}{GenPoint};
  \draw[medlinep] (Omega.east) -- ++(1.0em,0.0em) |- (Gener.east);
  \node[above left] at(Gener.west){$g(t)$};
  % --- Integr
  \bXBlocrL{Integr}{$k_i \int{}$}{Gener};
  % % --- param
  \draw[semiboldlinep] (Integr.west) -| (M1.south);
  %
  \TikzAddPadding
  %
\end{tikzpicture}

\end{center}
\caption{Адаптивно-пошукова ідентифікація зі збуренням параметра однієї моделі}
\label{atu:f:apid1}
\end{figure}

Пошукова частота визначається значенням функції якості:
\( \omega = \omega_0 ( 1 + k_\omega F ) \).
Керований генератор прямокутних коливань
(КГПК, російською --- УГПК, CRO, Controlled rectangular oscillator)
створює прямокутний сигнал $g(t)$ заданої частоти і одиничної
амплітуди. Надалі даний сигнал надходить на інтегратор, формуючи
поточне значення параметра (трикутний пошуковий сигнал) $p(t)$:
%
\begin{equation}
 p(t) = a_p + k_i \int\limits_{0}^{t} g(t) \, dt ,
\label{atu:eq:api_integr}
\end{equation}

Швидкість пошуку
$ v = dp / dt $ при малих змінах параметра оцінюється як
%
\begin{equation}
\label{atu:eq:vasi}
  v
  \approx
  \frac{k_i^2 k_\omega \sin(\omega T_d) }{\omega} \nabla {F},
\end{equation}
%
де
$ k_i $ --- коефіцієнт при інтеграторі,
$ k_\omega $ --- коефіцієнт впливу функції якості на частоту,
$ T_d $ --- параметричне запізнювання.

Важливою відмінністю даного методу від методу зі змінною
частотою пробних впливів є відсутність
ланки, яка реалізую функцію
диференціювання,
що підвищує стійкість методу до шумів вимірювання. Для
оцінювання усередненого градієнта функції якості
використовуються динамічні властивості самого
ідентифікованого об'єкта, а саме --- запізнювання по параметру.

Адаптивні властивості методу виявляються в двох ефектах. При
видаленні від шуканого значення параметра частота пошукових
коливань зменшується, що збільшує інтервал усереднення,
і, як наслідок --- дозволяє виділити корисний сигнал на тлі
перешкод. Зменшення пошукової частоти автоматично збільшує
амплітуду пошукових коливань, що дозволяє зменшити ймовірність
попадання в локальний екстремум. При наближенні до питомого
значення амплітуда пошукових коливань зменшується, що
призводить до підвищення точності
ідентифікації~\cite{mai_sear_meth_akt_id_ns,mai_syntez_adop_id,bodyan_adapt_viyavl}.
Існує модифікація методу, яка дозволяє ідентифікувати
одночасно кілька параметрів за рахунок застосування
випадково-детерменірованних елементів~\cite{mich_92, mich_upr_prost_mech}.

Проте, даний метод також не позбавлений недоліків. Постійні
пошукові коливання призводять до збурення динаміки моделі,
яке не характерно для об'єкта. При певних умовах таке збурення
може повністю порушити процес пошуку. Наприклад, такий
ефект спостерігається при близькості пошукової частоти до
характерних частотах вхідного сигналу або власних коливань
об'єкта.

Працездатність цього методу багато в чому визначається
значенням множника
$ \sin(\omega T_d)$, який входить до~(\ref{atu:eq:vasi}).
Він означає, що напрямок пошуку задається
співвідношенням між пошуковою частотою та параметричним запізнюванням.
З одного боку це відображує той факт, що у якості сховища історії,
необхідної для метода, використовується власна динаміка об'єкту,
тим самим --- вводить певні
обмеження на можливість використання цього методу.
Пошукова частота $\omega$ повинна як задавати інтервал усереднення,
так і забезпечувати
позитивну визначеність виразу $ \sin(\omega T_d)$.
Не усі об'єкти дозволяють
задовольнити ці дві вимоги одночасно.

% }}}2


\subsection{Адаптивно-пошукова ідентифікація з парою моделей і двома генераторами із загальним скиданням}%{{{2

Запропонований в роботі~\cite{atu_phd_thesis} метод ідентифікації
нелінійних динамічних систем з використанням пари моделей
і двох генераторів із загальним скиданням~(рис.~\ref{atu:f:apid2})
позбавлений основних недоліків попереднього методу.

\begin{figure}[htb!]
\begin{center}
% vi:syntax=tex
% use: УГПК =  CRO (controlled rectangular oscillator)
\begin{tikzpicture}
  % \draw[hair,step=1.0em] (0,-4) grid (12.0,4.0);
  \bXStyleBloc{semiboldline,inner sep=2pt};
  \bXLineStyle{medline};
  % --- U
  \bXInput{U};
  \path (U.center) ++(2.5em,0.0em) coordinate (UxM);
  \draw (U.center) -- (UxM);
  \fill (UxM) circle[radius=0.05];
  \node[above right] at(U) {$u(t)$};
  % --- Obj
  \bXBranchy[0]{UxM}{U0};
  % \fill[red](U0) circle[radius=0.05];
  \bXBlocL[1.0]{Obj}{$\mathbf{O}$}{U0};
  \bXLinkyx{UxM}{Obj};
  % --- Ml
  \bXBranchy[-5]{UxM}{U1};
  %\fill[green](U1) circle[radius=0.05];
  \bXBlocL[1.0]{Ml}{$\mathbf{M_l}$}{U1};
  \bXLinkyx{UxM}{Ml};
  \node[above right] at(Ml.north east) {$p_l(t)$};
  % --- Mr
  \bXBranchy[5]{UxM}{U2};
  %\fill[green](U2) circle[radius=0.05];
  \bXBlocL[1.0]{Mr}{$\mathbf{M_r}$}{U2};
  \bXLinkyx{UxM}{Mr};
  \node[below right] at(Mr.south east) {$p_r(t)$};
  % --- W
  \bXCompSum[5.0]{W}{Obj}{}{}{}{};
  \bXLink[$x_o(t)$]{Obj}{W};
  \path (W.north) ++(0.0em,2.0em) coordinate (Win);
  \draw[medlinep] (Win) -- (W.north);
  \node[below right] at(Win) {$w(t)$};
  % --- Errl
  \bXBranchy[2.5]{Ml}{Err1Point};
  \bXCompSum[8.0]{Errl}{Err1Point}{-}{+}{}{};
  \bXLinkxy{W}{Errl};
  \bXLinkxy{Ml}{Errl};
  \node[above right] at(Ml.east) {$x_{l}(t)$};
  % --- Errr
  \bXBranchy[-2.5]{Mr}{ErrrPoint};
  \bXCompSum[8.0]{Errr}{ErrrPoint}{+}{-}{}{};
  \bXLinkxy{W}{Errr};
  \bXLinkxy{Mr}{Errr};
  \node[above right] at(Mr.east) {$x_{r}(t)$};
  % --- Fl
  \bXBloc{Fl}{$F(e)$}{Errl};
  \bXLink[$e_l(t)$]{Errl}{Fl};
  % --- Fr
  \bXBloc{Fr}{$F(e)$}{Errr};
  \bXLink[$e_r(t)$]{Errr}{Fr};
  % --- Omegal
  \bXBloc{Omegal}{$\omega\left(F\right)$}{Fl};
  \bXLink[$F_l(t)$]{Fl}{Omegal};
  \node[above right] at(Omegal.east) {$\omega_l$};
  % --- Omegar
  \bXBloc{Omegar}{$\omega\left(F\right)$}{Fr};
  \bXLink[$F_r(t)$]{Fr}{Omegar};
  \node[above right] at(Omegar.east) {$\omega_r$};
  % --- Generl - CRO
  \bXBloc{Generl}{\textbf{CRO}}{Omegal};
  \node[above right] at(Generl.east) {$g_l(t)$};
  \bXLink{Omegal}{Generl};
  % --- Generr - CRO
  \bXBloc{Generr}{\textbf{CRO}}{Omegar};
  \node[below right] at(Generr.east) {$g_r(t)$};
  \bXLink{Omegar}{Generr};
  \draw[semiboldline,<->] (Generr.north) -- ( Generl.south);
  % --- dG
  \bXCompSum[21.0]{dG}{W}{+}{-}{}{};
  \bXLinkxy{Generr}{dG};
  \bXLinkxy{Generl}{dG};
  \node[left] at(dG.west) {rst};
  % --- Al
  \bXBranchy[-5.5]{Omegal}{AlPoint};
  \bXCompSum[0.0]{Al}{AlPoint}{}{-}{}{+};
  \draw[medlinep] (dG.east) -- ++(1.0em,0.0em) |- (Al.east);
  \draw[medline,<-] (Al.south) -- ++(0.0em,-1.0em);
  \node[below right] at(Al.south) {$A$};
  \draw[semiboldlinep] (Al.west) -| (Ml.north);
  % --- Ar
  \bXBranchy[5.5]{Omegar}{ArPoint};
  \bXCompSum[0.0]{Ar}{ArPoint}{+}{}{}{+};
  \draw[medlinep] (dG.east) -- ++(1.0em,0.0em) |- (Ar.east);
  \draw[medline,<-] (Ar.north) -- ++(0.0em,1.0em);
  \node[above right] at(Ar.north) {$A$};
  \draw[semiboldlinep] (Ar.west) -| (Mr.south);
  %
  \TikzAddPadding
  %
\end{tikzpicture}

\end{center}
\caption{Система адаптивно-пошукової ідентифікація з парою моделей і двома генераторами із загальним скиданням}
\label{atu:f:apid2}
\end{figure}


У квазістаціонарному випадку швидкість пошуку визначається
наступним чином:
%
\begin{equation}
\label{atu:eq:vugpk2}
  v
  \approx
  \frac{k_i (\tau_l - \tau_r)}{\tau}
  =
  2 k_i k_\omega A \nabla F .
\end{equation}

За рахунок збільшення витрат, пов'язаних з подвоєнням кількості
основних елементів системи ідентифікації, вдалося значно
підвищити як точність, так і швидкість ідентифікації.
Також розірвано жорсткий зв'язок пошукової частоти з
параметричним запізнюванням. Це дозволяє задавати
пошукову частоту незалежно від швидкості реакції об'єкту
на зміни параметру, що значно розширює можливості методу.

Але і цей метод має певні недоліки, пов'язані
з протиріччям ``швидкість -- точність''.
При малих значеннях параметра ``A''
стає можливим отримати максимально точні результати,
але швидкість пошуку пропорційно зменшується,
і навпаки.

Також, цей метод, як і більшість перерахованих,
не здатен проводити ідентифікацію  систем хаотичної динаміки,
також як і інших складних систем, для яких немає
прямого зв'язку між
близькістю значень параметрів з близькістю вихідних сигналів.


% }}}2







% }}}1

% \section{Свойства систем хаотической динамики применительно к задаче идентификации}  % {{{1
%
% Одним из основных свойств систем хаотической динамики является чувствительность
% к малым возмущениям сходных сигналов, состояния системы или же параметров.
% В линейном приближении это соответствует положительным значениям
% коэффициентов Ляпунова~\cite{magni_theory_dyn_chaos,moon_chaotic_vibr}.
%
%
% % }}}1

\section{Постановка задачі дослідження}%{{{1



У зв'язку з цим, метою дисертаційної роботи є створення
нових методів ідентифікації нелінійних динамічних систем,
для яких характерна хаотична динаміка,
моделей процесів ідентифікації параметрів таких систем.

Відповідно до поставленої мети у дисертаційній роботі необхідно
вирішити такі задачі:

\begin{itemize}

  \item
    за рахунок аналізу характеристик, які зберігаються навіть
    у хаотичних режимах,
    розробити нові критерії ідентифікації, які, на відміну від тих, що
    існують, були б придатні для аналізу стану та динаміки
    систем з хаотичною динамікою, і створять базис для обґрунтування працездатності систем
    ідентифікації;

  \item
    на підставі аналізу існуючих методів, їх недоліків та властивостей,
    розвинути існуючі та розробити нові методи пошукової параметричної ідентифікації,
    які б у повної мірі використовували можливості
    паралельних обчислювань та переваг використання ансамблю
    синергірованних моделей;


  \item
    створити моделі процесів
    ідентифікації хаотичних систем з використанням запропонованих методів,
    провести комп'ютерне моделювання процесів ідентифікації систем
    хаотичної динаміки та дослідити їх працездатність, можливості та
    характеристики;

  \item
    з ціллю створити підстави для перевірки адекватності розроблених
    моделей процесів ідентифікації,
    розробити нову фізичну систему з хаотичною поведінкою
    на підставі системи зв'язаних релаксаційних елементів,
    що надасть можливість дослідити та підтвердити переваги
    розроблених методів ідентифікації;

  \item
    для перевірки адекватності розроблених моделей та методів ідентифікації,
    провести натурне моделювання процесів ідентифікації запропонованого
    генератора хаосу, та порівняти з результатами комп'ютерного моделювання;

  \item
    розробити програмне забезпечення, придатне для моделювання систем
    хаотичної динаміки, систем ідентифікації, предікції та управління.

\end{itemize}


% }}}1


\section{Висновки по розділу \thechapter}%{{{1


В цьому розділі проведено аналіз існуючих визначень і методів
ідентифікації нелінійних динамічних систем.

В результаті проведеного аналізу зроблені висновки про
необхідність розвитку теоретичних основ, пов'язаних із
задачею ідентифікації систем, для яких характерна хаотична
динаміка.

Виходячи з аналізу властивостей, недоліків та переваг існуючих методів,
поставлені задачі дослідження, спрямовані на створення і
дослідження нових методів ідентифікації і моделей процесів
їх функціонування.

Список використаних джерел у даному розділі наведено у повному
списку використаних джерел під номерами
\cite{adopt_cont_sys},
\cite{andronov_vitt_haikin},
\cite{atu_phd_thesis},
\cite{auto_optim_intask},
\cite{bodyan_adapt_viyavl},
\cite{borcov},
\cite{chen_nls_id_radial_basis},
\cite{chen_nn_for_nls_mod},
\cite{chernavskii_syn_info},
\cite{deith_method_id_ds},
\cite{dmitriev_trans_chaos_lowpower},
\cite{eykhoff_id_base},
\cite{eykhoff_modern_id},
\cite{gladkov_optim_nongrad},
\cite{gropp_methods_id},
\cite{ivah_int_meth_direct},
\cite{ivanov_stoh_alg_int},
\cite{karabutov_adapt_id_sys},
\cite{koltsova_nl_dyn_chem},
\cite{kozlov_nosearch_sns},
\cite{kras_dyn_nsn},
\cite{kubik_nlsc},
\cite{kuznetsov_dyn_chaos},
\cite{leondes_modern_tu},
\cite{lung_id_sys},
\cite{magni_theory_dyn_chaos},
\cite{mai_adopt_meth_direct},
\cite{mai_iss_adop_alg_etalon},
\cite{mai_sear_meth_akt_id_ns},
\cite{mai_syntez_adop_id},
\cite{medved_nepar_alg_id_nds},
\cite{mich_92},
\cite{mich_upr_prost_mech},
\cite{moon_chaotic_vibr},
\cite{narendra_id_ctl_ds_nn},
\cite{nelles_nlsys_id},
\cite{patra_nds_id_cheb},
\cite{prigogine_from_existent},
\cite{raibman_id_obj_ctl},
\cite{rastr_adop_complex_sys},
\cite{rastr_intro},
\cite{rastr_rand_search},
\cite{rastr_rand_search_adopt},
\cite{rastr_seu},
\cite{rastr_stat_meth_search},
\cite{saliga_id_ctl_black},
\cite{vukobr_nonadopt},
\cite{zadeh_id_1956}.


% }}}1


% vim: fdm=marker foldlevel=0 foldignore="%#" fdc=4 ft=tex
