\chapter{Огляд стану проблеми і постановка завдання дослідження}

Сучасний рівень науки и технологий характерізується широким
Використання нелінійніх дінамічніх систем~\cite{andronov_vitt_haikin,
anisch_nonlin_eff, mishenko_du_small_relax, nonlin_vibro, malinetskii_modern_methods_nl_dyn}. При цьому
нелінійні ефекту притаманні як об'єктам, так і системам
управління~\cite{kubik_nlsc, vukobr_nonadopt}. Динаміка нелінійних систем,
характеризується загальним рівнем складності моделей,
в більшості випадків не допускає точного аналітичного
уявлення. Крім цього, було виявлено, і останнім часом представляє
постійний інтерес для дослідників клас систем, які проявляють
хаотичну динаміку~\cite{moon_chaotic_vibr, magni_theory_dyn_chaos, kuznetsov_dyn_chaos,
neimark_stoch_chaos_vibro, anisch_reg_and_chaotic_vibro}. Для таких систем, внаслідок
чутливості до початкових умов і параметрів, характерним
властивістю є наявність обмеженого горизонту прогнозу як в
минуле, так і в майбутнє, зростання ентропії і незворотність
процесів~\cite{chernavskii_syn_info, prigogine_from_existent, koltsova_nl_dyn_chem}. Ця
властивість робить непридатними більшість методів, розроблених
для опису, управління та ідентифікації динамічних систем. Також
було відмічено, що деякі відомі динамічні системи, в тому числі
широко використовуються в науковій, виробничій та технологічній
практиці, зарекомендували себе як системи регулярної динаміки,
при певних умовах демонструють хаотична поведінка. Хаотично
властивості проявляють системи як класичної, так і квантової
механіки. Нові властивості подібних систем вимагають створення
нових методів для вирішення задач моделювання, управління та
ідентифікації~\cite{karabutov_adapt_id_sys, dmitriev_trans_chaos_lowpower}.


\section{Основні поняття завдання ідентифікації динамічних систем} % {{{1

Одне з перших визначень завдання ідентифікації було дано
Л.~Заде~\cite{zadeh_id_1956}. Згідно з цим визначенням,
``Ідентифікація
полягає у знаходженні по вхідних і вихідних сигналам деякої
системи еквівалентної їй системи з деякого заданого
класу''~\cite{eykhoff_id_base, eykhoff_modern_id, lung_id_sys}.


Н.С.~Райбман в передмові до відомої роботі
П.~Ейкхоффа~\cite{eykhoff_id_base} дав наступне визначення ідентифікації
``за результатами спостережень над вхідними та вихідними
змінними системи повинна бути побудована оптимальна в
деякому сенсі модель, тобто формалізоване представлення цієї
системи ''. Дане визначення відрізняється граничною широтою
охоплення за рахунок того, що під визначення ``оптимальна в
деякому сенсі '' потрапляє практично будь-який, в тому числі не
відноситься до власне ідентифікації умова. В одній з наступних
робіт~\cite{raibman_id_obj_ctl} вид цього визначення став більш суворим і
формальним:``В якості критерію оптимальності використовується
функція вихідних змінних об'єкта
$ y (t) $ і моделі
$ y ^{*} (t) = A_t ^{*} x (s) $, і на ЇЇ математичне очікування накладається
умова
$ M \{\rho [y_t, y_t ^{*}] \} \to{\min} $''. Це визначення чинне лише
відповідає завданню ідентифікації ціною введення істотних
обмежень:можливості подання динаміки системи у вигляді
оператора, існування і можливості визначення математичного
очікування, і відповідності мінімуму цього очікування
близькості динаміки системи і її моделі.



% Определение Растригина.



%Критерии

При описі і аналізі динаміки хаотичних систем
використовуються спеціалізовані характеристики і
показники. Наприклад, перетин Пуанкаре~\cite{moon_chaotic_vibr,
anisch_complex_vibrations_in_simple_systems, atu_st105} дозволяє візуально визначити
ознаки хаотичної динаміки. Однак, використання візуальних
ознак в якості критерію вельми складно.

Іншою важливою ознакою є фрактальна розмірність. На відміну від
перетину Пуанкаре, вона має скалярний вид, що робить можливим
її використання в якості критерію. Однак, діапазон зміни цієї
величини для більшості сигналів досить малий, і в умовах шумів
вимірювання виділити корисну для ідентифікації складову
вельми скрутно.

% }}}1

\section{Існуючі методи ідентифікації нелінійних динамічних систем}%{{{1

З огляду на те, що завдання ідентифікації складних динамічних
систем завжди була актуальною, і привертала увагу багатьох
дослідників, було створено значну кількість методів і систем
ідентифікації~\cite{eykhoff_id_base, leondes_modern_tu, nelles_nlsys_id}. Створення
універсальної системи ідентифікації, безумовно підходящої
для будь-якої динамічної системи, представляється практично
нездійсненним завданням. Отже, кожен з існуючих методів має
свою область застосування~\cite{rastr_intro}. Найбільш математично
обґрунтованим є методи ідентифікації лінійних систем.

Для дуже вузького класу нелінійних систем є можливим отримання
аналітичного виразу для значення параметра за вимірюваннями
вхідних і вихідних сигналів. Найчастіше, для складних динамічних
систем отримати таку аналітичну залежність неможливо. Тому
широкого поширення набули методи з використанням паралельної
моделі~\cite{lung_id_sys, gropp_methods_id, deith_method_id_ds}, в яких на вхід і моделі,
і об'єкта подається один і той же вхідний сигнал, і параметри
моделі налаштовуються таким чином, щоб вихід моделі був
найбільш близьким до виходу об'єкта. В якійсь мірі саме такі
методи відображають визначення ідентифікації. Між собою ці
методи відрізняються способами настройки параметрів моделі.


\subsection{Синхронний детектор}%{{{2

Принцип работы синхронного детектора %  \cite{adopt_cont_sys}
(рис.~\ref{atu:f:syncdet})
заключается в том,
что значение коэффициента модели модулируется либо гармоническим,
либо случайным сигналом. Для выявления влияния
изменения коэффициента на ошибку идентификации
производится перемножение модулирующего сигнала с ошибкой
идентификации. После усреднения данного сигнала
получаем оценку значения градиента функции ошибки на
пространстве параметров.

\begin{figure}[htb!]
\begin{center}
% vi:syntax=tex
\begin{tikzpicture}
  % \draw[hair,step=1.0em] (0,-3) grid (12.0,3.0);
  \bXStyleBloc{semiboldline,inner sep=2pt};
  \bXLineStyle{medline};
  % --- U
  \bXInput{U};
  \path (U.center) ++(2.5em,0.0em) coordinate (UxM);
  \draw (U.center) -- (UxM);
  \fill (UxM) circle[radius=0.05];
  \node[above right] at(U) {$u(t)$};
  % --- Obj
  \bXBranchy[-3]{UxM}{U0};
  % \fill[red](U0) circle[radius=0.05];
  \bXBlocL[1.0]{Obj}{$\mathbf{O}$}{U0};
  \bXLinkyx{UxM}{Obj};
  % --- M1
  \bXBranchy[3]{UxM}{U1};
  %\fill[green](U1) circle[radius=0.05];
  \bXBlocL[1.0]{M1}{$\mathbf{M}$}{U1};
  \bXLinkyx{UxM}{M1};
  % --- W
  \bXCompSum[5.0]{W}{Obj}{}{}{}{};
  \bXLink[$x_o(t)$]{Obj}{W};
  \path (W.north) ++(0.0em,2.0em) coordinate (Win);
  \draw[medlinep] (Win) -- (W.north);
  \node[below right] at(Win) {$w(t)$};
  % --- Err
  \bXCompSum[9.5]{Err}{UxM}{+}{-}{}{};
  \bXLinkxy{W}{Err};
  \bXLinkxy{M1}{Err};
  \node[above right] at(M1.east) {$x_m(t)$};
  % --- Mult
  \bXCompSum*[9.0]{Mult}{Err}{}{}{}{};
  \bXLink[$e(t)$]{Err}{Mult};
  %\fill[black](Mult) circle[radius=0.07];
  \node at(Mult) {$\times$};
  % --- LPF
  \bXBloc[2.0]{LPF}{\textbf{LPF}}{Mult};
  \bXLink{Mult}{LPF};
  % --- Mixer
  \bXBranchy[6]{Mult}{MixPoint};
  \bXCompSum[0.0]{Mixer}{MixPoint}{}{}{}{};
  % \bXReturn{LPF}{Mixer}{};
  \bXLinkyx{LPF}{Mixer};
  % --- param
  \draw[semiboldlinep] (Mixer.west) -| (M1.south);
  \node[below right] at(M1.south east) {$p(t)$};
  % --- Gener
  \bXBranchy[3]{Err}{GenPoint};
  \bXBloc[2]{Gener}{\textbf{G}}{GenPoint};
  \bXLinkxy[$g(t)$]{Gener}{Mult};
  \bXLinkxy{Gener}{Mixer};
  \fill (Mixer) ++(0.0em,3.0em) circle[radius=0.05];
  %
  \TikzAddPadding
  %
\end{tikzpicture}

\end{center}
\caption{Система ідентифікації з синхронним детектором}
\label{atu:f:syncdet}
\end{figure}

%В
%\cite{Eykhoff,adopt_cont_sys}
показано, что в квазистационарных условиях
происходит смещение рабочей точки системы идентификации
в направлении градиента функции качества,
со скоростью, пропорциональной значению модуля этого градиента.

% }}}2

\subsection{Метод ідентифікації зі змінною частотою пробного впливу}%{{{2

Принцип работы данного метода
(рис.~\ref{atu:f:varfreq})
основан на том, что на параметрический вход модели подаётся треугольный поисковый сигнал.
Этот сигнал генерируется специальным генератором,
состоящего из управляемого генератора прямоугольных колебаний,
счётного триггера и интегратора~\cite{adopt_cont_sys,rastr_adop_complex_sys}.


\begin{figure}[htb!]
\begin{center}
% vi:syntax=tex
\begin{tikzpicture}
  % \draw[hair,step=1.0em] (0,-3) grid (12.0,3.0);
  \bXStyleBloc{semiboldline,inner sep=2pt};
  \bXLineStyle{medline};
  % --- U
  \bXInput{U};
  \path (U.center) ++(2.5em,0.0em) coordinate (UxM);
  \draw (U.center) -- (UxM);
  \fill (UxM) circle[radius=0.05];
  \node[above right] at(U) {$u(t)$};
  % --- Obj
  \bXBranchy[-3]{UxM}{U0};
  % \fill[red](U0) circle[radius=0.05];
  \bXBlocL[1.0]{Obj}{$\mathbf{O}$}{U0};
  \bXLinkyx{UxM}{Obj};
  % --- M1
  \bXBranchy[3]{UxM}{U1};
  %\fill[green](U1) circle[radius=0.05];
  \bXBlocL[1.0]{M1}{$\mathbf{M}$}{U1};
  \bXLinkyx{UxM}{M1};
  \node[below right] at(M1.south east) {$p(t)$};
  % --- W
  \bXCompSum[5.0]{W}{Obj}{}{}{}{};
  \bXLink[$x_o(t)$]{Obj}{W};
  \path (W.north) ++(0.0em,2.0em) coordinate (Win);
  \draw[medlinep] (Win) -- (W.north);
  \node[below right] at(Win) {$w(t)$};
  % --- Err
  \bXCompSum[9.5]{Err}{UxM}{+}{-}{}{};
  \bXLinkxy{W}{Err};
  \bXLinkxy{M1}{Err};
  \node[above right] at(M1.east) {$x_m(t)$};
  % --- F
  \bXBloc{F}{$F(e)$}{Err};
  \bXLink[$e(t)$]{Err}{F};
  % --- Diff
  \bXBloc{Diff}{$\nabla$}{F};
  \bXLink[$F(t)$]{F}{Diff};
  % --- Omega
  \bXBlocL{Omega}{$\omega\left(\frac{\mathrm{d}F}{\mathrm{d}t}\right)$}{Diff};
  \node[above right] at(Omega.east) {$\omega(t)$};
  % --- Gener
  \bXBranchy[6]{Omega}{GenPoint};
  \bXBlocr[-1.5]{Gener}{\textbf{G}}{GenPoint};
  \draw[medlinep] (Omega.east) -- ++(1.0em,0.0em) |- (Gener.east);
  % --- Trigg
  \bXBlocr{Trigg}{\textbf{T}}{Gener};
  \bXLink{Gener}{Trigg};
  \node[above right] at(Trigg.east){$g(t)$};
  % --- Integr
  \bXBlocrL{Integr}{$k_i \int{}$}{Trigg};
  % % --- param
  \draw[semiboldlinep] (Integr.west) -| (M1.south);
  %
  \TikzAddPadding
  %
\end{tikzpicture}

\end{center}
\caption{Система ідентифікації зі змінною частотою пробного впливу}
\label{atu:f:varfreq}
\end{figure}


Результаты предварительного моделирования показали,
что данный метод, хоть и показывает хорошие результаты
в задачах оптимизации, но для решения задач идентификации
практически не применим. Наличие дифференцирующего
звена делает данный метод особо чувствительным к шумам
измерения, а в задачах идентификации с параллельной моделью
влияние входного сигнала \( u(t) \)
на ошибку измерения \( e(t) \)
имеет вид мультипликативного шума,
что полностью нарушает процесс поиска.
Диапазон применимости этого метода расширяют разнообразные методы
адаптации~\cite{auto_optim_intask}.

% }}}2

\subsection{Метод випадкового пошуку}%{{{2

Основа данного метода заключается в том, что
очередном шаге сравнивается текущее и предыдущее
значения ошибки идентификации, и решение о смешении
рабочей точки принимается в соответствии с выбранной
тактикой случайного поиска~\cite{rastr_rand_search,rastr_rand_search_adopt,ivanov_stoh_alg_int}.

Наиболее известными и распространённым
являются линейная и нелинейная тактика
\cite{rastr_rand_search,gladkov_optim_nongrad}.
При использовании линейной тактики
в случае уменьшения ошибки идентификации
производится шаг в том же направлении, что и предыдущий.
В противном случае производится случайный шаг.
При использовании нелинейной тактики
случайный шаг делается в случае уменьшения ошибки.
В противном случае делается шаг назад.

Метод имеет широкий диапазон применимости,
но проявляет свои преимущества на сложных
системах с большим количеством идентифицируемых параметров.
В более простых случаях проявляется такой недостаток,
как малая скорость поиска.

% }}}2

\subsection{Непараметрические методы} % {{{2

~\cite{medved_nepar_alg_id_nds}



Беспоисковые самонастраивающиеся системы~\cite{kozlov_nosearch_sns}

Козлов~\cite{kozlov_nosearch_sns} -- беспоисковые.

Карабутов~\cite{karabutov_adapt_id_sys,saliga_id_ctl_black}

Борцов~\cite{borcov} -- эле-мех.

Статистические методы поиска~\cite{rastr_stat_meth_search}

Нейронка~\cite{chen_nn_for_nls_mod,chen_nls_id_radial_basis,patra_nds_id_cheb,narendra_id_ctl_ds_nn,bodyan_adapt_viyavl}

% }}}2

\subsection{Адаптивно-пошукова ідентифікація з обуренням параметра однієї моделі}%{{{2

Метод адаптивно-поисковой идентификации с возмущением параметров одной модели
изначально был разработан как метод поисковой оптимизации
\cite{mich_92,ivah_int_meth_direct,mai_adopt_meth_direct,rastr_seu,kras_dyn_nsn,mai_iss_adop_alg_etalon,rastr_seu}.
Минимальные требования к априорной информации, хорошая помехоустойчивость и
простота схемотехнической реализации позволили применить данный метод не только
для решения задач оптимизации, но и для задач идентификации параметров сложных
систем
(рис.~\ref{atu:f:apid1})

\begin{figure}[htb!]
\begin{center}
% vi:syntax=tex
% use: УГПК =  CRO (controlled rectangular oscillator)
\begin{tikzpicture}
  % \draw[hair,step=1.0em] (0,-3) grid (12.0,3.0);
  \bXStyleBloc{semiboldline,inner sep=2pt};
  \bXLineStyle{medline};
  % --- U
  \bXInput{U};
  \path (U.center) ++(2.5em,0.0em) coordinate (UxM);
  \draw (U.center) -- (UxM);
  \fill (UxM) circle[radius=0.05];
  \node[above right] at(U) {$u(t)$};
  % --- Obj
  \bXBranchy[-3]{UxM}{U0};
  % \fill[red](U0) circle[radius=0.05];
  \bXBlocL[1.0]{Obj}{$\mathbf{O}$}{U0};
  \bXLinkyx{UxM}{Obj};
  % --- M1
  \bXBranchy[3]{UxM}{U1};
  %\fill[green](U1) circle[radius=0.05];
  \bXBlocL[1.0]{M1}{$\mathbf{M}$}{U1};
  \bXLinkyx{UxM}{M1};
  \node[below right] at(M1.south east) {$p(t)$};
  % --- W
  \bXCompSum[5.0]{W}{Obj}{}{}{}{};
  \bXLink[$x_o(t)$]{Obj}{W};
  \path (W.north) ++(0.0em,2.0em) coordinate (Win);
  \draw[medlinep] (Win) -- (W.north);
  \node[below right] at(Win) {$w(t)$};
  % --- Err
  \bXCompSum[9.5]{Err}{UxM}{+}{-}{}{};
  \bXLinkxy{W}{Err};
  \bXLinkxy{M1}{Err};
  \node[above right] at(M1.east) {$x_m(t)$};
  % --- F
  \bXBloc{F}{$F(e)$}{Err};
  \bXLink[$e(t)$]{Err}{F};
  % --- Omega
  \bXBloc{Omega}{$\omega\left(F\right)$}{F};
  \bXLink[$F(t)$]{F}{Omega};
  \node[above right] at(Omega.east) {$\omega(t)$};
  % --- Gener - CRO
  \bXBranchy[6]{Omega}{GenPoint};
  \bXBlocr[-1.5]{Gener}{\textbf{CRO}}{GenPoint};
  \draw[medlinep] (Omega.east) -- ++(1.0em,0.0em) |- (Gener.east);
  \node[above left] at(Gener.west){$g(t)$};
  % --- Integr
  \bXBlocrL{Integr}{$k_i \int{}$}{Gener};
  % % --- param
  \draw[semiboldlinep] (Integr.west) -| (M1.south);
  %
  \TikzAddPadding
  %
\end{tikzpicture}

\end{center}
\caption{Адаптивно-пошукова ідентифікація зі збуренням параметра однієї моделі}
\label{atu:f:apid1}
\end{figure}

Поисковой частота определяется
значением функции качества:
\( \omega = \omega_0 ( 1 + k_\omega F ) \).
Управляемый генератор прямоугольных колебаний (УГПК, CRO, Controlled rectangular oscillator)
создаёт прямоугольный сигнал \( g(t) \)
заданной частоты и единичной амплитуды.
В дальнейшем данный сигнал поступает на интегратор,
формируя текущее значение параметра (поисковый сигнал) \( p(t) \):
%
\begin{equation}
 p(t) = a_p + k_i \int\limits_{0}^{t} g(t) \, dt ,
\label{atu:eq:api_integr}
\end{equation}

Скорость поиска $v = dp / dt$ при малых изменениях параметра
оценивается как
%
\begin{equation}
\label{atu:eq:vasi}
  v
  \approx
  \frac{k_i^2 k_\omega \sin(\omega T_d) }{\omega} \nabla {F},
\end{equation}
%
где
$k_i$ --- коэффициент при интеграторе,
$k_\omega$ --- коэффициент влияния функции качества на частоту,
$T_d$ --- параметрическое запаздывание.

Важным отличием данного метода от метода с переменной частотой пробных воздействий является
отсутствие дифференцирующего звена, что повышает устойчивость метода к шумам измерения.
Для оценивания усреднённого градиента функции качества используются
динамические свойства самого идентифицируемого объекта, а именно --- запаздывание
по параметру. Адаптивные свойства метода
проявляются в двух эффектах.
При удалении от искомого значения параметра частота поисковых колебаний
уменьшается, что увеличивает интервал усреднения, и, как следствие ---
позволяет выделить полезный сигнал на фоне помех.
Уменьшение поисковой частоты автоматически увеличивает амплитуду поисковых колебаний,
что позволяет уменьшить вероятность попадания в локальный экстремум.
При приближении к искомому значению амплитуда поисковых колебаний уменьшается,
что приводит к повышению точности идентификации~\cite{mai_sear_meth_akt_id_ns,mai_syntez_adop_id,bodyan_adapt_viyavl}.
Сущесвцет модификация метода,
позволяющая идентифицировать одноверменно несколько параметров
за счёт применения случайно-детерменированных элементов~\cite{mich_92,mich_upr_prost_mech}.

Тем не менее, данный метод также не лишён недостатков.
Постоянные поисковые колебания приводят к возмущению динамики модели,
которое не характерно для объекта. При определённых условиях
такое возмущение может полностью нарушить процесс поиска.
Например, такой эффект наблюдается при близости поисковой частоты к характерным частотам входного сигнала или
собственных колебаний объекта.

\Cmt{Связь поисковой частоты с временем оценивания и параметрическим запаздыванием}


% }}}2

\subsection{Адаптивно-поисковая идентификация с парой моделей и двумя генераторами с общим сбросом} % {{{2

Предложенный в работе~\cite{atu_phd_thesis} метод идентификации
нелинейных динамических систем с использованием
пары моделей и двух генераторов с общим сбросом~(рис.~\ref{atu:f:apid2})
лишён основных недостатков предыдущего метода.

\begin{figure}[htb!]
\begin{center}
% vi:syntax=tex
% use: УГПК =  CRO (controlled rectangular oscillator)
\begin{tikzpicture}
  % \draw[hair,step=1.0em] (0,-4) grid (12.0,4.0);
  \bXStyleBloc{semiboldline,inner sep=2pt};
  \bXLineStyle{medline};
  % --- U
  \bXInput{U};
  \path (U.center) ++(2.5em,0.0em) coordinate (UxM);
  \draw (U.center) -- (UxM);
  \fill (UxM) circle[radius=0.05];
  \node[above right] at(U) {$u(t)$};
  % --- Obj
  \bXBranchy[0]{UxM}{U0};
  % \fill[red](U0) circle[radius=0.05];
  \bXBlocL[1.0]{Obj}{$\mathbf{O}$}{U0};
  \bXLinkyx{UxM}{Obj};
  % --- Ml
  \bXBranchy[-5]{UxM}{U1};
  %\fill[green](U1) circle[radius=0.05];
  \bXBlocL[1.0]{Ml}{$\mathbf{M_l}$}{U1};
  \bXLinkyx{UxM}{Ml};
  \node[above right] at(Ml.north east) {$p_l(t)$};
  % --- Mr
  \bXBranchy[5]{UxM}{U2};
  %\fill[green](U2) circle[radius=0.05];
  \bXBlocL[1.0]{Mr}{$\mathbf{M_r}$}{U2};
  \bXLinkyx{UxM}{Mr};
  \node[below right] at(Mr.south east) {$p_r(t)$};
  % --- W
  \bXCompSum[5.0]{W}{Obj}{}{}{}{};
  \bXLink[$x_o(t)$]{Obj}{W};
  \path (W.north) ++(0.0em,2.0em) coordinate (Win);
  \draw[medlinep] (Win) -- (W.north);
  \node[below right] at(Win) {$w(t)$};
  % --- Errl
  \bXBranchy[2.5]{Ml}{Err1Point};
  \bXCompSum[8.0]{Errl}{Err1Point}{-}{+}{}{};
  \bXLinkxy{W}{Errl};
  \bXLinkxy{Ml}{Errl};
  \node[above right] at(Ml.east) {$x_{l}(t)$};
  % --- Errr
  \bXBranchy[-2.5]{Mr}{ErrrPoint};
  \bXCompSum[8.0]{Errr}{ErrrPoint}{+}{-}{}{};
  \bXLinkxy{W}{Errr};
  \bXLinkxy{Mr}{Errr};
  \node[above right] at(Mr.east) {$x_{r}(t)$};
  % --- Fl
  \bXBloc{Fl}{$F(e)$}{Errl};
  \bXLink[$e_l(t)$]{Errl}{Fl};
  % --- Fr
  \bXBloc{Fr}{$F(e)$}{Errr};
  \bXLink[$e_r(t)$]{Errr}{Fr};
  % --- Omegal
  \bXBloc{Omegal}{$\omega\left(F\right)$}{Fl};
  \bXLink[$F_l(t)$]{Fl}{Omegal};
  \node[above right] at(Omegal.east) {$\omega_l$};
  % --- Omegar
  \bXBloc{Omegar}{$\omega\left(F\right)$}{Fr};
  \bXLink[$F_r(t)$]{Fr}{Omegar};
  \node[above right] at(Omegar.east) {$\omega_r$};
  % --- Generl - CRO
  \bXBloc{Generl}{\textbf{CRO}}{Omegal};
  \node[above right] at(Generl.east) {$g_l(t)$};
  \bXLink{Omegal}{Generl};
  % --- Generr - CRO
  \bXBloc{Generr}{\textbf{CRO}}{Omegar};
  \node[below right] at(Generr.east) {$g_r(t)$};
  \bXLink{Omegar}{Generr};
  \draw[semiboldline,<->] (Generr.north) -- ( Generl.south);
  % --- dG
  \bXCompSum[21.0]{dG}{W}{+}{-}{}{};
  \bXLinkxy{Generr}{dG};
  \bXLinkxy{Generl}{dG};
  \node[left] at(dG.west) {rst};
  % --- Al
  \bXBranchy[-5.5]{Omegal}{AlPoint};
  \bXCompSum[0.0]{Al}{AlPoint}{}{-}{}{+};
  \draw[medlinep] (dG.east) -- ++(1.0em,0.0em) |- (Al.east);
  \draw[medline,<-] (Al.south) -- ++(0.0em,-1.0em);
  \node[below right] at(Al.south) {$A$};
  \draw[semiboldlinep] (Al.west) -| (Ml.north);
  % --- Ar
  \bXBranchy[5.5]{Omegar}{ArPoint};
  \bXCompSum[0.0]{Ar}{ArPoint}{+}{}{}{+};
  \draw[medlinep] (dG.east) -- ++(1.0em,0.0em) |- (Ar.east);
  \draw[medline,<-] (Ar.north) -- ++(0.0em,1.0em);
  \node[above right] at(Ar.north) {$A$};
  \draw[semiboldlinep] (Ar.west) -| (Mr.south);
  %
  \TikzAddPadding
  %
\end{tikzpicture}

\end{center}
\caption{Адаптивно-поисковая идентификация с парой моделей и двумя генераторами с общим сбросом}
\label{atu:f:apid2}
\end{figure}


В квазистационарном случае скорость поиска определяется следующим образом:
%
\begin{equation}
\label{atu:eq:vugpk2}
  v
  \approx
  \frac{k_i (\tau_l - \tau_r)}{\tau}
  =
  2 k_i k_\omega A \nabla F .
\end{equation}

За счёт увеличения затрат, связанных с удвоением количества основных элементов системы идентификации,
удалось значительно повысить как точность, так и скорость идентификации.



% }}}2







% }}}1

\section{Свойства систем хаотической динамики применительно к задаче идентификации}  % {{{1

Одним из основных свойств систем хаотической динамики является чувствительность
к малым возмущениям сходных сигналов, состояния системы или же параметров.
В линейном приближении это соответствует положительным значениям
коэффициентов Ляпунова~\cite{magni_theory_dyn_chaos,moon_chaotic_vibr}.


% }}}1

\section{Постановка задачи исследования}  % {{{1



У зв’язку з цим, метою дисертаційної роботи є створення
нових методів ідентифікації нелінійних динамічних систем,
для яких характерна хаотична динаміка,
моделей процесов ідентификації параметрів таких систем.

Відповідно до поставленої мети у дисертаційній роботі необхідно
вирішити такі завдання:

\begin{itemize}

  \item
    за рахунок аналізу характристик, які зберігаються навіть
    у хаотичних режимах,
    розробити нові критерії ідентифікації,
    які були б придатні для аналізу стану та динаміки
    хаотичних систем, що створить обґрунтування працездатності систем
    ідентифікації;

  \item
  розвинути існуючи та розробити нові методи пошуку, які б не мали
  обмежень тих, що існують, та у повної мірі використовували можливості
  паралельних обчислювань та переваг використання ансамблю
  синергированних моделей;

  \item
  розвинути існуючи та розробити нові методи адаптації параметрів
  системи ідентифікації, здатні пристосуватися до зміни режимів роботи
  системи;

  \item
  розробити програмне забезпечення, придатне для моделювання як систем
  хаотичної динаміки, так і систем ідентифікації;

  \item
  провести комп'ютерне моделювання процесів ідентифікації систем
  хаотичної динаміки та дослідити їх працездатність, можливості та
  характеристики.

  \item
    розробити нову модель хаотичної системи

  \item
    провести натурне моделювання

\end{itemize}



% }}}1








% }}}1


\section{Выводы по разделу \thechapter}  % {{{1


% }}}1


% vim: fdm=marker foldlevel=0 foldignore="%#" fdc=4 ft=tex
