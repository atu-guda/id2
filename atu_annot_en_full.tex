
\clearpage

\phantomsection
\pdfbookmark[0]{АBSTRACT}{annotarnbookmark}

\begin{center}
\textbf{АBSTRACT}
\end{center}

\medskip

\textit{\dissauthorEn} \booknameEn.
---
Qualifying scientific work as a manuscript.

Thesis for obtaining the doctoral degree in technical sciences in the specialty
\dissSpecId ``\dissSpecEn''. --- \institutionEn, \belongEn, \bookyear.

The dissertation is devoted to the actual scientific and technical problem of
identifying the complex technical systems parameters in the chaotic dynamics regime.
Nonlinear dynamic systems are usual in modern technological and natural processes,
and can exhibit chaotic properties in their dynamics.
The existing systems for identifying nonlinear dynamic objects are practically
useless in such conditions.

The existing identification methods and search dynamics models are analyzed.
Identified common features of identification systems, investigated factors that
do not allow them to be used in the parameter identification task for chaotic systems.
The necessity of creating new identification systems is grounded.

It is shown that the use of integral criteria is necessary
for the identification systems creation, that would be workable under conditions of chaotic
dynamics. Considered a couple of physical justification for the criteria syntesys,
both on the direct use of physical laws, and on empirical principles.

In order to improve the identification system speed,
a structure. which consist of a variety search agents and a search coordinatoris is proposed.
Methods for estimating the parameter of the triplet of search agents are created, both with
the direct use of the criterion and with the use of the quality function. The
dependence of the identification error on the method and its parameters is
investigated.

A group of methods for the work of the search coordinators has been created,
the dependencies of the identification quality on the parameters of the
coordinator's interaction with the agents have been determined. A
classification of system identification systems has been created, which allows
not only briefly describe the set of methods used, but also allows us to
create new identification systems when new methods of agents and coordinators
appear. The properties of the set of proposed methods for an artificial
quasistationary test system are investigated. A subset of methods is selected,
which it is expedient to use in further research.

The ``qontrol'' software was created, which is intended for modeling complex
nonlinear dynamical systems, including in the analysis of chaotic dynamics systems processes identification.

The working capacity and properties of the developed methods have been verified
in solving the problem of identifying several nonlinear systems, both known
chaotic dynamics systems, and new ones. It was determined that there are
systems that are not chaotic, but from the point of view of identification they
have common features. The influence of the parameters of the identification
system on the quality of its work is investigated, which gives grounds for the
adaptation of these parameters. It is determined that the type of criterion and
the dynamics of agents have the greatest influence.

The properties of the proposed identification methods are investigated on two
real physical objects: the Kolpitts oscillator and the system of coupled
relaxation elements. The methods have confirmed their efficiency, and it is
found that the dependence of the quality of identification on its parameters
for real systems has the same form as for the test ones.


\textbf{Keywords:}
chaotic systems,
identification criterion,
searching agent,
searching coordinator,
ensemble methods of the searching indetification.
