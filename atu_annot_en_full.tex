
\clearpage

\phantomsection
\pdfbookmark[0]{АBSTRACT}{annotarnbookmark}

\begin{center}
\textbf{АBSTRACT}
\end{center}

\medskip

\textit{\dissauthorEn} \booknameEn.
---
Qualifying scientific work as a manuscript.

Thesis for obtaining the doctoral degree in technical sciences in the specialty
\dissSpecId ``\dissSpecEn''. --- \institutionEn, \belongEn, \bookyear.

The dissertation is devoted to the actual scientific and technical problem of
identifying the complex technical systems parameters in the chaotic dynamics regime.
Nonlinear dynamic systems are usual in modern technological and natural processes,
and can exhibit chaotic properties in their dynamics.
The existing systems for identifying nonlinear dynamic objects are practically
useless in such conditions.

The existing identification methods and search dynamics models are analyzed.
Identified common features of identification systems, investigated factors that
do not allow them to be used in the parameter identification task for chaotic systems.
The necessity of creating new identification systems is grounded.

It is shown that the use of integral criteria is necessary
for the identification systems creation, that would be workable under conditions of chaotic
dynamics. Considered a couple of physical justification for the criteria syntesys,
both on the direct use of physical laws, and on empirical principles.

The object of research is the technical systems, which in the process of their operation may enter into chaotic regimes.

The subject of the study is mathematical models of processes and methods of
ensemble identification of technical systems with chaotic dynamics.

Methods of research: To solve the problems, the mathematical apparatus of the
theory of management and identification of nonlinear systems, dynamic chaos,
computational methods, fuzzy logic, information theory, etc. was used.

The practical value of the results obtained is that the elaborated
identification methods were used in designing, creating, adjusting the
parameters of the test bench for vibrating and acoustic effects. Analysis of
data results from this stand gave the opportunity to specify the necessary
nonlinear properties of the system and a range of parameters that collectively
provide a broadband spectrum of oscillations.

The created software environment for simulation of nonlinear dynamic systems is
used during practical work on disciplines ``Modeling of systems'', ``Modern
control systems'' at the department of information technologies and systems of
the National Metallurgical Academy of Ukraine.

In dissertation work the criteria of identification of nonlinear dynamical
systems, which, unlike existing ones, allow to estimate their state and chaotic
dynamics, and also give grounds for creation of effective algorithms of
adjustment of parameters of models of identification systems.

Identification methods based on the adaptive-search paradigm are created using
a group of search agents that interact with each other, which, unlike methods
that use one model or pair of models, greatly increase the search speed and are
able to adjust at a minimal time with a sharp change of parameters, and Unlike
in-game algorithms, new methods require a much smaller number of models and
provide certain search warranties.

A new classification of systems for the identification of dynamic systems has
been created, which, as it absorbs existing methods, allows the creation of new
methods of identification through the combination of their constituent parts.

The model of the chaotic dynamics system with the use of coupled relaxation
generators is proposed, which differs from the existing lack of inductive
components, low voltage operation and the possibility of controlling the
frequency range over a wide range, which contributes to the process of
analyzing the chaotic dynamics of a physical object, to check the adequacy of
the mathematical model and properties of the identification system.

Methods for assessing the quality of identification are improved, which, unlike
those existing, take into account the use of a plurality of agents.

Further development approaches to the adaptation of ensemble identification
systems parameters that are suitable for using current information from the
ensemble of synergy models and adjusting the global search parameters in a
priori and current uncertainty conditions.

The model of the Kolpitza generator is improved, which takes into account a
large number of nonlinear effects, which provides more adequate results of the
process of identifying its parameters by the proposed methods.

It is established that in order to solve the problem of identification it is
necessary to exist a criterion that reflects the similarity of the dynamics of
the object and models. In cases where it is difficult to formulate a criterion
based on strict physical principles, it makes sense to obtain the criterion of
a criterion obtained by interrogation, based on a limited set of
representations of integral quantities.
The notion of integral criterion implies a kind of averaging, which corresponds
to a certain extent with the use of low-frequency filtration.

The use of identification quality functions is necessary to determine the
achievement of the purpose of identification, obtaining the value of the
identified parameter by the search coordinator. Some methods of work of search
agents require values of these functions as an input signal.

The basic requirements for the properties of identification quality functions
are established, the basic methods of determination are considered.

A set of identification system structures based on the interaction of search
agents and search coordinators has been identified. The subclass of the
identification system used in the work of the dynamics of the ensemble of
search agents is determined.

The methods of the local estimation of the value of the identified parameter
for both search engines using the value as a criterion and the quality function
are determined.

The methods of task dynamics of search agents are determined. Methods for
obtaining the value of the identified parameter by the search coordinator are
determined.

The classification of ensemble search systems is created, which allows a short
and unambiguous determination of structure and dynamics, as well as the
creation of new methods, provided that the rules of operation of the elements
of search agents and search coordinators are defined.

On simple non-dynamic test tasks, the basic properties as separate elements of
ensemble identification systems and systems in general are studied.


In order to improve the identification system speed,
a structure. which consist of a variety search agents and a search coordinatoris is proposed.
Methods for estimating the parameter of the triplet of search agents are created, both with
the direct use of the criterion and with the use of the quality function. The
dependence of the identification error on the method and its parameters is
investigated.

A group of methods for the work of the search coordinators has been created,
the dependencies of the identification quality on the parameters of the
coordinator's interaction with the agents have been determined. A
classification of system identification systems has been created, which allows
not only briefly describe the set of methods used, but also allows us to
create new identification systems when new methods of agents and coordinators
appear. The properties of the set of proposed methods for an artificial
quasistationary test system are investigated. A subset of methods is selected,
which it is expedient to use in further research.



The ``qontrol'' software was created, which is intended for modeling complex
nonlinear dynamical systems, including in the analysis of chaotic dynamics systems processes identification.

The working capacity and properties of the developed methods have been verified
in solving the problem of identifying several nonlinear systems, both known
chaotic dynamics systems, and new ones. It was determined that there are
systems that are not chaotic, but from the point of view of identification they
have common features. The influence of the parameters of the identification
system on the quality of its work is investigated, which gives grounds for the
adaptation of these parameters. It is determined that the type of criterion and
the dynamics of agents have the greatest influence.

The properties of the proposed identification methods are investigated on two
real physical objects: the Kolpitts oscillator and the system of coupled
relaxation elements. The methods have confirmed their efficiency, and it is
found that the dependence of the quality of identification on its parameters
for real systems has the same form as for the test ones.


\textbf{Keywords:}
chaotic systems,
identification criterion,
searching agent,
searching coordinator,
ensemble methods of the searching indetification.
